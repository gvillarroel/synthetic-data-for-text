


\chapter{Listado de figuras}
\section{Lista completa de figura pairwise kingcounty}
\label{A-pairwise-kingcounty-top2-a-1}
\begin{figure}[H]
    \centering
    \includesvg[scale=.6,inkscapelatex=false]{datasets/kingcounty-a-1/pairwise/copulagan.svg}
    \caption{Correlación de conjunto original de entrenamiento y Copulagan, King county (A-1)}
    \label{pairwise-king county-a-1-copulagan}
\end{figure}
\begin{figure}[H]
    \centering
    \includesvg[scale=.6,inkscapelatex=false]{datasets/kingcounty-a-1/pairwise/tvae.svg}
    \caption{Correlación de conjunto original de entrenamiento y Tvae}
    \label{pairwise-king county-a-1-tvae}
\end{figure}
\begin{figure}[H]
    \centering
    \includesvg[scale=.6,inkscapelatex=false]{datasets/kingcounty-a-1/pairwise/gaussiancopula.svg}
    \caption{Correlación de conjunto original de entrenamiento y Gaussiancopula, King county (A-1)}
    \label{pairwise-king county-a-1-gaussiancopula}
\end{figure}
\begin{figure}[H]
    \centering
    \includesvg[scale=.5,inkscapelatex=false]{datasets/kingcounty-a-1/pairwise/ctgan.svg}
    \caption{Correlación de conjunto Real y Modelo: ctgan}
    \label{pairwise-king county-a-1-ctgan}
\end{figure}
\begin{figure}[H]
    \centering
    \includesvg[scale=.7,inkscapelatex=false]{datasets/kingcounty-a-1/pairwise/tablepreset.svg}
    \caption{Correlación de conjunto Real y Modelo: tablepreset}
    \label{pairwise-king county-a-1-tablepreset}
\end{figure}
\begin{figure}[H]
    \centering
    \includesvg[scale=.6,inkscapelatex=false]{datasets/kingcounty-a-1/pairwise/smote-enc.svg}
    \caption{Correlación de conjunto original de entrenamiento y Smote}
    \label{pairwise-king county-a-1-smote-enc}
\end{figure}
\begin{figure}[H]
    \centering
    \includesvg[scale=.6,inkscapelatex=false]{datasets/kingcounty-a-1/pairwise/tddpm_mlp.svg}
    \caption{Correlación de conjunto original de entrenamiento y Tddpm}
    \label{pairwise-king county-a-1-tddpm_mlp}
\end{figure}


\section{Smote y Tddpm en KingCounty Gráficas por Columnas}

\begin{figure}[H]
    \centering
    \includesvg[scale=.5,inkscapelatex=false]{datasets/economicos-a-3/smote-enc/privacy.svg}
    \caption{Frecuencia del campo privacy en el modelo real y smote-enc}
    \label{frecuency-Privacy-smote-enc}
\end{figure}
\begin{figure}[H]
    \centering
    \includesvg[scale=.7,inkscapelatex=false]{datasets/economicos-b-3/top2/rooms.svg}
    \caption{Frecuencia del campo Rooms en el modelo real y Top 2, Economicos (B-3)}
    \label{frecuency-Rooms-top2}
\end{figure}
\begin{figure}[H]
    \centering
    \includesvg[scale=.7,inkscapelatex=false]{datasets/economicos-a-3/smote-enc/publication_date.svg}
    \caption{Frecuencia del campo Publication date en el modelo real y smote, Economicos (A-3)}
    \label{frecuency-Publication Date-smote-enc}
\end{figure}
\begin{figure}[H]
    \centering
    \includesvg[scale=.7,inkscapelatex=false]{datasets/economicos-b-1/top2+1/_price.svg}
    \caption{Frecuencia del campo  price en el modelo real y top2+1, Economicos (B-1)}
    \label{frecuency- Price-top2+1}
\end{figure}
\begin{figure}[H]
    \centering
    \includesvg[scale=.7,inkscapelatex=false]{datasets/economicos-b-3/top2/property_type.svg}
    \caption{Frecuencia del campo Property type en el modelo real y Top 2, Economicos (B-3)}
    \label{frecuency-Property Type-top2}
\end{figure}
\begin{figure}[H]
    \centering
    \includesvg[scale=.7,inkscapelatex=false]{datasets/economicos-a-2/top2/bathrooms.svg}
    \caption{Frecuencia del campo Bathrooms en el modelo real y Top 2, Economicos (A-2)}
    \label{frecuency-Bathrooms-top2}
\end{figure}
\begin{figure}[H]
    \centering
    \includesvg[scale=.7,inkscapelatex=false]{datasets/economicos-b-1/tddpm_mlp/m_built.svg}
    \caption{Frecuencia del campo M built en el modelo real y tddpm, Economicos (B-1)}
    \label{frecuency-M Built-tddpm_mlp}
\end{figure}
\begin{figure}[H]
    \centering
    \includesvg[scale=.7,inkscapelatex=false]{datasets/economicos-a-1/tddpm_mlp/transaction_type.svg}
    \caption{Frecuencia del campo Transaction type en el modelo real y tddpm, Economicos (A-1)}
    \label{frecuency-Transaction Type-tddpm_mlp}
\end{figure}
\begin{figure}[H]
    \centering
    \includesvg[scale=.7,inkscapelatex=false]{datasets/economicos-b-3/top2/county.svg}
    \caption{Frecuencia del campo County en el modelo real y Top 2, Economicos (B-3)}
    \label{frecuency-County-top2}
\end{figure}
\begin{figure}[H]
    \centering
    \includesvg[scale=.7,inkscapelatex=false]{datasets/economicos-b-3/tddpm_mlp/state.svg}
    \caption{Frecuencia del campo State en el modelo real y tddpm}
    \label{frecuency-State-tddpm_mlp}
\end{figure}
\begin{figure}[H]
    \centering
    \includesvg[scale=.5,inkscapelatex=false]{datasets/economicos-a-1/top2+1/m_size.svg}
    \caption{Frecuencia del campo m size en el modelo real y top2+1}
    \label{frecuency-M Size-top2+1}
\end{figure}


\section{Figuras de correlación Económicos - Conjunto A}
\label{pairwise-full-a}
\begin{figure}[H]
    \centering
    \includesvg[scale=.6,inkscapelatex=false]{datasets/kingcounty-a-1/pairwise/copulagan.svg}
    \caption{Correlación de conjunto original de entrenamiento y Copulagan, King county (A-1)}
    \label{pairwise-king county-a-1-copulagan}
\end{figure}
\begin{figure}[H]
    \centering
    \includesvg[scale=.6,inkscapelatex=false]{datasets/kingcounty-a-1/pairwise/tvae.svg}
    \caption{Correlación de conjunto original de entrenamiento y Tvae}
    \label{pairwise-king county-a-1-tvae}
\end{figure}
\begin{figure}[H]
    \centering
    \includesvg[scale=.6,inkscapelatex=false]{datasets/kingcounty-a-1/pairwise/gaussiancopula.svg}
    \caption{Correlación de conjunto original de entrenamiento y Gaussiancopula, King county (A-1)}
    \label{pairwise-king county-a-1-gaussiancopula}
\end{figure}
\begin{figure}[H]
    \centering
    \includesvg[scale=.5,inkscapelatex=false]{datasets/kingcounty-a-1/pairwise/ctgan.svg}
    \caption{Correlación de conjunto Real y Modelo: ctgan}
    \label{pairwise-king county-a-1-ctgan}
\end{figure}
\begin{figure}[H]
    \centering
    \includesvg[scale=.7,inkscapelatex=false]{datasets/kingcounty-a-1/pairwise/tablepreset.svg}
    \caption{Correlación de conjunto Real y Modelo: tablepreset}
    \label{pairwise-king county-a-1-tablepreset}
\end{figure}
\begin{figure}[H]
    \centering
    \includesvg[scale=.6,inkscapelatex=false]{datasets/kingcounty-a-1/pairwise/smote-enc.svg}
    \caption{Correlación de conjunto original de entrenamiento y Smote}
    \label{pairwise-king county-a-1-smote-enc}
\end{figure}
\begin{figure}[H]
    \centering
    \includesvg[scale=.6,inkscapelatex=false]{datasets/kingcounty-a-1/pairwise/tddpm_mlp.svg}
    \caption{Correlación de conjunto original de entrenamiento y Tddpm}
    \label{pairwise-king county-a-1-tddpm_mlp}
\end{figure}



\section{Figuras de correlación Económicos - Conjunto B}
\label{pairwise-full-a}
\begin{figure}[H]
    \centering
    \includesvg[scale=.6,inkscapelatex=false]{datasets/kingcounty-a-1/pairwise/copulagan.svg}
    \caption{Correlación de conjunto original de entrenamiento y Copulagan, King county (A-1)}
    \label{pairwise-king county-a-1-copulagan}
\end{figure}
\begin{figure}[H]
    \centering
    \includesvg[scale=.6,inkscapelatex=false]{datasets/kingcounty-a-1/pairwise/tvae.svg}
    \caption{Correlación de conjunto original de entrenamiento y Tvae}
    \label{pairwise-king county-a-1-tvae}
\end{figure}
\begin{figure}[H]
    \centering
    \includesvg[scale=.6,inkscapelatex=false]{datasets/kingcounty-a-1/pairwise/gaussiancopula.svg}
    \caption{Correlación de conjunto original de entrenamiento y Gaussiancopula, King county (A-1)}
    \label{pairwise-king county-a-1-gaussiancopula}
\end{figure}
\begin{figure}[H]
    \centering
    \includesvg[scale=.5,inkscapelatex=false]{datasets/kingcounty-a-1/pairwise/ctgan.svg}
    \caption{Correlación de conjunto Real y Modelo: ctgan}
    \label{pairwise-king county-a-1-ctgan}
\end{figure}
\begin{figure}[H]
    \centering
    \includesvg[scale=.7,inkscapelatex=false]{datasets/kingcounty-a-1/pairwise/tablepreset.svg}
    \caption{Correlación de conjunto Real y Modelo: tablepreset}
    \label{pairwise-king county-a-1-tablepreset}
\end{figure}
\begin{figure}[H]
    \centering
    \includesvg[scale=.6,inkscapelatex=false]{datasets/kingcounty-a-1/pairwise/smote-enc.svg}
    \caption{Correlación de conjunto original de entrenamiento y Smote}
    \label{pairwise-king county-a-1-smote-enc}
\end{figure}
\begin{figure}[H]
    \centering
    \includesvg[scale=.6,inkscapelatex=false]{datasets/kingcounty-a-1/pairwise/tddpm_mlp.svg}
    \caption{Correlación de conjunto original de entrenamiento y Tddpm}
    \label{pairwise-king county-a-1-tddpm_mlp}
\end{figure}


\chapter{Ejemplos de textos generados}

\section{Ejemplos de 5 Registros Generados Aleatoriamente en Descripciones Económicas A-1}
\label{ejemplo-10-aleatoreos-a}
\begin{table}[H]
\centering
\fontsize{8}{14}\selectfont
\caption{Ejemplos de textos aleatoreos del modelo tddpm\_mlp, conjunto Economicos a-1}
\label{table-sample10-economicos-a-1-tddpm_mlp-text}
\begin{tabular}{|m{50em}|}
\hline
\rowcolor[gray]{0.8}
description \\
\hline Departamento de 2 pisos, 1 baño con salida a terraza en living comedor que comparte un gran jardín para una amplia vista panorámica al barrio del Barrio Los Barneches (principal) el edificio cuenta con lavandería completamente equipada por las ventanas termopanel/calefacción central-conserjería techada +56 9 7 8 6 5 4 \\
\hline Piso de madera en el primer piso, living comedor con salida a terraza que tienen 2 baños completos (incluye una sala de estar) Living comedor amplio para lavadora separados por un bodega altura del hall de entrada; logia equipada como estacionamiento techado cerrado! Calefacción central controlada: Horno termopanel/Horno Termopanel-Calefacción radiador trasero Condominio Amplio Hall de entrada Cocina amoblada Amueblado 3 estacionamientos 1 Baño Edificio cuenta con dos salas de reuniones + quincho \\
\hline Piso de madera en el primer piso, living comedor con salida a terraza que tienen 2 baños completos (incluye una sala de estar) Living comedor amplio para dos autos (2 estacionamientos) Cocina equipada completamente equipada Con vista panorámica al jardín El tercer piso cuenta con un gran hall de entrada por lavandería Calefacción central Azotea Techada: 1 bodega \\
\hline Piso de madera en el living comedor con salida a la terraza, 2 baños completos que comparten una gran sala de estar para dos autos (incluye un bodega). \\
\hline Departamento de 2 pisos, 1 baño completo, living comedor con salida a terraza en el primer piso: Living comedor amplio (cocina equipada) Cocina completamente equipada Con vista al jardín El departamento cuenta con una sala de estar Amplio Hall de entrada Calefacción por radiadores Ventanas termopanel Termopanel Azotea Techada Lavandería Terraza 3 estacionamientos 4 bodega Sala de reuniones Espacio para lavadora Recepción Baño principal Encimera Horno Depósitos \\
\hline Departamento de 2 pisos, 1 baño con salida a terraza en living comedor que comparten un gran jardín para una amplia vista panorámica al barrio del Barrio Los Barneches (principal) el edificio cuenta con lavandería completamente equipada por las ventanas termopanel/calefacción central-congreso techado; sala de reuniones + quincho amplio: logia cerrada! \\
\hline Piso de madera en el primer piso, living comedor con salida a terraza que tiene una vista panorámica al centro del barrio Concepción (condominio). Calefacción por radiadores para la ventanas termopaneles 2 bodega 3 estacionamientos 4 estacionamientos 1 baño completo Living Comedor Cocina equipada Amplia sala de estar Ascensor: Sala de Estar Edificio cuenta con un gran jardín Techado La casa está construida como uno de los dos pisos principals separados El tercer dormitorio es amplio hall de entrada Terraza completamente iluminada Encimera Ventanas Termopanel Horno Derecho Terreno Total Construcción Construida Completamente Equipada! Gran Jardín Propiedad Se Vende esta propiedad consta de 5 habitaciones (2 baños) Hall de acceso Baño Principal Superficie total aproximada \$100.000 \\
\hline Piso de madera en el primer piso, living comedor con salida a terraza que tienen 2 baños completos (incluye una sala de estar) Living comedor amplio Condominio Amplio Hall de entrada Calefacción por radiadores Ventanas termopanel Termopanel Cocina amoblada Edificio cuenta con lavandería Lavadora Azotea Techada: 1 bodega \\
\hline Excelente conectividad, cercana a la ciudad de Coquimbo con acceso directo al centro comercial del sector principal en el primer piso: 2 baños completos (incluye una sala de estar) Living comedor amplio que cuenta con salida a terraza completamente equipada para dos vehículos; 1 bodega techada por un gran jardín como estacionamiento privado se entregan los siguientes elementos: 3 estacionamientos + quincho \\
\hline Piso de madera en el primer piso, living comedor con salida a terraza que tiene una vista panorámica al barrio del Barrio Los Barneches por la zona principal (con baño) Living comedor amplio para 2 vehículos (2 autos), dos estacionamientos (1 bodega). Cocina equipada completamente equipada como sala de reuniones 1 Baño completo Amplio Hall de entrada Consta un hall de entrada Calefacción central Azotea Techada Encimera Ventanas termopanel Termopanel Lavandería Jardín Edificio cuenta con: Sala de Estar Terraza Gran jardín 3 pisos 4 estacionamiento 5 bodegas \\
\hline
\end{tabular}
\end{table}


\section{Ejemplos de 5 Registros Generados Aleatoriamente en Descripciones Económicas B-1}
\label{ejemplo-10-aleatoreos-b}
\begin{table}[H]
\centering
\fontsize{8}{14}\selectfont
\caption{Ejemplos de textos aleatoreos del modelo Tddpm, conjunto Economicos (B-1)}
\label{table-sample10-economicos-b-1-tddpm_mlp-text}
\begin{tabular}{|m{50em}|}
\hline
\rowcolor[gray]{0.8}
description \\
\hline Casa 3 dormitorios, 2 baños (principal en suite) Living comedor Cocina Americana bodega Antejardín Trasero Terraza Cercana al supermercado Jumbo. A pasos de locomoción colectiva Los Cuevas con Avda Quilpué Sector tranquilo Lo Ovalle sector La Paz Cuenta con estacionamiento techado para dos autos más 1 portón pequeñísimo Te invitamos! Contactar: +56947923434 Fonó:993450547 Corredora Propiedades Mónica Alcántara\-Corregidora +56 9 9629 9917 Email contacto@propiedadesalcantaracorredoreshipotecariosdelcampohotmailcvnrlazdelimavalco Estamos interesados \\
\hline Departamento de 55 m2 en San Francisco, Las Condes. Consta con living comedor por separado uno amplio para ampliar un baño completo; Cocina amoblada equipada (hornos empotrados), hornito \& campana incluidos 2 estacionamientos 1 bodega Valor: \$35.000 mensuales Código Axtep 4642 Edificio cuenta entre las siguientes funciones principales que incluyen acceso controlado 24 horas Cercano al metro Manquehue Además se reciben visitas cercanas lavandería Sala multiuso Quincho Salón MultiusOS Lavadora Gasto Común \$35.000 aprox Contactar directo sin comisión Corredora Para coordinación diurno durante el día \\
\hline CASA 4 DORMITORIOS, 2 BAÑOS ANTEJARDIN ENTRADA DE AUTO SECTOR A PASO DEL CENTRU DÍA Y NOCHE. OPORTUNIDAD! 982411601 C/C RUA PROPIEDADES Gestión Inmobiliaria www,inmuebleschile@gmail.com contacto: +56995811450-9-93604547 – 996519395 +56938945099 56.000UF M2 Construidos 100M2. Propiedades Chile \#818 VENDE SU HERMOSO CONDOMINIO SIN ESTADIO EN LA COMUNA (RUTA 3 SUR), VILLA HERRERA BUEN TERRENOS ESTRUCTURAS CONSTRUIDAS 670.000FUNDACION AMPLIAS Casa 1er Pasillo Amplio living comedor con amplios terrazas gran jardín entrada de auto amplia portón cerrado \\
\hline Cómodo depto amoblado en Santiago, La Comuna. Valor \$75.000 + gastos comunes (consumo agua incluidos) Baño con tina Cocina Equipada Tel: 942261144/991481133-/994171341 Acop@gmail.com Casa central completamente equipada para 3 personas 1 baños Amplie living comedor Amobladas todo tipo suministro al interior del DPTO En el momento que se solicita la entrega Se pide garantía por arriendo realmente solo si es una persona especial.... Soy buscando un lugar tranquilo cercanó las estaciones Metro Manquehuen Colegio Santa Isabel Las Condes Centro Propiedades Llamar sólidos interesados sin compromiso indefinición \\
\hline Arriendo de derechos al certificado comercial (patente en vivo) Se vende exclusivo cafetería para persona sola, con instalaciones el año 2018 que cuentan incluyendo café + barra+ terraza. Atención por sus principales necesidades: acceso directo directamente mediante televisión telefónica ou whatsapp 5697688270 994323000 Dueño https://www.instagram.com/la\_plazaybarcl?utm\#action=opensearch\&document\%3F\%26r=9G9Z4N0R8B1O-tj5A2K7LgJ *No se aceptan transacciones* Precio venta conversable Valor arriendo negociables es solo una oferta individual... ¡Sin estacionamiento ni propiedad!!! No tiene bodega(libre), no necesita pagar gastos comunes! \\
\hline
\end{tabular}
\end{table}


\chapter{Estadísticos}
\section{Estadísticos KingCounty}
\label{propiedades-estadisticas-kingCounty}
\begin{table}[H]
\centering
\fontsize{8}{14}\selectfont
\caption{Propiedades  estadisticas de variable rooms, Economicos (A-2)}
\label{table-stats-economicos-a-2-rooms}
\begin{tabular}{|l|m{10em}|m{10em}|m{10em}|m{10em}|}
\hline
 \rowcolor[gray]{0.8}
Variable/Modelo & Real & tddpm\_mlp & smote-enc & ctgan \\
\hline top5 & [3. 2. 4. 1. 5.] & [3. 2. 4. 1. 5.] & [3. 2. 4. 1. 5.] & [2. 3. 4. 5. 6.] \\
\hline top5\_freq & [6355 4614 4168 2671 2232] & [8216 5798 5162 3327 2911] & [8191 5804 5468 3245 2819] & [6541 5670 4070 2674 2058] \\
\hline top5\_prob & [0.28809103 0.20916633 0.18894782 0.12108436 0.10118319] & [0.29796185 0.21027054 0.18720534 0.12065714 0.10557046] & [0.2970552  0.21048814 0.19830275 0.11768332 0.10223399] & [0.23721622 0.20562849 0.14760281 0.09697541 0.07463553] \\
\hline nobs & 22059 & 27574 & 27574 & 27574 \\
\hline missing & 0.000 & 0.000 & 0.000 & 0.000 \\
\hline mean & 3.446 & \bfseries 3.347 & 3.310 & \cellcolor[rgb]{0.9, 0.54, 0.52} 5.000 \\
\hline std\_err & 0.026 & \bfseries 0.024 & 0.012 & \cellcolor[rgb]{0.9, 0.54, 0.52} 0.047 \\
\hline upper\_ci & 3.497 & \bfseries 3.393 & 3.334 & \cellcolor[rgb]{0.9, 0.54, 0.52} 5.092 \\
\hline lower\_ci & 3.395 & \bfseries 3.301 & 3.286 & \cellcolor[rgb]{0.9, 0.54, 0.52} 4.909 \\
\hline std & 3.881 & \bfseries 3.924 & 1.993 & \cellcolor[rgb]{0.9, 0.54, 0.52} 7.749 \\
\hline iqr & 2.000 & \bfseries 2.000 & \bfseries 2.000 & \cellcolor[rgb]{0.9, 0.54, 0.52} 3.000 \\
\hline iqr\_normal & 1.483 & \bfseries 1.483 & \bfseries 1.483 & \cellcolor[rgb]{0.9, 0.54, 0.52} 2.224 \\
\hline mad & 1.454 & \bfseries 1.340 & 1.279 & \cellcolor[rgb]{0.9, 0.54, 0.52} 3.061 \\
\hline mad\_normal & 1.822 & \bfseries 1.679 & 1.603 & \cellcolor[rgb]{0.9, 0.54, 0.52} 3.836 \\
\hline coef\_var & 1.126 & \bfseries 1.172 & \cellcolor[rgb]{0.9, 0.54, 0.52} 0.602 & 1.550 \\
\hline range & 399.000 & \bfseries 399.000 & \cellcolor[rgb]{0.9, 0.54, 0.52} 56.000 & \bfseries 399.000 \\
\hline max & 400.000 & \bfseries 400.000 & \cellcolor[rgb]{0.9, 0.54, 0.52} 57.000 & \bfseries 400.000 \\
\hline min & 1.000 & 1.000 & 1.000 & 1.000 \\
\hline skew & 57.785 & \bfseries 75.482 & \cellcolor[rgb]{0.9, 0.54, 0.52} 4.962 & 25.543 \\
\hline kurtosis & 5331 & \bfseries 7575 & \cellcolor[rgb]{0.9, 0.54, 0.52} 68 & 1106 \\
\hline jarque\_bera & 2.61061e+10 & \cellcolor[rgb]{0.9, 0.54, 0.52} 6.59026e+10 & 4.96441e+06 & \bfseries 1.40103e+09 \\
\hline jarque\_bera\_pval & 0.000 & 0.000 & 0.000 & 0.000 \\
\hline mode & 3.000 & \bfseries 3.000 & \bfseries 3.000 & \cellcolor[rgb]{0.9, 0.54, 0.52} 2.000 \\
\hline mode\_freq & 0.288 & 0.298 & \bfseries 0.297 & \cellcolor[rgb]{0.9, 0.54, 0.52} 0.237 \\
\hline median & 3.000 & 3.000 & 3.000 & 3.000 \\
\hline 0.1\% & 1.000 & 1.000 & 1.000 & 1.000 \\
\hline 1.0\% & 1.000 & 1.000 & 1.000 & 1.000 \\
\hline 5.0\% & 1.000 & 1.000 & 1.000 & 1.000 \\
\hline 25.0\% & 2.000 & 2.000 & 2.000 & 2.000 \\
\hline 75.0\% & 4.000 & \bfseries 4.000 & \bfseries 4.000 & \cellcolor[rgb]{0.9, 0.54, 0.52} 5.000 \\
\hline 95.0\% & 6.000 & \bfseries 6.000 & \bfseries 6.000 & \cellcolor[rgb]{0.9, 0.54, 0.52} 14.000 \\
\hline 99.0\% & 12.000 & \bfseries 10.000 & \bfseries 10.000 & \cellcolor[rgb]{0.9, 0.54, 0.52} 25.000 \\
\hline 99.9\% & 25.000 & 24.000 & \bfseries 25.000 & \cellcolor[rgb]{0.9, 0.54, 0.52} 57.000 \\
\hline
\end{tabular}
\end{table}

\begin{table}[H]
\centering
\fontsize{8}{14}\selectfont
\caption{Propiedades  estadisticas de variable bathrooms, Economicos (A-2)}
\label{table-stats-economicos-a-2-bathrooms}
\begin{tabular}{|l|m{10em}|m{10em}|m{10em}|m{10em}|}
\hline
 \rowcolor[gray]{0.8}
Variable/Modelo & Real & tddpm\_mlp & smote-enc & ctgan \\
\hline top5 & [2. 1. 3. 4. 5.] & [2. 1. 3. 4. 5.] & [2. 1. 3. 4. 5.] & [3. 1. 2. 4. 5.] \\
\hline top5\_freq & [7511 5440 4486 2665 1084] & [9481 6610 5861 3425 1331] & [9438 6742 5612 3270 1387] & [7474 5459 4858 3762 3251] \\
\hline top5\_prob & [0.34049594 0.24661136 0.20336371 0.12081237 0.04914094] & [0.3438384  0.23971858 0.21255531 0.12421121 0.04827011] & [0.34227896 0.24450569 0.20352506 0.11858998 0.05030101] & [0.27105244 0.19797635 0.17618046 0.13643287 0.11790092] \\
\hline nobs & 22059 & 27574 & 27574 & 27574 \\
\hline missing & 0.000 & 0.000 & 0.000 & 0.000 \\
\hline mean & 2.604 & 2.544 & \bfseries 2.593 & \cellcolor[rgb]{0.9, 0.54, 0.52} 3.679 \\
\hline std\_err & 0.025 & \bfseries 0.018 & 0.013 & \cellcolor[rgb]{0.9, 0.54, 0.52} 0.058 \\
\hline upper\_ci & 2.652 & 2.579 & \bfseries 2.618 & \cellcolor[rgb]{0.9, 0.54, 0.52} 3.792 \\
\hline lower\_ci & 2.556 & 2.508 & \bfseries 2.568 & \cellcolor[rgb]{0.9, 0.54, 0.52} 3.566 \\
\hline std & 3.655 & \bfseries 3.001 & 2.133 & \cellcolor[rgb]{0.9, 0.54, 0.52} 9.573 \\
\hline iqr & 1.000 & \bfseries 1.000 & \bfseries 1.000 & \cellcolor[rgb]{0.9, 0.54, 0.52} 2.000 \\
\hline iqr\_normal & 0.741 & \bfseries 0.741 & \bfseries 0.741 & \cellcolor[rgb]{0.9, 0.54, 0.52} 1.483 \\
\hline mad & 1.203 & 1.114 & \bfseries 1.185 & \cellcolor[rgb]{0.9, 0.54, 0.52} 2.020 \\
\hline mad\_normal & 1.507 & 1.396 & \bfseries 1.485 & \cellcolor[rgb]{0.9, 0.54, 0.52} 2.532 \\
\hline coef\_var & 1.404 & \bfseries 1.180 & 0.822 & \cellcolor[rgb]{0.9, 0.54, 0.52} 2.602 \\
\hline range & 435.000 & \bfseries 435.000 & \cellcolor[rgb]{0.9, 0.54, 0.52} 179.000 & \bfseries 435.000 \\
\hline max & 436.000 & \bfseries 436.000 & \cellcolor[rgb]{0.9, 0.54, 0.52} 180.000 & \bfseries 436.000 \\
\hline min & 1.000 & 1.000 & 1.000 & 1.000 \\
\hline skew & 82.448 & \bfseries 109.783 & \cellcolor[rgb]{0.9, 0.54, 0.52} 28.339 & 38.404 \\
\hline kurtosis & 9252 & \bfseries 15795 & 2009 & \cellcolor[rgb]{0.9, 0.54, 0.52} 1685 \\
\hline jarque\_bera & 7.86518e+10 & \cellcolor[rgb]{0.9, 0.54, 0.52} 2.86587e+11 & \bfseries 4.62894e+09 & 3.25889e+09 \\
\hline jarque\_bera\_pval & 0.000 & 0.000 & 0.000 & 0.000 \\
\hline mode & 2.000 & \bfseries 2.000 & \bfseries 2.000 & \cellcolor[rgb]{0.9, 0.54, 0.52} 3.000 \\
\hline mode\_freq & 0.340 & 0.344 & \bfseries 0.342 & \cellcolor[rgb]{0.9, 0.54, 0.52} 0.271 \\
\hline median & 2.000 & \bfseries 2.000 & \bfseries 2.000 & \cellcolor[rgb]{0.9, 0.54, 0.52} 3.000 \\
\hline 0.1\% & 1.000 & 1.000 & 1.000 & 1.000 \\
\hline 1.0\% & 1.000 & 1.000 & 1.000 & 1.000 \\
\hline 5.0\% & 1.000 & 1.000 & 1.000 & 1.000 \\
\hline 25.0\% & 2.000 & 2.000 & 2.000 & 2.000 \\
\hline 75.0\% & 3.000 & \bfseries 3.000 & \bfseries 3.000 & \cellcolor[rgb]{0.9, 0.54, 0.52} 4.000 \\
\hline 95.0\% & 5.000 & \bfseries 5.000 & \bfseries 5.000 & \cellcolor[rgb]{0.9, 0.54, 0.52} 8.000 \\
\hline 99.0\% & 8.000 & 7.000 & \bfseries 8.000 & \cellcolor[rgb]{0.9, 0.54, 0.52} 14.000 \\
\hline 99.9\% & 17.942 & 13.000 & \bfseries 17.427 & \cellcolor[rgb]{0.9, 0.54, 0.52} 57.000 \\
\hline
\end{tabular}
\end{table}

\begin{table}[H]
\centering
\fontsize{8}{14}\selectfont
\caption{Propiedades  estadisticas de variable \_price, Economicos (A-2)}
\label{table-stats-economicos-a-2-_price}
\begin{tabular}{|l|m{10em}|m{10em}|m{10em}|m{10em}|}
\hline
 \rowcolor[gray]{0.8}
Variable/Modelo & Real & tddpm\_mlp & smote-enc & ctgan \\
\hline top5 & [12500. 10500. 11500.  8500.  9000.] & [11500. 12500.  8500. 20000. 13500.] & [10500. 11500.  2100.  9000.  9500.] & [    0.          1873.91702513 41042.06971567 41062.90431098
 41066.23801271] \\
\hline top5\_freq & [104  99  91  86  85] & [102 101  99  99  94] & [165 148 142 141 139] & [17440     2     1     1     1] \\
\hline top5\_prob & [0.00471463 0.00448796 0.0041253  0.00389864 0.0038533 ] & [0.00369914 0.00366287 0.00359034 0.00359034 0.00340901] & [0.0059839  0.00536738 0.00514978 0.00511351 0.00504098] & [6.32479872e-01 7.25320955e-05 3.62660477e-05 3.62660477e-05
 3.62660477e-05] \\
\hline nobs & 22059 & 27574 & 27574 & 27574 \\
\hline missing & 0.000 & 0.000 & 0.000 & 0.000 \\
\hline mean & 110379 & \bfseries 52905 & 30808 & \cellcolor[rgb]{0.9, 0.54, 0.52} 12067 \\
\hline std\_err & 32746 & \bfseries 19395 & 12251 & \cellcolor[rgb]{0.9, 0.54, 0.52} 130 \\
\hline upper\_ci & 174559 & \bfseries 90919 & 54820 & \cellcolor[rgb]{0.9, 0.54, 0.52} 12322 \\
\hline lower\_ci & 46199 & \bfseries 14891 & \cellcolor[rgb]{0.9, 0.54, 0.52} 6797 & 11812 \\
\hline std & 4863477 & \bfseries 3220683 & 2034332 & \cellcolor[rgb]{0.9, 0.54, 0.52} 21633 \\
\hline iqr & 9959 & \bfseries 9948 & 10425 & \cellcolor[rgb]{0.9, 0.54, 0.52} 16851 \\
\hline iqr\_normal & 7383 & \bfseries 7374 & 7728 & \cellcolor[rgb]{0.9, 0.54, 0.52} 12492 \\
\hline mad & 202281 & \bfseries 88001 & 44740 & \cellcolor[rgb]{0.9, 0.54, 0.52} 16342 \\
\hline mad\_normal & 253522 & \bfseries 110293 & 56073 & \cellcolor[rgb]{0.9, 0.54, 0.52} 20481 \\
\hline coef\_var & 44.062 & \bfseries 60.877 & 66.032 & \cellcolor[rgb]{0.9, 0.54, 0.52} 1.793 \\
\hline range & 390000000 & \bfseries 362636196 & 279000000 & \cellcolor[rgb]{0.9, 0.54, 0.52} 124189 \\
\hline max & 390000000 & \bfseries 362636196 & 279000000 & \cellcolor[rgb]{0.9, 0.54, 0.52} 124189 \\
\hline min & 0.000 & 0.000 & 0.000 & 0.000 \\
\hline skew & 60.579 & \bfseries 88.292 & 114.793 & \cellcolor[rgb]{0.9, 0.54, 0.52} 1.959 \\
\hline kurtosis & 4067 & 8564 & \cellcolor[rgb]{0.9, 0.54, 0.52} 14362 & \bfseries 6 \\
\hline jarque\_bera & 1.51936e+10 & 8.42493e+10 & \cellcolor[rgb]{0.9, 0.54, 0.52} 2.36952e+11 & \bfseries 2.97507e+04 \\
\hline jarque\_bera\_pval & 0.000 & 0.000 & 0.000 & 0.000 \\
\hline mode & 12500 & \bfseries 11500 & 10500 & \cellcolor[rgb]{0.9, 0.54, 0.52} 0 \\
\hline mode\_freq & 0.005 & \bfseries 0.004 & 0.006 & \cellcolor[rgb]{0.9, 0.54, 0.52} 0.632 \\
\hline median & 5084 & 5194 & \bfseries 5085 & \cellcolor[rgb]{0.9, 0.54, 0.52} 0 \\
\hline 0.1\% & 0.263 & 0.417 & \bfseries 0.287 & \cellcolor[rgb]{0.9, 0.54, 0.52} 0.000 \\
\hline 1.0\% & 6.270 & 7.795 & \bfseries 7.060 & \cellcolor[rgb]{0.9, 0.54, 0.52} 0.000 \\
\hline 5.0\% & 11.760 & 12.100 & \bfseries 11.902 & \cellcolor[rgb]{0.9, 0.54, 0.52} 0.000 \\
\hline 25.0\% & 2041 & 2162 & \bfseries 2065 & \cellcolor[rgb]{0.9, 0.54, 0.52} 0 \\
\hline 75.0\% & 12000 & \bfseries 12109 & 12490 & \cellcolor[rgb]{0.9, 0.54, 0.52} 16851 \\
\hline 95.0\% & 32000 & 30839 & \bfseries 32000 & \cellcolor[rgb]{0.9, 0.54, 0.52} 62115 \\
\hline 99.0\% & 58942 & 53875 & \bfseries 55000 & \cellcolor[rgb]{0.9, 0.54, 0.52} 88157 \\
\hline 99.9\% & 262695 & 110019 & \bfseries 120000 & \cellcolor[rgb]{0.9, 0.54, 0.52} 109231 \\
\hline
\end{tabular}
\end{table}

\begin{table}[H]
\centering
\fontsize{8}{14}\selectfont
\caption{Propiedades  estadisticas de variable transaction\_type, Economicos (A-2)}
\label{table-stats-economicos-a-2-transaction_type}
\begin{tabular}{|l|m{10em}|m{10em}|m{10em}|m{10em}|}
\hline
 \rowcolor[gray]{0.8}
Variable/Modelo & Real & tddpm\_mlp & smote-enc & ctgan \\
\hline top5 & ['Venta' 'Arriendo' 'Busco arriendo' 'Compro'] & ['Venta' 'Arriendo'] & ['Venta' 'Arriendo'] & ['Venta' 'Arriendo' 'Compro' 'Busco arriendo'] \\
\hline top5\_freq & [17540  4517     1     1] & [22168  5406] & [21978  5596] & [20557  5163  1830    24] \\
\hline top5\_prob & [7.95140306e-01 2.04769029e-01 4.53329707e-05 4.53329707e-05] & [0.80394575 0.19605425] & [0.7970552 0.2029448] & [0.74552114 0.1872416  0.06636687 0.00087039] \\
\hline nobs & 22059 & 27574 & 27574 & 27574 \\
\hline missing & 22059 & 0 & 0 & 0 \\
\hline
\end{tabular}
\end{table}

\begin{table}[H]
\centering
\fontsize{8}{14}\selectfont
\caption{Propiedades  estadisticas de variable m\_size, Economicos (A-2)}
\label{table-stats-economicos-a-2-m_size}
\begin{tabular}{|l|m{10em}|m{10em}|m{10em}|m{10em}|}
\hline
 \rowcolor[gray]{0.8}
Variable/Modelo & Real & tddpm\_mlp & smote-enc & ctgan \\
\hline top5 & [5000.   50.   60.  200.   70.] & [5000.   50.   60.  200.  100.] & [5000.   40.   30.   35.   50.] & [     0.   196535.56 264489.37 276876.36 189529.51] \\
\hline top5\_freq & [601 342 321 285 281] & [750 415 363 354 349] & [164  73  69  61  61] & [13794     2     2     2     2] \\
\hline top5\_prob & [0.02724512 0.01550388 0.01455188 0.0129199  0.01273856] & [0.02719954 0.01505041 0.01316458 0.01283818 0.01265685] & [0.00594763 0.00264742 0.00250236 0.00221223 0.00221223] & [5.00253862e-01 7.25320955e-05 7.25320955e-05 7.25320955e-05
 7.25320955e-05] \\
\hline nobs & 22059 & 27574 & 27574 & 27574 \\
\hline missing & 0.000 & 0.000 & 0.000 & 0.000 \\
\hline mean & 146269 & \cellcolor[rgb]{0.9, 0.54, 0.52} 3544115 & 254142 & \bfseries 69905 \\
\hline std\_err & 105454 & \cellcolor[rgb]{0.9, 0.54, 0.52} 1526171 & \bfseries 111829 & 576 \\
\hline upper\_ci & 352956 & \cellcolor[rgb]{0.9, 0.54, 0.52} 6535355 & \bfseries 473322 & 71034 \\
\hline lower\_ci & -60417 & \cellcolor[rgb]{0.9, 0.54, 0.52} 552876 & \bfseries 34962 & 68776 \\
\hline std & 15662334 & \cellcolor[rgb]{0.9, 0.54, 0.52} 253427077 & \bfseries 18569598 & 95655 \\
\hline iqr & 340.500 & 314.542 & \bfseries 362.825 & \cellcolor[rgb]{0.9, 0.54, 0.52} 127954.560 \\
\hline iqr\_normal & 252.413 & 233.171 & \bfseries 268.963 & \cellcolor[rgb]{0.9, 0.54, 0.52} 94852.857 \\
\hline mad & 290635 & \cellcolor[rgb]{0.9, 0.54, 0.52} 7084741 & 506460 & \bfseries 79853 \\
\hline mad\_normal & 364257 & \cellcolor[rgb]{0.9, 0.54, 0.52} 8879406 & 634753 & \bfseries 100081 \\
\hline coef\_var & 107.079 & 71.506 & \bfseries 73.068 & \cellcolor[rgb]{0.9, 0.54, 0.52} 1.368 \\
\hline range & 2.24100e+09 & \cellcolor[rgb]{0.9, 0.54, 0.52} 3.09336e+10 & \bfseries 1.76077e+09 & 4.05049e+05 \\
\hline max & 2.24100e+09 & \cellcolor[rgb]{0.9, 0.54, 0.52} 3.09336e+10 & \bfseries 1.76077e+09 & 4.05049e+05 \\
\hline min & 0.000 & 0.000 & \cellcolor[rgb]{0.9, 0.54, 0.52} 1.000 & \bfseries 0.000 \\
\hline skew & 134.762 & \bfseries 91.081 & 82.386 & \cellcolor[rgb]{0.9, 0.54, 0.52} 1.220 \\
\hline kurtosis & 19053 & \bfseries 9527 & 6994 & \cellcolor[rgb]{0.9, 0.54, 0.52} 3 \\
\hline jarque\_bera & 3.33616e+11 & \bfseries 1.04259e+11 & 5.61901e+10 & \cellcolor[rgb]{0.9, 0.54, 0.52} 6.98701e+03 \\
\hline jarque\_bera\_pval & 0.000 & 0.000 & 0.000 & 0.000 \\
\hline mode & 5000 & \bfseries 5000 & \bfseries 5000 & \cellcolor[rgb]{0.9, 0.54, 0.52} 0 \\
\hline mode\_freq & 0.027 & \bfseries 0.027 & 0.006 & \cellcolor[rgb]{0.9, 0.54, 0.52} 0.500 \\
\hline median & 145.000 & \bfseries 141.875 & 148.714 & \cellcolor[rgb]{0.9, 0.54, 0.52} 0.000 \\
\hline 0.1\% & 2.000 & \bfseries 3.272 & \cellcolor[rgb]{0.9, 0.54, 0.52} 13.633 & 0.000 \\
\hline 1.0\% & 22.000 & \bfseries 23.048 & 25.428 & \cellcolor[rgb]{0.9, 0.54, 0.52} 0.000 \\
\hline 5.0\% & 35.000 & \bfseries 35.000 & 35.011 & \cellcolor[rgb]{0.9, 0.54, 0.52} 0.000 \\
\hline 25.0\% & 66.000 & \bfseries 66.000 & 67.454 & \cellcolor[rgb]{0.9, 0.54, 0.52} 0.000 \\
\hline 75.0\% & 406.500 & 380.542 & \bfseries 430.279 & \cellcolor[rgb]{0.9, 0.54, 0.52} 127954.560 \\
\hline 95.0\% & 5000 & \bfseries 5000 & 4241 & \cellcolor[rgb]{0.9, 0.54, 0.52} 273416 \\
\hline 99.0\% & 10200 & 8114 & \bfseries 8945 & \cellcolor[rgb]{0.9, 0.54, 0.52} 338370 \\
\hline 99.9\% & 70000 & 36560 & \bfseries 51349 & \cellcolor[rgb]{0.9, 0.54, 0.52} 379858 \\
\hline
\end{tabular}
\end{table}

\begin{table}[H]
\centering
\fontsize{8}{14}\selectfont
\caption{Propiedades  estadisticas de variable county, Economicos (A-2)}
\label{table-stats-economicos-a-2-county}
\begin{tabular}{|l|m{10em}|m{10em}|m{10em}|m{10em}|}
\hline
 \rowcolor[gray]{0.8}
Variable/Modelo & Real & tddpm\_mlp & smote-enc & ctgan \\
\hline top5 & ['Las Condes' 'Santiago' 'Providencia' 'Vitacura' 'Lo Barnechea'] & ['Las Condes' 'Santiago' 'Providencia' 'Vitacura' 'Lo Barnechea'] & ['Las Condes' 'Santiago' 'Providencia' 'Vitacura' 'Lo Barnechea'] & ['Las Condes' 'Viña del Mar' 'Santiago' 'Vitacura' 'Providencia'] \\
\hline top5\_freq & [3233 2703 1481 1415 1322] & [4220 3346 1908 1905 1783] & [4590 3812 1909 1902 1859] & [3009 2080 2003 1705 1610] \\
\hline top5\_prob & [0.14656149 0.12253502 0.06713813 0.06414615 0.05993019] & [0.15304272 0.1213462  0.06919562 0.06908682 0.06466236] & [0.16646116 0.13824617 0.06923189 0.06897802 0.06741858] & [0.10912454 0.07543338 0.07264089 0.06183361 0.05838834] \\
\hline nobs & 22059 & 27574 & 27574 & 27574 \\
\hline missing & 22059 & 0 & 0 & 0 \\
\hline
\end{tabular}
\end{table}

\begin{table}[H]
\centering
\fontsize{8}{14}\selectfont
\caption{Propiedades  estadisticas de variable publication\_date, Economicos (A-2)}
\label{table-stats-economicos-a-2-publication_date}
\begin{tabular}{|l|m{10em}|m{10em}|m{10em}|m{10em}|}
\hline
 \rowcolor[gray]{0.8}
Variable/Modelo & Real & tddpm\_mlp & smote-enc & ctgan \\
\hline top5 & [1545 1693 1546 1549  721] & [1545. 1693. 1546. 1549.  721.] & [1545. 1693. 1546. 1549.  721.] & [1693 1550 1549 1548 1551] \\
\hline top5\_freq & [10883  6103   895   320   125] & [14363  7572  1044   190   114] & [13090  7531   516   114    82] & [6007  827  821  812  781] \\
\hline top5\_prob & [0.49335872 0.27666712 0.04057301 0.01450655 0.00566662] & [0.52088924 0.27460651 0.03786175 0.00689055 0.00413433] & [0.47472256 0.27311961 0.01871328 0.00413433 0.00297382] & [0.21785015 0.02999202 0.02977443 0.02944803 0.02832378] \\
\hline nobs & 22059 & 27574 & 27574 & 27574 \\
\hline missing & 0.000 & 0.000 & 0.000 & 0.000 \\
\hline mean & 1471 & 1474 & \bfseries 1471 & \cellcolor[rgb]{0.9, 0.54, 0.52} 1488 \\
\hline std\_err & 2.056 & \cellcolor[rgb]{0.9, 0.54, 0.52} 1.815 & 1.842 & \bfseries 1.849 \\
\hline upper\_ci & 1475 & 1477 & \bfseries 1474 & \cellcolor[rgb]{0.9, 0.54, 0.52} 1491 \\
\hline lower\_ci & 1467 & 1470 & \bfseries 1467 & \cellcolor[rgb]{0.9, 0.54, 0.52} 1484 \\
\hline std & 305.403 & \cellcolor[rgb]{0.9, 0.54, 0.52} 301.398 & \bfseries 305.857 & 307.103 \\
\hline iqr & 148.000 & \bfseries 148.000 & \bfseries 148.000 & \cellcolor[rgb]{0.9, 0.54, 0.52} 149.000 \\
\hline iqr\_normal & 109.713 & \bfseries 109.713 & \bfseries 109.713 & \cellcolor[rgb]{0.9, 0.54, 0.52} 110.454 \\
\hline mad & 206.755 & \cellcolor[rgb]{0.9, 0.54, 0.52} 201.911 & \bfseries 207.198 & 206.070 \\
\hline mad\_normal & 259.128 & \cellcolor[rgb]{0.9, 0.54, 0.52} 253.058 & \bfseries 259.684 & 258.271 \\
\hline coef\_var & 0.208 & \cellcolor[rgb]{0.9, 0.54, 0.52} 0.205 & \bfseries 0.208 & 0.206 \\
\hline range & 1489 & 1490 & \bfseries 1488 & \cellcolor[rgb]{0.9, 0.54, 0.52} 1312 \\
\hline max & 1693 & 1693 & 1693 & 1693 \\
\hline min & 204.000 & 202.742 & \bfseries 205.000 & \cellcolor[rgb]{0.9, 0.54, 0.52} 381.000 \\
\hline skew & -2.019 & -2.047 & \bfseries -2.015 & \cellcolor[rgb]{0.9, 0.54, 0.52} -2.062 \\
\hline kurtosis & 5.892 & 6.003 & \bfseries 5.852 & \cellcolor[rgb]{0.9, 0.54, 0.52} 6.098 \\
\hline jarque\_bera & 22679 & 29629 & \bfseries 28015 & \cellcolor[rgb]{0.9, 0.54, 0.52} 30563 \\
\hline jarque\_bera\_pval & 0.000 & 0.000 & 0.000 & 0.000 \\
\hline mode & 1545 & \bfseries 1545 & \bfseries 1545 & \cellcolor[rgb]{0.9, 0.54, 0.52} 1693 \\
\hline mode\_freq & 0.493 & 0.521 & \bfseries 0.475 & \cellcolor[rgb]{0.9, 0.54, 0.52} 0.218 \\
\hline median & 1545 & \bfseries 1545 & \bfseries 1545 & \cellcolor[rgb]{0.9, 0.54, 0.52} 1551 \\
\hline 0.1\% & 450.696 & \cellcolor[rgb]{0.9, 0.54, 0.52} 497.193 & 483.807 & \bfseries 424.573 \\
\hline 1.0\% & 531.000 & 538.101 & \bfseries 535.709 & \cellcolor[rgb]{0.9, 0.54, 0.52} 501.000 \\
\hline 5.0\% & 628.900 & \bfseries 631.192 & 619.511 & \cellcolor[rgb]{0.9, 0.54, 0.52} 659.000 \\
\hline 25.0\% & 1545 & \bfseries 1545 & \bfseries 1545 & \cellcolor[rgb]{0.9, 0.54, 0.52} 1541 \\
\hline 75.0\% & 1693 & \bfseries 1693 & \bfseries 1693 & \cellcolor[rgb]{0.9, 0.54, 0.52} 1690 \\
\hline 95.0\% & 1693 & 1693 & 1693 & 1693 \\
\hline 99.0\% & 1693 & 1693 & 1693 & 1693 \\
\hline 99.9\% & 1693 & 1693 & 1693 & 1693 \\
\hline
\end{tabular}
\end{table}

\begin{table}[H]
\centering
\fontsize{8}{14}\selectfont
\caption{Propiedades  estadisticas de variable state, Economicos (A-2)}
\label{table-stats-economicos-a-2-state}
\begin{tabular}{|l|m{10em}|m{10em}|m{10em}|m{10em}|}
\hline
 \rowcolor[gray]{0.8}
Variable/Modelo & Real & tddpm\_mlp & smote-enc & ctgan \\
\hline top5 & ['Metropolitana de Santiago' 'Valparaíso' 'Coquimbo' 'Araucanía'
 "Libertador General Bernardo O'higgins"] & ['Metropolitana de Santiago' 'Valparaíso' 'Coquimbo' 'Araucanía' 'Biobío'] & ['Metropolitana de Santiago' 'Valparaíso' 'Coquimbo' 'Araucanía' 'Maule'] & ['Metropolitana de Santiago' 'Valparaíso' 'Araucanía'
 "Libertador General Bernardo O'higgins" 'Coquimbo'] \\
\hline top5\_freq & [17248  2014   567   558   305] & [22014  2495   683   629   316] & [22573  2299   673   596   295] & [12809  4245  2972  1303  1258] \\
\hline top5\_prob & [0.78190308 0.0913006  0.02570379 0.0252958  0.01382656] & [0.79836077 0.09048379 0.02476971 0.02281134 0.01146007] & [0.8186335  0.08337564 0.02440705 0.02161456 0.01069848] & [0.46453181 0.15394937 0.10778269 0.04725466 0.04562269] \\
\hline nobs & 22059 & 27574 & 27574 & 27574 \\
\hline missing & 22059 & 0 & 0 & 0 \\
\hline
\end{tabular}
\end{table}

\begin{table}[H]
\centering
\fontsize{8}{14}\selectfont
\caption{Propiedades  estadisticas de variable m\_built, Economicos (A-2)}
\label{table-stats-economicos-a-2-m_built}
\begin{tabular}{|l|m{10em}|m{10em}|m{10em}|m{10em}|}
\hline
 \rowcolor[gray]{0.8}
Variable/Modelo & Real & tddpm\_mlp & smote-enc & ctgan \\
\hline top5 & [140.  60. 120.  50.  70.] & [140.  60.  50. 100. 120.] & [140.  60.  70.  50.  40.] & [1.00000e+00 4.87560e+02 6.95840e+02 2.33830e+02 1.83236e+03] \\
\hline top5\_freq & [700 467 444 431 415] & [861 555 539 505 497] & [349 162 152 148 144] & [16891     4     3     2     2] \\
\hline top5\_prob & [0.03173308 0.0211705  0.02012784 0.01953851 0.01881318] & [0.03122507 0.02012766 0.0195474  0.01831435 0.01802423] & [0.01265685 0.0058751  0.00551244 0.00536738 0.00522231] & [6.12569812e-01 1.45064191e-04 1.08798143e-04 7.25320955e-05
 7.25320955e-05] \\
\hline nobs & 22059 & 27574 & 27574 & 27574 \\
\hline missing & 0.000 & 0.000 & 0.000 & 0.000 \\
\hline mean & 1771 & 2230 & \bfseries 1383 & \cellcolor[rgb]{0.9, 0.54, 0.52} 453 \\
\hline std\_err & 664.365 & \bfseries 707.978 & 377.735 & \cellcolor[rgb]{0.9, 0.54, 0.52} 8.752 \\
\hline upper\_ci & 3073 & \bfseries 3618 & 2123 & \cellcolor[rgb]{0.9, 0.54, 0.52} 470 \\
\hline lower\_ci & 469.205 & \cellcolor[rgb]{0.9, 0.54, 0.52} 842.298 & 642.508 & \bfseries 436.022 \\
\hline std & 98673 & \bfseries 117563 & 62724 & \cellcolor[rgb]{0.9, 0.54, 0.52} 1453 \\
\hline iqr & 140.000 & 134.369 & \bfseries 139.781 & \cellcolor[rgb]{0.9, 0.54, 0.52} 544.865 \\
\hline iqr\_normal & 103.782 & 99.608 & \bfseries 103.620 & \cellcolor[rgb]{0.9, 0.54, 0.52} 403.909 \\
\hline mad & 3202 & 4128 & \bfseries 2435 & \cellcolor[rgb]{0.9, 0.54, 0.52} 609 \\
\hline mad\_normal & 4013 & 5174 & \bfseries 3052 & \cellcolor[rgb]{0.9, 0.54, 0.52} 763 \\
\hline coef\_var & 55.706 & \bfseries 52.721 & 45.359 & \cellcolor[rgb]{0.9, 0.54, 0.52} 3.207 \\
\hline range & 11999999 & \bfseries 11997684 & 5491367 & \cellcolor[rgb]{0.9, 0.54, 0.52} 46828 \\
\hline max & 12000000 & \bfseries 11997685 & 5491368 & \cellcolor[rgb]{0.9, 0.54, 0.52} 46829 \\
\hline min & 1.000 & 1.000 & 1.000 & 1.000 \\
\hline skew & 96.078 & \bfseries 70.629 & 67.003 & \cellcolor[rgb]{0.9, 0.54, 0.52} 15.230 \\
\hline kurtosis & 10659 & \bfseries 5724 & 4957 & \cellcolor[rgb]{0.9, 0.54, 0.52} 327 \\
\hline jarque\_bera & 1.04399e+11 & \bfseries 3.76215e+10 & 2.82200e+10 & \cellcolor[rgb]{0.9, 0.54, 0.52} 1.21319e+08 \\
\hline jarque\_bera\_pval & 0.000 & 0.000 & 0.000 & 0.000 \\
\hline mode & 140.000 & \bfseries 140.000 & \bfseries 140.000 & \cellcolor[rgb]{0.9, 0.54, 0.52} 1.000 \\
\hline mode\_freq & 0.032 & \bfseries 0.031 & 0.013 & \cellcolor[rgb]{0.9, 0.54, 0.52} 0.613 \\
\hline median & 107.000 & 105.000 & \bfseries 107.984 & \cellcolor[rgb]{0.9, 0.54, 0.52} 1.000 \\
\hline 0.1\% & 2.000 & \bfseries 2.256 & \cellcolor[rgb]{0.9, 0.54, 0.52} 16.265 & 1.000 \\
\hline 1.0\% & 23.000 & \bfseries 23.000 & 25.547 & \cellcolor[rgb]{0.9, 0.54, 0.52} 1.000 \\
\hline 5.0\% & 33.000 & \bfseries 33.351 & 33.891 & \cellcolor[rgb]{0.9, 0.54, 0.52} 1.000 \\
\hline 25.0\% & 60.000 & \bfseries 60.000 & 60.089 & \cellcolor[rgb]{0.9, 0.54, 0.52} 1.000 \\
\hline 75.0\% & 200.000 & 194.369 & \bfseries 199.870 & \cellcolor[rgb]{0.9, 0.54, 0.52} 545.865 \\
\hline 95.0\% & 490.000 & 450.107 & \bfseries 475.689 & \cellcolor[rgb]{0.9, 0.54, 0.52} 2006.255 \\
\hline 99.0\% & 1200 & 917 & \bfseries 1111 & \cellcolor[rgb]{0.9, 0.54, 0.52} 2534 \\
\hline 99.9\% & 37947 & \cellcolor[rgb]{0.9, 0.54, 0.52} 20788 & \bfseries 42146 & 24096 \\
\hline
\end{tabular}
\end{table}

\begin{table}[H]
\centering
\fontsize{8}{14}\selectfont
\caption{Propiedades  estadisticas de variable property\_type, Economicos (A-2)}
\label{table-stats-economicos-a-2-property_type}
\begin{tabular}{|l|m{10em}|m{10em}|m{10em}|m{10em}|}
\hline
 \rowcolor[gray]{0.8}
Variable/Modelo & Real & tddpm\_mlp & smote-enc & ctgan \\
\hline top5 & ['Departamento' 'Casa' 'Oficina o Casa Oficina' 'Parcela o Chacra'
 'Local o Casa comercial'] & ['Departamento' 'Casa' 'Oficina o Casa Oficina' 'Parcela o Chacra'
 'Local o Casa comercial'] & ['Departamento' 'Casa' 'Oficina o Casa Oficina' 'Parcela o Chacra'
 'Departamento Amoblado'] & ['Casa' 'Departamento' 'Oficina o Casa Oficina' 'Parcela o Chacra'
 'Local o Casa comercial'] \\
\hline top5\_freq & [10592  8911  1553   413   255] & [13595 11267  1869   404   194] & [13570 11704  1837   261   124] & [18606  5166  2132   625   416] \\
\hline top5\_prob & [0.48016683 0.4039621  0.0704021  0.01872252 0.01155991] & [0.49303692 0.40860956 0.06778124 0.01465148 0.00703561] & [0.49213027 0.42445782 0.06662073 0.00946544 0.00449699] & [0.67476608 0.1873504  0.07731921 0.02266628 0.01508668] \\
\hline nobs & 22059 & 27574 & 27574 & 27574 \\
\hline missing & 22059 & 0 & 0 & 0 \\
\hline
\end{tabular}
\end{table}



\section{Estadísticos Económicos - Conjunto A}
\label{propiedades-estadisticas-economicos-A}
\begin{table}[H]
\centering
\fontsize{8}{14}\selectfont
\caption{Propiedades  estadisticas de variable rooms, Economicos (A-2)}
\label{table-stats-economicos-a-2-rooms}
\begin{tabular}{|l|m{10em}|m{10em}|m{10em}|m{10em}|}
\hline
 \rowcolor[gray]{0.8}
Variable/Modelo & Real & tddpm\_mlp & smote-enc & ctgan \\
\hline top5 & [3. 2. 4. 1. 5.] & [3. 2. 4. 1. 5.] & [3. 2. 4. 1. 5.] & [2. 3. 4. 5. 6.] \\
\hline top5\_freq & [6355 4614 4168 2671 2232] & [8216 5798 5162 3327 2911] & [8191 5804 5468 3245 2819] & [6541 5670 4070 2674 2058] \\
\hline top5\_prob & [0.28809103 0.20916633 0.18894782 0.12108436 0.10118319] & [0.29796185 0.21027054 0.18720534 0.12065714 0.10557046] & [0.2970552  0.21048814 0.19830275 0.11768332 0.10223399] & [0.23721622 0.20562849 0.14760281 0.09697541 0.07463553] \\
\hline nobs & 22059 & 27574 & 27574 & 27574 \\
\hline missing & 0.000 & 0.000 & 0.000 & 0.000 \\
\hline mean & 3.446 & \bfseries 3.347 & 3.310 & \cellcolor[rgb]{0.9, 0.54, 0.52} 5.000 \\
\hline std\_err & 0.026 & \bfseries 0.024 & 0.012 & \cellcolor[rgb]{0.9, 0.54, 0.52} 0.047 \\
\hline upper\_ci & 3.497 & \bfseries 3.393 & 3.334 & \cellcolor[rgb]{0.9, 0.54, 0.52} 5.092 \\
\hline lower\_ci & 3.395 & \bfseries 3.301 & 3.286 & \cellcolor[rgb]{0.9, 0.54, 0.52} 4.909 \\
\hline std & 3.881 & \bfseries 3.924 & 1.993 & \cellcolor[rgb]{0.9, 0.54, 0.52} 7.749 \\
\hline iqr & 2.000 & \bfseries 2.000 & \bfseries 2.000 & \cellcolor[rgb]{0.9, 0.54, 0.52} 3.000 \\
\hline iqr\_normal & 1.483 & \bfseries 1.483 & \bfseries 1.483 & \cellcolor[rgb]{0.9, 0.54, 0.52} 2.224 \\
\hline mad & 1.454 & \bfseries 1.340 & 1.279 & \cellcolor[rgb]{0.9, 0.54, 0.52} 3.061 \\
\hline mad\_normal & 1.822 & \bfseries 1.679 & 1.603 & \cellcolor[rgb]{0.9, 0.54, 0.52} 3.836 \\
\hline coef\_var & 1.126 & \bfseries 1.172 & \cellcolor[rgb]{0.9, 0.54, 0.52} 0.602 & 1.550 \\
\hline range & 399.000 & \bfseries 399.000 & \cellcolor[rgb]{0.9, 0.54, 0.52} 56.000 & \bfseries 399.000 \\
\hline max & 400.000 & \bfseries 400.000 & \cellcolor[rgb]{0.9, 0.54, 0.52} 57.000 & \bfseries 400.000 \\
\hline min & 1.000 & 1.000 & 1.000 & 1.000 \\
\hline skew & 57.785 & \bfseries 75.482 & \cellcolor[rgb]{0.9, 0.54, 0.52} 4.962 & 25.543 \\
\hline kurtosis & 5331 & \bfseries 7575 & \cellcolor[rgb]{0.9, 0.54, 0.52} 68 & 1106 \\
\hline jarque\_bera & 2.61061e+10 & \cellcolor[rgb]{0.9, 0.54, 0.52} 6.59026e+10 & 4.96441e+06 & \bfseries 1.40103e+09 \\
\hline jarque\_bera\_pval & 0.000 & 0.000 & 0.000 & 0.000 \\
\hline mode & 3.000 & \bfseries 3.000 & \bfseries 3.000 & \cellcolor[rgb]{0.9, 0.54, 0.52} 2.000 \\
\hline mode\_freq & 0.288 & 0.298 & \bfseries 0.297 & \cellcolor[rgb]{0.9, 0.54, 0.52} 0.237 \\
\hline median & 3.000 & 3.000 & 3.000 & 3.000 \\
\hline 0.1\% & 1.000 & 1.000 & 1.000 & 1.000 \\
\hline 1.0\% & 1.000 & 1.000 & 1.000 & 1.000 \\
\hline 5.0\% & 1.000 & 1.000 & 1.000 & 1.000 \\
\hline 25.0\% & 2.000 & 2.000 & 2.000 & 2.000 \\
\hline 75.0\% & 4.000 & \bfseries 4.000 & \bfseries 4.000 & \cellcolor[rgb]{0.9, 0.54, 0.52} 5.000 \\
\hline 95.0\% & 6.000 & \bfseries 6.000 & \bfseries 6.000 & \cellcolor[rgb]{0.9, 0.54, 0.52} 14.000 \\
\hline 99.0\% & 12.000 & \bfseries 10.000 & \bfseries 10.000 & \cellcolor[rgb]{0.9, 0.54, 0.52} 25.000 \\
\hline 99.9\% & 25.000 & 24.000 & \bfseries 25.000 & \cellcolor[rgb]{0.9, 0.54, 0.52} 57.000 \\
\hline
\end{tabular}
\end{table}

\begin{table}[H]
\centering
\fontsize{8}{14}\selectfont
\caption{Propiedades  estadisticas de variable bathrooms, Economicos (A-2)}
\label{table-stats-economicos-a-2-bathrooms}
\begin{tabular}{|l|m{10em}|m{10em}|m{10em}|m{10em}|}
\hline
 \rowcolor[gray]{0.8}
Variable/Modelo & Real & tddpm\_mlp & smote-enc & ctgan \\
\hline top5 & [2. 1. 3. 4. 5.] & [2. 1. 3. 4. 5.] & [2. 1. 3. 4. 5.] & [3. 1. 2. 4. 5.] \\
\hline top5\_freq & [7511 5440 4486 2665 1084] & [9481 6610 5861 3425 1331] & [9438 6742 5612 3270 1387] & [7474 5459 4858 3762 3251] \\
\hline top5\_prob & [0.34049594 0.24661136 0.20336371 0.12081237 0.04914094] & [0.3438384  0.23971858 0.21255531 0.12421121 0.04827011] & [0.34227896 0.24450569 0.20352506 0.11858998 0.05030101] & [0.27105244 0.19797635 0.17618046 0.13643287 0.11790092] \\
\hline nobs & 22059 & 27574 & 27574 & 27574 \\
\hline missing & 0.000 & 0.000 & 0.000 & 0.000 \\
\hline mean & 2.604 & 2.544 & \bfseries 2.593 & \cellcolor[rgb]{0.9, 0.54, 0.52} 3.679 \\
\hline std\_err & 0.025 & \bfseries 0.018 & 0.013 & \cellcolor[rgb]{0.9, 0.54, 0.52} 0.058 \\
\hline upper\_ci & 2.652 & 2.579 & \bfseries 2.618 & \cellcolor[rgb]{0.9, 0.54, 0.52} 3.792 \\
\hline lower\_ci & 2.556 & 2.508 & \bfseries 2.568 & \cellcolor[rgb]{0.9, 0.54, 0.52} 3.566 \\
\hline std & 3.655 & \bfseries 3.001 & 2.133 & \cellcolor[rgb]{0.9, 0.54, 0.52} 9.573 \\
\hline iqr & 1.000 & \bfseries 1.000 & \bfseries 1.000 & \cellcolor[rgb]{0.9, 0.54, 0.52} 2.000 \\
\hline iqr\_normal & 0.741 & \bfseries 0.741 & \bfseries 0.741 & \cellcolor[rgb]{0.9, 0.54, 0.52} 1.483 \\
\hline mad & 1.203 & 1.114 & \bfseries 1.185 & \cellcolor[rgb]{0.9, 0.54, 0.52} 2.020 \\
\hline mad\_normal & 1.507 & 1.396 & \bfseries 1.485 & \cellcolor[rgb]{0.9, 0.54, 0.52} 2.532 \\
\hline coef\_var & 1.404 & \bfseries 1.180 & 0.822 & \cellcolor[rgb]{0.9, 0.54, 0.52} 2.602 \\
\hline range & 435.000 & \bfseries 435.000 & \cellcolor[rgb]{0.9, 0.54, 0.52} 179.000 & \bfseries 435.000 \\
\hline max & 436.000 & \bfseries 436.000 & \cellcolor[rgb]{0.9, 0.54, 0.52} 180.000 & \bfseries 436.000 \\
\hline min & 1.000 & 1.000 & 1.000 & 1.000 \\
\hline skew & 82.448 & \bfseries 109.783 & \cellcolor[rgb]{0.9, 0.54, 0.52} 28.339 & 38.404 \\
\hline kurtosis & 9252 & \bfseries 15795 & 2009 & \cellcolor[rgb]{0.9, 0.54, 0.52} 1685 \\
\hline jarque\_bera & 7.86518e+10 & \cellcolor[rgb]{0.9, 0.54, 0.52} 2.86587e+11 & \bfseries 4.62894e+09 & 3.25889e+09 \\
\hline jarque\_bera\_pval & 0.000 & 0.000 & 0.000 & 0.000 \\
\hline mode & 2.000 & \bfseries 2.000 & \bfseries 2.000 & \cellcolor[rgb]{0.9, 0.54, 0.52} 3.000 \\
\hline mode\_freq & 0.340 & 0.344 & \bfseries 0.342 & \cellcolor[rgb]{0.9, 0.54, 0.52} 0.271 \\
\hline median & 2.000 & \bfseries 2.000 & \bfseries 2.000 & \cellcolor[rgb]{0.9, 0.54, 0.52} 3.000 \\
\hline 0.1\% & 1.000 & 1.000 & 1.000 & 1.000 \\
\hline 1.0\% & 1.000 & 1.000 & 1.000 & 1.000 \\
\hline 5.0\% & 1.000 & 1.000 & 1.000 & 1.000 \\
\hline 25.0\% & 2.000 & 2.000 & 2.000 & 2.000 \\
\hline 75.0\% & 3.000 & \bfseries 3.000 & \bfseries 3.000 & \cellcolor[rgb]{0.9, 0.54, 0.52} 4.000 \\
\hline 95.0\% & 5.000 & \bfseries 5.000 & \bfseries 5.000 & \cellcolor[rgb]{0.9, 0.54, 0.52} 8.000 \\
\hline 99.0\% & 8.000 & 7.000 & \bfseries 8.000 & \cellcolor[rgb]{0.9, 0.54, 0.52} 14.000 \\
\hline 99.9\% & 17.942 & 13.000 & \bfseries 17.427 & \cellcolor[rgb]{0.9, 0.54, 0.52} 57.000 \\
\hline
\end{tabular}
\end{table}

\begin{table}[H]
\centering
\fontsize{8}{14}\selectfont
\caption{Propiedades  estadisticas de variable \_price, Economicos (A-2)}
\label{table-stats-economicos-a-2-_price}
\begin{tabular}{|l|m{10em}|m{10em}|m{10em}|m{10em}|}
\hline
 \rowcolor[gray]{0.8}
Variable/Modelo & Real & tddpm\_mlp & smote-enc & ctgan \\
\hline top5 & [12500. 10500. 11500.  8500.  9000.] & [11500. 12500.  8500. 20000. 13500.] & [10500. 11500.  2100.  9000.  9500.] & [    0.          1873.91702513 41042.06971567 41062.90431098
 41066.23801271] \\
\hline top5\_freq & [104  99  91  86  85] & [102 101  99  99  94] & [165 148 142 141 139] & [17440     2     1     1     1] \\
\hline top5\_prob & [0.00471463 0.00448796 0.0041253  0.00389864 0.0038533 ] & [0.00369914 0.00366287 0.00359034 0.00359034 0.00340901] & [0.0059839  0.00536738 0.00514978 0.00511351 0.00504098] & [6.32479872e-01 7.25320955e-05 3.62660477e-05 3.62660477e-05
 3.62660477e-05] \\
\hline nobs & 22059 & 27574 & 27574 & 27574 \\
\hline missing & 0.000 & 0.000 & 0.000 & 0.000 \\
\hline mean & 110379 & \bfseries 52905 & 30808 & \cellcolor[rgb]{0.9, 0.54, 0.52} 12067 \\
\hline std\_err & 32746 & \bfseries 19395 & 12251 & \cellcolor[rgb]{0.9, 0.54, 0.52} 130 \\
\hline upper\_ci & 174559 & \bfseries 90919 & 54820 & \cellcolor[rgb]{0.9, 0.54, 0.52} 12322 \\
\hline lower\_ci & 46199 & \bfseries 14891 & \cellcolor[rgb]{0.9, 0.54, 0.52} 6797 & 11812 \\
\hline std & 4863477 & \bfseries 3220683 & 2034332 & \cellcolor[rgb]{0.9, 0.54, 0.52} 21633 \\
\hline iqr & 9959 & \bfseries 9948 & 10425 & \cellcolor[rgb]{0.9, 0.54, 0.52} 16851 \\
\hline iqr\_normal & 7383 & \bfseries 7374 & 7728 & \cellcolor[rgb]{0.9, 0.54, 0.52} 12492 \\
\hline mad & 202281 & \bfseries 88001 & 44740 & \cellcolor[rgb]{0.9, 0.54, 0.52} 16342 \\
\hline mad\_normal & 253522 & \bfseries 110293 & 56073 & \cellcolor[rgb]{0.9, 0.54, 0.52} 20481 \\
\hline coef\_var & 44.062 & \bfseries 60.877 & 66.032 & \cellcolor[rgb]{0.9, 0.54, 0.52} 1.793 \\
\hline range & 390000000 & \bfseries 362636196 & 279000000 & \cellcolor[rgb]{0.9, 0.54, 0.52} 124189 \\
\hline max & 390000000 & \bfseries 362636196 & 279000000 & \cellcolor[rgb]{0.9, 0.54, 0.52} 124189 \\
\hline min & 0.000 & 0.000 & 0.000 & 0.000 \\
\hline skew & 60.579 & \bfseries 88.292 & 114.793 & \cellcolor[rgb]{0.9, 0.54, 0.52} 1.959 \\
\hline kurtosis & 4067 & 8564 & \cellcolor[rgb]{0.9, 0.54, 0.52} 14362 & \bfseries 6 \\
\hline jarque\_bera & 1.51936e+10 & 8.42493e+10 & \cellcolor[rgb]{0.9, 0.54, 0.52} 2.36952e+11 & \bfseries 2.97507e+04 \\
\hline jarque\_bera\_pval & 0.000 & 0.000 & 0.000 & 0.000 \\
\hline mode & 12500 & \bfseries 11500 & 10500 & \cellcolor[rgb]{0.9, 0.54, 0.52} 0 \\
\hline mode\_freq & 0.005 & \bfseries 0.004 & 0.006 & \cellcolor[rgb]{0.9, 0.54, 0.52} 0.632 \\
\hline median & 5084 & 5194 & \bfseries 5085 & \cellcolor[rgb]{0.9, 0.54, 0.52} 0 \\
\hline 0.1\% & 0.263 & 0.417 & \bfseries 0.287 & \cellcolor[rgb]{0.9, 0.54, 0.52} 0.000 \\
\hline 1.0\% & 6.270 & 7.795 & \bfseries 7.060 & \cellcolor[rgb]{0.9, 0.54, 0.52} 0.000 \\
\hline 5.0\% & 11.760 & 12.100 & \bfseries 11.902 & \cellcolor[rgb]{0.9, 0.54, 0.52} 0.000 \\
\hline 25.0\% & 2041 & 2162 & \bfseries 2065 & \cellcolor[rgb]{0.9, 0.54, 0.52} 0 \\
\hline 75.0\% & 12000 & \bfseries 12109 & 12490 & \cellcolor[rgb]{0.9, 0.54, 0.52} 16851 \\
\hline 95.0\% & 32000 & 30839 & \bfseries 32000 & \cellcolor[rgb]{0.9, 0.54, 0.52} 62115 \\
\hline 99.0\% & 58942 & 53875 & \bfseries 55000 & \cellcolor[rgb]{0.9, 0.54, 0.52} 88157 \\
\hline 99.9\% & 262695 & 110019 & \bfseries 120000 & \cellcolor[rgb]{0.9, 0.54, 0.52} 109231 \\
\hline
\end{tabular}
\end{table}

\begin{table}[H]
\centering
\fontsize{8}{14}\selectfont
\caption{Propiedades  estadisticas de variable transaction\_type, Economicos (A-2)}
\label{table-stats-economicos-a-2-transaction_type}
\begin{tabular}{|l|m{10em}|m{10em}|m{10em}|m{10em}|}
\hline
 \rowcolor[gray]{0.8}
Variable/Modelo & Real & tddpm\_mlp & smote-enc & ctgan \\
\hline top5 & ['Venta' 'Arriendo' 'Busco arriendo' 'Compro'] & ['Venta' 'Arriendo'] & ['Venta' 'Arriendo'] & ['Venta' 'Arriendo' 'Compro' 'Busco arriendo'] \\
\hline top5\_freq & [17540  4517     1     1] & [22168  5406] & [21978  5596] & [20557  5163  1830    24] \\
\hline top5\_prob & [7.95140306e-01 2.04769029e-01 4.53329707e-05 4.53329707e-05] & [0.80394575 0.19605425] & [0.7970552 0.2029448] & [0.74552114 0.1872416  0.06636687 0.00087039] \\
\hline nobs & 22059 & 27574 & 27574 & 27574 \\
\hline missing & 22059 & 0 & 0 & 0 \\
\hline
\end{tabular}
\end{table}

\begin{table}[H]
\centering
\fontsize{8}{14}\selectfont
\caption{Propiedades  estadisticas de variable m\_size, Economicos (A-2)}
\label{table-stats-economicos-a-2-m_size}
\begin{tabular}{|l|m{10em}|m{10em}|m{10em}|m{10em}|}
\hline
 \rowcolor[gray]{0.8}
Variable/Modelo & Real & tddpm\_mlp & smote-enc & ctgan \\
\hline top5 & [5000.   50.   60.  200.   70.] & [5000.   50.   60.  200.  100.] & [5000.   40.   30.   35.   50.] & [     0.   196535.56 264489.37 276876.36 189529.51] \\
\hline top5\_freq & [601 342 321 285 281] & [750 415 363 354 349] & [164  73  69  61  61] & [13794     2     2     2     2] \\
\hline top5\_prob & [0.02724512 0.01550388 0.01455188 0.0129199  0.01273856] & [0.02719954 0.01505041 0.01316458 0.01283818 0.01265685] & [0.00594763 0.00264742 0.00250236 0.00221223 0.00221223] & [5.00253862e-01 7.25320955e-05 7.25320955e-05 7.25320955e-05
 7.25320955e-05] \\
\hline nobs & 22059 & 27574 & 27574 & 27574 \\
\hline missing & 0.000 & 0.000 & 0.000 & 0.000 \\
\hline mean & 146269 & \cellcolor[rgb]{0.9, 0.54, 0.52} 3544115 & 254142 & \bfseries 69905 \\
\hline std\_err & 105454 & \cellcolor[rgb]{0.9, 0.54, 0.52} 1526171 & \bfseries 111829 & 576 \\
\hline upper\_ci & 352956 & \cellcolor[rgb]{0.9, 0.54, 0.52} 6535355 & \bfseries 473322 & 71034 \\
\hline lower\_ci & -60417 & \cellcolor[rgb]{0.9, 0.54, 0.52} 552876 & \bfseries 34962 & 68776 \\
\hline std & 15662334 & \cellcolor[rgb]{0.9, 0.54, 0.52} 253427077 & \bfseries 18569598 & 95655 \\
\hline iqr & 340.500 & 314.542 & \bfseries 362.825 & \cellcolor[rgb]{0.9, 0.54, 0.52} 127954.560 \\
\hline iqr\_normal & 252.413 & 233.171 & \bfseries 268.963 & \cellcolor[rgb]{0.9, 0.54, 0.52} 94852.857 \\
\hline mad & 290635 & \cellcolor[rgb]{0.9, 0.54, 0.52} 7084741 & 506460 & \bfseries 79853 \\
\hline mad\_normal & 364257 & \cellcolor[rgb]{0.9, 0.54, 0.52} 8879406 & 634753 & \bfseries 100081 \\
\hline coef\_var & 107.079 & 71.506 & \bfseries 73.068 & \cellcolor[rgb]{0.9, 0.54, 0.52} 1.368 \\
\hline range & 2.24100e+09 & \cellcolor[rgb]{0.9, 0.54, 0.52} 3.09336e+10 & \bfseries 1.76077e+09 & 4.05049e+05 \\
\hline max & 2.24100e+09 & \cellcolor[rgb]{0.9, 0.54, 0.52} 3.09336e+10 & \bfseries 1.76077e+09 & 4.05049e+05 \\
\hline min & 0.000 & 0.000 & \cellcolor[rgb]{0.9, 0.54, 0.52} 1.000 & \bfseries 0.000 \\
\hline skew & 134.762 & \bfseries 91.081 & 82.386 & \cellcolor[rgb]{0.9, 0.54, 0.52} 1.220 \\
\hline kurtosis & 19053 & \bfseries 9527 & 6994 & \cellcolor[rgb]{0.9, 0.54, 0.52} 3 \\
\hline jarque\_bera & 3.33616e+11 & \bfseries 1.04259e+11 & 5.61901e+10 & \cellcolor[rgb]{0.9, 0.54, 0.52} 6.98701e+03 \\
\hline jarque\_bera\_pval & 0.000 & 0.000 & 0.000 & 0.000 \\
\hline mode & 5000 & \bfseries 5000 & \bfseries 5000 & \cellcolor[rgb]{0.9, 0.54, 0.52} 0 \\
\hline mode\_freq & 0.027 & \bfseries 0.027 & 0.006 & \cellcolor[rgb]{0.9, 0.54, 0.52} 0.500 \\
\hline median & 145.000 & \bfseries 141.875 & 148.714 & \cellcolor[rgb]{0.9, 0.54, 0.52} 0.000 \\
\hline 0.1\% & 2.000 & \bfseries 3.272 & \cellcolor[rgb]{0.9, 0.54, 0.52} 13.633 & 0.000 \\
\hline 1.0\% & 22.000 & \bfseries 23.048 & 25.428 & \cellcolor[rgb]{0.9, 0.54, 0.52} 0.000 \\
\hline 5.0\% & 35.000 & \bfseries 35.000 & 35.011 & \cellcolor[rgb]{0.9, 0.54, 0.52} 0.000 \\
\hline 25.0\% & 66.000 & \bfseries 66.000 & 67.454 & \cellcolor[rgb]{0.9, 0.54, 0.52} 0.000 \\
\hline 75.0\% & 406.500 & 380.542 & \bfseries 430.279 & \cellcolor[rgb]{0.9, 0.54, 0.52} 127954.560 \\
\hline 95.0\% & 5000 & \bfseries 5000 & 4241 & \cellcolor[rgb]{0.9, 0.54, 0.52} 273416 \\
\hline 99.0\% & 10200 & 8114 & \bfseries 8945 & \cellcolor[rgb]{0.9, 0.54, 0.52} 338370 \\
\hline 99.9\% & 70000 & 36560 & \bfseries 51349 & \cellcolor[rgb]{0.9, 0.54, 0.52} 379858 \\
\hline
\end{tabular}
\end{table}

\begin{table}[H]
\centering
\fontsize{8}{14}\selectfont
\caption{Propiedades  estadisticas de variable county, Economicos (A-2)}
\label{table-stats-economicos-a-2-county}
\begin{tabular}{|l|m{10em}|m{10em}|m{10em}|m{10em}|}
\hline
 \rowcolor[gray]{0.8}
Variable/Modelo & Real & tddpm\_mlp & smote-enc & ctgan \\
\hline top5 & ['Las Condes' 'Santiago' 'Providencia' 'Vitacura' 'Lo Barnechea'] & ['Las Condes' 'Santiago' 'Providencia' 'Vitacura' 'Lo Barnechea'] & ['Las Condes' 'Santiago' 'Providencia' 'Vitacura' 'Lo Barnechea'] & ['Las Condes' 'Viña del Mar' 'Santiago' 'Vitacura' 'Providencia'] \\
\hline top5\_freq & [3233 2703 1481 1415 1322] & [4220 3346 1908 1905 1783] & [4590 3812 1909 1902 1859] & [3009 2080 2003 1705 1610] \\
\hline top5\_prob & [0.14656149 0.12253502 0.06713813 0.06414615 0.05993019] & [0.15304272 0.1213462  0.06919562 0.06908682 0.06466236] & [0.16646116 0.13824617 0.06923189 0.06897802 0.06741858] & [0.10912454 0.07543338 0.07264089 0.06183361 0.05838834] \\
\hline nobs & 22059 & 27574 & 27574 & 27574 \\
\hline missing & 22059 & 0 & 0 & 0 \\
\hline
\end{tabular}
\end{table}

\begin{table}[H]
\centering
\fontsize{8}{14}\selectfont
\caption{Propiedades  estadisticas de variable publication\_date, Economicos (A-2)}
\label{table-stats-economicos-a-2-publication_date}
\begin{tabular}{|l|m{10em}|m{10em}|m{10em}|m{10em}|}
\hline
 \rowcolor[gray]{0.8}
Variable/Modelo & Real & tddpm\_mlp & smote-enc & ctgan \\
\hline top5 & [1545 1693 1546 1549  721] & [1545. 1693. 1546. 1549.  721.] & [1545. 1693. 1546. 1549.  721.] & [1693 1550 1549 1548 1551] \\
\hline top5\_freq & [10883  6103   895   320   125] & [14363  7572  1044   190   114] & [13090  7531   516   114    82] & [6007  827  821  812  781] \\
\hline top5\_prob & [0.49335872 0.27666712 0.04057301 0.01450655 0.00566662] & [0.52088924 0.27460651 0.03786175 0.00689055 0.00413433] & [0.47472256 0.27311961 0.01871328 0.00413433 0.00297382] & [0.21785015 0.02999202 0.02977443 0.02944803 0.02832378] \\
\hline nobs & 22059 & 27574 & 27574 & 27574 \\
\hline missing & 0.000 & 0.000 & 0.000 & 0.000 \\
\hline mean & 1471 & 1474 & \bfseries 1471 & \cellcolor[rgb]{0.9, 0.54, 0.52} 1488 \\
\hline std\_err & 2.056 & \cellcolor[rgb]{0.9, 0.54, 0.52} 1.815 & 1.842 & \bfseries 1.849 \\
\hline upper\_ci & 1475 & 1477 & \bfseries 1474 & \cellcolor[rgb]{0.9, 0.54, 0.52} 1491 \\
\hline lower\_ci & 1467 & 1470 & \bfseries 1467 & \cellcolor[rgb]{0.9, 0.54, 0.52} 1484 \\
\hline std & 305.403 & \cellcolor[rgb]{0.9, 0.54, 0.52} 301.398 & \bfseries 305.857 & 307.103 \\
\hline iqr & 148.000 & \bfseries 148.000 & \bfseries 148.000 & \cellcolor[rgb]{0.9, 0.54, 0.52} 149.000 \\
\hline iqr\_normal & 109.713 & \bfseries 109.713 & \bfseries 109.713 & \cellcolor[rgb]{0.9, 0.54, 0.52} 110.454 \\
\hline mad & 206.755 & \cellcolor[rgb]{0.9, 0.54, 0.52} 201.911 & \bfseries 207.198 & 206.070 \\
\hline mad\_normal & 259.128 & \cellcolor[rgb]{0.9, 0.54, 0.52} 253.058 & \bfseries 259.684 & 258.271 \\
\hline coef\_var & 0.208 & \cellcolor[rgb]{0.9, 0.54, 0.52} 0.205 & \bfseries 0.208 & 0.206 \\
\hline range & 1489 & 1490 & \bfseries 1488 & \cellcolor[rgb]{0.9, 0.54, 0.52} 1312 \\
\hline max & 1693 & 1693 & 1693 & 1693 \\
\hline min & 204.000 & 202.742 & \bfseries 205.000 & \cellcolor[rgb]{0.9, 0.54, 0.52} 381.000 \\
\hline skew & -2.019 & -2.047 & \bfseries -2.015 & \cellcolor[rgb]{0.9, 0.54, 0.52} -2.062 \\
\hline kurtosis & 5.892 & 6.003 & \bfseries 5.852 & \cellcolor[rgb]{0.9, 0.54, 0.52} 6.098 \\
\hline jarque\_bera & 22679 & 29629 & \bfseries 28015 & \cellcolor[rgb]{0.9, 0.54, 0.52} 30563 \\
\hline jarque\_bera\_pval & 0.000 & 0.000 & 0.000 & 0.000 \\
\hline mode & 1545 & \bfseries 1545 & \bfseries 1545 & \cellcolor[rgb]{0.9, 0.54, 0.52} 1693 \\
\hline mode\_freq & 0.493 & 0.521 & \bfseries 0.475 & \cellcolor[rgb]{0.9, 0.54, 0.52} 0.218 \\
\hline median & 1545 & \bfseries 1545 & \bfseries 1545 & \cellcolor[rgb]{0.9, 0.54, 0.52} 1551 \\
\hline 0.1\% & 450.696 & \cellcolor[rgb]{0.9, 0.54, 0.52} 497.193 & 483.807 & \bfseries 424.573 \\
\hline 1.0\% & 531.000 & 538.101 & \bfseries 535.709 & \cellcolor[rgb]{0.9, 0.54, 0.52} 501.000 \\
\hline 5.0\% & 628.900 & \bfseries 631.192 & 619.511 & \cellcolor[rgb]{0.9, 0.54, 0.52} 659.000 \\
\hline 25.0\% & 1545 & \bfseries 1545 & \bfseries 1545 & \cellcolor[rgb]{0.9, 0.54, 0.52} 1541 \\
\hline 75.0\% & 1693 & \bfseries 1693 & \bfseries 1693 & \cellcolor[rgb]{0.9, 0.54, 0.52} 1690 \\
\hline 95.0\% & 1693 & 1693 & 1693 & 1693 \\
\hline 99.0\% & 1693 & 1693 & 1693 & 1693 \\
\hline 99.9\% & 1693 & 1693 & 1693 & 1693 \\
\hline
\end{tabular}
\end{table}

\begin{table}[H]
\centering
\fontsize{8}{14}\selectfont
\caption{Propiedades  estadisticas de variable state, Economicos (A-2)}
\label{table-stats-economicos-a-2-state}
\begin{tabular}{|l|m{10em}|m{10em}|m{10em}|m{10em}|}
\hline
 \rowcolor[gray]{0.8}
Variable/Modelo & Real & tddpm\_mlp & smote-enc & ctgan \\
\hline top5 & ['Metropolitana de Santiago' 'Valparaíso' 'Coquimbo' 'Araucanía'
 "Libertador General Bernardo O'higgins"] & ['Metropolitana de Santiago' 'Valparaíso' 'Coquimbo' 'Araucanía' 'Biobío'] & ['Metropolitana de Santiago' 'Valparaíso' 'Coquimbo' 'Araucanía' 'Maule'] & ['Metropolitana de Santiago' 'Valparaíso' 'Araucanía'
 "Libertador General Bernardo O'higgins" 'Coquimbo'] \\
\hline top5\_freq & [17248  2014   567   558   305] & [22014  2495   683   629   316] & [22573  2299   673   596   295] & [12809  4245  2972  1303  1258] \\
\hline top5\_prob & [0.78190308 0.0913006  0.02570379 0.0252958  0.01382656] & [0.79836077 0.09048379 0.02476971 0.02281134 0.01146007] & [0.8186335  0.08337564 0.02440705 0.02161456 0.01069848] & [0.46453181 0.15394937 0.10778269 0.04725466 0.04562269] \\
\hline nobs & 22059 & 27574 & 27574 & 27574 \\
\hline missing & 22059 & 0 & 0 & 0 \\
\hline
\end{tabular}
\end{table}

\begin{table}[H]
\centering
\fontsize{8}{14}\selectfont
\caption{Propiedades  estadisticas de variable m\_built, Economicos (A-2)}
\label{table-stats-economicos-a-2-m_built}
\begin{tabular}{|l|m{10em}|m{10em}|m{10em}|m{10em}|}
\hline
 \rowcolor[gray]{0.8}
Variable/Modelo & Real & tddpm\_mlp & smote-enc & ctgan \\
\hline top5 & [140.  60. 120.  50.  70.] & [140.  60.  50. 100. 120.] & [140.  60.  70.  50.  40.] & [1.00000e+00 4.87560e+02 6.95840e+02 2.33830e+02 1.83236e+03] \\
\hline top5\_freq & [700 467 444 431 415] & [861 555 539 505 497] & [349 162 152 148 144] & [16891     4     3     2     2] \\
\hline top5\_prob & [0.03173308 0.0211705  0.02012784 0.01953851 0.01881318] & [0.03122507 0.02012766 0.0195474  0.01831435 0.01802423] & [0.01265685 0.0058751  0.00551244 0.00536738 0.00522231] & [6.12569812e-01 1.45064191e-04 1.08798143e-04 7.25320955e-05
 7.25320955e-05] \\
\hline nobs & 22059 & 27574 & 27574 & 27574 \\
\hline missing & 0.000 & 0.000 & 0.000 & 0.000 \\
\hline mean & 1771 & 2230 & \bfseries 1383 & \cellcolor[rgb]{0.9, 0.54, 0.52} 453 \\
\hline std\_err & 664.365 & \bfseries 707.978 & 377.735 & \cellcolor[rgb]{0.9, 0.54, 0.52} 8.752 \\
\hline upper\_ci & 3073 & \bfseries 3618 & 2123 & \cellcolor[rgb]{0.9, 0.54, 0.52} 470 \\
\hline lower\_ci & 469.205 & \cellcolor[rgb]{0.9, 0.54, 0.52} 842.298 & 642.508 & \bfseries 436.022 \\
\hline std & 98673 & \bfseries 117563 & 62724 & \cellcolor[rgb]{0.9, 0.54, 0.52} 1453 \\
\hline iqr & 140.000 & 134.369 & \bfseries 139.781 & \cellcolor[rgb]{0.9, 0.54, 0.52} 544.865 \\
\hline iqr\_normal & 103.782 & 99.608 & \bfseries 103.620 & \cellcolor[rgb]{0.9, 0.54, 0.52} 403.909 \\
\hline mad & 3202 & 4128 & \bfseries 2435 & \cellcolor[rgb]{0.9, 0.54, 0.52} 609 \\
\hline mad\_normal & 4013 & 5174 & \bfseries 3052 & \cellcolor[rgb]{0.9, 0.54, 0.52} 763 \\
\hline coef\_var & 55.706 & \bfseries 52.721 & 45.359 & \cellcolor[rgb]{0.9, 0.54, 0.52} 3.207 \\
\hline range & 11999999 & \bfseries 11997684 & 5491367 & \cellcolor[rgb]{0.9, 0.54, 0.52} 46828 \\
\hline max & 12000000 & \bfseries 11997685 & 5491368 & \cellcolor[rgb]{0.9, 0.54, 0.52} 46829 \\
\hline min & 1.000 & 1.000 & 1.000 & 1.000 \\
\hline skew & 96.078 & \bfseries 70.629 & 67.003 & \cellcolor[rgb]{0.9, 0.54, 0.52} 15.230 \\
\hline kurtosis & 10659 & \bfseries 5724 & 4957 & \cellcolor[rgb]{0.9, 0.54, 0.52} 327 \\
\hline jarque\_bera & 1.04399e+11 & \bfseries 3.76215e+10 & 2.82200e+10 & \cellcolor[rgb]{0.9, 0.54, 0.52} 1.21319e+08 \\
\hline jarque\_bera\_pval & 0.000 & 0.000 & 0.000 & 0.000 \\
\hline mode & 140.000 & \bfseries 140.000 & \bfseries 140.000 & \cellcolor[rgb]{0.9, 0.54, 0.52} 1.000 \\
\hline mode\_freq & 0.032 & \bfseries 0.031 & 0.013 & \cellcolor[rgb]{0.9, 0.54, 0.52} 0.613 \\
\hline median & 107.000 & 105.000 & \bfseries 107.984 & \cellcolor[rgb]{0.9, 0.54, 0.52} 1.000 \\
\hline 0.1\% & 2.000 & \bfseries 2.256 & \cellcolor[rgb]{0.9, 0.54, 0.52} 16.265 & 1.000 \\
\hline 1.0\% & 23.000 & \bfseries 23.000 & 25.547 & \cellcolor[rgb]{0.9, 0.54, 0.52} 1.000 \\
\hline 5.0\% & 33.000 & \bfseries 33.351 & 33.891 & \cellcolor[rgb]{0.9, 0.54, 0.52} 1.000 \\
\hline 25.0\% & 60.000 & \bfseries 60.000 & 60.089 & \cellcolor[rgb]{0.9, 0.54, 0.52} 1.000 \\
\hline 75.0\% & 200.000 & 194.369 & \bfseries 199.870 & \cellcolor[rgb]{0.9, 0.54, 0.52} 545.865 \\
\hline 95.0\% & 490.000 & 450.107 & \bfseries 475.689 & \cellcolor[rgb]{0.9, 0.54, 0.52} 2006.255 \\
\hline 99.0\% & 1200 & 917 & \bfseries 1111 & \cellcolor[rgb]{0.9, 0.54, 0.52} 2534 \\
\hline 99.9\% & 37947 & \cellcolor[rgb]{0.9, 0.54, 0.52} 20788 & \bfseries 42146 & 24096 \\
\hline
\end{tabular}
\end{table}

\begin{table}[H]
\centering
\fontsize{8}{14}\selectfont
\caption{Propiedades  estadisticas de variable property\_type, Economicos (A-2)}
\label{table-stats-economicos-a-2-property_type}
\begin{tabular}{|l|m{10em}|m{10em}|m{10em}|m{10em}|}
\hline
 \rowcolor[gray]{0.8}
Variable/Modelo & Real & tddpm\_mlp & smote-enc & ctgan \\
\hline top5 & ['Departamento' 'Casa' 'Oficina o Casa Oficina' 'Parcela o Chacra'
 'Local o Casa comercial'] & ['Departamento' 'Casa' 'Oficina o Casa Oficina' 'Parcela o Chacra'
 'Local o Casa comercial'] & ['Departamento' 'Casa' 'Oficina o Casa Oficina' 'Parcela o Chacra'
 'Departamento Amoblado'] & ['Casa' 'Departamento' 'Oficina o Casa Oficina' 'Parcela o Chacra'
 'Local o Casa comercial'] \\
\hline top5\_freq & [10592  8911  1553   413   255] & [13595 11267  1869   404   194] & [13570 11704  1837   261   124] & [18606  5166  2132   625   416] \\
\hline top5\_prob & [0.48016683 0.4039621  0.0704021  0.01872252 0.01155991] & [0.49303692 0.40860956 0.06778124 0.01465148 0.00703561] & [0.49213027 0.42445782 0.06662073 0.00946544 0.00449699] & [0.67476608 0.1873504  0.07731921 0.02266628 0.01508668] \\
\hline nobs & 22059 & 27574 & 27574 & 27574 \\
\hline missing & 22059 & 0 & 0 & 0 \\
\hline
\end{tabular}
\end{table}




\section{Estadísticos Económicos - Conjunto B}
\label{propiedades-estadisticas-economicos-B}
\begin{table}[H]
\centering
\fontsize{8}{14}\selectfont
\caption{Propiedades  estadisticas de variable rooms, Economicos (A-2)}
\label{table-stats-economicos-a-2-rooms}
\begin{tabular}{|l|m{10em}|m{10em}|m{10em}|m{10em}|}
\hline
 \rowcolor[gray]{0.8}
Variable/Modelo & Real & tddpm\_mlp & smote-enc & ctgan \\
\hline top5 & [3. 2. 4. 1. 5.] & [3. 2. 4. 1. 5.] & [3. 2. 4. 1. 5.] & [2. 3. 4. 5. 6.] \\
\hline top5\_freq & [6355 4614 4168 2671 2232] & [8216 5798 5162 3327 2911] & [8191 5804 5468 3245 2819] & [6541 5670 4070 2674 2058] \\
\hline top5\_prob & [0.28809103 0.20916633 0.18894782 0.12108436 0.10118319] & [0.29796185 0.21027054 0.18720534 0.12065714 0.10557046] & [0.2970552  0.21048814 0.19830275 0.11768332 0.10223399] & [0.23721622 0.20562849 0.14760281 0.09697541 0.07463553] \\
\hline nobs & 22059 & 27574 & 27574 & 27574 \\
\hline missing & 0.000 & 0.000 & 0.000 & 0.000 \\
\hline mean & 3.446 & \bfseries 3.347 & 3.310 & \cellcolor[rgb]{0.9, 0.54, 0.52} 5.000 \\
\hline std\_err & 0.026 & \bfseries 0.024 & 0.012 & \cellcolor[rgb]{0.9, 0.54, 0.52} 0.047 \\
\hline upper\_ci & 3.497 & \bfseries 3.393 & 3.334 & \cellcolor[rgb]{0.9, 0.54, 0.52} 5.092 \\
\hline lower\_ci & 3.395 & \bfseries 3.301 & 3.286 & \cellcolor[rgb]{0.9, 0.54, 0.52} 4.909 \\
\hline std & 3.881 & \bfseries 3.924 & 1.993 & \cellcolor[rgb]{0.9, 0.54, 0.52} 7.749 \\
\hline iqr & 2.000 & \bfseries 2.000 & \bfseries 2.000 & \cellcolor[rgb]{0.9, 0.54, 0.52} 3.000 \\
\hline iqr\_normal & 1.483 & \bfseries 1.483 & \bfseries 1.483 & \cellcolor[rgb]{0.9, 0.54, 0.52} 2.224 \\
\hline mad & 1.454 & \bfseries 1.340 & 1.279 & \cellcolor[rgb]{0.9, 0.54, 0.52} 3.061 \\
\hline mad\_normal & 1.822 & \bfseries 1.679 & 1.603 & \cellcolor[rgb]{0.9, 0.54, 0.52} 3.836 \\
\hline coef\_var & 1.126 & \bfseries 1.172 & \cellcolor[rgb]{0.9, 0.54, 0.52} 0.602 & 1.550 \\
\hline range & 399.000 & \bfseries 399.000 & \cellcolor[rgb]{0.9, 0.54, 0.52} 56.000 & \bfseries 399.000 \\
\hline max & 400.000 & \bfseries 400.000 & \cellcolor[rgb]{0.9, 0.54, 0.52} 57.000 & \bfseries 400.000 \\
\hline min & 1.000 & 1.000 & 1.000 & 1.000 \\
\hline skew & 57.785 & \bfseries 75.482 & \cellcolor[rgb]{0.9, 0.54, 0.52} 4.962 & 25.543 \\
\hline kurtosis & 5331 & \bfseries 7575 & \cellcolor[rgb]{0.9, 0.54, 0.52} 68 & 1106 \\
\hline jarque\_bera & 2.61061e+10 & \cellcolor[rgb]{0.9, 0.54, 0.52} 6.59026e+10 & 4.96441e+06 & \bfseries 1.40103e+09 \\
\hline jarque\_bera\_pval & 0.000 & 0.000 & 0.000 & 0.000 \\
\hline mode & 3.000 & \bfseries 3.000 & \bfseries 3.000 & \cellcolor[rgb]{0.9, 0.54, 0.52} 2.000 \\
\hline mode\_freq & 0.288 & 0.298 & \bfseries 0.297 & \cellcolor[rgb]{0.9, 0.54, 0.52} 0.237 \\
\hline median & 3.000 & 3.000 & 3.000 & 3.000 \\
\hline 0.1\% & 1.000 & 1.000 & 1.000 & 1.000 \\
\hline 1.0\% & 1.000 & 1.000 & 1.000 & 1.000 \\
\hline 5.0\% & 1.000 & 1.000 & 1.000 & 1.000 \\
\hline 25.0\% & 2.000 & 2.000 & 2.000 & 2.000 \\
\hline 75.0\% & 4.000 & \bfseries 4.000 & \bfseries 4.000 & \cellcolor[rgb]{0.9, 0.54, 0.52} 5.000 \\
\hline 95.0\% & 6.000 & \bfseries 6.000 & \bfseries 6.000 & \cellcolor[rgb]{0.9, 0.54, 0.52} 14.000 \\
\hline 99.0\% & 12.000 & \bfseries 10.000 & \bfseries 10.000 & \cellcolor[rgb]{0.9, 0.54, 0.52} 25.000 \\
\hline 99.9\% & 25.000 & 24.000 & \bfseries 25.000 & \cellcolor[rgb]{0.9, 0.54, 0.52} 57.000 \\
\hline
\end{tabular}
\end{table}

\begin{table}[H]
\centering
\fontsize{8}{14}\selectfont
\caption{Propiedades  estadisticas de variable bathrooms, Economicos (A-2)}
\label{table-stats-economicos-a-2-bathrooms}
\begin{tabular}{|l|m{10em}|m{10em}|m{10em}|m{10em}|}
\hline
 \rowcolor[gray]{0.8}
Variable/Modelo & Real & tddpm\_mlp & smote-enc & ctgan \\
\hline top5 & [2. 1. 3. 4. 5.] & [2. 1. 3. 4. 5.] & [2. 1. 3. 4. 5.] & [3. 1. 2. 4. 5.] \\
\hline top5\_freq & [7511 5440 4486 2665 1084] & [9481 6610 5861 3425 1331] & [9438 6742 5612 3270 1387] & [7474 5459 4858 3762 3251] \\
\hline top5\_prob & [0.34049594 0.24661136 0.20336371 0.12081237 0.04914094] & [0.3438384  0.23971858 0.21255531 0.12421121 0.04827011] & [0.34227896 0.24450569 0.20352506 0.11858998 0.05030101] & [0.27105244 0.19797635 0.17618046 0.13643287 0.11790092] \\
\hline nobs & 22059 & 27574 & 27574 & 27574 \\
\hline missing & 0.000 & 0.000 & 0.000 & 0.000 \\
\hline mean & 2.604 & 2.544 & \bfseries 2.593 & \cellcolor[rgb]{0.9, 0.54, 0.52} 3.679 \\
\hline std\_err & 0.025 & \bfseries 0.018 & 0.013 & \cellcolor[rgb]{0.9, 0.54, 0.52} 0.058 \\
\hline upper\_ci & 2.652 & 2.579 & \bfseries 2.618 & \cellcolor[rgb]{0.9, 0.54, 0.52} 3.792 \\
\hline lower\_ci & 2.556 & 2.508 & \bfseries 2.568 & \cellcolor[rgb]{0.9, 0.54, 0.52} 3.566 \\
\hline std & 3.655 & \bfseries 3.001 & 2.133 & \cellcolor[rgb]{0.9, 0.54, 0.52} 9.573 \\
\hline iqr & 1.000 & \bfseries 1.000 & \bfseries 1.000 & \cellcolor[rgb]{0.9, 0.54, 0.52} 2.000 \\
\hline iqr\_normal & 0.741 & \bfseries 0.741 & \bfseries 0.741 & \cellcolor[rgb]{0.9, 0.54, 0.52} 1.483 \\
\hline mad & 1.203 & 1.114 & \bfseries 1.185 & \cellcolor[rgb]{0.9, 0.54, 0.52} 2.020 \\
\hline mad\_normal & 1.507 & 1.396 & \bfseries 1.485 & \cellcolor[rgb]{0.9, 0.54, 0.52} 2.532 \\
\hline coef\_var & 1.404 & \bfseries 1.180 & 0.822 & \cellcolor[rgb]{0.9, 0.54, 0.52} 2.602 \\
\hline range & 435.000 & \bfseries 435.000 & \cellcolor[rgb]{0.9, 0.54, 0.52} 179.000 & \bfseries 435.000 \\
\hline max & 436.000 & \bfseries 436.000 & \cellcolor[rgb]{0.9, 0.54, 0.52} 180.000 & \bfseries 436.000 \\
\hline min & 1.000 & 1.000 & 1.000 & 1.000 \\
\hline skew & 82.448 & \bfseries 109.783 & \cellcolor[rgb]{0.9, 0.54, 0.52} 28.339 & 38.404 \\
\hline kurtosis & 9252 & \bfseries 15795 & 2009 & \cellcolor[rgb]{0.9, 0.54, 0.52} 1685 \\
\hline jarque\_bera & 7.86518e+10 & \cellcolor[rgb]{0.9, 0.54, 0.52} 2.86587e+11 & \bfseries 4.62894e+09 & 3.25889e+09 \\
\hline jarque\_bera\_pval & 0.000 & 0.000 & 0.000 & 0.000 \\
\hline mode & 2.000 & \bfseries 2.000 & \bfseries 2.000 & \cellcolor[rgb]{0.9, 0.54, 0.52} 3.000 \\
\hline mode\_freq & 0.340 & 0.344 & \bfseries 0.342 & \cellcolor[rgb]{0.9, 0.54, 0.52} 0.271 \\
\hline median & 2.000 & \bfseries 2.000 & \bfseries 2.000 & \cellcolor[rgb]{0.9, 0.54, 0.52} 3.000 \\
\hline 0.1\% & 1.000 & 1.000 & 1.000 & 1.000 \\
\hline 1.0\% & 1.000 & 1.000 & 1.000 & 1.000 \\
\hline 5.0\% & 1.000 & 1.000 & 1.000 & 1.000 \\
\hline 25.0\% & 2.000 & 2.000 & 2.000 & 2.000 \\
\hline 75.0\% & 3.000 & \bfseries 3.000 & \bfseries 3.000 & \cellcolor[rgb]{0.9, 0.54, 0.52} 4.000 \\
\hline 95.0\% & 5.000 & \bfseries 5.000 & \bfseries 5.000 & \cellcolor[rgb]{0.9, 0.54, 0.52} 8.000 \\
\hline 99.0\% & 8.000 & 7.000 & \bfseries 8.000 & \cellcolor[rgb]{0.9, 0.54, 0.52} 14.000 \\
\hline 99.9\% & 17.942 & 13.000 & \bfseries 17.427 & \cellcolor[rgb]{0.9, 0.54, 0.52} 57.000 \\
\hline
\end{tabular}
\end{table}

\begin{table}[H]
\centering
\fontsize{8}{14}\selectfont
\caption{Propiedades  estadisticas de variable \_price, Economicos (A-2)}
\label{table-stats-economicos-a-2-_price}
\begin{tabular}{|l|m{10em}|m{10em}|m{10em}|m{10em}|}
\hline
 \rowcolor[gray]{0.8}
Variable/Modelo & Real & tddpm\_mlp & smote-enc & ctgan \\
\hline top5 & [12500. 10500. 11500.  8500.  9000.] & [11500. 12500.  8500. 20000. 13500.] & [10500. 11500.  2100.  9000.  9500.] & [    0.          1873.91702513 41042.06971567 41062.90431098
 41066.23801271] \\
\hline top5\_freq & [104  99  91  86  85] & [102 101  99  99  94] & [165 148 142 141 139] & [17440     2     1     1     1] \\
\hline top5\_prob & [0.00471463 0.00448796 0.0041253  0.00389864 0.0038533 ] & [0.00369914 0.00366287 0.00359034 0.00359034 0.00340901] & [0.0059839  0.00536738 0.00514978 0.00511351 0.00504098] & [6.32479872e-01 7.25320955e-05 3.62660477e-05 3.62660477e-05
 3.62660477e-05] \\
\hline nobs & 22059 & 27574 & 27574 & 27574 \\
\hline missing & 0.000 & 0.000 & 0.000 & 0.000 \\
\hline mean & 110379 & \bfseries 52905 & 30808 & \cellcolor[rgb]{0.9, 0.54, 0.52} 12067 \\
\hline std\_err & 32746 & \bfseries 19395 & 12251 & \cellcolor[rgb]{0.9, 0.54, 0.52} 130 \\
\hline upper\_ci & 174559 & \bfseries 90919 & 54820 & \cellcolor[rgb]{0.9, 0.54, 0.52} 12322 \\
\hline lower\_ci & 46199 & \bfseries 14891 & \cellcolor[rgb]{0.9, 0.54, 0.52} 6797 & 11812 \\
\hline std & 4863477 & \bfseries 3220683 & 2034332 & \cellcolor[rgb]{0.9, 0.54, 0.52} 21633 \\
\hline iqr & 9959 & \bfseries 9948 & 10425 & \cellcolor[rgb]{0.9, 0.54, 0.52} 16851 \\
\hline iqr\_normal & 7383 & \bfseries 7374 & 7728 & \cellcolor[rgb]{0.9, 0.54, 0.52} 12492 \\
\hline mad & 202281 & \bfseries 88001 & 44740 & \cellcolor[rgb]{0.9, 0.54, 0.52} 16342 \\
\hline mad\_normal & 253522 & \bfseries 110293 & 56073 & \cellcolor[rgb]{0.9, 0.54, 0.52} 20481 \\
\hline coef\_var & 44.062 & \bfseries 60.877 & 66.032 & \cellcolor[rgb]{0.9, 0.54, 0.52} 1.793 \\
\hline range & 390000000 & \bfseries 362636196 & 279000000 & \cellcolor[rgb]{0.9, 0.54, 0.52} 124189 \\
\hline max & 390000000 & \bfseries 362636196 & 279000000 & \cellcolor[rgb]{0.9, 0.54, 0.52} 124189 \\
\hline min & 0.000 & 0.000 & 0.000 & 0.000 \\
\hline skew & 60.579 & \bfseries 88.292 & 114.793 & \cellcolor[rgb]{0.9, 0.54, 0.52} 1.959 \\
\hline kurtosis & 4067 & 8564 & \cellcolor[rgb]{0.9, 0.54, 0.52} 14362 & \bfseries 6 \\
\hline jarque\_bera & 1.51936e+10 & 8.42493e+10 & \cellcolor[rgb]{0.9, 0.54, 0.52} 2.36952e+11 & \bfseries 2.97507e+04 \\
\hline jarque\_bera\_pval & 0.000 & 0.000 & 0.000 & 0.000 \\
\hline mode & 12500 & \bfseries 11500 & 10500 & \cellcolor[rgb]{0.9, 0.54, 0.52} 0 \\
\hline mode\_freq & 0.005 & \bfseries 0.004 & 0.006 & \cellcolor[rgb]{0.9, 0.54, 0.52} 0.632 \\
\hline median & 5084 & 5194 & \bfseries 5085 & \cellcolor[rgb]{0.9, 0.54, 0.52} 0 \\
\hline 0.1\% & 0.263 & 0.417 & \bfseries 0.287 & \cellcolor[rgb]{0.9, 0.54, 0.52} 0.000 \\
\hline 1.0\% & 6.270 & 7.795 & \bfseries 7.060 & \cellcolor[rgb]{0.9, 0.54, 0.52} 0.000 \\
\hline 5.0\% & 11.760 & 12.100 & \bfseries 11.902 & \cellcolor[rgb]{0.9, 0.54, 0.52} 0.000 \\
\hline 25.0\% & 2041 & 2162 & \bfseries 2065 & \cellcolor[rgb]{0.9, 0.54, 0.52} 0 \\
\hline 75.0\% & 12000 & \bfseries 12109 & 12490 & \cellcolor[rgb]{0.9, 0.54, 0.52} 16851 \\
\hline 95.0\% & 32000 & 30839 & \bfseries 32000 & \cellcolor[rgb]{0.9, 0.54, 0.52} 62115 \\
\hline 99.0\% & 58942 & 53875 & \bfseries 55000 & \cellcolor[rgb]{0.9, 0.54, 0.52} 88157 \\
\hline 99.9\% & 262695 & 110019 & \bfseries 120000 & \cellcolor[rgb]{0.9, 0.54, 0.52} 109231 \\
\hline
\end{tabular}
\end{table}

\begin{table}[H]
\centering
\fontsize{8}{14}\selectfont
\caption{Propiedades  estadisticas de variable transaction\_type, Economicos (A-2)}
\label{table-stats-economicos-a-2-transaction_type}
\begin{tabular}{|l|m{10em}|m{10em}|m{10em}|m{10em}|}
\hline
 \rowcolor[gray]{0.8}
Variable/Modelo & Real & tddpm\_mlp & smote-enc & ctgan \\
\hline top5 & ['Venta' 'Arriendo' 'Busco arriendo' 'Compro'] & ['Venta' 'Arriendo'] & ['Venta' 'Arriendo'] & ['Venta' 'Arriendo' 'Compro' 'Busco arriendo'] \\
\hline top5\_freq & [17540  4517     1     1] & [22168  5406] & [21978  5596] & [20557  5163  1830    24] \\
\hline top5\_prob & [7.95140306e-01 2.04769029e-01 4.53329707e-05 4.53329707e-05] & [0.80394575 0.19605425] & [0.7970552 0.2029448] & [0.74552114 0.1872416  0.06636687 0.00087039] \\
\hline nobs & 22059 & 27574 & 27574 & 27574 \\
\hline missing & 22059 & 0 & 0 & 0 \\
\hline
\end{tabular}
\end{table}

\begin{table}[H]
\centering
\fontsize{8}{14}\selectfont
\caption{Propiedades  estadisticas de variable m\_size, Economicos (A-2)}
\label{table-stats-economicos-a-2-m_size}
\begin{tabular}{|l|m{10em}|m{10em}|m{10em}|m{10em}|}
\hline
 \rowcolor[gray]{0.8}
Variable/Modelo & Real & tddpm\_mlp & smote-enc & ctgan \\
\hline top5 & [5000.   50.   60.  200.   70.] & [5000.   50.   60.  200.  100.] & [5000.   40.   30.   35.   50.] & [     0.   196535.56 264489.37 276876.36 189529.51] \\
\hline top5\_freq & [601 342 321 285 281] & [750 415 363 354 349] & [164  73  69  61  61] & [13794     2     2     2     2] \\
\hline top5\_prob & [0.02724512 0.01550388 0.01455188 0.0129199  0.01273856] & [0.02719954 0.01505041 0.01316458 0.01283818 0.01265685] & [0.00594763 0.00264742 0.00250236 0.00221223 0.00221223] & [5.00253862e-01 7.25320955e-05 7.25320955e-05 7.25320955e-05
 7.25320955e-05] \\
\hline nobs & 22059 & 27574 & 27574 & 27574 \\
\hline missing & 0.000 & 0.000 & 0.000 & 0.000 \\
\hline mean & 146269 & \cellcolor[rgb]{0.9, 0.54, 0.52} 3544115 & 254142 & \bfseries 69905 \\
\hline std\_err & 105454 & \cellcolor[rgb]{0.9, 0.54, 0.52} 1526171 & \bfseries 111829 & 576 \\
\hline upper\_ci & 352956 & \cellcolor[rgb]{0.9, 0.54, 0.52} 6535355 & \bfseries 473322 & 71034 \\
\hline lower\_ci & -60417 & \cellcolor[rgb]{0.9, 0.54, 0.52} 552876 & \bfseries 34962 & 68776 \\
\hline std & 15662334 & \cellcolor[rgb]{0.9, 0.54, 0.52} 253427077 & \bfseries 18569598 & 95655 \\
\hline iqr & 340.500 & 314.542 & \bfseries 362.825 & \cellcolor[rgb]{0.9, 0.54, 0.52} 127954.560 \\
\hline iqr\_normal & 252.413 & 233.171 & \bfseries 268.963 & \cellcolor[rgb]{0.9, 0.54, 0.52} 94852.857 \\
\hline mad & 290635 & \cellcolor[rgb]{0.9, 0.54, 0.52} 7084741 & 506460 & \bfseries 79853 \\
\hline mad\_normal & 364257 & \cellcolor[rgb]{0.9, 0.54, 0.52} 8879406 & 634753 & \bfseries 100081 \\
\hline coef\_var & 107.079 & 71.506 & \bfseries 73.068 & \cellcolor[rgb]{0.9, 0.54, 0.52} 1.368 \\
\hline range & 2.24100e+09 & \cellcolor[rgb]{0.9, 0.54, 0.52} 3.09336e+10 & \bfseries 1.76077e+09 & 4.05049e+05 \\
\hline max & 2.24100e+09 & \cellcolor[rgb]{0.9, 0.54, 0.52} 3.09336e+10 & \bfseries 1.76077e+09 & 4.05049e+05 \\
\hline min & 0.000 & 0.000 & \cellcolor[rgb]{0.9, 0.54, 0.52} 1.000 & \bfseries 0.000 \\
\hline skew & 134.762 & \bfseries 91.081 & 82.386 & \cellcolor[rgb]{0.9, 0.54, 0.52} 1.220 \\
\hline kurtosis & 19053 & \bfseries 9527 & 6994 & \cellcolor[rgb]{0.9, 0.54, 0.52} 3 \\
\hline jarque\_bera & 3.33616e+11 & \bfseries 1.04259e+11 & 5.61901e+10 & \cellcolor[rgb]{0.9, 0.54, 0.52} 6.98701e+03 \\
\hline jarque\_bera\_pval & 0.000 & 0.000 & 0.000 & 0.000 \\
\hline mode & 5000 & \bfseries 5000 & \bfseries 5000 & \cellcolor[rgb]{0.9, 0.54, 0.52} 0 \\
\hline mode\_freq & 0.027 & \bfseries 0.027 & 0.006 & \cellcolor[rgb]{0.9, 0.54, 0.52} 0.500 \\
\hline median & 145.000 & \bfseries 141.875 & 148.714 & \cellcolor[rgb]{0.9, 0.54, 0.52} 0.000 \\
\hline 0.1\% & 2.000 & \bfseries 3.272 & \cellcolor[rgb]{0.9, 0.54, 0.52} 13.633 & 0.000 \\
\hline 1.0\% & 22.000 & \bfseries 23.048 & 25.428 & \cellcolor[rgb]{0.9, 0.54, 0.52} 0.000 \\
\hline 5.0\% & 35.000 & \bfseries 35.000 & 35.011 & \cellcolor[rgb]{0.9, 0.54, 0.52} 0.000 \\
\hline 25.0\% & 66.000 & \bfseries 66.000 & 67.454 & \cellcolor[rgb]{0.9, 0.54, 0.52} 0.000 \\
\hline 75.0\% & 406.500 & 380.542 & \bfseries 430.279 & \cellcolor[rgb]{0.9, 0.54, 0.52} 127954.560 \\
\hline 95.0\% & 5000 & \bfseries 5000 & 4241 & \cellcolor[rgb]{0.9, 0.54, 0.52} 273416 \\
\hline 99.0\% & 10200 & 8114 & \bfseries 8945 & \cellcolor[rgb]{0.9, 0.54, 0.52} 338370 \\
\hline 99.9\% & 70000 & 36560 & \bfseries 51349 & \cellcolor[rgb]{0.9, 0.54, 0.52} 379858 \\
\hline
\end{tabular}
\end{table}

\begin{table}[H]
\centering
\fontsize{8}{14}\selectfont
\caption{Propiedades  estadisticas de variable county, Economicos (A-2)}
\label{table-stats-economicos-a-2-county}
\begin{tabular}{|l|m{10em}|m{10em}|m{10em}|m{10em}|}
\hline
 \rowcolor[gray]{0.8}
Variable/Modelo & Real & tddpm\_mlp & smote-enc & ctgan \\
\hline top5 & ['Las Condes' 'Santiago' 'Providencia' 'Vitacura' 'Lo Barnechea'] & ['Las Condes' 'Santiago' 'Providencia' 'Vitacura' 'Lo Barnechea'] & ['Las Condes' 'Santiago' 'Providencia' 'Vitacura' 'Lo Barnechea'] & ['Las Condes' 'Viña del Mar' 'Santiago' 'Vitacura' 'Providencia'] \\
\hline top5\_freq & [3233 2703 1481 1415 1322] & [4220 3346 1908 1905 1783] & [4590 3812 1909 1902 1859] & [3009 2080 2003 1705 1610] \\
\hline top5\_prob & [0.14656149 0.12253502 0.06713813 0.06414615 0.05993019] & [0.15304272 0.1213462  0.06919562 0.06908682 0.06466236] & [0.16646116 0.13824617 0.06923189 0.06897802 0.06741858] & [0.10912454 0.07543338 0.07264089 0.06183361 0.05838834] \\
\hline nobs & 22059 & 27574 & 27574 & 27574 \\
\hline missing & 22059 & 0 & 0 & 0 \\
\hline
\end{tabular}
\end{table}

\begin{table}[H]
\centering
\fontsize{8}{14}\selectfont
\caption{Propiedades  estadisticas de variable publication\_date, Economicos (A-2)}
\label{table-stats-economicos-a-2-publication_date}
\begin{tabular}{|l|m{10em}|m{10em}|m{10em}|m{10em}|}
\hline
 \rowcolor[gray]{0.8}
Variable/Modelo & Real & tddpm\_mlp & smote-enc & ctgan \\
\hline top5 & [1545 1693 1546 1549  721] & [1545. 1693. 1546. 1549.  721.] & [1545. 1693. 1546. 1549.  721.] & [1693 1550 1549 1548 1551] \\
\hline top5\_freq & [10883  6103   895   320   125] & [14363  7572  1044   190   114] & [13090  7531   516   114    82] & [6007  827  821  812  781] \\
\hline top5\_prob & [0.49335872 0.27666712 0.04057301 0.01450655 0.00566662] & [0.52088924 0.27460651 0.03786175 0.00689055 0.00413433] & [0.47472256 0.27311961 0.01871328 0.00413433 0.00297382] & [0.21785015 0.02999202 0.02977443 0.02944803 0.02832378] \\
\hline nobs & 22059 & 27574 & 27574 & 27574 \\
\hline missing & 0.000 & 0.000 & 0.000 & 0.000 \\
\hline mean & 1471 & 1474 & \bfseries 1471 & \cellcolor[rgb]{0.9, 0.54, 0.52} 1488 \\
\hline std\_err & 2.056 & \cellcolor[rgb]{0.9, 0.54, 0.52} 1.815 & 1.842 & \bfseries 1.849 \\
\hline upper\_ci & 1475 & 1477 & \bfseries 1474 & \cellcolor[rgb]{0.9, 0.54, 0.52} 1491 \\
\hline lower\_ci & 1467 & 1470 & \bfseries 1467 & \cellcolor[rgb]{0.9, 0.54, 0.52} 1484 \\
\hline std & 305.403 & \cellcolor[rgb]{0.9, 0.54, 0.52} 301.398 & \bfseries 305.857 & 307.103 \\
\hline iqr & 148.000 & \bfseries 148.000 & \bfseries 148.000 & \cellcolor[rgb]{0.9, 0.54, 0.52} 149.000 \\
\hline iqr\_normal & 109.713 & \bfseries 109.713 & \bfseries 109.713 & \cellcolor[rgb]{0.9, 0.54, 0.52} 110.454 \\
\hline mad & 206.755 & \cellcolor[rgb]{0.9, 0.54, 0.52} 201.911 & \bfseries 207.198 & 206.070 \\
\hline mad\_normal & 259.128 & \cellcolor[rgb]{0.9, 0.54, 0.52} 253.058 & \bfseries 259.684 & 258.271 \\
\hline coef\_var & 0.208 & \cellcolor[rgb]{0.9, 0.54, 0.52} 0.205 & \bfseries 0.208 & 0.206 \\
\hline range & 1489 & 1490 & \bfseries 1488 & \cellcolor[rgb]{0.9, 0.54, 0.52} 1312 \\
\hline max & 1693 & 1693 & 1693 & 1693 \\
\hline min & 204.000 & 202.742 & \bfseries 205.000 & \cellcolor[rgb]{0.9, 0.54, 0.52} 381.000 \\
\hline skew & -2.019 & -2.047 & \bfseries -2.015 & \cellcolor[rgb]{0.9, 0.54, 0.52} -2.062 \\
\hline kurtosis & 5.892 & 6.003 & \bfseries 5.852 & \cellcolor[rgb]{0.9, 0.54, 0.52} 6.098 \\
\hline jarque\_bera & 22679 & 29629 & \bfseries 28015 & \cellcolor[rgb]{0.9, 0.54, 0.52} 30563 \\
\hline jarque\_bera\_pval & 0.000 & 0.000 & 0.000 & 0.000 \\
\hline mode & 1545 & \bfseries 1545 & \bfseries 1545 & \cellcolor[rgb]{0.9, 0.54, 0.52} 1693 \\
\hline mode\_freq & 0.493 & 0.521 & \bfseries 0.475 & \cellcolor[rgb]{0.9, 0.54, 0.52} 0.218 \\
\hline median & 1545 & \bfseries 1545 & \bfseries 1545 & \cellcolor[rgb]{0.9, 0.54, 0.52} 1551 \\
\hline 0.1\% & 450.696 & \cellcolor[rgb]{0.9, 0.54, 0.52} 497.193 & 483.807 & \bfseries 424.573 \\
\hline 1.0\% & 531.000 & 538.101 & \bfseries 535.709 & \cellcolor[rgb]{0.9, 0.54, 0.52} 501.000 \\
\hline 5.0\% & 628.900 & \bfseries 631.192 & 619.511 & \cellcolor[rgb]{0.9, 0.54, 0.52} 659.000 \\
\hline 25.0\% & 1545 & \bfseries 1545 & \bfseries 1545 & \cellcolor[rgb]{0.9, 0.54, 0.52} 1541 \\
\hline 75.0\% & 1693 & \bfseries 1693 & \bfseries 1693 & \cellcolor[rgb]{0.9, 0.54, 0.52} 1690 \\
\hline 95.0\% & 1693 & 1693 & 1693 & 1693 \\
\hline 99.0\% & 1693 & 1693 & 1693 & 1693 \\
\hline 99.9\% & 1693 & 1693 & 1693 & 1693 \\
\hline
\end{tabular}
\end{table}

\begin{table}[H]
\centering
\fontsize{8}{14}\selectfont
\caption{Propiedades  estadisticas de variable state, Economicos (A-2)}
\label{table-stats-economicos-a-2-state}
\begin{tabular}{|l|m{10em}|m{10em}|m{10em}|m{10em}|}
\hline
 \rowcolor[gray]{0.8}
Variable/Modelo & Real & tddpm\_mlp & smote-enc & ctgan \\
\hline top5 & ['Metropolitana de Santiago' 'Valparaíso' 'Coquimbo' 'Araucanía'
 "Libertador General Bernardo O'higgins"] & ['Metropolitana de Santiago' 'Valparaíso' 'Coquimbo' 'Araucanía' 'Biobío'] & ['Metropolitana de Santiago' 'Valparaíso' 'Coquimbo' 'Araucanía' 'Maule'] & ['Metropolitana de Santiago' 'Valparaíso' 'Araucanía'
 "Libertador General Bernardo O'higgins" 'Coquimbo'] \\
\hline top5\_freq & [17248  2014   567   558   305] & [22014  2495   683   629   316] & [22573  2299   673   596   295] & [12809  4245  2972  1303  1258] \\
\hline top5\_prob & [0.78190308 0.0913006  0.02570379 0.0252958  0.01382656] & [0.79836077 0.09048379 0.02476971 0.02281134 0.01146007] & [0.8186335  0.08337564 0.02440705 0.02161456 0.01069848] & [0.46453181 0.15394937 0.10778269 0.04725466 0.04562269] \\
\hline nobs & 22059 & 27574 & 27574 & 27574 \\
\hline missing & 22059 & 0 & 0 & 0 \\
\hline
\end{tabular}
\end{table}

\begin{table}[H]
\centering
\fontsize{8}{14}\selectfont
\caption{Propiedades  estadisticas de variable m\_built, Economicos (A-2)}
\label{table-stats-economicos-a-2-m_built}
\begin{tabular}{|l|m{10em}|m{10em}|m{10em}|m{10em}|}
\hline
 \rowcolor[gray]{0.8}
Variable/Modelo & Real & tddpm\_mlp & smote-enc & ctgan \\
\hline top5 & [140.  60. 120.  50.  70.] & [140.  60.  50. 100. 120.] & [140.  60.  70.  50.  40.] & [1.00000e+00 4.87560e+02 6.95840e+02 2.33830e+02 1.83236e+03] \\
\hline top5\_freq & [700 467 444 431 415] & [861 555 539 505 497] & [349 162 152 148 144] & [16891     4     3     2     2] \\
\hline top5\_prob & [0.03173308 0.0211705  0.02012784 0.01953851 0.01881318] & [0.03122507 0.02012766 0.0195474  0.01831435 0.01802423] & [0.01265685 0.0058751  0.00551244 0.00536738 0.00522231] & [6.12569812e-01 1.45064191e-04 1.08798143e-04 7.25320955e-05
 7.25320955e-05] \\
\hline nobs & 22059 & 27574 & 27574 & 27574 \\
\hline missing & 0.000 & 0.000 & 0.000 & 0.000 \\
\hline mean & 1771 & 2230 & \bfseries 1383 & \cellcolor[rgb]{0.9, 0.54, 0.52} 453 \\
\hline std\_err & 664.365 & \bfseries 707.978 & 377.735 & \cellcolor[rgb]{0.9, 0.54, 0.52} 8.752 \\
\hline upper\_ci & 3073 & \bfseries 3618 & 2123 & \cellcolor[rgb]{0.9, 0.54, 0.52} 470 \\
\hline lower\_ci & 469.205 & \cellcolor[rgb]{0.9, 0.54, 0.52} 842.298 & 642.508 & \bfseries 436.022 \\
\hline std & 98673 & \bfseries 117563 & 62724 & \cellcolor[rgb]{0.9, 0.54, 0.52} 1453 \\
\hline iqr & 140.000 & 134.369 & \bfseries 139.781 & \cellcolor[rgb]{0.9, 0.54, 0.52} 544.865 \\
\hline iqr\_normal & 103.782 & 99.608 & \bfseries 103.620 & \cellcolor[rgb]{0.9, 0.54, 0.52} 403.909 \\
\hline mad & 3202 & 4128 & \bfseries 2435 & \cellcolor[rgb]{0.9, 0.54, 0.52} 609 \\
\hline mad\_normal & 4013 & 5174 & \bfseries 3052 & \cellcolor[rgb]{0.9, 0.54, 0.52} 763 \\
\hline coef\_var & 55.706 & \bfseries 52.721 & 45.359 & \cellcolor[rgb]{0.9, 0.54, 0.52} 3.207 \\
\hline range & 11999999 & \bfseries 11997684 & 5491367 & \cellcolor[rgb]{0.9, 0.54, 0.52} 46828 \\
\hline max & 12000000 & \bfseries 11997685 & 5491368 & \cellcolor[rgb]{0.9, 0.54, 0.52} 46829 \\
\hline min & 1.000 & 1.000 & 1.000 & 1.000 \\
\hline skew & 96.078 & \bfseries 70.629 & 67.003 & \cellcolor[rgb]{0.9, 0.54, 0.52} 15.230 \\
\hline kurtosis & 10659 & \bfseries 5724 & 4957 & \cellcolor[rgb]{0.9, 0.54, 0.52} 327 \\
\hline jarque\_bera & 1.04399e+11 & \bfseries 3.76215e+10 & 2.82200e+10 & \cellcolor[rgb]{0.9, 0.54, 0.52} 1.21319e+08 \\
\hline jarque\_bera\_pval & 0.000 & 0.000 & 0.000 & 0.000 \\
\hline mode & 140.000 & \bfseries 140.000 & \bfseries 140.000 & \cellcolor[rgb]{0.9, 0.54, 0.52} 1.000 \\
\hline mode\_freq & 0.032 & \bfseries 0.031 & 0.013 & \cellcolor[rgb]{0.9, 0.54, 0.52} 0.613 \\
\hline median & 107.000 & 105.000 & \bfseries 107.984 & \cellcolor[rgb]{0.9, 0.54, 0.52} 1.000 \\
\hline 0.1\% & 2.000 & \bfseries 2.256 & \cellcolor[rgb]{0.9, 0.54, 0.52} 16.265 & 1.000 \\
\hline 1.0\% & 23.000 & \bfseries 23.000 & 25.547 & \cellcolor[rgb]{0.9, 0.54, 0.52} 1.000 \\
\hline 5.0\% & 33.000 & \bfseries 33.351 & 33.891 & \cellcolor[rgb]{0.9, 0.54, 0.52} 1.000 \\
\hline 25.0\% & 60.000 & \bfseries 60.000 & 60.089 & \cellcolor[rgb]{0.9, 0.54, 0.52} 1.000 \\
\hline 75.0\% & 200.000 & 194.369 & \bfseries 199.870 & \cellcolor[rgb]{0.9, 0.54, 0.52} 545.865 \\
\hline 95.0\% & 490.000 & 450.107 & \bfseries 475.689 & \cellcolor[rgb]{0.9, 0.54, 0.52} 2006.255 \\
\hline 99.0\% & 1200 & 917 & \bfseries 1111 & \cellcolor[rgb]{0.9, 0.54, 0.52} 2534 \\
\hline 99.9\% & 37947 & \cellcolor[rgb]{0.9, 0.54, 0.52} 20788 & \bfseries 42146 & 24096 \\
\hline
\end{tabular}
\end{table}

\begin{table}[H]
\centering
\fontsize{8}{14}\selectfont
\caption{Propiedades  estadisticas de variable property\_type, Economicos (A-2)}
\label{table-stats-economicos-a-2-property_type}
\begin{tabular}{|l|m{10em}|m{10em}|m{10em}|m{10em}|}
\hline
 \rowcolor[gray]{0.8}
Variable/Modelo & Real & tddpm\_mlp & smote-enc & ctgan \\
\hline top5 & ['Departamento' 'Casa' 'Oficina o Casa Oficina' 'Parcela o Chacra'
 'Local o Casa comercial'] & ['Departamento' 'Casa' 'Oficina o Casa Oficina' 'Parcela o Chacra'
 'Local o Casa comercial'] & ['Departamento' 'Casa' 'Oficina o Casa Oficina' 'Parcela o Chacra'
 'Departamento Amoblado'] & ['Casa' 'Departamento' 'Oficina o Casa Oficina' 'Parcela o Chacra'
 'Local o Casa comercial'] \\
\hline top5\_freq & [10592  8911  1553   413   255] & [13595 11267  1869   404   194] & [13570 11704  1837   261   124] & [18606  5166  2132   625   416] \\
\hline top5\_prob & [0.48016683 0.4039621  0.0704021  0.01872252 0.01155991] & [0.49303692 0.40860956 0.06778124 0.01465148 0.00703561] & [0.49213027 0.42445782 0.06662073 0.00946544 0.00449699] & [0.67476608 0.1873504  0.07731921 0.02266628 0.01508668] \\
\hline nobs & 22059 & 27574 & 27574 & 27574 \\
\hline missing & 22059 & 0 & 0 & 0 \\
\hline
\end{tabular}
\end{table}



\chapter{Ejemplos de código y configuraciones}
Este capítulo de Anexos proporciona información adicional y detallada que respalda la investigación realizada en esta tesis. Aunque estos detalles son esenciales para el completo entendimiento de la investigación, se han incluido en los anexos para mantener la fluidez del cuerpo principal de la tesis.

En las siguientes secciones, se presentan diversos elementos suplementarios. El código de entrenamiento de modelos económicos se proporciona para dar visibilidad a los métodos de aprendizaje automático utilizados. Se incluyen gráficos detallados de correlaciones y estadísticas para los conjuntos de datos utilizados, aportando un análisis más profundo de las características y estructuras de estos conjuntos de datos. También se proporcionan ejemplos de registros generados, ofreciendo una visión tangible de los resultados de la generación de datos.

Por favor, refiérase a estos anexos para una comprensión más completa y detallada de la investigación y los métodos utilizados en este trabajo.
\section{Código de entrenamiento de económicos}

\begin{listing}[H]
\inputminted[
    framesep=5pt, rulecolor=gray,
    fontsize=\tiny,
    linenos=true, 
    breaklines=true,xleftmargin=1.0cm
    ]{python}{../../notebooks/economicos_train.py}
\caption{Código de ejemplo en Python para sumar dos números. Fuente: Autor.}
\label{anexo-economicos-cl}
\end{listing}

\section{Archivo Devcontainer}
\label{devcontainer-anexo}
\begin{listing}[H]
\inputminted[
    framesep=5pt, rulecolor=gray,
    fontsize=\small,
    linenos=true, 
    breaklines=true,xleftmargin=1.0cm
    ]{json}{../../.devcontainer/devcontainer.json}
\caption{Devcontainer del proyecto en curso.}
\label{devcontainer-file}
\end{listing}

\section{Ejemplos de código con fines de reproducibilidad}
\label{anexo:reproducibilidad}

En el Código \ref{codigo-show-score}, se muestra cómo se calcula y se muestra el puntaje promedio para una selección específica de modelos. El código utiliza la función "sort\_values" para ordenar los resultados en orden descendente según el puntaje. Luego, se filtran los resultados para incluir solo los modelos seleccionados y las columnas que muestran el puntaje y la Distancia al registro más cercano (DCR) en los tres umbrales \emph{Synthetic vs Train} (ST), \emph{Synthetic vs Hold} (SH) y \emph{Train vs Hold} TH.
\begin{listing}[H]
    \begin{minted}[linenos=true,frame=lines,framesep=2mm,baselinestretch=1.2]{python}
avg = syn.scores[syn.scores["type"] == "avg"]
avg.sort_values("score", ascending=False).loc[ ["tddpm_mlp","smote-enc","gaussiancopula","tvae","gaussiancopula", "copulagan","ctgan"], ["score", "DCR ST 5th", "DCR SH 5th", "DCR TH 5th"]]
    \end{minted}
\caption{Mostrando Puntajes Promedios Calculados}
\label{codigo-show-score}
\end{listing}

En el ejemplo presentado en el Código \ref{code-economicos-synthetic}, se crea una instancia de la clase \emph{Synthetic} utilizando un pandas dataframe previamente pre procesado. Se especifican las columnas que se considerarán como categorías, las que se considerarán como texto y las que se excluirán del análisis. Además, se indica el directorio donde se almacenarán los archivos temporales, se seleccionan los modelos a utilizar, se establece el número de registros sintéticos deseados y se define una columna objetiva para realizar pruebas con aprendizaje automático y estratificar los conjuntos parciales de datos que se utilizarán. De esta manera, se configura de forma flexible el proceso de generación de datos sintéticos según las necesidades específicas del usuario.

\begin{listing}[H]
\inputminted[
    firstline=45, lastline=54
    ]{python}{../../notebooks/economicos_train.py}
\caption{Instanciando clase Synthetic}
\label{code-economicos-synthetic}
\end{listing}