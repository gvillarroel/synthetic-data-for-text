\chapter{Anexos}
Este capítulo de Anexos proporciona información adicional y detallada que respalda la investigación realizada en esta tesis. Aunque estos detalles son esenciales para el completo entendimiento de la investigación, se han incluido en los anexos para mantener la fluidez del cuerpo principal de la tesis.

En las siguientes secciones, se presentan diversos elementos suplementarios. El código de entrenamiento de modelos económicos se proporciona para dar visibilidad a los métodos de aprendizaje automático utilizados. Se incluyen gráficos detallados de correlaciones y estadísticas para los conjuntos de datos utilizados, aportando un análisis más profundo de las características y estructuras de estos conjuntos de datos. También se proporcionan ejemplos de registros generados, ofreciendo una visión tangible de los resultados de la generación de datos.

Por favor, refiérase a estos anexos para una comprensión más completa y detallada de la investigación y los métodos utilizados en este trabajo.
\section{Código de entrenamiento de económicos}

\begin{listing}[H]
\inputminted[
    framesep=5pt, rulecolor=gray,
    fontsize=\tiny,
    linenos=true, 
    breaklines=true,xleftmargin=1.0cm
    ]{python}{../../notebooks/economicos_train.py}
\caption{Código de ejemplo en Python para sumar dos números. Fuente: Autor.}
\label{anexo-economicos-cl}
\end{listing}

\section{Archivo Devcontainer}
\label{devcontainer-anexo}
\begin{listing}[H]
\inputminted[
    framesep=5pt, rulecolor=gray,
    fontsize=\small,
    linenos=true, 
    breaklines=true,xleftmargin=1.0cm
    ]{json}{../../.devcontainer/devcontainer.json}
\caption{Devcontainer del proyecto en curso.}
\label{devcontainer-file}
\end{listing}

\section{Lista completa de figura pairwise kingcounty}
\label{A-pairwise-kingcounty-top2-a-1}
\begin{figure}[H]
    \centering
    \includesvg[scale=.7,inkscapelatex=false]{datasets/kingcounty-a-2/pairwise/copulagan.svg}
    \caption{Correlación de conjunto Real y Modelo: copulagan}
    \label{pairwise-king county-a-2-copulagan}
\end{figure}
\begin{figure}[H]
    \centering
    \includesvg[scale=.5,inkscapelatex=false]{datasets/kingcounty-a-2/pairwise/tvae.svg}
    \caption{Correlación de conjunto Real y Modelo: tvae}
    \label{pairwise-king county-a-2-tvae}
\end{figure}
\begin{figure}[H]
    \centering
    \includesvg[scale=.6,inkscapelatex=false]{datasets/kingcounty-a-2/pairwise/gaussiancopula.svg}
    \caption{Correlación de conjunto original de entrenamiento y Gaussiancopula}
    \label{pairwise-king county-a-2-gaussiancopula}
\end{figure}
\begin{figure}[H]
    \centering
    \includesvg[scale=.6,inkscapelatex=false]{datasets/kingcounty-a-2/pairwise/ctgan.svg}
    \caption{Correlación de conjunto original de entrenamiento y Ctgan}
    \label{pairwise-king county-a-2-ctgan}
\end{figure}
\begin{figure}[H]
    \centering
    \includesvg[scale=.5,inkscapelatex=false]{datasets/kingcounty-a-2/pairwise/tablepreset.svg}
    \caption{Correlación de conjunto Real y Modelo: tablepreset}
    \label{pairwise-king county-a-2-tablepreset}
\end{figure}
\begin{figure}[H]
    \centering
    \includesvg[scale=.7,inkscapelatex=false]{datasets/kingcounty-a-2/pairwise/smote-enc.svg}
    \caption{Correlación de conjunto Real y Modelo: smote-enc}
    \label{pairwise-king county-a-2-smote-enc}
\end{figure}
\begin{figure}[H]
    \centering
    \includesvg[scale=.7,inkscapelatex=false]{datasets/kingcounty-a-2/pairwise/tddpm_mlp.svg}
    \caption{Correlación de conjunto Real y Modelo: tddpm\_mlp}
    \label{pairwise-king county-a-2-tddpm_mlp}
\end{figure}


\section{Smote y Tddpm en KingCounty Gráficas por Columnas}

\begin{figure}[H]
    \centering
    \includesvg[scale=.5,inkscapelatex=false]{datasets/economicos-b-1/top2/privacy.svg}
    \caption{Frecuencia del campo privacy en el modelo real y top2}
    \label{frecuency-Privacy-top2}
\end{figure}
\begin{figure}[H]
    \centering
    \includesvg[scale=.7,inkscapelatex=false]{datasets/economicos-b-1/smote-enc/m_size.svg}
    \caption{Frecuencia del campo M size en el modelo real y smote}
    \label{frecuency-M Size-smote-enc}
\end{figure}
\begin{figure}[H]
    \centering
    \includesvg[scale=.7,inkscapelatex=false]{datasets/economicos-a-2/smote-enc/publication_date.svg}
    \caption{Frecuencia del campo publication date en el modelo real y smote-enc}
    \label{frecuency-Publication Date-smote-enc}
\end{figure}
\begin{figure}[H]
    \centering
    \includesvg[scale=.7,inkscapelatex=false]{datasets/economicos-a-2/top2+1/property_type.svg}
    \caption{Frecuencia del campo Property type en el modelo real y top2+1}
    \label{frecuency-Property Type-top2+1}
\end{figure}
\begin{figure}[H]
    \centering
    \includesvg[scale=.7,inkscapelatex=false]{datasets/economicos-a-2/top2/transaction_type.svg}
    \caption{Frecuencia del campo transaction type en el modelo real y top2}
    \label{frecuency-Transaction Type-top2}
\end{figure}
\begin{figure}[H]
    \centering
    \includesvg[scale=.7,inkscapelatex=false]{datasets/economicos-b-1/smote-enc/bathrooms.svg}
    \caption{Frecuencia del campo bathrooms en el modelo real y smote-enc}
    \label{frecuency-Bathrooms-smote-enc}
\end{figure}
\begin{figure}[H]
    \centering
    \includesvg[scale=.7,inkscapelatex=false]{datasets/economicos-a-2/smote-enc/rooms.svg}
    \caption{Frecuencia del campo Rooms en el modelo real y smote}
    \label{frecuency-Rooms-smote-enc}
\end{figure}
\begin{figure}[H]
    \centering
    \includesvg[scale=.5,inkscapelatex=false]{datasets/economicos-a-2/top2+1/county.svg}
    \caption{Frecuencia del campo county en el modelo real y top2+1}
    \label{frecuency-County-top2+1}
\end{figure}
\begin{figure}[H]
    \centering
    \includesvg[scale=.7,inkscapelatex=false]{datasets/economicos-a-3/top2/_price.svg}
    \caption{Frecuencia del campo  price en el modelo real y top2}
    \label{frecuency- Price-top2}
\end{figure}
\begin{figure}[H]
    \centering
    \includesvg[scale=.5,inkscapelatex=false]{datasets/economicos-a-1/tddpm_mlp/m_built.svg}
    \caption{Frecuencia del campo m built en el modelo real y tddpm\_mlp}
    \label{frecuency-M Built-tddpm_mlp}
\end{figure}
\begin{figure}[H]
    \centering
    \includesvg[scale=.7,inkscapelatex=false]{datasets/economicos-a-1/top2+1/state.svg}
    \caption{Frecuencia del campo state en el modelo real y top2+1}
    \label{frecuency-State-top2+1}
\end{figure}


\section{Figuras de correlación Económicos - Conjunto A}
\label{pairwise-full-a}
\begin{figure}[H]
    \centering
    \includesvg[scale=.7,inkscapelatex=false]{datasets/kingcounty-a-2/pairwise/copulagan.svg}
    \caption{Correlación de conjunto Real y Modelo: copulagan}
    \label{pairwise-king county-a-2-copulagan}
\end{figure}
\begin{figure}[H]
    \centering
    \includesvg[scale=.5,inkscapelatex=false]{datasets/kingcounty-a-2/pairwise/tvae.svg}
    \caption{Correlación de conjunto Real y Modelo: tvae}
    \label{pairwise-king county-a-2-tvae}
\end{figure}
\begin{figure}[H]
    \centering
    \includesvg[scale=.6,inkscapelatex=false]{datasets/kingcounty-a-2/pairwise/gaussiancopula.svg}
    \caption{Correlación de conjunto original de entrenamiento y Gaussiancopula}
    \label{pairwise-king county-a-2-gaussiancopula}
\end{figure}
\begin{figure}[H]
    \centering
    \includesvg[scale=.6,inkscapelatex=false]{datasets/kingcounty-a-2/pairwise/ctgan.svg}
    \caption{Correlación de conjunto original de entrenamiento y Ctgan}
    \label{pairwise-king county-a-2-ctgan}
\end{figure}
\begin{figure}[H]
    \centering
    \includesvg[scale=.5,inkscapelatex=false]{datasets/kingcounty-a-2/pairwise/tablepreset.svg}
    \caption{Correlación de conjunto Real y Modelo: tablepreset}
    \label{pairwise-king county-a-2-tablepreset}
\end{figure}
\begin{figure}[H]
    \centering
    \includesvg[scale=.7,inkscapelatex=false]{datasets/kingcounty-a-2/pairwise/smote-enc.svg}
    \caption{Correlación de conjunto Real y Modelo: smote-enc}
    \label{pairwise-king county-a-2-smote-enc}
\end{figure}
\begin{figure}[H]
    \centering
    \includesvg[scale=.7,inkscapelatex=false]{datasets/kingcounty-a-2/pairwise/tddpm_mlp.svg}
    \caption{Correlación de conjunto Real y Modelo: tddpm\_mlp}
    \label{pairwise-king county-a-2-tddpm_mlp}
\end{figure}



\section{Figuras de correlación Económicos - Conjunto B}
\label{pairwise-full-a}
\begin{figure}[H]
    \centering
    \includesvg[scale=.7,inkscapelatex=false]{datasets/kingcounty-a-2/pairwise/copulagan.svg}
    \caption{Correlación de conjunto Real y Modelo: copulagan}
    \label{pairwise-king county-a-2-copulagan}
\end{figure}
\begin{figure}[H]
    \centering
    \includesvg[scale=.5,inkscapelatex=false]{datasets/kingcounty-a-2/pairwise/tvae.svg}
    \caption{Correlación de conjunto Real y Modelo: tvae}
    \label{pairwise-king county-a-2-tvae}
\end{figure}
\begin{figure}[H]
    \centering
    \includesvg[scale=.6,inkscapelatex=false]{datasets/kingcounty-a-2/pairwise/gaussiancopula.svg}
    \caption{Correlación de conjunto original de entrenamiento y Gaussiancopula}
    \label{pairwise-king county-a-2-gaussiancopula}
\end{figure}
\begin{figure}[H]
    \centering
    \includesvg[scale=.6,inkscapelatex=false]{datasets/kingcounty-a-2/pairwise/ctgan.svg}
    \caption{Correlación de conjunto original de entrenamiento y Ctgan}
    \label{pairwise-king county-a-2-ctgan}
\end{figure}
\begin{figure}[H]
    \centering
    \includesvg[scale=.5,inkscapelatex=false]{datasets/kingcounty-a-2/pairwise/tablepreset.svg}
    \caption{Correlación de conjunto Real y Modelo: tablepreset}
    \label{pairwise-king county-a-2-tablepreset}
\end{figure}
\begin{figure}[H]
    \centering
    \includesvg[scale=.7,inkscapelatex=false]{datasets/kingcounty-a-2/pairwise/smote-enc.svg}
    \caption{Correlación de conjunto Real y Modelo: smote-enc}
    \label{pairwise-king county-a-2-smote-enc}
\end{figure}
\begin{figure}[H]
    \centering
    \includesvg[scale=.7,inkscapelatex=false]{datasets/kingcounty-a-2/pairwise/tddpm_mlp.svg}
    \caption{Correlación de conjunto Real y Modelo: tddpm\_mlp}
    \label{pairwise-king county-a-2-tddpm_mlp}
\end{figure}


\section{Ejemplos de 10 Registros Generados Aleatoriamente en Descripciones Económicas A-1}
\label{ejemplo-10-aleatoreos-a}
\begin{table}[H]
\centering
\fontsize{8}{14}\selectfont
\caption{Ejemplos de textos aleatoreos del modelo Tddpm, conjunto Economicos (A-1)}
\label{table-sample10-economicos-a-1-tddpm_mlp-text}
\begin{tabular}{|m{50em}|}
\hline
\rowcolor[gray]{0.8}
description \\
\hline Departamento de 2 pisos, 1 baño con walk in closet, living comedor en su primer piso, sala de estar, quincho, lavandería, gimnasio, estacionamiento para dos vehículos, 3 estacionamientos, terraza privada, logia techada, bodega por todo el sector del barrio Las Condes, supermercados, bancos, centros comerciales, hospitales etc. \\
\hline Piso de madera en el primer piso, living comedor con salida a terraza que tienen 2 baños completos (incluye una sala de estar) Living comedor amplio Condominio Amplio Hall de entrada Calefacción por radiadores Ventanas termopanel Termopanel Cocina amoblada Amueblada: 1 bodega Lavandería 3 estacionamientos Terrazas Techada para lavadora Edificio cuenta con estacionamiento privado \\
\hline Departamento de 2 pisos, 1 baño con salida a terraza en living comedor que comparten un gran jardín para una amplia vista panorámica por el sector del barrio Las Casas (Cuenta con lavandería), sala de estar, quincho, gimnasio, estacionamiento privado, 3 estacionamientos, logia, bodega, cocina completamente equipada, horno termopanel, campana techada, amplio hall de entrada, acceso controlado al exterior, ventanas flotantes, calefacción centralizada, agua potable, refrigerador/calefacción central, cortinas fluorescentes, cubierta electrónica, lavavajillas, jardines comunes, supermercados, centros comerciales, comercio localizado, cercano a metro San Miguel, Supermercados, Centros Comerciales, Colegios, Hospitales, Plazas Comerciales, Mall La Paz, Metro Santa Rosa, Avenida Santo Cristorio, Cocina amoblada, Sala de reuniones, salas de cine, salón multiuso, restaurante, cafetería, etc. \\
\hline Piso de madera en el primer piso, living comedor con salida a terraza que tienen 2 baños completos (incluye una sala de estar) Living comedor amplio para lavadora separados por los dos dormitorios del hall de entrada al jardín como estacionamiento techado A pasos de metro Metro San Francisco La Casa cuenta con: 1 bodega + 3 estacionamientos \\
\hline Piso de madera en el primer piso, living comedor con salida a terraza que tienen 2 baños completos (incluye una sala de estar) Living comedor amplio para dos autos (2 estacionamientos) 1 bodega \\
\hline Piso de madera en el living comedor con salida a la terraza, 2 baños completos que comparten una gran sala de estar para dos autos (incluye un bodega). \\
\hline Excelente conectividad, cercana a la metro de Vitacura, supermercados, bancos, hospitales, centros comerciales, comercio en general con estacionamiento para los visitantes que buscan una casa completamente equipada por el sector del barrio rural (principal) +569 9 7 6 5 8 \\
\hline Departamento de 2 pisos, 1 baño con salida a terraza en living comedor que comparte un gran jardín para dos vehículos (incluye una bodega). Cuenta con 3 estacionamientos del metro Las Condes; 4 estacionamiento completo: 5 autos + logia \\
\hline Excelente conectividad, cercana a metro de Metros Metropolitanos, supermercados, centros comerciales, hospitales, bancos, comercios, etc. \\
\hline Departamento de 2 pisos, 1 baño con salida a terraza en living comedor que comparte un amplio hall de entrada para una gran vista panorámica por el sector del metro Los Angeles (centros comerciales), cercano a supermercados, centros comerciales, hospitales etc. \\
\hline
\end{tabular}
\end{table}


\section{Ejemplos de 10 Registros Generados Aleatoriamente en Descripciones Económicas}
\label{ejemplo-10-aleatoreos-b}
\begin{table}[H]
\centering
\fontsize{8}{14}\selectfont
\caption{Ejemplos de textos aleatoreos del modelo Tddpm, conjunto Economicos (B-1)}
\label{table-sample10-economicos-b-1-tddpm_mlp-text}
\begin{tabular}{|m{50em}|}
\hline
\rowcolor[gray]{0.8}
description \\
\hline Cuenta con 3 dormitorios, 2 baños, living comedor, cocina amoblada, logia, estacionamiento techado para dos vehículos en el primer piso de la casa se encuentra al lado del centro de Lampue (Lima). \\
\hline Departamento de 85 m2, con vista despejada hacia la cordillera en el primer piso (con ventanas termopanel) 2 dormitorios 1 baño Cocina amoblada Logia Terraza Living Comedor Condominio cuenta con gimnasio Sala multiuso Lavandería Conserjería 24 horas Gastos comunes \$50.000 aprox \\
\hline  \\
\hline Departamento de dos dormitorios, 2 baños, living comedor, cocina amoblada con encimera, horno empotrado, campana, terraza techada, estacionamiento subterráneo \\
\hline Cuenta con: 2 dormitorios, 1 baño, living comedor, cocina amoblada, logia, estacionamiento para dos vehículos \\
\hline Excelente casa de dos pisos, con 2 dormitorios, 1 baño, living comedor, cocina amoblada, logia, estacionamiento techado para 3 vehículos en el primer piso (con ventanas termopanel) \\
\hline Vendo parcela de 5000 m2 en Frutillar, con vista al mar por la cordillera del Lago Los Lobos (Frutillar). Condominio cuenta con accesos controlados las 24 hectareas que permiten disfrutar de un lugar tranquilo para disfrute de una hermosa paisaje natural! \\
\hline Departamento en San Joaquín, 1 dormitorio, 1 baño, living comedor con salida a terraza (con vista despejada) Cocina amoblada equipada con campana, horno empotrado, cubiertas de granito para lavadora Calefacción por radiadores Edificio cuenta con gimnasio, sala multiuso, quincho, estacionamiento visitas \\
\hline Departamento amoblado, 2 dormitorios, 1 baño, living comedor con salida a terraza en el primer piso (cocina equipada), logia, estacionamiento de visitas, sala multiuso, gimnasio, lavandería \\
\hline Departamento de 1 dormitorio en suite con walk in closet, living comedor separados (con salida a terraza), cocina equipada con horno eléctrico, campana empotrada, cubierta de granito para lavadora; logia cerrada completamente amoblada por radiadores). El edificio cuenta con gimnasio, sala multiuso, quincho, salón de eventos, estacionamiento subterráneo \\
\hline
\end{tabular}
\end{table}


\section{Estadísticos KingCounty}
\label{propiedades-estadisticas-kingCounty}
\begin{table}[H]
\centering
\fontsize{8}{14}\selectfont
\caption{Propiedades  estadisticas de variable yr\_built, King county (A-2)}
\label{table-stats-king county-a-2-yr_built}
\begin{tabular}{|l|m{10em}|m{10em}|m{10em}|m{10em}|}
\hline
 \rowcolor[gray]{0.8}
Variable/Modelo & Real & tddpm\_mlp & smote-enc & ctgan \\
\hline top5 & [2014 2005 2006 2004 2007] & [2006. 2005. 2004. 1977. 2003.] & [2014. 2006. 2005. 2007. 2003.] & [1900 1976 1977 1975 1974] \\
\hline top5\_freq & [449 371 366 350 347] & [463 451 442 439 415] & [247 113 102  97  93] & [639 355 343 330 329] \\
\hline top5\_prob & [0.02596877 0.02145749 0.02116831 0.02024291 0.0200694 ] & [0.02142229 0.02086707 0.02045065 0.02031185 0.01920141] & [0.01142778 0.00522809 0.00471916 0.00448783 0.00430277] & [0.02956554 0.0164253  0.01587008 0.01526859 0.01522232] \\
\hline nobs & 17290 & \bfseries 21613 & \cellcolor[rgb]{0.9, 0.54, 0.52} 21614 & \bfseries 21613 \\
\hline missing & 0.000 & 0.000 & 0.000 & 0.000 \\
\hline mean & 1971 & 1972 & \bfseries 1971 & \cellcolor[rgb]{0.9, 0.54, 0.52} 1961 \\
\hline std\_err & 0.224 & \cellcolor[rgb]{0.9, 0.54, 0.52} 0.190 & 0.198 & \bfseries 0.204 \\
\hline upper\_ci & 1972 & 1972 & \bfseries 1972 & \cellcolor[rgb]{0.9, 0.54, 0.52} 1961 \\
\hline lower\_ci & 1971 & 1972 & \bfseries 1971 & \cellcolor[rgb]{0.9, 0.54, 0.52} 1960 \\
\hline std & 29.436 & \cellcolor[rgb]{0.9, 0.54, 0.52} 27.911 & \bfseries 29.169 & 29.969 \\
\hline iqr & 46.000 & \cellcolor[rgb]{0.9, 0.54, 0.52} 42.578 & \bfseries 45.832 & 45.000 \\
\hline iqr\_normal & 34.100 & \cellcolor[rgb]{0.9, 0.54, 0.52} 31.563 & \bfseries 33.975 & 33.359 \\
\hline mad & 24.632 & \cellcolor[rgb]{0.9, 0.54, 0.52} 23.153 & \bfseries 24.459 & 24.966 \\
\hline mad\_normal & 30.872 & \cellcolor[rgb]{0.9, 0.54, 0.52} 29.018 & \bfseries 30.655 & 31.290 \\
\hline coef\_var & 0.015 & \cellcolor[rgb]{0.9, 0.54, 0.52} 0.014 & \bfseries 0.015 & 0.015 \\
\hline range & 115.000 & 115.000 & 115.000 & 115.000 \\
\hline max & 2015 & \bfseries 2015 & \cellcolor[rgb]{0.9, 0.54, 0.52} 2015 & \bfseries 2015 \\
\hline min & 1900 & \bfseries 1900 & \cellcolor[rgb]{0.9, 0.54, 0.52} 1900 & \bfseries 1900 \\
\hline skew & -0.472 & \bfseries -0.475 & -0.464 & \cellcolor[rgb]{0.9, 0.54, 0.52} -0.305 \\
\hline kurtosis & 2.337 & 2.438 & \bfseries 2.309 & \cellcolor[rgb]{0.9, 0.54, 0.52} 2.155 \\
\hline jarque\_bera & 957.631 & 1095.917 & \cellcolor[rgb]{0.9, 0.54, 0.52} 1205.886 & \bfseries 978.229 \\
\hline jarque\_bera\_pval & 0.000 & \cellcolor[rgb]{0.9, 0.54, 0.52} 0.000 & \cellcolor[rgb]{0.9, 0.54, 0.52} 0.000 & \bfseries 0.000 \\
\hline mode & 2014 & 2006 & \bfseries 2014 & \cellcolor[rgb]{0.9, 0.54, 0.52} 1900 \\
\hline mode\_freq & 0.026 & 0.021 & \cellcolor[rgb]{0.9, 0.54, 0.52} 0.011 & \bfseries 0.030 \\
\hline median & 1975 & \bfseries 1975 & 1974 & \cellcolor[rgb]{0.9, 0.54, 0.52} 1963 \\
\hline 0.1\% & 1900 & \bfseries 1900 & \cellcolor[rgb]{0.9, 0.54, 0.52} 1901 & \bfseries 1900 \\
\hline 1.0\% & 1904 & 1906 & \bfseries 1905 & \cellcolor[rgb]{0.9, 0.54, 0.52} 1900 \\
\hline 5.0\% & 1915 & 1919 & \bfseries 1916 & \cellcolor[rgb]{0.9, 0.54, 0.52} 1908 \\
\hline 25.0\% & 1951 & 1954 & \bfseries 1952 & \cellcolor[rgb]{0.9, 0.54, 0.52} 1939 \\
\hline 75.0\% & 1997 & \bfseries 1997 & 1998 & \cellcolor[rgb]{0.9, 0.54, 0.52} 1984 \\
\hline 95.0\% & 2011 & 2009 & \bfseries 2010 & \cellcolor[rgb]{0.9, 0.54, 0.52} 2005 \\
\hline 99.0\% & 2014 & \bfseries 2014 & \bfseries 2014 & \cellcolor[rgb]{0.9, 0.54, 0.52} 2013 \\
\hline 99.9\% & 2015 & \bfseries 2015 & \cellcolor[rgb]{0.9, 0.54, 0.52} 2014 & \bfseries 2015 \\
\hline
\end{tabular}
\end{table}

\begin{table}[H]
\centering
\fontsize{8}{14}\selectfont
\caption{Propiedades  estadisticas de variable sqft\_above, King county (A-2)}
\label{table-stats-king county-a-2-sqft_above}
\begin{tabular}{|l|m{10em}|m{10em}|m{10em}|m{10em}|}
\hline
 \rowcolor[gray]{0.8}
Variable/Modelo & Real & tddpm\_mlp & smote-enc & ctgan \\
\hline top5 & [1300 1010 1200 1220 1140] & [1300. 1010. 1220. 1140. 1340.] & [1290. 1160. 1830. 1010. 1320.] & [1563 1241 1203 1170 1197] \\
\hline top5\_freq & [166 165 160 152 148] & [203 185 180 171 162] & [15 15 14 13 12] & [20 20 20 19 19] \\
\hline top5\_prob & [0.00960093 0.00954309 0.0092539  0.00879121 0.00855986] & [0.0093925  0.00855966 0.00832832 0.0079119  0.00749549] & [0.00069399 0.00069399 0.00064773 0.00060146 0.0005552 ] & [0.00092537 0.00092537 0.00092537 0.0008791  0.0008791 ] \\
\hline nobs & 17290 & \bfseries 21613 & \cellcolor[rgb]{0.9, 0.54, 0.52} 21614 & \bfseries 21613 \\
\hline missing & 0.000 & 0.000 & 0.000 & 0.000 \\
\hline mean & 1786 & \bfseries 1778 & 1769 & \cellcolor[rgb]{0.9, 0.54, 0.52} 2018 \\
\hline std\_err & 6.249 & \bfseries 5.620 & 5.283 & \cellcolor[rgb]{0.9, 0.54, 0.52} 7.466 \\
\hline upper\_ci & 1798 & \bfseries 1789 & 1780 & \cellcolor[rgb]{0.9, 0.54, 0.52} 2033 \\
\hline lower\_ci & 1774 & \bfseries 1767 & 1759 & \cellcolor[rgb]{0.9, 0.54, 0.52} 2004 \\
\hline std & 821.626 & \bfseries 826.164 & 776.654 & \cellcolor[rgb]{0.9, 0.54, 0.52} 1097.579 \\
\hline iqr & 1000.000 & \bfseries 975.555 & 972.791 & \cellcolor[rgb]{0.9, 0.54, 0.52} 1434.000 \\
\hline iqr\_normal & 741.301 & \bfseries 723.180 & 721.131 & \cellcolor[rgb]{0.9, 0.54, 0.52} 1063.026 \\
\hline mad & 635.012 & \bfseries 615.572 & 608.256 & \cellcolor[rgb]{0.9, 0.54, 0.52} 872.298 \\
\hline mad\_normal & 795.870 & \bfseries 771.505 & 762.336 & \cellcolor[rgb]{0.9, 0.54, 0.52} 1093.263 \\
\hline coef\_var & 0.460 & \bfseries 0.465 & 0.439 & \cellcolor[rgb]{0.9, 0.54, 0.52} 0.544 \\
\hline range & 8570 & 9120 & \bfseries 8368 & \cellcolor[rgb]{0.9, 0.54, 0.52} 7269 \\
\hline max & 8860 & 9410 & \bfseries 8670 & \cellcolor[rgb]{0.9, 0.54, 0.52} 7559 \\
\hline min & 290.000 & \bfseries 290.000 & \cellcolor[rgb]{0.9, 0.54, 0.52} 301.856 & \bfseries 290.000 \\
\hline skew & 1.428 & \cellcolor[rgb]{0.9, 0.54, 0.52} 2.236 & \bfseries 1.320 & 1.097 \\
\hline kurtosis & 6.260 & \cellcolor[rgb]{0.9, 0.54, 0.52} 15.164 & \bfseries 5.306 & 3.945 \\
\hline jarque\_bera & 13530 & \cellcolor[rgb]{0.9, 0.54, 0.52} 151241 & \bfseries 11064 & 5139 \\
\hline jarque\_bera\_pval & 0.000 & 0.000 & 0.000 & 0.000 \\
\hline mode & 1300 & \bfseries 1300 & \cellcolor[rgb]{0.9, 0.54, 0.52} 1160 & 1203 \\
\hline mode\_freq & 0.010 & \bfseries 0.009 & \cellcolor[rgb]{0.9, 0.54, 0.52} 0.001 & 0.001 \\
\hline median & 1560 & \bfseries 1560 & 1540 & \cellcolor[rgb]{0.9, 0.54, 0.52} 1721 \\
\hline 0.1\% & 520.000 & \bfseries 544.640 & 603.805 & \cellcolor[rgb]{0.9, 0.54, 0.52} 360.224 \\
\hline 1.0\% & 700.000 & \bfseries 720.516 & 740.191 & \cellcolor[rgb]{0.9, 0.54, 0.52} 512.120 \\
\hline 5.0\% & 850.000 & \bfseries 878.073 & 888.091 & \cellcolor[rgb]{0.9, 0.54, 0.52} 731.000 \\
\hline 25.0\% & 1200 & 1207 & \cellcolor[rgb]{0.9, 0.54, 0.52} 1207 & \bfseries 1198 \\
\hline 75.0\% & 2200 & \bfseries 2182 & 2180 & \cellcolor[rgb]{0.9, 0.54, 0.52} 2632 \\
\hline 95.0\% & 3380 & 3269 & \bfseries 3295 & \cellcolor[rgb]{0.9, 0.54, 0.52} 4226 \\
\hline 99.0\% & 4371 & 4194 & \bfseries 4199 & \cellcolor[rgb]{0.9, 0.54, 0.52} 5324 \\
\hline 99.9\% & 6070 & \cellcolor[rgb]{0.9, 0.54, 0.52} 9410 & 5258 & \bfseries 6386 \\
\hline
\end{tabular}
\end{table}

\begin{table}[H]
\centering
\fontsize{8}{14}\selectfont
\caption{Propiedades  estadisticas de variable yr\_renovated, King county (A-2)}
\label{table-stats-king county-a-2-yr_renovated}
\begin{tabular}{|l|m{10em}|m{10em}|m{10em}|m{10em}|}
\hline
 \rowcolor[gray]{0.8}
Variable/Modelo & Real & tddpm\_mlp & smote-enc & ctgan \\
\hline top5 & [   0 2014 2005 2000 2003] & [   0.         2014.         2015.          557.16404191 2005.93527463] & [   0. 2014. 2005. 2006. 2013.] & [   0 2015    2    3    4] \\
\hline top5\_freq & [16571    76    32    30    29] & [20892    62    53     1     1] & [20753    23     4     3     2] & [7471 1690 1337 1320 1309] \\
\hline top5\_prob & [0.95841527 0.0043956  0.00185078 0.00173511 0.00167727] & [9.66640448e-01 2.86864387e-03 2.45222783e-03 4.62684495e-05
 4.62684495e-05] & [9.60164708e-01 1.06412510e-03 1.85065235e-04 1.38798927e-04
 9.25326177e-05] & [0.34567159 0.07819368 0.06186092 0.06107435 0.0605654 ] \\
\hline nobs & 17290 & \bfseries 21613 & \cellcolor[rgb]{0.9, 0.54, 0.52} 21614 & \bfseries 21613 \\
\hline missing & 0.000 & 0.000 & 0.000 & 0.000 \\
\hline mean & 83.003 & 66.202 & \bfseries 74.187 & \cellcolor[rgb]{0.9, 0.54, 0.52} 280.061 \\
\hline std\_err & 3.031 & 2.431 & \bfseries 2.533 & \cellcolor[rgb]{0.9, 0.54, 0.52} 4.664 \\
\hline upper\_ci & 88.943 & 70.967 & \bfseries 79.151 & \cellcolor[rgb]{0.9, 0.54, 0.52} 289.203 \\
\hline lower\_ci & 77.063 & 61.438 & \bfseries 69.222 & \cellcolor[rgb]{0.9, 0.54, 0.52} 270.920 \\
\hline std & 398.503 & 357.413 & \bfseries 372.384 & \cellcolor[rgb]{0.9, 0.54, 0.52} 685.680 \\
\hline iqr & 0.000 & \bfseries 0.000 & \bfseries 0.000 & \cellcolor[rgb]{0.9, 0.54, 0.52} 7.000 \\
\hline iqr\_normal & 0.000 & \bfseries 0.000 & \bfseries 0.000 & \cellcolor[rgb]{0.9, 0.54, 0.52} 5.189 \\
\hline mad & 159.103 & 127.988 & \bfseries 142.469 & \cellcolor[rgb]{0.9, 0.54, 0.52} 476.288 \\
\hline mad\_normal & 199.407 & 160.409 & \bfseries 178.559 & \cellcolor[rgb]{0.9, 0.54, 0.52} 596.939 \\
\hline coef\_var & 4.801 & 5.399 & \bfseries 5.020 & \cellcolor[rgb]{0.9, 0.54, 0.52} 2.448 \\
\hline range & 2015 & \bfseries 2015 & \cellcolor[rgb]{0.9, 0.54, 0.52} 2015 & \bfseries 2015 \\
\hline max & 2015 & \bfseries 2015 & \cellcolor[rgb]{0.9, 0.54, 0.52} 2015 & \bfseries 2015 \\
\hline min & 0.000 & 0.000 & 0.000 & 0.000 \\
\hline skew & 4.593 & 5.224 & \bfseries 4.886 & \cellcolor[rgb]{0.9, 0.54, 0.52} 2.074 \\
\hline kurtosis & 22.096 & 28.309 & \bfseries 25.024 & \cellcolor[rgb]{0.9, 0.54, 0.52} 5.308 \\
\hline jarque\_bera & 323506 & \cellcolor[rgb]{0.9, 0.54, 0.52} 675119 & \bfseries 522838 & 20287 \\
\hline jarque\_bera\_pval & 0.000 & 0.000 & 0.000 & 0.000 \\
\hline mode & 0.000 & 0.000 & 0.000 & 0.000 \\
\hline mode\_freq & 0.958 & 0.967 & \bfseries 0.960 & \cellcolor[rgb]{0.9, 0.54, 0.52} 0.346 \\
\hline median & 0.000 & \bfseries 0.000 & \bfseries 0.000 & \cellcolor[rgb]{0.9, 0.54, 0.52} 3.000 \\
\hline 0.1\% & 0.000 & 0.000 & 0.000 & 0.000 \\
\hline 1.0\% & 0.000 & 0.000 & 0.000 & 0.000 \\
\hline 5.0\% & 0.000 & 0.000 & 0.000 & 0.000 \\
\hline 25.0\% & 0.000 & 0.000 & 0.000 & 0.000 \\
\hline 75.0\% & 0.000 & \bfseries 0.000 & \bfseries 0.000 & \cellcolor[rgb]{0.9, 0.54, 0.52} 7.000 \\
\hline 95.0\% & 0.000 & \bfseries 0.000 & \bfseries 0.000 & \cellcolor[rgb]{0.9, 0.54, 0.52} 2015.000 \\
\hline 99.0\% & 2008 & \bfseries 2010 & 2005 & \cellcolor[rgb]{0.9, 0.54, 0.52} 2015 \\
\hline 99.9\% & 2014 & \cellcolor[rgb]{0.9, 0.54, 0.52} 2015 & \bfseries 2014 & \cellcolor[rgb]{0.9, 0.54, 0.52} 2015 \\
\hline
\end{tabular}
\end{table}

\begin{table}[H]
\centering
\fontsize{8}{14}\selectfont
\caption{Propiedades  estadisticas de variable sqft\_lot, King county (A-2)}
\label{table-stats-king county-a-2-sqft_lot}
\begin{tabular}{|l|m{10em}|m{10em}|m{10em}|m{10em}|}
\hline
 \rowcolor[gray]{0.8}
Variable/Modelo & Real & tddpm\_mlp & smote-enc & ctgan \\
\hline top5 & [5000 4000 6000 7200 4800] & [5000. 4000. 6000. 7200. 4500.] & [5000. 4000. 4080. 6000. 8000.] & [  520 12941 13597 12637 12255] \\
\hline top5\_freq & [301 209 208 179  98] & [356 248 230 195 109] & [56 40 22 17 16] & [321   9   9   7   7] \\
\hline top5\_prob & [0.01740891 0.01208791 0.01203008 0.01035281 0.00566802] & [0.01647157 0.01147458 0.01064174 0.00902235 0.00504326] & [0.00259091 0.00185065 0.00101786 0.00078653 0.00074026] & [0.01485217 0.00041642 0.00041642 0.00032388 0.00032388] \\
\hline nobs & 17290 & \bfseries 21613 & \cellcolor[rgb]{0.9, 0.54, 0.52} 21614 & \bfseries 21613 \\
\hline missing & 0.000 & 0.000 & 0.000 & 0.000 \\
\hline mean & 14799 & 16628 & \bfseries 14059 & \cellcolor[rgb]{0.9, 0.54, 0.52} 18126 \\
\hline std\_err & 295.375 & \cellcolor[rgb]{0.9, 0.54, 0.52} 611.901 & \bfseries 216.074 & 206.230 \\
\hline upper\_ci & 15378 & 17827 & \bfseries 14482 & \cellcolor[rgb]{0.9, 0.54, 0.52} 18530 \\
\hline lower\_ci & 14220 & 15428 & \bfseries 13635 & \cellcolor[rgb]{0.9, 0.54, 0.52} 17722 \\
\hline std & 38839 & \cellcolor[rgb]{0.9, 0.54, 0.52} 89958 & \bfseries 31766 & 30319 \\
\hline iqr & 5606 & 4905 & \bfseries 5584 & \cellcolor[rgb]{0.9, 0.54, 0.52} 7113 \\
\hline iqr\_normal & 4155 & 3636 & \bfseries 4139 & \cellcolor[rgb]{0.9, 0.54, 0.52} 5273 \\
\hline mad & 13382 & \cellcolor[rgb]{0.9, 0.54, 0.52} 17162 & 11949 & \bfseries 13481 \\
\hline mad\_normal & 16772 & \cellcolor[rgb]{0.9, 0.54, 0.52} 21510 & 14976 & \bfseries 16896 \\
\hline coef\_var & 2.624 & \cellcolor[rgb]{0.9, 0.54, 0.52} 5.410 & \bfseries 2.260 & 1.673 \\
\hline range & 1164274 & 1650839 & \bfseries 962591 & \cellcolor[rgb]{0.9, 0.54, 0.52} 333705 \\
\hline max & 1164794 & 1651359 & \bfseries 963282 & \cellcolor[rgb]{0.9, 0.54, 0.52} 334225 \\
\hline min & 520.000 & \bfseries 520.000 & \cellcolor[rgb]{0.9, 0.54, 0.52} 690.676 & \bfseries 520.000 \\
\hline skew & 11.588 & 16.678 & \bfseries 8.762 & \cellcolor[rgb]{0.9, 0.54, 0.52} 6.046 \\
\hline kurtosis & 215.591 & \bfseries 297.135 & 123.595 & \cellcolor[rgb]{0.9, 0.54, 0.52} 46.341 \\
\hline jarque\_bera & 32946220 & \cellcolor[rgb]{0.9, 0.54, 0.52} 78912817 & \bfseries 13373770 & 1823319 \\
\hline jarque\_bera\_pval & 0.000 & 0.000 & 0.000 & 0.000 \\
\hline mode & 5000 & \bfseries 5000 & \bfseries 5000 & \cellcolor[rgb]{0.9, 0.54, 0.52} 520 \\
\hline mode\_freq & 0.017 & \bfseries 0.016 & \cellcolor[rgb]{0.9, 0.54, 0.52} 0.003 & 0.015 \\
\hline median & 7600 & \bfseries 7491 & 7769 & \cellcolor[rgb]{0.9, 0.54, 0.52} 11577 \\
\hline 0.1\% & 737.156 & \bfseries 812.333 & 833.904 & \cellcolor[rgb]{0.9, 0.54, 0.52} 520.000 \\
\hline 1.0\% & 1005 & \bfseries 1019 & 1067 & \cellcolor[rgb]{0.9, 0.54, 0.52} 520 \\
\hline 5.0\% & 1756 & 1589 & \bfseries 1773 & \cellcolor[rgb]{0.9, 0.54, 0.52} 2927 \\
\hline 25.0\% & 5001 & \bfseries 5000 & 5136 & \cellcolor[rgb]{0.9, 0.54, 0.52} 7973 \\
\hline 75.0\% & 10607 & 9905 & \bfseries 10720 & \cellcolor[rgb]{0.9, 0.54, 0.52} 15086 \\
\hline 95.0\% & 42999 & 36194 & \bfseries 40503 & \cellcolor[rgb]{0.9, 0.54, 0.52} 56973 \\
\hline 99.0\% & 212192 & \cellcolor[rgb]{0.9, 0.54, 0.52} 181619 & 191877 & \bfseries 196506 \\
\hline 99.9\% & 435600 & \cellcolor[rgb]{0.9, 0.54, 0.52} 1651359 & \bfseries 377395 & 300056 \\
\hline
\end{tabular}
\end{table}

\begin{table}[H]
\centering
\fontsize{8}{14}\selectfont
\caption{Propiedades  estadisticas de variable view, King county (A-2)}
\label{table-stats-king county-a-2-view}
\begin{tabular}{|l|m{10em}|m{10em}|m{10em}|m{10em}|}
\hline
 \rowcolor[gray]{0.8}
Variable/Modelo & Real & tddpm\_mlp & smote-enc & ctgan \\
\hline top5 & [0 2 3 1 4] & [0 2 3 1 4] & [0 2 3 4 1] & [0 2 4 3 1] \\
\hline top5\_freq & [15586   783   396   275   250] & [20619   514   224   147   109] & [20891   336   176   140    71] & [18103  1617   748   604   541] \\
\hline top5\_prob & [0.90144592 0.04528629 0.02290341 0.01590515 0.01445922] & [0.95400916 0.02378198 0.01036413 0.00680146 0.00504326] & [0.96654946 0.01554548 0.00814287 0.00647728 0.00328491] & [0.83759774 0.07481608 0.0346088  0.02794614 0.02503123] \\
\hline nobs & 17290 & \bfseries 21613 & \cellcolor[rgb]{0.9, 0.54, 0.52} 21614 & \bfseries 21613 \\
\hline missing & 0.000 & 0.000 & 0.000 & 0.000 \\
\hline mean & 0.233 & \bfseries 0.106 & 0.085 & \cellcolor[rgb]{0.9, 0.54, 0.52} 0.397 \\
\hline std\_err & 0.006 & 0.003 & \cellcolor[rgb]{0.9, 0.54, 0.52} 0.003 & \bfseries 0.007 \\
\hline upper\_ci & 0.244 & \bfseries 0.112 & 0.091 & \cellcolor[rgb]{0.9, 0.54, 0.52} 0.410 \\
\hline lower\_ci & 0.222 & \bfseries 0.099 & 0.078 & \cellcolor[rgb]{0.9, 0.54, 0.52} 0.384 \\
\hline std & 0.762 & 0.515 & \cellcolor[rgb]{0.9, 0.54, 0.52} 0.485 & \bfseries 0.986 \\
\hline iqr & 0.000 & 0.000 & 0.000 & 0.000 \\
\hline iqr\_normal & 0.000 & 0.000 & 0.000 & 0.000 \\
\hline mad & 0.420 & \bfseries 0.202 & \cellcolor[rgb]{0.9, 0.54, 0.52} 0.164 & 0.665 \\
\hline mad\_normal & 0.527 & \bfseries 0.253 & \cellcolor[rgb]{0.9, 0.54, 0.52} 0.205 & 0.833 \\
\hline coef\_var & 3.269 & 4.871 & \cellcolor[rgb]{0.9, 0.54, 0.52} 5.725 & \bfseries 2.484 \\
\hline range & 4.000 & 4.000 & 4.000 & 4.000 \\
\hline max & 4.000 & 4.000 & 4.000 & 4.000 \\
\hline min & 0.000 & 0.000 & 0.000 & 0.000 \\
\hline skew & 3.402 & 5.246 & \cellcolor[rgb]{0.9, 0.54, 0.52} 6.151 & \bfseries 2.476 \\
\hline kurtosis & 13.971 & 31.362 & \cellcolor[rgb]{0.9, 0.54, 0.52} 41.967 & \bfseries 8.080 \\
\hline jarque\_bera & 120072 & 823540 & \cellcolor[rgb]{0.9, 0.54, 0.52} 1503760 & \bfseries 45333 \\
\hline jarque\_bera\_pval & 0.000 & 0.000 & 0.000 & 0.000 \\
\hline mode & 0.000 & 0.000 & 0.000 & 0.000 \\
\hline mode\_freq & 0.901 & \bfseries 0.954 & \cellcolor[rgb]{0.9, 0.54, 0.52} 0.967 & 0.838 \\
\hline median & 0.000 & 0.000 & 0.000 & 0.000 \\
\hline 0.1\% & 0.000 & 0.000 & 0.000 & 0.000 \\
\hline 1.0\% & 0.000 & 0.000 & 0.000 & 0.000 \\
\hline 5.0\% & 0.000 & 0.000 & 0.000 & 0.000 \\
\hline 25.0\% & 0.000 & 0.000 & 0.000 & 0.000 \\
\hline 75.0\% & 0.000 & 0.000 & 0.000 & 0.000 \\
\hline 95.0\% & 2.000 & \cellcolor[rgb]{0.9, 0.54, 0.52} 0.000 & \cellcolor[rgb]{0.9, 0.54, 0.52} 0.000 & \bfseries 3.000 \\
\hline 99.0\% & 4.000 & \cellcolor[rgb]{0.9, 0.54, 0.52} 3.000 & \cellcolor[rgb]{0.9, 0.54, 0.52} 3.000 & \bfseries 4.000 \\
\hline 99.9\% & 4.000 & 4.000 & 4.000 & 4.000 \\
\hline
\end{tabular}
\end{table}

\begin{table}[H]
\centering
\fontsize{8}{14}\selectfont
\caption{Propiedades  estadisticas de variable floors, King county (A-2)}
\label{table-stats-king county-a-2-floors}
\begin{tabular}{|l|m{10em}|m{10em}|m{10em}|m{10em}|}
\hline
 \rowcolor[gray]{0.8}
Variable/Modelo & Real & tddpm\_mlp & smote-enc & ctgan \\
\hline top5 & [1.  2.  1.5 3.  2.5] & [1.  2.  1.5 3.  2.5] & [1.  2.  1.5 3.  2.5] & [1.  2.  1.5 3.  2.5] \\
\hline top5\_freq & [8488 6628 1523  517  128] & [11201  8302  1488   584    36] & [11267  8351  1329   623    44] & [12847  4573  2944   693   398] \\
\hline top5\_prob & [0.49091961 0.38334297 0.0880856  0.02990168 0.00740312] & [0.5182529  0.38412067 0.06884745 0.02702077 0.00166566] & [0.5212825  0.38636995 0.06148792 0.02882391 0.00203572] & [0.59441077 0.21158562 0.13621432 0.03206404 0.01841484] \\
\hline nobs & 17290 & \bfseries 21613 & \cellcolor[rgb]{0.9, 0.54, 0.52} 21614 & \bfseries 21613 \\
\hline missing & 0.000 & 0.000 & 0.000 & 0.000 \\
\hline mean & 1.499 & 1.475 & \bfseries 1.478 & \cellcolor[rgb]{0.9, 0.54, 0.52} 1.390 \\
\hline std\_err & 0.004 & \cellcolor[rgb]{0.9, 0.54, 0.52} 0.004 & 0.004 & \bfseries 0.004 \\
\hline upper\_ci & 1.507 & 1.482 & \bfseries 1.485 & \cellcolor[rgb]{0.9, 0.54, 0.52} 1.397 \\
\hline lower\_ci & 1.491 & 1.468 & \bfseries 1.471 & \cellcolor[rgb]{0.9, 0.54, 0.52} 1.382 \\
\hline std & 0.543 & 0.536 & \bfseries 0.542 & \cellcolor[rgb]{0.9, 0.54, 0.52} 0.556 \\
\hline iqr & 1.000 & 1.000 & 1.000 & 1.000 \\
\hline iqr\_normal & 0.741 & 0.741 & 0.741 & 0.741 \\
\hline mad & 0.490 & \bfseries 0.493 & 0.498 & \cellcolor[rgb]{0.9, 0.54, 0.52} 0.463 \\
\hline mad\_normal & 0.614 & \bfseries 0.617 & 0.624 & \cellcolor[rgb]{0.9, 0.54, 0.52} 0.581 \\
\hline coef\_var & 0.362 & \bfseries 0.364 & 0.366 & \cellcolor[rgb]{0.9, 0.54, 0.52} 0.400 \\
\hline range & 2.500 & \bfseries 2.500 & \cellcolor[rgb]{0.9, 0.54, 0.52} 2.000 & \bfseries 2.500 \\
\hline max & 3.500 & \bfseries 3.500 & \cellcolor[rgb]{0.9, 0.54, 0.52} 3.000 & \bfseries 3.500 \\
\hline min & 1.000 & 1.000 & 1.000 & 1.000 \\
\hline skew & 0.615 & \bfseries 0.636 & 0.642 & \cellcolor[rgb]{0.9, 0.54, 0.52} 1.401 \\
\hline kurtosis & 2.526 & \bfseries 2.474 & 2.474 & \cellcolor[rgb]{0.9, 0.54, 0.52} 4.517 \\
\hline jarque\_bera & 1252 & \bfseries 1705 & 1734 & \cellcolor[rgb]{0.9, 0.54, 0.52} 9146 \\
\hline jarque\_bera\_pval & 0.000 & 0.000 & 0.000 & 0.000 \\
\hline mode & 1.000 & 1.000 & 1.000 & 1.000 \\
\hline mode\_freq & 0.491 & \bfseries 0.518 & 0.521 & \cellcolor[rgb]{0.9, 0.54, 0.52} 0.594 \\
\hline median & 1.500 & 1.000 & 1.000 & 1.000 \\
\hline 0.1\% & 1.000 & 1.000 & 1.000 & 1.000 \\
\hline 1.0\% & 1.000 & 1.000 & 1.000 & 1.000 \\
\hline 5.0\% & 1.000 & 1.000 & 1.000 & 1.000 \\
\hline 25.0\% & 1.000 & 1.000 & 1.000 & 1.000 \\
\hline 75.0\% & 2.000 & 2.000 & 2.000 & 2.000 \\
\hline 95.0\% & 2.000 & \bfseries 2.000 & \bfseries 2.000 & \cellcolor[rgb]{0.9, 0.54, 0.52} 2.500 \\
\hline 99.0\% & 3.000 & 3.000 & 3.000 & 3.000 \\
\hline 99.9\% & 3.000 & \bfseries 3.000 & \bfseries 3.000 & \cellcolor[rgb]{0.9, 0.54, 0.52} 3.500 \\
\hline
\end{tabular}
\end{table}

\begin{table}[H]
\centering
\fontsize{8}{14}\selectfont
\caption{Propiedades  estadisticas de variable bedrooms, King county (A-2)}
\label{table-stats-king county-a-2-bedrooms}
\begin{tabular}{|l|m{10em}|m{10em}|m{10em}|m{10em}|}
\hline
 \rowcolor[gray]{0.8}
Variable/Modelo & Real & tddpm\_mlp & smote-enc & ctgan \\
\hline top5 & [3 4 2 5 6] & [3 4 2 5 6] & [3 4 2 5 1] & [4 3 2 5 6] \\
\hline top5\_freq & [7865 5477 2237 1292  212] & [10728  6929  2646  1132    86] & [11430  7061  2430   621    47] & [8624 4895 4079 1578  679] \\
\hline top5\_prob & [0.45488722 0.3167727  0.12938115 0.07472527 0.01226142] & [0.49636793 0.32059409 0.12242632 0.05237588 0.00397909] & [0.52882391 0.32668641 0.11242713 0.02873138 0.00217452] & [0.39901911 0.22648406 0.18872901 0.07301161 0.03141628] \\
\hline nobs & 17290 & \bfseries 21613 & \cellcolor[rgb]{0.9, 0.54, 0.52} 21614 & \bfseries 21613 \\
\hline missing & 0.000 & 0.000 & 0.000 & 0.000 \\
\hline mean & 3.368 & \bfseries 3.307 & 3.271 & \cellcolor[rgb]{0.9, 0.54, 0.52} 3.975 \\
\hline std\_err & 0.007 & \bfseries 0.005 & 0.005 & \cellcolor[rgb]{0.9, 0.54, 0.52} 0.025 \\
\hline upper\_ci & 3.382 & \bfseries 3.318 & 3.280 & \cellcolor[rgb]{0.9, 0.54, 0.52} 4.025 \\
\hline lower\_ci & 3.354 & \bfseries 3.297 & 3.261 & \cellcolor[rgb]{0.9, 0.54, 0.52} 3.925 \\
\hline std & 0.931 & \bfseries 0.785 & 0.707 & \cellcolor[rgb]{0.9, 0.54, 0.52} 3.732 \\
\hline iqr & 1.000 & 1.000 & 1.000 & 1.000 \\
\hline iqr\_normal & 0.741 & 0.741 & 0.741 & 0.741 \\
\hline mad & 0.734 & \bfseries 0.645 & 0.582 & \cellcolor[rgb]{0.9, 0.54, 0.52} 1.395 \\
\hline mad\_normal & 0.920 & \bfseries 0.808 & 0.730 & \cellcolor[rgb]{0.9, 0.54, 0.52} 1.748 \\
\hline coef\_var & 0.277 & \bfseries 0.237 & 0.216 & \cellcolor[rgb]{0.9, 0.54, 0.52} 0.939 \\
\hline range & 33.000 & 9.000 & \cellcolor[rgb]{0.9, 0.54, 0.52} 5.000 & \bfseries 33.000 \\
\hline max & 33.000 & 9.000 & \cellcolor[rgb]{0.9, 0.54, 0.52} 6.000 & \bfseries 33.000 \\
\hline min & 0.000 & \bfseries 0.000 & \cellcolor[rgb]{0.9, 0.54, 0.52} 1.000 & \bfseries 0.000 \\
\hline skew & 2.304 & \bfseries 0.243 & 0.091 & \cellcolor[rgb]{0.9, 0.54, 0.52} 6.612 \\
\hline kurtosis & 63.268 & 3.485 & \cellcolor[rgb]{0.9, 0.54, 0.52} 3.072 & \bfseries 51.539 \\
\hline jarque\_bera & 2631992 & 424 & \cellcolor[rgb]{0.9, 0.54, 0.52} 34 & \bfseries 2279187 \\
\hline jarque\_bera\_pval & 0.000 & 0.000 & \cellcolor[rgb]{0.9, 0.54, 0.52} 0.000 & \bfseries 0.000 \\
\hline mode & 3.000 & \bfseries 3.000 & \bfseries 3.000 & \cellcolor[rgb]{0.9, 0.54, 0.52} 4.000 \\
\hline mode\_freq & 0.455 & \bfseries 0.496 & \cellcolor[rgb]{0.9, 0.54, 0.52} 0.529 & 0.399 \\
\hline median & 3.000 & \bfseries 3.000 & \bfseries 3.000 & \cellcolor[rgb]{0.9, 0.54, 0.52} 4.000 \\
\hline 0.1\% & 1.000 & \bfseries 1.000 & \bfseries 1.000 & \cellcolor[rgb]{0.9, 0.54, 0.52} 0.000 \\
\hline 1.0\% & 2.000 & \bfseries 2.000 & \bfseries 2.000 & \cellcolor[rgb]{0.9, 0.54, 0.52} 0.000 \\
\hline 5.0\% & 2.000 & 2.000 & 2.000 & 2.000 \\
\hline 25.0\% & 3.000 & 3.000 & 3.000 & 3.000 \\
\hline 75.0\% & 4.000 & 4.000 & 4.000 & 4.000 \\
\hline 95.0\% & 5.000 & \bfseries 5.000 & 4.000 & \cellcolor[rgb]{0.9, 0.54, 0.52} 7.000 \\
\hline 99.0\% & 6.000 & \bfseries 5.000 & \bfseries 5.000 & \cellcolor[rgb]{0.9, 0.54, 0.52} 33.000 \\
\hline 99.9\% & 7.000 & \bfseries 6.000 & \bfseries 6.000 & \cellcolor[rgb]{0.9, 0.54, 0.52} 33.000 \\
\hline
\end{tabular}
\end{table}

\begin{table}[H]
\centering
\fontsize{8}{14}\selectfont
\caption{Propiedades  estadisticas de variable zipcode, King county (A-2)}
\label{table-stats-king county-a-2-zipcode}
\begin{tabular}{|l|m{10em}|m{10em}|m{10em}|m{10em}|}
\hline
 \rowcolor[gray]{0.8}
Variable/Modelo & Real & tddpm\_mlp & smote-enc & ctgan \\
\hline top5 & [98103 98038 98115 98052 98117] & [98052 98103 98038 98115 98042] & [98103 98115 98052 98038 98117] & [98034 98118 98006 98023 98103] \\
\hline top5\_freq & [489 473 462 459 455] & [714 658 645 634 601] & [650 623 574 570 557] & [804 711 638 630 612] \\
\hline top5\_prob & [0.02828224 0.02735685 0.02672065 0.02654714 0.02631579] & [0.03303567 0.03044464 0.02984315 0.0293342  0.02780734] & [0.0300731  0.02882391 0.02655686 0.0263718  0.02577033] & [0.03719983 0.03289687 0.02951927 0.02914912 0.02831629] \\
\hline nobs & 17290 & \bfseries 21613 & \cellcolor[rgb]{0.9, 0.54, 0.52} 21614 & \bfseries 21613 \\
\hline missing & 0.000 & 0.000 & 0.000 & 0.000 \\
\hline mean & 98078 & 98077 & \bfseries 98078 & \cellcolor[rgb]{0.9, 0.54, 0.52} 98079 \\
\hline std\_err & 0.406 & \cellcolor[rgb]{0.9, 0.54, 0.52} 0.360 & 0.363 & \bfseries 0.370 \\
\hline upper\_ci & 98079 & 98078 & \bfseries 98079 & \cellcolor[rgb]{0.9, 0.54, 0.52} 98080 \\
\hline lower\_ci & 98077 & 98077 & \bfseries 98077 & \cellcolor[rgb]{0.9, 0.54, 0.52} 98078 \\
\hline std & 53.326 & 52.955 & \bfseries 53.304 & \cellcolor[rgb]{0.9, 0.54, 0.52} 54.353 \\
\hline iqr & 84.000 & \bfseries 84.000 & \bfseries 84.000 & \cellcolor[rgb]{0.9, 0.54, 0.52} 85.000 \\
\hline iqr\_normal & 62.269 & \bfseries 62.269 & \bfseries 62.269 & \cellcolor[rgb]{0.9, 0.54, 0.52} 63.011 \\
\hline mad & 46.554 & 46.116 & \bfseries 46.585 & \cellcolor[rgb]{0.9, 0.54, 0.52} 47.661 \\
\hline mad\_normal & 58.347 & 57.798 & \bfseries 58.385 & \cellcolor[rgb]{0.9, 0.54, 0.52} 59.734 \\
\hline coef\_var & 0.001 & 0.001 & \bfseries 0.001 & \cellcolor[rgb]{0.9, 0.54, 0.52} 0.001 \\
\hline range & 198.000 & 198.000 & 198.000 & 198.000 \\
\hline max & 98199 & 98199 & 98199 & 98199 \\
\hline min & 98001 & 98001 & 98001 & 98001 \\
\hline skew & 0.402 & \cellcolor[rgb]{0.9, 0.54, 0.52} 0.423 & \bfseries 0.390 & 0.417 \\
\hline kurtosis & 2.153 & 2.182 & \bfseries 2.142 & \cellcolor[rgb]{0.9, 0.54, 0.52} 2.101 \\
\hline jarque\_bera & 983.027 & 1246.547 & \bfseries 1210.461 & \cellcolor[rgb]{0.9, 0.54, 0.52} 1354.816 \\
\hline jarque\_bera\_pval & 0.000 & 0.000 & 0.000 & 0.000 \\
\hline mode & 98103 & 98052 & \bfseries 98103 & \cellcolor[rgb]{0.9, 0.54, 0.52} 98034 \\
\hline mode\_freq & 0.028 & 0.033 & \bfseries 0.030 & \cellcolor[rgb]{0.9, 0.54, 0.52} 0.037 \\
\hline median & 98065 & \bfseries 98065 & \cellcolor[rgb]{0.9, 0.54, 0.52} 98070 & \cellcolor[rgb]{0.9, 0.54, 0.52} 98070 \\
\hline 0.1\% & 98001 & 98001 & 98001 & 98001 \\
\hline 1.0\% & 98001 & 98001 & 98001 & 98001 \\
\hline 5.0\% & 98004 & \bfseries 98004 & \bfseries 98004 & \cellcolor[rgb]{0.9, 0.54, 0.52} 98005 \\
\hline 25.0\% & 98033 & 98033 & 98033 & 98033 \\
\hline 75.0\% & 98117 & \bfseries 98117 & \bfseries 98117 & \cellcolor[rgb]{0.9, 0.54, 0.52} 98118 \\
\hline 95.0\% & 98177 & \bfseries 98177 & \bfseries 98177 & \cellcolor[rgb]{0.9, 0.54, 0.52} 98178 \\
\hline 99.0\% & 98199 & 98199 & 98199 & 98199 \\
\hline 99.9\% & 98199 & 98199 & 98199 & 98199 \\
\hline
\end{tabular}
\end{table}

\begin{table}[H]
\centering
\fontsize{8}{14}\selectfont
\caption{Propiedades  estadisticas de variable bathrooms, King county (A-2)}
\label{table-stats-king county-a-2-bathrooms}
\begin{tabular}{|l|m{10em}|m{10em}|m{10em}|m{10em}|}
\hline
 \rowcolor[gray]{0.8}
Variable/Modelo & Real & tddpm\_mlp & smote-enc & ctgan \\
\hline top5 & [2.5  1.   1.75 2.25 2.  ] & [2.5  1.   1.75 2.25 2.  ] & [2.5  1.   1.75 2.25 2.  ] & [2.5  1.75 1.   2.75 3.25] \\
\hline top5\_freq & [4333 3088 2425 1621 1526] & [6256 3998 3308 2019 1764] & [6751 4830 3303 1979 1320] & [4571 3423 2963 2032 1228] \\
\hline top5\_prob & [0.25060729 0.17860035 0.14025448 0.09375361 0.08825911] & [0.28945542 0.18498126 0.15305603 0.093416   0.08161754] & [0.31234385 0.22346627 0.15281762 0.09156103 0.06107153] & [0.21149308 0.1583769  0.13709342 0.09401749 0.05681766] \\
\hline nobs & 17290 & \bfseries 21613 & \cellcolor[rgb]{0.9, 0.54, 0.52} 21614 & \bfseries 21613 \\
\hline missing & 0.000 & 0.000 & 0.000 & 0.000 \\
\hline mean & 2.114 & \bfseries 2.080 & 2.011 & \cellcolor[rgb]{0.9, 0.54, 0.52} 2.291 \\
\hline std\_err & 0.006 & 0.005 & \cellcolor[rgb]{0.9, 0.54, 0.52} 0.005 & \bfseries 0.006 \\
\hline upper\_ci & 2.125 & \bfseries 2.089 & 2.020 & \cellcolor[rgb]{0.9, 0.54, 0.52} 2.304 \\
\hline lower\_ci & 2.102 & \bfseries 2.070 & 2.001 & \cellcolor[rgb]{0.9, 0.54, 0.52} 2.279 \\
\hline std & 0.767 & \bfseries 0.713 & 0.706 & \cellcolor[rgb]{0.9, 0.54, 0.52} 0.916 \\
\hline iqr & 1.000 & \cellcolor[rgb]{0.9, 0.54, 0.52} 0.750 & \bfseries 1.000 & \bfseries 1.000 \\
\hline iqr\_normal & 0.741 & \cellcolor[rgb]{0.9, 0.54, 0.52} 0.556 & \bfseries 0.741 & \bfseries 0.741 \\
\hline mad & 0.615 & 0.582 & \bfseries 0.592 & \cellcolor[rgb]{0.9, 0.54, 0.52} 0.728 \\
\hline mad\_normal & 0.771 & 0.730 & \bfseries 0.742 & \cellcolor[rgb]{0.9, 0.54, 0.52} 0.913 \\
\hline coef\_var & 0.363 & 0.343 & \bfseries 0.351 & \cellcolor[rgb]{0.9, 0.54, 0.52} 0.400 \\
\hline range & 8.000 & 7.750 & \cellcolor[rgb]{0.9, 0.54, 0.52} 4.750 & \bfseries 8.000 \\
\hline max & 8.000 & 7.750 & \cellcolor[rgb]{0.9, 0.54, 0.52} 5.500 & \bfseries 8.000 \\
\hline min & 0.000 & \bfseries 0.000 & \cellcolor[rgb]{0.9, 0.54, 0.52} 0.750 & \bfseries 0.000 \\
\hline skew & 0.464 & 0.234 & \cellcolor[rgb]{0.9, 0.54, 0.52} 0.146 & \bfseries 0.441 \\
\hline kurtosis & 3.989 & 3.439 & \cellcolor[rgb]{0.9, 0.54, 0.52} 2.843 & \bfseries 3.888 \\
\hline jarque\_bera & 1326 & 371 & \cellcolor[rgb]{0.9, 0.54, 0.52} 99 & \bfseries 1409 \\
\hline jarque\_bera\_pval & 0.000 & 0.000 & \cellcolor[rgb]{0.9, 0.54, 0.52} 0.000 & \bfseries 0.000 \\
\hline mode & 2.500 & 2.500 & 2.500 & 2.500 \\
\hline mode\_freq & 0.251 & \bfseries 0.289 & \cellcolor[rgb]{0.9, 0.54, 0.52} 0.312 & 0.211 \\
\hline median & 2.250 & \bfseries 2.250 & \bfseries 2.250 & \cellcolor[rgb]{0.9, 0.54, 0.52} 2.500 \\
\hline 0.1\% & 0.750 & \bfseries 1.000 & \bfseries 1.000 & \cellcolor[rgb]{0.9, 0.54, 0.52} 0.000 \\
\hline 1.0\% & 1.000 & \bfseries 1.000 & \bfseries 1.000 & \cellcolor[rgb]{0.9, 0.54, 0.52} 0.750 \\
\hline 5.0\% & 1.000 & 1.000 & 1.000 & 1.000 \\
\hline 25.0\% & 1.500 & \cellcolor[rgb]{0.9, 0.54, 0.52} 1.750 & \bfseries 1.500 & \cellcolor[rgb]{0.9, 0.54, 0.52} 1.750 \\
\hline 75.0\% & 2.500 & \bfseries 2.500 & \bfseries 2.500 & \cellcolor[rgb]{0.9, 0.54, 0.52} 2.750 \\
\hline 95.0\% & 3.500 & \bfseries 3.250 & \bfseries 3.250 & \cellcolor[rgb]{0.9, 0.54, 0.52} 4.000 \\
\hline 99.0\% & 4.250 & \bfseries 3.750 & \cellcolor[rgb]{0.9, 0.54, 0.52} 3.500 & \bfseries 4.750 \\
\hline 99.9\% & 5.428 & \bfseries 5.000 & 4.750 & \cellcolor[rgb]{0.9, 0.54, 0.52} 6.250 \\
\hline
\end{tabular}
\end{table}

\begin{table}[H]
\centering
\fontsize{8}{14}\selectfont
\caption{Propiedades  estadisticas de variable sqft\_living15, King county (A-2)}
\label{table-stats-king county-a-2-sqft_living15}
\begin{tabular}{|l|m{10em}|m{10em}|m{10em}|m{10em}|}
\hline
 \rowcolor[gray]{0.8}
Variable/Modelo & Real & tddpm\_mlp & smote-enc & ctgan \\
\hline top5 & [1440 1540 1560 1500 1610] & [1440. 1550. 1540. 1560. 1530.] & [1440. 1830. 1670. 1078. 2370.] & [1827 1696 1594 1790 1621] \\
\hline top5\_freq & [156 154 152 137 136] & [200 183 165 162 162] & [24 22 21 18 17] & [28 26 25 25 24] \\
\hline top5\_prob & [0.00902256 0.00890688 0.00879121 0.00792366 0.00786582] & [0.00925369 0.00846713 0.00763429 0.00749549 0.00749549] & [0.00111039 0.00101786 0.00097159 0.00083279 0.00078653] & [0.00129552 0.00120298 0.00115671 0.00115671 0.00111044] \\
\hline nobs & 17290 & \bfseries 21613 & \cellcolor[rgb]{0.9, 0.54, 0.52} 21614 & \bfseries 21613 \\
\hline missing & 0.000 & 0.000 & 0.000 & 0.000 \\
\hline mean & 1983 & 1971 & \bfseries 1974 & \cellcolor[rgb]{0.9, 0.54, 0.52} 1952 \\
\hline std\_err & 5.181 & \cellcolor[rgb]{0.9, 0.54, 0.52} 4.381 & 4.422 & \bfseries 5.044 \\
\hline upper\_ci & 1993 & 1980 & \bfseries 1982 & \cellcolor[rgb]{0.9, 0.54, 0.52} 1962 \\
\hline lower\_ci & 1973 & 1962 & \bfseries 1965 & \cellcolor[rgb]{0.9, 0.54, 0.52} 1942 \\
\hline std & 681.232 & 644.021 & \bfseries 650.115 & \cellcolor[rgb]{0.9, 0.54, 0.52} 741.571 \\
\hline iqr & 880.000 & \cellcolor[rgb]{0.9, 0.54, 0.52} 827.821 & 832.491 & \bfseries 853.000 \\
\hline iqr\_normal & 652.345 & \cellcolor[rgb]{0.9, 0.54, 0.52} 613.665 & 617.127 & \bfseries 632.330 \\
\hline mad & 533.237 & \cellcolor[rgb]{0.9, 0.54, 0.52} 504.400 & \bfseries 509.430 & 557.482 \\
\hline mad\_normal & 668.313 & \cellcolor[rgb]{0.9, 0.54, 0.52} 632.171 & \bfseries 638.476 & 698.700 \\
\hline coef\_var & 0.344 & 0.327 & \bfseries 0.329 & \cellcolor[rgb]{0.9, 0.54, 0.52} 0.380 \\
\hline range & 5811 & 5811 & \cellcolor[rgb]{0.9, 0.54, 0.52} 5320 & \bfseries 5811 \\
\hline max & 6210 & 6210 & \cellcolor[rgb]{0.9, 0.54, 0.52} 5993 & \bfseries 6210 \\
\hline min & 399.000 & \bfseries 399.000 & \cellcolor[rgb]{0.9, 0.54, 0.52} 672.953 & \bfseries 399.000 \\
\hline skew & 1.095 & \cellcolor[rgb]{0.9, 0.54, 0.52} 1.067 & \bfseries 1.083 & 1.083 \\
\hline kurtosis & 4.572 & \bfseries 4.604 & 4.443 & \cellcolor[rgb]{0.9, 0.54, 0.52} 5.613 \\
\hline jarque\_bera & 5237 & 6416 & \bfseries 6104 & \cellcolor[rgb]{0.9, 0.54, 0.52} 10375 \\
\hline jarque\_bera\_pval & 0.000 & 0.000 & 0.000 & 0.000 \\
\hline mode & 1440 & \bfseries 1440 & 1440 & \cellcolor[rgb]{0.9, 0.54, 0.52} 1827 \\
\hline mode\_freq & 0.009 & \bfseries 0.009 & \cellcolor[rgb]{0.9, 0.54, 0.52} 0.001 & 0.001 \\
\hline median & 1840 & 1830 & \bfseries 1835 & \cellcolor[rgb]{0.9, 0.54, 0.52} 1859 \\
\hline 0.1\% & 740.000 & \cellcolor[rgb]{0.9, 0.54, 0.52} 399.000 & \bfseries 791.697 & 468.060 \\
\hline 1.0\% & 950.000 & \bfseries 980.000 & 988.057 & \cellcolor[rgb]{0.9, 0.54, 0.52} 648.120 \\
\hline 5.0\% & 1140 & 1179 & \bfseries 1166 & \cellcolor[rgb]{0.9, 0.54, 0.52} 899 \\
\hline 25.0\% & 1480 & \cellcolor[rgb]{0.9, 0.54, 0.52} 1502 & 1495 & \bfseries 1469 \\
\hline 75.0\% & 2360 & \bfseries 2330 & 2328 & \cellcolor[rgb]{0.9, 0.54, 0.52} 2322 \\
\hline 95.0\% & 3280 & \cellcolor[rgb]{0.9, 0.54, 0.52} 3204 & 3222 & \bfseries 3307 \\
\hline 99.0\% & 4050 & 3868 & \bfseries 3939 & \cellcolor[rgb]{0.9, 0.54, 0.52} 4270 \\
\hline 99.9\% & 4986 & \bfseries 4889 & 4802 & \cellcolor[rgb]{0.9, 0.54, 0.52} 5989 \\
\hline
\end{tabular}
\end{table}

\begin{table}[H]
\centering
\fontsize{8}{14}\selectfont
\caption{Propiedades  estadisticas de variable price, King county (A-2)}
\label{table-stats-king county-a-2-price}
\begin{tabular}{|l|m{10em}|m{10em}|m{10em}|m{10em}|}
\hline
 \rowcolor[gray]{0.8}
Variable/Modelo & Real & tddpm\_mlp & smote-enc & ctgan \\
\hline top5 & [350000. 450000. 425000. 550000. 325000.] & [450000. 525000. 300000. 350000. 550000.] & [350000. 450000. 325000. 550000. 500000.] & [ 75000. 347056. 209754. 370839. 905913.] \\
\hline top5\_freq & [143 140 123 123 123] & [190 138 135 135 134] & [213 191 186 184 182] & [602   3   3   2   2] \\
\hline top5\_prob & [0.00827068 0.00809717 0.00711394 0.00711394 0.00711394] & [0.00879101 0.00638505 0.00624624 0.00624624 0.00619997] & [0.00985472 0.00883686 0.00860553 0.008513   0.00842047] & [2.78536066e-02 1.38805349e-04 1.38805349e-04 9.25368991e-05
 9.25368991e-05] \\
\hline nobs & 17290 & \bfseries 21613 & \cellcolor[rgb]{0.9, 0.54, 0.52} 21614 & \bfseries 21613 \\
\hline missing & 0.000 & 0.000 & 0.000 & 0.000 \\
\hline mean & 537768 & \bfseries 528328 & 519642 & \cellcolor[rgb]{0.9, 0.54, 0.52} 498555 \\
\hline std\_err & 2749 & 2828 & \cellcolor[rgb]{0.9, 0.54, 0.52} 2097 & \bfseries 2784 \\
\hline upper\_ci & 543156 & \bfseries 533872 & 523753 & \cellcolor[rgb]{0.9, 0.54, 0.52} 504011 \\
\hline lower\_ci & 532380 & \bfseries 522785 & 515532 & \cellcolor[rgb]{0.9, 0.54, 0.52} 493099 \\
\hline std & 361464 & \cellcolor[rgb]{0.9, 0.54, 0.52} 415801 & 308334 & \bfseries 409263 \\
\hline iqr & 319850 & 302163 & \bfseries 305500 & \cellcolor[rgb]{0.9, 0.54, 0.52} 359276 \\
\hline iqr\_normal & 237105 & 223994 & \bfseries 226467 & \cellcolor[rgb]{0.9, 0.54, 0.52} 266332 \\
\hline mad & 231680 & \bfseries 218397 & 210266 & \cellcolor[rgb]{0.9, 0.54, 0.52} 265501 \\
\hline mad\_normal & 290368 & \bfseries 273720 & 263529 & \cellcolor[rgb]{0.9, 0.54, 0.52} 332756 \\
\hline coef\_var & 0.672 & 0.787 & \bfseries 0.593 & \cellcolor[rgb]{0.9, 0.54, 0.52} 0.821 \\
\hline range & 7625000 & \bfseries 7625000 & \cellcolor[rgb]{0.9, 0.54, 0.52} 4130000 & 4404475 \\
\hline max & 7700000 & \bfseries 7700000 & \cellcolor[rgb]{0.9, 0.54, 0.52} 4208000 & 4479475 \\
\hline min & 75000 & \bfseries 75000 & \cellcolor[rgb]{0.9, 0.54, 0.52} 78000 & \bfseries 75000 \\
\hline skew & 4.032 & \cellcolor[rgb]{0.9, 0.54, 0.52} 9.124 & 2.762 & \bfseries 2.979 \\
\hline kurtosis & 39.678 & \cellcolor[rgb]{0.9, 0.54, 0.52} 141.899 & \bfseries 16.937 & 16.856 \\
\hline jarque\_bera & 1016020 & \cellcolor[rgb]{0.9, 0.54, 0.52} 17674040 & 202399 & \bfseries 204867 \\
\hline jarque\_bera\_pval & 0.000 & 0.000 & 0.000 & 0.000 \\
\hline mode & 350000 & 450000 & \bfseries 350000 & \cellcolor[rgb]{0.9, 0.54, 0.52} 75000 \\
\hline mode\_freq & 0.008 & \bfseries 0.009 & 0.010 & \cellcolor[rgb]{0.9, 0.54, 0.52} 0.028 \\
\hline median & 450000 & \bfseries 450000 & 450500 & \cellcolor[rgb]{0.9, 0.54, 0.52} 408171 \\
\hline 0.1\% & 95000 & 94193 & \bfseries 95000 & \cellcolor[rgb]{0.9, 0.54, 0.52} 75000 \\
\hline 1.0\% & 154467 & 161482 & \bfseries 158000 & \cellcolor[rgb]{0.9, 0.54, 0.52} 75000 \\
\hline 5.0\% & 210000 & 215000 & \bfseries 210000 & \cellcolor[rgb]{0.9, 0.54, 0.52} 100410 \\
\hline 25.0\% & 320150 & 322837 & \bfseries 319500 & \cellcolor[rgb]{0.9, 0.54, 0.52} 250152 \\
\hline 75.0\% & 640000 & \bfseries 625000 & \bfseries 625000 & \cellcolor[rgb]{0.9, 0.54, 0.52} 609428 \\
\hline 95.0\% & 1150000 & \cellcolor[rgb]{0.9, 0.54, 0.52} 1003061 & 1037350 & \bfseries 1209711 \\
\hline 99.0\% & 1950000 & \bfseries 1732519 & 1699856 & \cellcolor[rgb]{0.9, 0.54, 0.52} 2281066 \\
\hline 99.9\% & 3331995 & \cellcolor[rgb]{0.9, 0.54, 0.52} 7700000 & 2950000 & \bfseries 3517622 \\
\hline
\end{tabular}
\end{table}

\begin{table}[H]
\centering
\fontsize{8}{14}\selectfont
\caption{Propiedades  estadisticas de variable long, King county (A-2)}
\label{table-stats-king county-a-2-long}
\begin{tabular}{|l|m{10em}|m{10em}|m{10em}|m{10em}|}
\hline
 \rowcolor[gray]{0.8}
Variable/Modelo & Real & tddpm\_mlp & smote-enc & ctgan \\
\hline top5 & [-122.29  -122.362 -122.3   -122.372 -122.284] & [-122.3   -122.307 -122.301 -122.29  -122.299] & [-122.29  -122.34  -122.387 -122.375 -122.244] & [-122.519 -122.28  -122.332 -122.346 -122.323] \\
\hline top5\_freq & [100  88  81  81  81] & [137  84  84  83  81] & [25 21 21 20 18] & [108  98  86  83  81] \\
\hline top5\_prob & [0.00578369 0.00508965 0.00468479 0.00468479 0.00468479] & [0.00633878 0.00388655 0.00388655 0.00384028 0.00374774] & [0.00115666 0.00097159 0.00097159 0.00092533 0.00083279] & [0.00499699 0.00453431 0.00397909 0.00384028 0.00374774] \\
\hline nobs & 17290 & \bfseries 21613 & \cellcolor[rgb]{0.9, 0.54, 0.52} 21614 & \bfseries 21613 \\
\hline missing & 0.000 & 0.000 & 0.000 & 0.000 \\
\hline mean & -122.214 & \bfseries -122.213 & -122.216 & \cellcolor[rgb]{0.9, 0.54, 0.52} -122.211 \\
\hline std\_err & 0.001 & \cellcolor[rgb]{0.9, 0.54, 0.52} 0.001 & 0.001 & \bfseries 0.001 \\
\hline upper\_ci & -122.212 & \bfseries -122.211 & -122.214 & \cellcolor[rgb]{0.9, 0.54, 0.52} -122.209 \\
\hline lower\_ci & -122.216 & \bfseries -122.215 & -122.218 & \cellcolor[rgb]{0.9, 0.54, 0.52} -122.213 \\
\hline std & 0.140 & 0.136 & \bfseries 0.139 & \cellcolor[rgb]{0.9, 0.54, 0.52} 0.146 \\
\hline iqr & 0.204 & \cellcolor[rgb]{0.9, 0.54, 0.52} 0.196 & \bfseries 0.201 & 0.209 \\
\hline iqr\_normal & 0.151 & \cellcolor[rgb]{0.9, 0.54, 0.52} 0.145 & \bfseries 0.149 & 0.155 \\
\hline mad & 0.115 & \cellcolor[rgb]{0.9, 0.54, 0.52} 0.111 & \bfseries 0.115 & 0.119 \\
\hline mad\_normal & 0.144 & \cellcolor[rgb]{0.9, 0.54, 0.52} 0.140 & \bfseries 0.144 & 0.149 \\
\hline coef\_var & -0.001 & -0.001 & \bfseries -0.001 & \cellcolor[rgb]{0.9, 0.54, 0.52} -0.001 \\
\hline range & 1.204 & \bfseries 1.204 & 1.193 & \cellcolor[rgb]{0.9, 0.54, 0.52} 0.959 \\
\hline max & -121.315 & \bfseries -121.315 & -121.318 & \cellcolor[rgb]{0.9, 0.54, 0.52} -121.560 \\
\hline min & -122.519 & -122.519 & \cellcolor[rgb]{0.9, 0.54, 0.52} -122.511 & \bfseries -122.519 \\
\hline skew & 0.867 & 0.937 & \bfseries 0.844 & \cellcolor[rgb]{0.9, 0.54, 0.52} 0.754 \\
\hline kurtosis & 3.953 & \cellcolor[rgb]{0.9, 0.54, 0.52} 4.863 & \bfseries 3.780 & 3.753 \\
\hline jarque\_bera & 2819 & \cellcolor[rgb]{0.9, 0.54, 0.52} 6287 & 3116 & \bfseries 2558 \\
\hline jarque\_bera\_pval & 0.000 & 0.000 & 0.000 & 0.000 \\
\hline mode & -122.290 & -122.300 & \bfseries -122.290 & \cellcolor[rgb]{0.9, 0.54, 0.52} -122.519 \\
\hline mode\_freq & 0.006 & \bfseries 0.006 & \cellcolor[rgb]{0.9, 0.54, 0.52} 0.001 & 0.005 \\
\hline median & -122.231 & \cellcolor[rgb]{0.9, 0.54, 0.52} -122.221 & \bfseries -122.235 & -122.239 \\
\hline 0.1\% & -122.497 & \cellcolor[rgb]{0.9, 0.54, 0.52} -122.469 & \bfseries -122.483 & -122.519 \\
\hline 1.0\% & -122.408 & -122.399 & \bfseries -122.404 & \cellcolor[rgb]{0.9, 0.54, 0.52} -122.464 \\
\hline 5.0\% & -122.387 & -122.385 & \bfseries -122.386 & \cellcolor[rgb]{0.9, 0.54, 0.52} -122.406 \\
\hline 25.0\% & -122.329 & -122.322 & \bfseries -122.328 & \cellcolor[rgb]{0.9, 0.54, 0.52} -122.322 \\
\hline 75.0\% & -122.125 & \bfseries -122.126 & -122.128 & \cellcolor[rgb]{0.9, 0.54, 0.52} -122.113 \\
\hline 95.0\% & -121.979 & \cellcolor[rgb]{0.9, 0.54, 0.52} -121.996 & \bfseries -121.983 & -121.964 \\
\hline 99.0\% & -121.787 & \cellcolor[rgb]{0.9, 0.54, 0.52} -121.830 & -121.817 & \bfseries -121.758 \\
\hline 99.9\% & -121.699 & \cellcolor[rgb]{0.9, 0.54, 0.52} -121.469 & \bfseries -121.704 & -121.622 \\
\hline
\end{tabular}
\end{table}

\begin{table}[H]
\centering
\fontsize{8}{14}\selectfont
\caption{Propiedades  estadisticas de variable grade, King county (A-2)}
\label{table-stats-king county-a-2-grade}
\begin{tabular}{|l|m{10em}|m{10em}|m{10em}|m{10em}|}
\hline
 \rowcolor[gray]{0.8}
Variable/Modelo & Real & tddpm\_mlp & smote-enc & ctgan \\
\hline top5 & [ 7  8  9  6 10] & [ 7  8  9  6 10] & [ 7  8  9  6 10] & [ 8  7  9  6 10] \\
\hline top5\_freq & [7201 4879 2072 1620  915] & [9511 6272 2597 1701 1048] & [9815 6178 2584 1697  987] & [6536 3941 3726 3091 1408] \\
\hline top5\_prob & [0.41648352 0.28218623 0.11983806 0.09369578 0.05292076] & [0.44005922 0.29019572 0.12015916 0.07870263 0.04848934] & [0.45410382 0.28583326 0.11955214 0.07851393 0.04566485] & [0.30241059 0.18234396 0.17239624 0.14301578 0.06514598] \\
\hline nobs & 17290 & \bfseries 21613 & \cellcolor[rgb]{0.9, 0.54, 0.52} 21614 & \bfseries 21613 \\
\hline missing & 0.000 & 0.000 & 0.000 & 0.000 \\
\hline mean & 7.654 & \bfseries 7.650 & 7.621 & \cellcolor[rgb]{0.9, 0.54, 0.52} 7.929 \\
\hline std\_err & 0.009 & \bfseries 0.007 & 0.007 & \cellcolor[rgb]{0.9, 0.54, 0.52} 0.012 \\
\hline upper\_ci & 7.671 & \bfseries 7.664 & 7.635 & \cellcolor[rgb]{0.9, 0.54, 0.52} 7.952 \\
\hline lower\_ci & 7.636 & \bfseries 7.635 & 7.607 & \cellcolor[rgb]{0.9, 0.54, 0.52} 7.906 \\
\hline std & 1.170 & \bfseries 1.082 & 1.042 & \cellcolor[rgb]{0.9, 0.54, 0.52} 1.736 \\
\hline iqr & 1.000 & \bfseries 1.000 & \bfseries 1.000 & \cellcolor[rgb]{0.9, 0.54, 0.52} 2.000 \\
\hline iqr\_normal & 0.741 & \bfseries 0.741 & \bfseries 0.741 & \cellcolor[rgb]{0.9, 0.54, 0.52} 1.483 \\
\hline mad & 0.926 & \bfseries 0.867 & 0.844 & \cellcolor[rgb]{0.9, 0.54, 0.52} 1.275 \\
\hline mad\_normal & 1.160 & \bfseries 1.087 & 1.058 & \cellcolor[rgb]{0.9, 0.54, 0.52} 1.598 \\
\hline coef\_var & 0.153 & \bfseries 0.141 & 0.137 & \cellcolor[rgb]{0.9, 0.54, 0.52} 0.219 \\
\hline range & 12.000 & 9.000 & \cellcolor[rgb]{0.9, 0.54, 0.52} 8.000 & \bfseries 12.000 \\
\hline max & 13.000 & \bfseries 13.000 & \cellcolor[rgb]{0.9, 0.54, 0.52} 12.000 & \bfseries 13.000 \\
\hline min & 1.000 & \cellcolor[rgb]{0.9, 0.54, 0.52} 4.000 & \cellcolor[rgb]{0.9, 0.54, 0.52} 4.000 & \bfseries 1.000 \\
\hline skew & 0.758 & 0.820 & \bfseries 0.812 & \cellcolor[rgb]{0.9, 0.54, 0.52} 0.213 \\
\hline kurtosis & 4.209 & \bfseries 4.083 & 3.918 & \cellcolor[rgb]{0.9, 0.54, 0.52} 3.654 \\
\hline jarque\_bera & 2709 & 3479 & \bfseries 3133 & \cellcolor[rgb]{0.9, 0.54, 0.52} 549 \\
\hline jarque\_bera\_pval & 0.000 & \bfseries 0.000 & \bfseries 0.000 & \cellcolor[rgb]{0.9, 0.54, 0.52} 0.000 \\
\hline mode & 7.000 & \bfseries 7.000 & \bfseries 7.000 & \cellcolor[rgb]{0.9, 0.54, 0.52} 8.000 \\
\hline mode\_freq & 0.416 & \bfseries 0.440 & 0.454 & \cellcolor[rgb]{0.9, 0.54, 0.52} 0.302 \\
\hline median & 7.000 & \bfseries 7.000 & \bfseries 7.000 & \cellcolor[rgb]{0.9, 0.54, 0.52} 8.000 \\
\hline 0.1\% & 4.000 & 5.000 & 5.000 & 3.000 \\
\hline 1.0\% & 5.000 & 6.000 & 6.000 & 4.000 \\
\hline 5.0\% & 6.000 & \bfseries 6.000 & \bfseries 6.000 & \cellcolor[rgb]{0.9, 0.54, 0.52} 5.000 \\
\hline 25.0\% & 7.000 & 7.000 & 7.000 & 7.000 \\
\hline 75.0\% & 8.000 & \bfseries 8.000 & \bfseries 8.000 & \cellcolor[rgb]{0.9, 0.54, 0.52} 9.000 \\
\hline 95.0\% & 10.000 & \bfseries 10.000 & \bfseries 10.000 & \cellcolor[rgb]{0.9, 0.54, 0.52} 11.000 \\
\hline 99.0\% & 11.000 & \bfseries 11.000 & \bfseries 11.000 & \cellcolor[rgb]{0.9, 0.54, 0.52} 12.000 \\
\hline 99.9\% & 12.000 & \bfseries 12.000 & \bfseries 12.000 & \cellcolor[rgb]{0.9, 0.54, 0.52} 13.000 \\
\hline
\end{tabular}
\end{table}

\begin{table}[H]
\centering
\fontsize{8}{14}\selectfont
\caption{Propiedades  estadisticas de variable condition, King county (A-2)}
\label{table-stats-king county-a-2-condition}
\begin{tabular}{|l|m{10em}|m{10em}|m{10em}|m{10em}|}
\hline
 \rowcolor[gray]{0.8}
Variable/Modelo & Real & tddpm\_mlp & smote-enc & ctgan \\
\hline top5 & [3 4 5 2 1] & [3 4 5 2 1] & [3 4 5 2 1] & [3 4 5 2 1] \\
\hline top5\_freq & [11248  4512  1364   139    27] & [14872  5461  1228    49     3] & [15439  5203   945    26     1] & [12161  6512  2329   400   211] \\
\hline top5\_prob & [0.65054945 0.26096009 0.07888953 0.00803933 0.0015616 ] & [6.88104382e-01 2.52672003e-01 5.68176560e-02 2.26715403e-03
 1.38805349e-04] & [7.14305543e-01 2.40723605e-01 4.37216619e-02 1.20292403e-03
 4.62663089e-05] & [0.56267061 0.30130014 0.10775922 0.01850738 0.00976264] \\
\hline nobs & 17290 & \bfseries 21613 & \cellcolor[rgb]{0.9, 0.54, 0.52} 21614 & \bfseries 21613 \\
\hline missing & 0.000 & 0.000 & 0.000 & 0.000 \\
\hline mean & 3.408 & \bfseries 3.364 & \cellcolor[rgb]{0.9, 0.54, 0.52} 3.327 & 3.479 \\
\hline std\_err & 0.005 & 0.004 & \cellcolor[rgb]{0.9, 0.54, 0.52} 0.004 & \bfseries 0.005 \\
\hline upper\_ci & 3.417 & \bfseries 3.372 & \cellcolor[rgb]{0.9, 0.54, 0.52} 3.334 & 3.489 \\
\hline lower\_ci & 3.398 & \bfseries 3.356 & \cellcolor[rgb]{0.9, 0.54, 0.52} 3.319 & 3.469 \\
\hline std & 0.652 & \bfseries 0.592 & 0.557 & \cellcolor[rgb]{0.9, 0.54, 0.52} 0.749 \\
\hline iqr & 1.000 & 1.000 & 1.000 & 1.000 \\
\hline iqr\_normal & 0.741 & 0.741 & 0.741 & 0.741 \\
\hline mad & 0.560 & \bfseries 0.507 & \cellcolor[rgb]{0.9, 0.54, 0.52} 0.470 & 0.642 \\
\hline mad\_normal & 0.702 & \bfseries 0.636 & \cellcolor[rgb]{0.9, 0.54, 0.52} 0.590 & 0.805 \\
\hline coef\_var & 0.191 & \bfseries 0.176 & 0.167 & \cellcolor[rgb]{0.9, 0.54, 0.52} 0.215 \\
\hline range & 4.000 & 4.000 & 4.000 & 4.000 \\
\hline max & 5.000 & 5.000 & 5.000 & 5.000 \\
\hline min & 1.000 & 1.000 & 1.000 & 1.000 \\
\hline skew & 1.028 & \bfseries 1.317 & 1.447 & \cellcolor[rgb]{0.9, 0.54, 0.52} 0.361 \\
\hline kurtosis & 3.556 & 3.851 & \cellcolor[rgb]{0.9, 0.54, 0.52} 4.213 & \bfseries 3.455 \\
\hline jarque\_bera & 3269 & 6902 & \cellcolor[rgb]{0.9, 0.54, 0.52} 8863 & \bfseries 657 \\
\hline jarque\_bera\_pval & 0.000 & \bfseries 0.000 & \bfseries 0.000 & \cellcolor[rgb]{0.9, 0.54, 0.52} 0.000 \\
\hline mode & 3.000 & 3.000 & 3.000 & 3.000 \\
\hline mode\_freq & 0.651 & \bfseries 0.688 & 0.714 & \cellcolor[rgb]{0.9, 0.54, 0.52} 0.563 \\
\hline median & 3.000 & 3.000 & 3.000 & 3.000 \\
\hline 0.1\% & 1.000 & \cellcolor[rgb]{0.9, 0.54, 0.52} 2.000 & \cellcolor[rgb]{0.9, 0.54, 0.52} 2.000 & \bfseries 1.000 \\
\hline 1.0\% & 3.000 & \bfseries 3.000 & \bfseries 3.000 & \cellcolor[rgb]{0.9, 0.54, 0.52} 2.000 \\
\hline 5.0\% & 3.000 & 3.000 & 3.000 & 3.000 \\
\hline 25.0\% & 3.000 & 3.000 & 3.000 & 3.000 \\
\hline 75.0\% & 4.000 & 4.000 & 4.000 & 4.000 \\
\hline 95.0\% & 5.000 & \bfseries 5.000 & \cellcolor[rgb]{0.9, 0.54, 0.52} 4.000 & \bfseries 5.000 \\
\hline 99.0\% & 5.000 & 5.000 & 5.000 & 5.000 \\
\hline 99.9\% & 5.000 & 5.000 & 5.000 & 5.000 \\
\hline
\end{tabular}
\end{table}

\begin{table}[H]
\centering
\fontsize{8}{14}\selectfont
\caption{Propiedades  estadisticas de variable lat, King county (A-2)}
\label{table-stats-king county-a-2-lat}
\begin{tabular}{|l|m{10em}|m{10em}|m{10em}|m{10em}|}
\hline
 \rowcolor[gray]{0.8}
Variable/Modelo & Real & tddpm\_mlp & smote-enc & ctgan \\
\hline top5 & [47.5402 47.6914 47.6853 47.6624 47.6968] & [47.1559     47.7776     47.64939189 47.6493812  47.64937361] & [47.3363 47.6534 47.6647 47.6904 47.7438] & [47.1593 47.7776 47.6289 47.6879 47.643 ] \\
\hline top5\_freq & [14 13 13 13 13] & [16  5  1  1  1] & [7 6 6 5 4] & [430  56  18  16  16] \\
\hline top5\_prob & [0.00080972 0.00075188 0.00075188 0.00075188 0.00075188] & [7.40295193e-04 2.31342248e-04 4.62684495e-05 4.62684495e-05
 4.62684495e-05] & [0.00032386 0.0002776  0.0002776  0.00023133 0.00018507] & [0.01989543 0.00259103 0.00083283 0.0007403  0.0007403 ] \\
\hline nobs & 17290 & \bfseries 21613 & \cellcolor[rgb]{0.9, 0.54, 0.52} 21614 & \bfseries 21613 \\
\hline missing & 0.000 & 0.000 & 0.000 & 0.000 \\
\hline mean & 47.560 & 47.561 & \bfseries 47.561 & \cellcolor[rgb]{0.9, 0.54, 0.52} 47.527 \\
\hline std\_err & 0.001 & \cellcolor[rgb]{0.9, 0.54, 0.52} 0.001 & 0.001 & \bfseries 0.001 \\
\hline upper\_ci & 47.562 & 47.563 & \bfseries 47.563 & \cellcolor[rgb]{0.9, 0.54, 0.52} 47.529 \\
\hline lower\_ci & 47.558 & 47.559 & \bfseries 47.559 & \cellcolor[rgb]{0.9, 0.54, 0.52} 47.525 \\
\hline std & 0.138 & 0.137 & \bfseries 0.138 & \cellcolor[rgb]{0.9, 0.54, 0.52} 0.157 \\
\hline iqr & 0.206 & 0.208 & \bfseries 0.205 & \cellcolor[rgb]{0.9, 0.54, 0.52} 0.235 \\
\hline iqr\_normal & 0.153 & 0.154 & \bfseries 0.152 & \cellcolor[rgb]{0.9, 0.54, 0.52} 0.174 \\
\hline mad & 0.115 & 0.114 & \bfseries 0.114 & \cellcolor[rgb]{0.9, 0.54, 0.52} 0.131 \\
\hline mad\_normal & 0.144 & 0.143 & \bfseries 0.143 & \cellcolor[rgb]{0.9, 0.54, 0.52} 0.164 \\
\hline coef\_var & 0.003 & 0.003 & \bfseries 0.003 & \cellcolor[rgb]{0.9, 0.54, 0.52} 0.003 \\
\hline range & 0.618 & 0.622 & \cellcolor[rgb]{0.9, 0.54, 0.52} 0.606 & \bfseries 0.618 \\
\hline max & 47.778 & \bfseries 47.778 & \cellcolor[rgb]{0.9, 0.54, 0.52} 47.777 & \bfseries 47.778 \\
\hline min & 47.159 & 47.156 & \cellcolor[rgb]{0.9, 0.54, 0.52} 47.171 & \bfseries 47.159 \\
\hline skew & -0.487 & -0.459 & \bfseries -0.492 & \cellcolor[rgb]{0.9, 0.54, 0.52} -0.603 \\
\hline kurtosis & 2.328 & 2.267 & \bfseries 2.313 & \cellcolor[rgb]{0.9, 0.54, 0.52} 2.423 \\
\hline jarque\_bera & 1009 & \bfseries 1245 & 1298 & \cellcolor[rgb]{0.9, 0.54, 0.52} 1611 \\
\hline jarque\_bera\_pval & 0.000 & 0.000 & 0.000 & 0.000 \\
\hline mode & 47.540 & \cellcolor[rgb]{0.9, 0.54, 0.52} 47.156 & \bfseries 47.336 & 47.159 \\
\hline mode\_freq & 0.001 & \bfseries 0.001 & 0.000 & \cellcolor[rgb]{0.9, 0.54, 0.52} 0.020 \\
\hline median & 47.572 & 47.569 & \bfseries 47.572 & \cellcolor[rgb]{0.9, 0.54, 0.52} 47.547 \\
\hline 0.1\% & 47.193 & 47.171 & \bfseries 47.194 & \cellcolor[rgb]{0.9, 0.54, 0.52} 47.159 \\
\hline 1.0\% & 47.257 & 47.263 & \bfseries 47.258 & \cellcolor[rgb]{0.9, 0.54, 0.52} 47.159 \\
\hline 5.0\% & 47.311 & 47.313 & \bfseries 47.312 & \cellcolor[rgb]{0.9, 0.54, 0.52} 47.231 \\
\hline 25.0\% & 47.472 & \bfseries 47.472 & 47.473 & \cellcolor[rgb]{0.9, 0.54, 0.52} 47.424 \\
\hline 75.0\% & 47.678 & 47.680 & \bfseries 47.678 & \cellcolor[rgb]{0.9, 0.54, 0.52} 47.660 \\
\hline 95.0\% & 47.750 & 47.748 & \bfseries 47.750 & \cellcolor[rgb]{0.9, 0.54, 0.52} 47.724 \\
\hline 99.0\% & 47.773 & \bfseries 47.771 & 47.771 & \cellcolor[rgb]{0.9, 0.54, 0.52} 47.756 \\
\hline 99.9\% & 47.777 & \bfseries 47.777 & \cellcolor[rgb]{0.9, 0.54, 0.52} 47.776 & 47.778 \\
\hline
\end{tabular}
\end{table}

\begin{table}[H]
\centering
\fontsize{8}{14}\selectfont
\caption{Propiedades  estadisticas de variable sqft\_living, King county (A-2)}
\label{table-stats-king county-a-2-sqft_living}
\begin{tabular}{|l|m{10em}|m{10em}|m{10em}|m{10em}|}
\hline
 \rowcolor[gray]{0.8}
Variable/Modelo & Real & tddpm\_mlp & smote-enc & ctgan \\
\hline top5 & [1400 1300 1720 1250 1540] & [1440. 1300. 1800. 1320. 1820.] & [1250. 1280. 1160. 1690. 1800.] & [ 290 1925 2471 1406 2241] \\
\hline top5\_freq & [109 107 106 106 105] & [144 142 137 117 115] & [15 14 12 11 11] & [20 18 17 17 17] \\
\hline top5\_prob & [0.00630422 0.00618855 0.00613071 0.00613071 0.00607287] & [0.00666266 0.00657012 0.00633878 0.00541341 0.00532087] & [0.00069399 0.00064773 0.0005552  0.00050893 0.00050893] & [0.00092537 0.00083283 0.00078656 0.00078656 0.00078656] \\
\hline nobs & 17290 & \bfseries 21613 & \cellcolor[rgb]{0.9, 0.54, 0.52} 21614 & \bfseries 21613 \\
\hline missing & 0.000 & 0.000 & 0.000 & 0.000 \\
\hline mean & 2074 & 2032 & \bfseries 2049 & \cellcolor[rgb]{0.9, 0.54, 0.52} 2813 \\
\hline std\_err & 6.900 & \bfseries 6.968 & 5.741 & \cellcolor[rgb]{0.9, 0.54, 0.52} 10.591 \\
\hline upper\_ci & 2087 & 2046 & \bfseries 2060 & \cellcolor[rgb]{0.9, 0.54, 0.52} 2833 \\
\hline lower\_ci & 2060 & 2019 & \bfseries 2037 & \cellcolor[rgb]{0.9, 0.54, 0.52} 2792 \\
\hline std & 907.298 & 1024.446 & \bfseries 844.097 & \cellcolor[rgb]{0.9, 0.54, 0.52} 1556.997 \\
\hline iqr & 1110 & 1057 & \bfseries 1079 & \cellcolor[rgb]{0.9, 0.54, 0.52} 1810 \\
\hline iqr\_normal & 822.844 & 783.834 & \bfseries 799.642 & \cellcolor[rgb]{0.9, 0.54, 0.52} 1341.755 \\
\hline mad & 693.180 & \bfseries 679.548 & 657.789 & \cellcolor[rgb]{0.9, 0.54, 0.52} 1185.435 \\
\hline mad\_normal & 868.773 & \bfseries 851.687 & 824.416 & \cellcolor[rgb]{0.9, 0.54, 0.52} 1485.722 \\
\hline coef\_var & 0.437 & 0.504 & \bfseries 0.412 & \cellcolor[rgb]{0.9, 0.54, 0.52} 0.554 \\
\hline range & 11760 & \bfseries 13250 & \cellcolor[rgb]{0.9, 0.54, 0.52} 8998 & 10198 \\
\hline max & 12050 & \bfseries 13540 & \cellcolor[rgb]{0.9, 0.54, 0.52} 9404 & 10488 \\
\hline min & 290.000 & \bfseries 290.000 & \cellcolor[rgb]{0.9, 0.54, 0.52} 405.289 & \bfseries 290.000 \\
\hline skew & 1.371 & \cellcolor[rgb]{0.9, 0.54, 0.52} 4.333 & 1.117 & \bfseries 1.334 \\
\hline kurtosis & 7.167 & \cellcolor[rgb]{0.9, 0.54, 0.52} 44.030 & 5.084 & \bfseries 5.144 \\
\hline jarque\_bera & 17922 & \cellcolor[rgb]{0.9, 0.54, 0.52} 1583687 & 8402 & \bfseries 10545 \\
\hline jarque\_bera\_pval & 0.000 & 0.000 & 0.000 & 0.000 \\
\hline mode & 1400 & \bfseries 1440 & 1250 & \cellcolor[rgb]{0.9, 0.54, 0.52} 290 \\
\hline mode\_freq & 0.006 & \bfseries 0.007 & \cellcolor[rgb]{0.9, 0.54, 0.52} 0.001 & 0.001 \\
\hline median & 1910 & 1850 & \bfseries 1896 & \cellcolor[rgb]{0.9, 0.54, 0.52} 2442 \\
\hline 0.1\% & 522.890 & \bfseries 482.093 & 595.171 & \cellcolor[rgb]{0.9, 0.54, 0.52} 294.612 \\
\hline 1.0\% & 720.000 & \bfseries 726.418 & 757.374 & \cellcolor[rgb]{0.9, 0.54, 0.52} 592.120 \\
\hline 5.0\% & 940.000 & \bfseries 938.703 & 969.440 & \cellcolor[rgb]{0.9, 0.54, 0.52} 979.600 \\
\hline 25.0\% & 1430 & 1400 & \bfseries 1428 & \cellcolor[rgb]{0.9, 0.54, 0.52} 1715 \\
\hline 75.0\% & 2540 & 2457 & \bfseries 2507 & \cellcolor[rgb]{0.9, 0.54, 0.52} 3525 \\
\hline 95.0\% & 3740 & 3571 & \bfseries 3640 & \cellcolor[rgb]{0.9, 0.54, 0.52} 5727 \\
\hline 99.0\% & 4921 & \bfseries 4768 & 4590 & \cellcolor[rgb]{0.9, 0.54, 0.52} 8012 \\
\hline 99.9\% & 6966 & \cellcolor[rgb]{0.9, 0.54, 0.52} 13540 & \bfseries 6129 & 9755 \\
\hline
\end{tabular}
\end{table}

\begin{table}[H]
\centering
\fontsize{8}{14}\selectfont
\caption{Propiedades  estadisticas de variable waterfront, King county (A-2)}
\label{table-stats-king county-a-2-waterfront}
\begin{tabular}{|l|m{10em}|m{10em}|m{10em}|m{10em}|}
\hline
 \rowcolor[gray]{0.8}
Variable/Modelo & Real & tddpm\_mlp & smote-enc & ctgan \\
\hline top5 & [0 1] & [0 1] & [0 1] & [0 1] \\
\hline top5\_freq & [17166   124] & [21565    48] & [21582    32] & [20038  1575] \\
\hline top5\_prob & [0.99282822 0.00717178] & [0.99777911 0.00222089] & [0.99851948 0.00148052] & [0.92712719 0.07287281] \\
\hline nobs & 17290 & \bfseries 21613 & \cellcolor[rgb]{0.9, 0.54, 0.52} 21614 & \bfseries 21613 \\
\hline missing & 0.000 & 0.000 & 0.000 & 0.000 \\
\hline mean & 0.007 & \bfseries 0.002 & 0.001 & \cellcolor[rgb]{0.9, 0.54, 0.52} 0.073 \\
\hline std\_err & 0.001 & \bfseries 0.000 & 0.000 & \cellcolor[rgb]{0.9, 0.54, 0.52} 0.002 \\
\hline upper\_ci & 0.008 & \bfseries 0.003 & 0.002 & \cellcolor[rgb]{0.9, 0.54, 0.52} 0.076 \\
\hline lower\_ci & 0.006 & \bfseries 0.002 & 0.001 & \cellcolor[rgb]{0.9, 0.54, 0.52} 0.069 \\
\hline std & 0.084 & \bfseries 0.047 & 0.038 & \cellcolor[rgb]{0.9, 0.54, 0.52} 0.260 \\
\hline iqr & 0.000 & 0.000 & 0.000 & 0.000 \\
\hline iqr\_normal & 0.000 & 0.000 & 0.000 & 0.000 \\
\hline mad & 0.014 & \bfseries 0.004 & 0.003 & \cellcolor[rgb]{0.9, 0.54, 0.52} 0.135 \\
\hline mad\_normal & 0.018 & \bfseries 0.006 & 0.004 & \cellcolor[rgb]{0.9, 0.54, 0.52} 0.169 \\
\hline coef\_var & 11.766 & 21.197 & \cellcolor[rgb]{0.9, 0.54, 0.52} 25.971 & \bfseries 3.567 \\
\hline range & 1.000 & 1.000 & 1.000 & 1.000 \\
\hline max & 1.000 & 1.000 & 1.000 & 1.000 \\
\hline min & 0.000 & 0.000 & 0.000 & 0.000 \\
\hline skew & 11.681 & 21.149 & \cellcolor[rgb]{0.9, 0.54, 0.52} 25.931 & \bfseries 3.287 \\
\hline kurtosis & 137.443 & 448.273 & \cellcolor[rgb]{0.9, 0.54, 0.52} 673.439 & \bfseries 11.801 \\
\hline jarque\_bera & 13414600 & 180159835 & \cellcolor[rgb]{0.9, 0.54, 0.52} 407224138 & \bfseries 108664 \\
\hline jarque\_bera\_pval & 0.000 & 0.000 & 0.000 & 0.000 \\
\hline mode & 0.000 & 0.000 & 0.000 & 0.000 \\
\hline mode\_freq & 0.993 & \bfseries 0.998 & 0.999 & \cellcolor[rgb]{0.9, 0.54, 0.52} 0.927 \\
\hline median & 0.000 & 0.000 & 0.000 & 0.000 \\
\hline 0.1\% & 0.000 & 0.000 & 0.000 & 0.000 \\
\hline 1.0\% & 0.000 & 0.000 & 0.000 & 0.000 \\
\hline 5.0\% & 0.000 & 0.000 & 0.000 & 0.000 \\
\hline 25.0\% & 0.000 & 0.000 & 0.000 & 0.000 \\
\hline 75.0\% & 0.000 & 0.000 & 0.000 & 0.000 \\
\hline 95.0\% & 0.000 & \bfseries 0.000 & \bfseries 0.000 & \cellcolor[rgb]{0.9, 0.54, 0.52} 1.000 \\
\hline 99.0\% & 0.000 & \bfseries 0.000 & \bfseries 0.000 & \cellcolor[rgb]{0.9, 0.54, 0.52} 1.000 \\
\hline 99.9\% & 1.000 & 1.000 & 1.000 & 1.000 \\
\hline
\end{tabular}
\end{table}

\begin{table}[H]
\centering
\fontsize{8}{14}\selectfont
\caption{Propiedades  estadisticas de variable sqft\_basement, King county (A-2)}
\label{table-stats-king county-a-2-sqft_basement}
\begin{tabular}{|l|m{10em}|m{10em}|m{10em}|m{10em}|}
\hline
 \rowcolor[gray]{0.8}
Variable/Modelo & Real & tddpm\_mlp & smote-enc & ctgan \\
\hline top5 & [  0 600 700 500 800] & [  0. 500. 600. 700. 800.] & [  0. 800. 600. 850. 500.] & [ 0 10 11  9 12] \\
\hline top5\_freq & [10553   182   169   167   164] & [13604   239   196   187   177] & [11761    16    11     9     8] & [2222  863  815  804  804] \\
\hline top5\_prob & [0.61035281 0.01052632 0.00977444 0.00965876 0.00948525] & [0.62943599 0.01105816 0.00906862 0.0086522  0.00818952] & [5.44138059e-01 7.40260942e-04 5.08929398e-04 4.16396780e-04
 3.70130471e-04] & [0.10280849 0.03992967 0.03770879 0.03719983 0.03719983] \\
\hline nobs & 17290 & \bfseries 21613 & \cellcolor[rgb]{0.9, 0.54, 0.52} 21614 & \bfseries 21613 \\
\hline missing & 0.000 & 0.000 & 0.000 & 0.000 \\
\hline mean & 287.933 & \bfseries 286.843 & 274.142 & \cellcolor[rgb]{0.9, 0.54, 0.52} 457.012 \\
\hline std\_err & 3.337 & \bfseries 3.255 & 2.755 & \cellcolor[rgb]{0.9, 0.54, 0.52} 4.625 \\
\hline upper\_ci & 294.472 & \bfseries 293.223 & 279.541 & \cellcolor[rgb]{0.9, 0.54, 0.52} 466.076 \\
\hline lower\_ci & 281.393 & \bfseries 280.463 & 268.743 & \cellcolor[rgb]{0.9, 0.54, 0.52} 447.948 \\
\hline std & 438.727 & 478.554 & \bfseries 404.963 & \cellcolor[rgb]{0.9, 0.54, 0.52} 679.898 \\
\hline iqr & 550.000 & \bfseries 550.000 & 517.020 & \cellcolor[rgb]{0.9, 0.54, 0.52} 881.000 \\
\hline iqr\_normal & 407.716 & \bfseries 407.716 & 383.267 & \cellcolor[rgb]{0.9, 0.54, 0.52} 653.086 \\
\hline mad & 360.277 & \bfseries 365.019 & 332.362 & \cellcolor[rgb]{0.9, 0.54, 0.52} 562.364 \\
\hline mad\_normal & 451.541 & \bfseries 457.483 & 416.555 & \cellcolor[rgb]{0.9, 0.54, 0.52} 704.819 \\
\hline coef\_var & 1.524 & \cellcolor[rgb]{0.9, 0.54, 0.52} 1.668 & 1.477 & \bfseries 1.488 \\
\hline range & 4820 & \bfseries 4820 & \cellcolor[rgb]{0.9, 0.54, 0.52} 2713 & 3238 \\
\hline max & 4820 & \bfseries 4820 & \cellcolor[rgb]{0.9, 0.54, 0.52} 2713 & 3238 \\
\hline min & 0.000 & 0.000 & 0.000 & 0.000 \\
\hline skew & 1.571 & \cellcolor[rgb]{0.9, 0.54, 0.52} 3.129 & \bfseries 1.467 & 1.451 \\
\hline kurtosis & 5.639 & \cellcolor[rgb]{0.9, 0.54, 0.52} 23.527 & \bfseries 4.575 & 4.359 \\
\hline jarque\_bera & 12126 & \cellcolor[rgb]{0.9, 0.54, 0.52} 414719 & \bfseries 9983 & 9250 \\
\hline jarque\_bera\_pval & 0.000 & 0.000 & 0.000 & 0.000 \\
\hline mode & 0.000 & 0.000 & 0.000 & 0.000 \\
\hline mode\_freq & 0.610 & \bfseries 0.629 & 0.544 & \cellcolor[rgb]{0.9, 0.54, 0.52} 0.103 \\
\hline median & 0.000 & \bfseries 0.000 & \bfseries 0.000 & \cellcolor[rgb]{0.9, 0.54, 0.52} 13.000 \\
\hline 0.1\% & 0.000 & 0.000 & 0.000 & 0.000 \\
\hline 1.0\% & 0.000 & 0.000 & 0.000 & 0.000 \\
\hline 5.0\% & 0.000 & 0.000 & 0.000 & 0.000 \\
\hline 25.0\% & 0.000 & \bfseries 0.000 & \bfseries 0.000 & \cellcolor[rgb]{0.9, 0.54, 0.52} 6.000 \\
\hline 75.0\% & 550.000 & \bfseries 550.000 & 517.020 & \cellcolor[rgb]{0.9, 0.54, 0.52} 887.000 \\
\hline 95.0\% & 1180 & \bfseries 1134 & 1097 & \cellcolor[rgb]{0.9, 0.54, 0.52} 1868 \\
\hline 99.0\% & 1650 & \bfseries 1594 & 1539 & \cellcolor[rgb]{0.9, 0.54, 0.52} 2679 \\
\hline 99.9\% & 2324 & \cellcolor[rgb]{0.9, 0.54, 0.52} 4820 & \bfseries 2061 & 3082 \\
\hline
\end{tabular}
\end{table}

\begin{table}[H]
\centering
\fontsize{8}{14}\selectfont
\caption{Propiedades  estadisticas de variable sqft\_lot15, King county (A-2)}
\label{table-stats-king county-a-2-sqft_lot15}
\begin{tabular}{|l|m{10em}|m{10em}|m{10em}|m{10em}|}
\hline
 \rowcolor[gray]{0.8}
Variable/Modelo & Real & tddpm\_mlp & smote-enc & ctgan \\
\hline top5 & [5000 4000 6000 7200 4800] & [5000. 4000. 6000. 7200. 4800.] & [5000. 4000. 5200. 6000. 4080.] & [ 651 7137 9396 8240 4547] \\
\hline top5\_freq & [349 289 224 160 120] & [407 322 228 191 125] & [74 71 33 32 26] & [890   7   7   7   6] \\
\hline top5\_prob & [0.02018508 0.01671486 0.01295547 0.0092539  0.00694043] & [0.01883126 0.01489844 0.01054921 0.00883727 0.00578356] & [0.00342371 0.00328491 0.00152679 0.00148052 0.00120292] & [0.04117892 0.00032388 0.00032388 0.00032388 0.00027761] \\
\hline nobs & 17290 & \bfseries 21613 & \cellcolor[rgb]{0.9, 0.54, 0.52} 21614 & \bfseries 21613 \\
\hline missing & 0.000 & 0.000 & 0.000 & 0.000 \\
\hline mean & 12725 & \bfseries 12168 & 11832 & \cellcolor[rgb]{0.9, 0.54, 0.52} 14318 \\
\hline std\_err & 209.331 & \bfseries 221.883 & \cellcolor[rgb]{0.9, 0.54, 0.52} 155.914 & 163.365 \\
\hline upper\_ci & 13135 & \bfseries 12603 & 12137 & \cellcolor[rgb]{0.9, 0.54, 0.52} 14638 \\
\hline lower\_ci & 12315 & \bfseries 11733 & 11526 & \cellcolor[rgb]{0.9, 0.54, 0.52} 13997 \\
\hline std & 27525 & \cellcolor[rgb]{0.9, 0.54, 0.52} 32620 & 22922 & \bfseries 24017 \\
\hline iqr & 4963 & 4663 & \bfseries 4930 & \cellcolor[rgb]{0.9, 0.54, 0.52} 8395 \\
\hline iqr\_normal & 3679 & 3457 & \bfseries 3654 & \cellcolor[rgb]{0.9, 0.54, 0.52} 6223 \\
\hline mad & 10095 & 9042 & \cellcolor[rgb]{0.9, 0.54, 0.52} 8618 & \bfseries 10910 \\
\hline mad\_normal & 12652 & 11332 & \cellcolor[rgb]{0.9, 0.54, 0.52} 10800 & \bfseries 13674 \\
\hline coef\_var & 2.163 & \cellcolor[rgb]{0.9, 0.54, 0.52} 2.681 & \bfseries 1.937 & 1.677 \\
\hline range & 870549 & \bfseries 870549 & 680511 & \cellcolor[rgb]{0.9, 0.54, 0.52} 309873 \\
\hline max & 871200 & \bfseries 871200 & 681231 & \cellcolor[rgb]{0.9, 0.54, 0.52} 310524 \\
\hline min & 651.000 & \bfseries 651.000 & \cellcolor[rgb]{0.9, 0.54, 0.52} 719.360 & \bfseries 651.000 \\
\hline skew & 9.701 & \cellcolor[rgb]{0.9, 0.54, 0.52} 16.907 & \bfseries 8.316 & 6.341 \\
\hline kurtosis & 163.253 & \cellcolor[rgb]{0.9, 0.54, 0.52} 383.753 & \bfseries 101.830 & 54.342 \\
\hline jarque\_bera & 18772189 & \cellcolor[rgb]{0.9, 0.54, 0.52} 131583876 & \bfseries 9045446 & 2518709 \\
\hline jarque\_bera\_pval & 0.000 & 0.000 & 0.000 & 0.000 \\
\hline mode & 5000 & \bfseries 5000 & 5000 & \cellcolor[rgb]{0.9, 0.54, 0.52} 651 \\
\hline mode\_freq & 0.020 & \bfseries 0.019 & 0.003 & \cellcolor[rgb]{0.9, 0.54, 0.52} 0.041 \\
\hline median & 7615 & 7723 & \bfseries 7644 & \cellcolor[rgb]{0.9, 0.54, 0.52} 9083 \\
\hline 0.1\% & 886.289 & \bfseries 878.456 & 913.044 & \cellcolor[rgb]{0.9, 0.54, 0.52} 651.000 \\
\hline 1.0\% & 1189 & \bfseries 1215 & 1230 & \cellcolor[rgb]{0.9, 0.54, 0.52} 651 \\
\hline 5.0\% & 1965 & 2545 & \bfseries 2106 & \cellcolor[rgb]{0.9, 0.54, 0.52} 970 \\
\hline 25.0\% & 5083 & 5183 & \bfseries 5100 & \cellcolor[rgb]{0.9, 0.54, 0.52} 5235 \\
\hline 75.0\% & 10046 & 9846 & \bfseries 10030 & \cellcolor[rgb]{0.9, 0.54, 0.52} 13630 \\
\hline 95.0\% & 36822 & 35077 & \bfseries 35134 & \cellcolor[rgb]{0.9, 0.54, 0.52} 45254 \\
\hline 99.0\% & 168296 & \cellcolor[rgb]{0.9, 0.54, 0.52} 107518 & 130809 & \bfseries 138474 \\
\hline 99.9\% & 306998 & \cellcolor[rgb]{0.9, 0.54, 0.52} 532713 & 263407 & \bfseries 269421 \\
\hline
\end{tabular}
\end{table}

\begin{table}[H]
\centering
\fontsize{8}{14}\selectfont
\caption{Propiedades  estadisticas de variable date, King county (A-2)}
\label{table-stats-king county-a-2-date}
\begin{tabular}{|l|m{10em}|m{10em}|m{10em}|m{10em}|}
\hline
 \rowcolor[gray]{0.8}
Variable/Modelo & Real & tddpm\_mlp & smote-enc & ctgan \\
\hline top5 & ['20140623T000000' '20140625T000000' '20140626T000000' '20150421T000000'
 '20150325T000000'] & ['20140623T000000' '20140625T000000' '20150427T000000' '20140520T000000'
 '20150421T000000'] & ['20140623T000000' '20140625T000000' '20150414T000000' '20140520T000000'
 '20150421T000000'] & ['20150310T000000' '20150327T000000' '20140603T000000' '20150226T000000'
 '20150329T000000'] \\
\hline top5\_freq & [123 105 101 101 101] & [195 167 163 161 151] & [151 146 139 136 135] & [489 421 337 288 283] \\
\hline top5\_prob & [0.00711394 0.00607287 0.00584153 0.00584153 0.00584153] & [0.00902235 0.00772683 0.00754176 0.00744922 0.00698654] & [0.00698621 0.00675488 0.00643102 0.00629222 0.00624595] & [0.02262527 0.01947902 0.01559247 0.01332531 0.01309397] \\
\hline nobs & 17290 & \bfseries 21613 & \cellcolor[rgb]{0.9, 0.54, 0.52} 21614 & \bfseries 21613 \\
\hline missing & 17290 & 0 & 0 & 0 \\
\hline
\end{tabular}
\end{table}



\section{Estadísticos Económicos - Conjunto A}
\label{propiedades-estadisticas-economicos-A}
\begin{table}[H]
\centering
\fontsize{8}{14}\selectfont
\caption{Propiedades  estadisticas de variable yr\_built, King county (A-2)}
\label{table-stats-king county-a-2-yr_built}
\begin{tabular}{|l|m{10em}|m{10em}|m{10em}|m{10em}|}
\hline
 \rowcolor[gray]{0.8}
Variable/Modelo & Real & tddpm\_mlp & smote-enc & ctgan \\
\hline top5 & [2014 2005 2006 2004 2007] & [2006. 2005. 2004. 1977. 2003.] & [2014. 2006. 2005. 2007. 2003.] & [1900 1976 1977 1975 1974] \\
\hline top5\_freq & [449 371 366 350 347] & [463 451 442 439 415] & [247 113 102  97  93] & [639 355 343 330 329] \\
\hline top5\_prob & [0.02596877 0.02145749 0.02116831 0.02024291 0.0200694 ] & [0.02142229 0.02086707 0.02045065 0.02031185 0.01920141] & [0.01142778 0.00522809 0.00471916 0.00448783 0.00430277] & [0.02956554 0.0164253  0.01587008 0.01526859 0.01522232] \\
\hline nobs & 17290 & \bfseries 21613 & \cellcolor[rgb]{0.9, 0.54, 0.52} 21614 & \bfseries 21613 \\
\hline missing & 0.000 & 0.000 & 0.000 & 0.000 \\
\hline mean & 1971 & 1972 & \bfseries 1971 & \cellcolor[rgb]{0.9, 0.54, 0.52} 1961 \\
\hline std\_err & 0.224 & \cellcolor[rgb]{0.9, 0.54, 0.52} 0.190 & 0.198 & \bfseries 0.204 \\
\hline upper\_ci & 1972 & 1972 & \bfseries 1972 & \cellcolor[rgb]{0.9, 0.54, 0.52} 1961 \\
\hline lower\_ci & 1971 & 1972 & \bfseries 1971 & \cellcolor[rgb]{0.9, 0.54, 0.52} 1960 \\
\hline std & 29.436 & \cellcolor[rgb]{0.9, 0.54, 0.52} 27.911 & \bfseries 29.169 & 29.969 \\
\hline iqr & 46.000 & \cellcolor[rgb]{0.9, 0.54, 0.52} 42.578 & \bfseries 45.832 & 45.000 \\
\hline iqr\_normal & 34.100 & \cellcolor[rgb]{0.9, 0.54, 0.52} 31.563 & \bfseries 33.975 & 33.359 \\
\hline mad & 24.632 & \cellcolor[rgb]{0.9, 0.54, 0.52} 23.153 & \bfseries 24.459 & 24.966 \\
\hline mad\_normal & 30.872 & \cellcolor[rgb]{0.9, 0.54, 0.52} 29.018 & \bfseries 30.655 & 31.290 \\
\hline coef\_var & 0.015 & \cellcolor[rgb]{0.9, 0.54, 0.52} 0.014 & \bfseries 0.015 & 0.015 \\
\hline range & 115.000 & 115.000 & 115.000 & 115.000 \\
\hline max & 2015 & \bfseries 2015 & \cellcolor[rgb]{0.9, 0.54, 0.52} 2015 & \bfseries 2015 \\
\hline min & 1900 & \bfseries 1900 & \cellcolor[rgb]{0.9, 0.54, 0.52} 1900 & \bfseries 1900 \\
\hline skew & -0.472 & \bfseries -0.475 & -0.464 & \cellcolor[rgb]{0.9, 0.54, 0.52} -0.305 \\
\hline kurtosis & 2.337 & 2.438 & \bfseries 2.309 & \cellcolor[rgb]{0.9, 0.54, 0.52} 2.155 \\
\hline jarque\_bera & 957.631 & 1095.917 & \cellcolor[rgb]{0.9, 0.54, 0.52} 1205.886 & \bfseries 978.229 \\
\hline jarque\_bera\_pval & 0.000 & \cellcolor[rgb]{0.9, 0.54, 0.52} 0.000 & \cellcolor[rgb]{0.9, 0.54, 0.52} 0.000 & \bfseries 0.000 \\
\hline mode & 2014 & 2006 & \bfseries 2014 & \cellcolor[rgb]{0.9, 0.54, 0.52} 1900 \\
\hline mode\_freq & 0.026 & 0.021 & \cellcolor[rgb]{0.9, 0.54, 0.52} 0.011 & \bfseries 0.030 \\
\hline median & 1975 & \bfseries 1975 & 1974 & \cellcolor[rgb]{0.9, 0.54, 0.52} 1963 \\
\hline 0.1\% & 1900 & \bfseries 1900 & \cellcolor[rgb]{0.9, 0.54, 0.52} 1901 & \bfseries 1900 \\
\hline 1.0\% & 1904 & 1906 & \bfseries 1905 & \cellcolor[rgb]{0.9, 0.54, 0.52} 1900 \\
\hline 5.0\% & 1915 & 1919 & \bfseries 1916 & \cellcolor[rgb]{0.9, 0.54, 0.52} 1908 \\
\hline 25.0\% & 1951 & 1954 & \bfseries 1952 & \cellcolor[rgb]{0.9, 0.54, 0.52} 1939 \\
\hline 75.0\% & 1997 & \bfseries 1997 & 1998 & \cellcolor[rgb]{0.9, 0.54, 0.52} 1984 \\
\hline 95.0\% & 2011 & 2009 & \bfseries 2010 & \cellcolor[rgb]{0.9, 0.54, 0.52} 2005 \\
\hline 99.0\% & 2014 & \bfseries 2014 & \bfseries 2014 & \cellcolor[rgb]{0.9, 0.54, 0.52} 2013 \\
\hline 99.9\% & 2015 & \bfseries 2015 & \cellcolor[rgb]{0.9, 0.54, 0.52} 2014 & \bfseries 2015 \\
\hline
\end{tabular}
\end{table}

\begin{table}[H]
\centering
\fontsize{8}{14}\selectfont
\caption{Propiedades  estadisticas de variable sqft\_above, King county (A-2)}
\label{table-stats-king county-a-2-sqft_above}
\begin{tabular}{|l|m{10em}|m{10em}|m{10em}|m{10em}|}
\hline
 \rowcolor[gray]{0.8}
Variable/Modelo & Real & tddpm\_mlp & smote-enc & ctgan \\
\hline top5 & [1300 1010 1200 1220 1140] & [1300. 1010. 1220. 1140. 1340.] & [1290. 1160. 1830. 1010. 1320.] & [1563 1241 1203 1170 1197] \\
\hline top5\_freq & [166 165 160 152 148] & [203 185 180 171 162] & [15 15 14 13 12] & [20 20 20 19 19] \\
\hline top5\_prob & [0.00960093 0.00954309 0.0092539  0.00879121 0.00855986] & [0.0093925  0.00855966 0.00832832 0.0079119  0.00749549] & [0.00069399 0.00069399 0.00064773 0.00060146 0.0005552 ] & [0.00092537 0.00092537 0.00092537 0.0008791  0.0008791 ] \\
\hline nobs & 17290 & \bfseries 21613 & \cellcolor[rgb]{0.9, 0.54, 0.52} 21614 & \bfseries 21613 \\
\hline missing & 0.000 & 0.000 & 0.000 & 0.000 \\
\hline mean & 1786 & \bfseries 1778 & 1769 & \cellcolor[rgb]{0.9, 0.54, 0.52} 2018 \\
\hline std\_err & 6.249 & \bfseries 5.620 & 5.283 & \cellcolor[rgb]{0.9, 0.54, 0.52} 7.466 \\
\hline upper\_ci & 1798 & \bfseries 1789 & 1780 & \cellcolor[rgb]{0.9, 0.54, 0.52} 2033 \\
\hline lower\_ci & 1774 & \bfseries 1767 & 1759 & \cellcolor[rgb]{0.9, 0.54, 0.52} 2004 \\
\hline std & 821.626 & \bfseries 826.164 & 776.654 & \cellcolor[rgb]{0.9, 0.54, 0.52} 1097.579 \\
\hline iqr & 1000.000 & \bfseries 975.555 & 972.791 & \cellcolor[rgb]{0.9, 0.54, 0.52} 1434.000 \\
\hline iqr\_normal & 741.301 & \bfseries 723.180 & 721.131 & \cellcolor[rgb]{0.9, 0.54, 0.52} 1063.026 \\
\hline mad & 635.012 & \bfseries 615.572 & 608.256 & \cellcolor[rgb]{0.9, 0.54, 0.52} 872.298 \\
\hline mad\_normal & 795.870 & \bfseries 771.505 & 762.336 & \cellcolor[rgb]{0.9, 0.54, 0.52} 1093.263 \\
\hline coef\_var & 0.460 & \bfseries 0.465 & 0.439 & \cellcolor[rgb]{0.9, 0.54, 0.52} 0.544 \\
\hline range & 8570 & 9120 & \bfseries 8368 & \cellcolor[rgb]{0.9, 0.54, 0.52} 7269 \\
\hline max & 8860 & 9410 & \bfseries 8670 & \cellcolor[rgb]{0.9, 0.54, 0.52} 7559 \\
\hline min & 290.000 & \bfseries 290.000 & \cellcolor[rgb]{0.9, 0.54, 0.52} 301.856 & \bfseries 290.000 \\
\hline skew & 1.428 & \cellcolor[rgb]{0.9, 0.54, 0.52} 2.236 & \bfseries 1.320 & 1.097 \\
\hline kurtosis & 6.260 & \cellcolor[rgb]{0.9, 0.54, 0.52} 15.164 & \bfseries 5.306 & 3.945 \\
\hline jarque\_bera & 13530 & \cellcolor[rgb]{0.9, 0.54, 0.52} 151241 & \bfseries 11064 & 5139 \\
\hline jarque\_bera\_pval & 0.000 & 0.000 & 0.000 & 0.000 \\
\hline mode & 1300 & \bfseries 1300 & \cellcolor[rgb]{0.9, 0.54, 0.52} 1160 & 1203 \\
\hline mode\_freq & 0.010 & \bfseries 0.009 & \cellcolor[rgb]{0.9, 0.54, 0.52} 0.001 & 0.001 \\
\hline median & 1560 & \bfseries 1560 & 1540 & \cellcolor[rgb]{0.9, 0.54, 0.52} 1721 \\
\hline 0.1\% & 520.000 & \bfseries 544.640 & 603.805 & \cellcolor[rgb]{0.9, 0.54, 0.52} 360.224 \\
\hline 1.0\% & 700.000 & \bfseries 720.516 & 740.191 & \cellcolor[rgb]{0.9, 0.54, 0.52} 512.120 \\
\hline 5.0\% & 850.000 & \bfseries 878.073 & 888.091 & \cellcolor[rgb]{0.9, 0.54, 0.52} 731.000 \\
\hline 25.0\% & 1200 & 1207 & \cellcolor[rgb]{0.9, 0.54, 0.52} 1207 & \bfseries 1198 \\
\hline 75.0\% & 2200 & \bfseries 2182 & 2180 & \cellcolor[rgb]{0.9, 0.54, 0.52} 2632 \\
\hline 95.0\% & 3380 & 3269 & \bfseries 3295 & \cellcolor[rgb]{0.9, 0.54, 0.52} 4226 \\
\hline 99.0\% & 4371 & 4194 & \bfseries 4199 & \cellcolor[rgb]{0.9, 0.54, 0.52} 5324 \\
\hline 99.9\% & 6070 & \cellcolor[rgb]{0.9, 0.54, 0.52} 9410 & 5258 & \bfseries 6386 \\
\hline
\end{tabular}
\end{table}

\begin{table}[H]
\centering
\fontsize{8}{14}\selectfont
\caption{Propiedades  estadisticas de variable yr\_renovated, King county (A-2)}
\label{table-stats-king county-a-2-yr_renovated}
\begin{tabular}{|l|m{10em}|m{10em}|m{10em}|m{10em}|}
\hline
 \rowcolor[gray]{0.8}
Variable/Modelo & Real & tddpm\_mlp & smote-enc & ctgan \\
\hline top5 & [   0 2014 2005 2000 2003] & [   0.         2014.         2015.          557.16404191 2005.93527463] & [   0. 2014. 2005. 2006. 2013.] & [   0 2015    2    3    4] \\
\hline top5\_freq & [16571    76    32    30    29] & [20892    62    53     1     1] & [20753    23     4     3     2] & [7471 1690 1337 1320 1309] \\
\hline top5\_prob & [0.95841527 0.0043956  0.00185078 0.00173511 0.00167727] & [9.66640448e-01 2.86864387e-03 2.45222783e-03 4.62684495e-05
 4.62684495e-05] & [9.60164708e-01 1.06412510e-03 1.85065235e-04 1.38798927e-04
 9.25326177e-05] & [0.34567159 0.07819368 0.06186092 0.06107435 0.0605654 ] \\
\hline nobs & 17290 & \bfseries 21613 & \cellcolor[rgb]{0.9, 0.54, 0.52} 21614 & \bfseries 21613 \\
\hline missing & 0.000 & 0.000 & 0.000 & 0.000 \\
\hline mean & 83.003 & 66.202 & \bfseries 74.187 & \cellcolor[rgb]{0.9, 0.54, 0.52} 280.061 \\
\hline std\_err & 3.031 & 2.431 & \bfseries 2.533 & \cellcolor[rgb]{0.9, 0.54, 0.52} 4.664 \\
\hline upper\_ci & 88.943 & 70.967 & \bfseries 79.151 & \cellcolor[rgb]{0.9, 0.54, 0.52} 289.203 \\
\hline lower\_ci & 77.063 & 61.438 & \bfseries 69.222 & \cellcolor[rgb]{0.9, 0.54, 0.52} 270.920 \\
\hline std & 398.503 & 357.413 & \bfseries 372.384 & \cellcolor[rgb]{0.9, 0.54, 0.52} 685.680 \\
\hline iqr & 0.000 & \bfseries 0.000 & \bfseries 0.000 & \cellcolor[rgb]{0.9, 0.54, 0.52} 7.000 \\
\hline iqr\_normal & 0.000 & \bfseries 0.000 & \bfseries 0.000 & \cellcolor[rgb]{0.9, 0.54, 0.52} 5.189 \\
\hline mad & 159.103 & 127.988 & \bfseries 142.469 & \cellcolor[rgb]{0.9, 0.54, 0.52} 476.288 \\
\hline mad\_normal & 199.407 & 160.409 & \bfseries 178.559 & \cellcolor[rgb]{0.9, 0.54, 0.52} 596.939 \\
\hline coef\_var & 4.801 & 5.399 & \bfseries 5.020 & \cellcolor[rgb]{0.9, 0.54, 0.52} 2.448 \\
\hline range & 2015 & \bfseries 2015 & \cellcolor[rgb]{0.9, 0.54, 0.52} 2015 & \bfseries 2015 \\
\hline max & 2015 & \bfseries 2015 & \cellcolor[rgb]{0.9, 0.54, 0.52} 2015 & \bfseries 2015 \\
\hline min & 0.000 & 0.000 & 0.000 & 0.000 \\
\hline skew & 4.593 & 5.224 & \bfseries 4.886 & \cellcolor[rgb]{0.9, 0.54, 0.52} 2.074 \\
\hline kurtosis & 22.096 & 28.309 & \bfseries 25.024 & \cellcolor[rgb]{0.9, 0.54, 0.52} 5.308 \\
\hline jarque\_bera & 323506 & \cellcolor[rgb]{0.9, 0.54, 0.52} 675119 & \bfseries 522838 & 20287 \\
\hline jarque\_bera\_pval & 0.000 & 0.000 & 0.000 & 0.000 \\
\hline mode & 0.000 & 0.000 & 0.000 & 0.000 \\
\hline mode\_freq & 0.958 & 0.967 & \bfseries 0.960 & \cellcolor[rgb]{0.9, 0.54, 0.52} 0.346 \\
\hline median & 0.000 & \bfseries 0.000 & \bfseries 0.000 & \cellcolor[rgb]{0.9, 0.54, 0.52} 3.000 \\
\hline 0.1\% & 0.000 & 0.000 & 0.000 & 0.000 \\
\hline 1.0\% & 0.000 & 0.000 & 0.000 & 0.000 \\
\hline 5.0\% & 0.000 & 0.000 & 0.000 & 0.000 \\
\hline 25.0\% & 0.000 & 0.000 & 0.000 & 0.000 \\
\hline 75.0\% & 0.000 & \bfseries 0.000 & \bfseries 0.000 & \cellcolor[rgb]{0.9, 0.54, 0.52} 7.000 \\
\hline 95.0\% & 0.000 & \bfseries 0.000 & \bfseries 0.000 & \cellcolor[rgb]{0.9, 0.54, 0.52} 2015.000 \\
\hline 99.0\% & 2008 & \bfseries 2010 & 2005 & \cellcolor[rgb]{0.9, 0.54, 0.52} 2015 \\
\hline 99.9\% & 2014 & \cellcolor[rgb]{0.9, 0.54, 0.52} 2015 & \bfseries 2014 & \cellcolor[rgb]{0.9, 0.54, 0.52} 2015 \\
\hline
\end{tabular}
\end{table}

\begin{table}[H]
\centering
\fontsize{8}{14}\selectfont
\caption{Propiedades  estadisticas de variable sqft\_lot, King county (A-2)}
\label{table-stats-king county-a-2-sqft_lot}
\begin{tabular}{|l|m{10em}|m{10em}|m{10em}|m{10em}|}
\hline
 \rowcolor[gray]{0.8}
Variable/Modelo & Real & tddpm\_mlp & smote-enc & ctgan \\
\hline top5 & [5000 4000 6000 7200 4800] & [5000. 4000. 6000. 7200. 4500.] & [5000. 4000. 4080. 6000. 8000.] & [  520 12941 13597 12637 12255] \\
\hline top5\_freq & [301 209 208 179  98] & [356 248 230 195 109] & [56 40 22 17 16] & [321   9   9   7   7] \\
\hline top5\_prob & [0.01740891 0.01208791 0.01203008 0.01035281 0.00566802] & [0.01647157 0.01147458 0.01064174 0.00902235 0.00504326] & [0.00259091 0.00185065 0.00101786 0.00078653 0.00074026] & [0.01485217 0.00041642 0.00041642 0.00032388 0.00032388] \\
\hline nobs & 17290 & \bfseries 21613 & \cellcolor[rgb]{0.9, 0.54, 0.52} 21614 & \bfseries 21613 \\
\hline missing & 0.000 & 0.000 & 0.000 & 0.000 \\
\hline mean & 14799 & 16628 & \bfseries 14059 & \cellcolor[rgb]{0.9, 0.54, 0.52} 18126 \\
\hline std\_err & 295.375 & \cellcolor[rgb]{0.9, 0.54, 0.52} 611.901 & \bfseries 216.074 & 206.230 \\
\hline upper\_ci & 15378 & 17827 & \bfseries 14482 & \cellcolor[rgb]{0.9, 0.54, 0.52} 18530 \\
\hline lower\_ci & 14220 & 15428 & \bfseries 13635 & \cellcolor[rgb]{0.9, 0.54, 0.52} 17722 \\
\hline std & 38839 & \cellcolor[rgb]{0.9, 0.54, 0.52} 89958 & \bfseries 31766 & 30319 \\
\hline iqr & 5606 & 4905 & \bfseries 5584 & \cellcolor[rgb]{0.9, 0.54, 0.52} 7113 \\
\hline iqr\_normal & 4155 & 3636 & \bfseries 4139 & \cellcolor[rgb]{0.9, 0.54, 0.52} 5273 \\
\hline mad & 13382 & \cellcolor[rgb]{0.9, 0.54, 0.52} 17162 & 11949 & \bfseries 13481 \\
\hline mad\_normal & 16772 & \cellcolor[rgb]{0.9, 0.54, 0.52} 21510 & 14976 & \bfseries 16896 \\
\hline coef\_var & 2.624 & \cellcolor[rgb]{0.9, 0.54, 0.52} 5.410 & \bfseries 2.260 & 1.673 \\
\hline range & 1164274 & 1650839 & \bfseries 962591 & \cellcolor[rgb]{0.9, 0.54, 0.52} 333705 \\
\hline max & 1164794 & 1651359 & \bfseries 963282 & \cellcolor[rgb]{0.9, 0.54, 0.52} 334225 \\
\hline min & 520.000 & \bfseries 520.000 & \cellcolor[rgb]{0.9, 0.54, 0.52} 690.676 & \bfseries 520.000 \\
\hline skew & 11.588 & 16.678 & \bfseries 8.762 & \cellcolor[rgb]{0.9, 0.54, 0.52} 6.046 \\
\hline kurtosis & 215.591 & \bfseries 297.135 & 123.595 & \cellcolor[rgb]{0.9, 0.54, 0.52} 46.341 \\
\hline jarque\_bera & 32946220 & \cellcolor[rgb]{0.9, 0.54, 0.52} 78912817 & \bfseries 13373770 & 1823319 \\
\hline jarque\_bera\_pval & 0.000 & 0.000 & 0.000 & 0.000 \\
\hline mode & 5000 & \bfseries 5000 & \bfseries 5000 & \cellcolor[rgb]{0.9, 0.54, 0.52} 520 \\
\hline mode\_freq & 0.017 & \bfseries 0.016 & \cellcolor[rgb]{0.9, 0.54, 0.52} 0.003 & 0.015 \\
\hline median & 7600 & \bfseries 7491 & 7769 & \cellcolor[rgb]{0.9, 0.54, 0.52} 11577 \\
\hline 0.1\% & 737.156 & \bfseries 812.333 & 833.904 & \cellcolor[rgb]{0.9, 0.54, 0.52} 520.000 \\
\hline 1.0\% & 1005 & \bfseries 1019 & 1067 & \cellcolor[rgb]{0.9, 0.54, 0.52} 520 \\
\hline 5.0\% & 1756 & 1589 & \bfseries 1773 & \cellcolor[rgb]{0.9, 0.54, 0.52} 2927 \\
\hline 25.0\% & 5001 & \bfseries 5000 & 5136 & \cellcolor[rgb]{0.9, 0.54, 0.52} 7973 \\
\hline 75.0\% & 10607 & 9905 & \bfseries 10720 & \cellcolor[rgb]{0.9, 0.54, 0.52} 15086 \\
\hline 95.0\% & 42999 & 36194 & \bfseries 40503 & \cellcolor[rgb]{0.9, 0.54, 0.52} 56973 \\
\hline 99.0\% & 212192 & \cellcolor[rgb]{0.9, 0.54, 0.52} 181619 & 191877 & \bfseries 196506 \\
\hline 99.9\% & 435600 & \cellcolor[rgb]{0.9, 0.54, 0.52} 1651359 & \bfseries 377395 & 300056 \\
\hline
\end{tabular}
\end{table}

\begin{table}[H]
\centering
\fontsize{8}{14}\selectfont
\caption{Propiedades  estadisticas de variable view, King county (A-2)}
\label{table-stats-king county-a-2-view}
\begin{tabular}{|l|m{10em}|m{10em}|m{10em}|m{10em}|}
\hline
 \rowcolor[gray]{0.8}
Variable/Modelo & Real & tddpm\_mlp & smote-enc & ctgan \\
\hline top5 & [0 2 3 1 4] & [0 2 3 1 4] & [0 2 3 4 1] & [0 2 4 3 1] \\
\hline top5\_freq & [15586   783   396   275   250] & [20619   514   224   147   109] & [20891   336   176   140    71] & [18103  1617   748   604   541] \\
\hline top5\_prob & [0.90144592 0.04528629 0.02290341 0.01590515 0.01445922] & [0.95400916 0.02378198 0.01036413 0.00680146 0.00504326] & [0.96654946 0.01554548 0.00814287 0.00647728 0.00328491] & [0.83759774 0.07481608 0.0346088  0.02794614 0.02503123] \\
\hline nobs & 17290 & \bfseries 21613 & \cellcolor[rgb]{0.9, 0.54, 0.52} 21614 & \bfseries 21613 \\
\hline missing & 0.000 & 0.000 & 0.000 & 0.000 \\
\hline mean & 0.233 & \bfseries 0.106 & 0.085 & \cellcolor[rgb]{0.9, 0.54, 0.52} 0.397 \\
\hline std\_err & 0.006 & 0.003 & \cellcolor[rgb]{0.9, 0.54, 0.52} 0.003 & \bfseries 0.007 \\
\hline upper\_ci & 0.244 & \bfseries 0.112 & 0.091 & \cellcolor[rgb]{0.9, 0.54, 0.52} 0.410 \\
\hline lower\_ci & 0.222 & \bfseries 0.099 & 0.078 & \cellcolor[rgb]{0.9, 0.54, 0.52} 0.384 \\
\hline std & 0.762 & 0.515 & \cellcolor[rgb]{0.9, 0.54, 0.52} 0.485 & \bfseries 0.986 \\
\hline iqr & 0.000 & 0.000 & 0.000 & 0.000 \\
\hline iqr\_normal & 0.000 & 0.000 & 0.000 & 0.000 \\
\hline mad & 0.420 & \bfseries 0.202 & \cellcolor[rgb]{0.9, 0.54, 0.52} 0.164 & 0.665 \\
\hline mad\_normal & 0.527 & \bfseries 0.253 & \cellcolor[rgb]{0.9, 0.54, 0.52} 0.205 & 0.833 \\
\hline coef\_var & 3.269 & 4.871 & \cellcolor[rgb]{0.9, 0.54, 0.52} 5.725 & \bfseries 2.484 \\
\hline range & 4.000 & 4.000 & 4.000 & 4.000 \\
\hline max & 4.000 & 4.000 & 4.000 & 4.000 \\
\hline min & 0.000 & 0.000 & 0.000 & 0.000 \\
\hline skew & 3.402 & 5.246 & \cellcolor[rgb]{0.9, 0.54, 0.52} 6.151 & \bfseries 2.476 \\
\hline kurtosis & 13.971 & 31.362 & \cellcolor[rgb]{0.9, 0.54, 0.52} 41.967 & \bfseries 8.080 \\
\hline jarque\_bera & 120072 & 823540 & \cellcolor[rgb]{0.9, 0.54, 0.52} 1503760 & \bfseries 45333 \\
\hline jarque\_bera\_pval & 0.000 & 0.000 & 0.000 & 0.000 \\
\hline mode & 0.000 & 0.000 & 0.000 & 0.000 \\
\hline mode\_freq & 0.901 & \bfseries 0.954 & \cellcolor[rgb]{0.9, 0.54, 0.52} 0.967 & 0.838 \\
\hline median & 0.000 & 0.000 & 0.000 & 0.000 \\
\hline 0.1\% & 0.000 & 0.000 & 0.000 & 0.000 \\
\hline 1.0\% & 0.000 & 0.000 & 0.000 & 0.000 \\
\hline 5.0\% & 0.000 & 0.000 & 0.000 & 0.000 \\
\hline 25.0\% & 0.000 & 0.000 & 0.000 & 0.000 \\
\hline 75.0\% & 0.000 & 0.000 & 0.000 & 0.000 \\
\hline 95.0\% & 2.000 & \cellcolor[rgb]{0.9, 0.54, 0.52} 0.000 & \cellcolor[rgb]{0.9, 0.54, 0.52} 0.000 & \bfseries 3.000 \\
\hline 99.0\% & 4.000 & \cellcolor[rgb]{0.9, 0.54, 0.52} 3.000 & \cellcolor[rgb]{0.9, 0.54, 0.52} 3.000 & \bfseries 4.000 \\
\hline 99.9\% & 4.000 & 4.000 & 4.000 & 4.000 \\
\hline
\end{tabular}
\end{table}

\begin{table}[H]
\centering
\fontsize{8}{14}\selectfont
\caption{Propiedades  estadisticas de variable floors, King county (A-2)}
\label{table-stats-king county-a-2-floors}
\begin{tabular}{|l|m{10em}|m{10em}|m{10em}|m{10em}|}
\hline
 \rowcolor[gray]{0.8}
Variable/Modelo & Real & tddpm\_mlp & smote-enc & ctgan \\
\hline top5 & [1.  2.  1.5 3.  2.5] & [1.  2.  1.5 3.  2.5] & [1.  2.  1.5 3.  2.5] & [1.  2.  1.5 3.  2.5] \\
\hline top5\_freq & [8488 6628 1523  517  128] & [11201  8302  1488   584    36] & [11267  8351  1329   623    44] & [12847  4573  2944   693   398] \\
\hline top5\_prob & [0.49091961 0.38334297 0.0880856  0.02990168 0.00740312] & [0.5182529  0.38412067 0.06884745 0.02702077 0.00166566] & [0.5212825  0.38636995 0.06148792 0.02882391 0.00203572] & [0.59441077 0.21158562 0.13621432 0.03206404 0.01841484] \\
\hline nobs & 17290 & \bfseries 21613 & \cellcolor[rgb]{0.9, 0.54, 0.52} 21614 & \bfseries 21613 \\
\hline missing & 0.000 & 0.000 & 0.000 & 0.000 \\
\hline mean & 1.499 & 1.475 & \bfseries 1.478 & \cellcolor[rgb]{0.9, 0.54, 0.52} 1.390 \\
\hline std\_err & 0.004 & \cellcolor[rgb]{0.9, 0.54, 0.52} 0.004 & 0.004 & \bfseries 0.004 \\
\hline upper\_ci & 1.507 & 1.482 & \bfseries 1.485 & \cellcolor[rgb]{0.9, 0.54, 0.52} 1.397 \\
\hline lower\_ci & 1.491 & 1.468 & \bfseries 1.471 & \cellcolor[rgb]{0.9, 0.54, 0.52} 1.382 \\
\hline std & 0.543 & 0.536 & \bfseries 0.542 & \cellcolor[rgb]{0.9, 0.54, 0.52} 0.556 \\
\hline iqr & 1.000 & 1.000 & 1.000 & 1.000 \\
\hline iqr\_normal & 0.741 & 0.741 & 0.741 & 0.741 \\
\hline mad & 0.490 & \bfseries 0.493 & 0.498 & \cellcolor[rgb]{0.9, 0.54, 0.52} 0.463 \\
\hline mad\_normal & 0.614 & \bfseries 0.617 & 0.624 & \cellcolor[rgb]{0.9, 0.54, 0.52} 0.581 \\
\hline coef\_var & 0.362 & \bfseries 0.364 & 0.366 & \cellcolor[rgb]{0.9, 0.54, 0.52} 0.400 \\
\hline range & 2.500 & \bfseries 2.500 & \cellcolor[rgb]{0.9, 0.54, 0.52} 2.000 & \bfseries 2.500 \\
\hline max & 3.500 & \bfseries 3.500 & \cellcolor[rgb]{0.9, 0.54, 0.52} 3.000 & \bfseries 3.500 \\
\hline min & 1.000 & 1.000 & 1.000 & 1.000 \\
\hline skew & 0.615 & \bfseries 0.636 & 0.642 & \cellcolor[rgb]{0.9, 0.54, 0.52} 1.401 \\
\hline kurtosis & 2.526 & \bfseries 2.474 & 2.474 & \cellcolor[rgb]{0.9, 0.54, 0.52} 4.517 \\
\hline jarque\_bera & 1252 & \bfseries 1705 & 1734 & \cellcolor[rgb]{0.9, 0.54, 0.52} 9146 \\
\hline jarque\_bera\_pval & 0.000 & 0.000 & 0.000 & 0.000 \\
\hline mode & 1.000 & 1.000 & 1.000 & 1.000 \\
\hline mode\_freq & 0.491 & \bfseries 0.518 & 0.521 & \cellcolor[rgb]{0.9, 0.54, 0.52} 0.594 \\
\hline median & 1.500 & 1.000 & 1.000 & 1.000 \\
\hline 0.1\% & 1.000 & 1.000 & 1.000 & 1.000 \\
\hline 1.0\% & 1.000 & 1.000 & 1.000 & 1.000 \\
\hline 5.0\% & 1.000 & 1.000 & 1.000 & 1.000 \\
\hline 25.0\% & 1.000 & 1.000 & 1.000 & 1.000 \\
\hline 75.0\% & 2.000 & 2.000 & 2.000 & 2.000 \\
\hline 95.0\% & 2.000 & \bfseries 2.000 & \bfseries 2.000 & \cellcolor[rgb]{0.9, 0.54, 0.52} 2.500 \\
\hline 99.0\% & 3.000 & 3.000 & 3.000 & 3.000 \\
\hline 99.9\% & 3.000 & \bfseries 3.000 & \bfseries 3.000 & \cellcolor[rgb]{0.9, 0.54, 0.52} 3.500 \\
\hline
\end{tabular}
\end{table}

\begin{table}[H]
\centering
\fontsize{8}{14}\selectfont
\caption{Propiedades  estadisticas de variable bedrooms, King county (A-2)}
\label{table-stats-king county-a-2-bedrooms}
\begin{tabular}{|l|m{10em}|m{10em}|m{10em}|m{10em}|}
\hline
 \rowcolor[gray]{0.8}
Variable/Modelo & Real & tddpm\_mlp & smote-enc & ctgan \\
\hline top5 & [3 4 2 5 6] & [3 4 2 5 6] & [3 4 2 5 1] & [4 3 2 5 6] \\
\hline top5\_freq & [7865 5477 2237 1292  212] & [10728  6929  2646  1132    86] & [11430  7061  2430   621    47] & [8624 4895 4079 1578  679] \\
\hline top5\_prob & [0.45488722 0.3167727  0.12938115 0.07472527 0.01226142] & [0.49636793 0.32059409 0.12242632 0.05237588 0.00397909] & [0.52882391 0.32668641 0.11242713 0.02873138 0.00217452] & [0.39901911 0.22648406 0.18872901 0.07301161 0.03141628] \\
\hline nobs & 17290 & \bfseries 21613 & \cellcolor[rgb]{0.9, 0.54, 0.52} 21614 & \bfseries 21613 \\
\hline missing & 0.000 & 0.000 & 0.000 & 0.000 \\
\hline mean & 3.368 & \bfseries 3.307 & 3.271 & \cellcolor[rgb]{0.9, 0.54, 0.52} 3.975 \\
\hline std\_err & 0.007 & \bfseries 0.005 & 0.005 & \cellcolor[rgb]{0.9, 0.54, 0.52} 0.025 \\
\hline upper\_ci & 3.382 & \bfseries 3.318 & 3.280 & \cellcolor[rgb]{0.9, 0.54, 0.52} 4.025 \\
\hline lower\_ci & 3.354 & \bfseries 3.297 & 3.261 & \cellcolor[rgb]{0.9, 0.54, 0.52} 3.925 \\
\hline std & 0.931 & \bfseries 0.785 & 0.707 & \cellcolor[rgb]{0.9, 0.54, 0.52} 3.732 \\
\hline iqr & 1.000 & 1.000 & 1.000 & 1.000 \\
\hline iqr\_normal & 0.741 & 0.741 & 0.741 & 0.741 \\
\hline mad & 0.734 & \bfseries 0.645 & 0.582 & \cellcolor[rgb]{0.9, 0.54, 0.52} 1.395 \\
\hline mad\_normal & 0.920 & \bfseries 0.808 & 0.730 & \cellcolor[rgb]{0.9, 0.54, 0.52} 1.748 \\
\hline coef\_var & 0.277 & \bfseries 0.237 & 0.216 & \cellcolor[rgb]{0.9, 0.54, 0.52} 0.939 \\
\hline range & 33.000 & 9.000 & \cellcolor[rgb]{0.9, 0.54, 0.52} 5.000 & \bfseries 33.000 \\
\hline max & 33.000 & 9.000 & \cellcolor[rgb]{0.9, 0.54, 0.52} 6.000 & \bfseries 33.000 \\
\hline min & 0.000 & \bfseries 0.000 & \cellcolor[rgb]{0.9, 0.54, 0.52} 1.000 & \bfseries 0.000 \\
\hline skew & 2.304 & \bfseries 0.243 & 0.091 & \cellcolor[rgb]{0.9, 0.54, 0.52} 6.612 \\
\hline kurtosis & 63.268 & 3.485 & \cellcolor[rgb]{0.9, 0.54, 0.52} 3.072 & \bfseries 51.539 \\
\hline jarque\_bera & 2631992 & 424 & \cellcolor[rgb]{0.9, 0.54, 0.52} 34 & \bfseries 2279187 \\
\hline jarque\_bera\_pval & 0.000 & 0.000 & \cellcolor[rgb]{0.9, 0.54, 0.52} 0.000 & \bfseries 0.000 \\
\hline mode & 3.000 & \bfseries 3.000 & \bfseries 3.000 & \cellcolor[rgb]{0.9, 0.54, 0.52} 4.000 \\
\hline mode\_freq & 0.455 & \bfseries 0.496 & \cellcolor[rgb]{0.9, 0.54, 0.52} 0.529 & 0.399 \\
\hline median & 3.000 & \bfseries 3.000 & \bfseries 3.000 & \cellcolor[rgb]{0.9, 0.54, 0.52} 4.000 \\
\hline 0.1\% & 1.000 & \bfseries 1.000 & \bfseries 1.000 & \cellcolor[rgb]{0.9, 0.54, 0.52} 0.000 \\
\hline 1.0\% & 2.000 & \bfseries 2.000 & \bfseries 2.000 & \cellcolor[rgb]{0.9, 0.54, 0.52} 0.000 \\
\hline 5.0\% & 2.000 & 2.000 & 2.000 & 2.000 \\
\hline 25.0\% & 3.000 & 3.000 & 3.000 & 3.000 \\
\hline 75.0\% & 4.000 & 4.000 & 4.000 & 4.000 \\
\hline 95.0\% & 5.000 & \bfseries 5.000 & 4.000 & \cellcolor[rgb]{0.9, 0.54, 0.52} 7.000 \\
\hline 99.0\% & 6.000 & \bfseries 5.000 & \bfseries 5.000 & \cellcolor[rgb]{0.9, 0.54, 0.52} 33.000 \\
\hline 99.9\% & 7.000 & \bfseries 6.000 & \bfseries 6.000 & \cellcolor[rgb]{0.9, 0.54, 0.52} 33.000 \\
\hline
\end{tabular}
\end{table}

\begin{table}[H]
\centering
\fontsize{8}{14}\selectfont
\caption{Propiedades  estadisticas de variable zipcode, King county (A-2)}
\label{table-stats-king county-a-2-zipcode}
\begin{tabular}{|l|m{10em}|m{10em}|m{10em}|m{10em}|}
\hline
 \rowcolor[gray]{0.8}
Variable/Modelo & Real & tddpm\_mlp & smote-enc & ctgan \\
\hline top5 & [98103 98038 98115 98052 98117] & [98052 98103 98038 98115 98042] & [98103 98115 98052 98038 98117] & [98034 98118 98006 98023 98103] \\
\hline top5\_freq & [489 473 462 459 455] & [714 658 645 634 601] & [650 623 574 570 557] & [804 711 638 630 612] \\
\hline top5\_prob & [0.02828224 0.02735685 0.02672065 0.02654714 0.02631579] & [0.03303567 0.03044464 0.02984315 0.0293342  0.02780734] & [0.0300731  0.02882391 0.02655686 0.0263718  0.02577033] & [0.03719983 0.03289687 0.02951927 0.02914912 0.02831629] \\
\hline nobs & 17290 & \bfseries 21613 & \cellcolor[rgb]{0.9, 0.54, 0.52} 21614 & \bfseries 21613 \\
\hline missing & 0.000 & 0.000 & 0.000 & 0.000 \\
\hline mean & 98078 & 98077 & \bfseries 98078 & \cellcolor[rgb]{0.9, 0.54, 0.52} 98079 \\
\hline std\_err & 0.406 & \cellcolor[rgb]{0.9, 0.54, 0.52} 0.360 & 0.363 & \bfseries 0.370 \\
\hline upper\_ci & 98079 & 98078 & \bfseries 98079 & \cellcolor[rgb]{0.9, 0.54, 0.52} 98080 \\
\hline lower\_ci & 98077 & 98077 & \bfseries 98077 & \cellcolor[rgb]{0.9, 0.54, 0.52} 98078 \\
\hline std & 53.326 & 52.955 & \bfseries 53.304 & \cellcolor[rgb]{0.9, 0.54, 0.52} 54.353 \\
\hline iqr & 84.000 & \bfseries 84.000 & \bfseries 84.000 & \cellcolor[rgb]{0.9, 0.54, 0.52} 85.000 \\
\hline iqr\_normal & 62.269 & \bfseries 62.269 & \bfseries 62.269 & \cellcolor[rgb]{0.9, 0.54, 0.52} 63.011 \\
\hline mad & 46.554 & 46.116 & \bfseries 46.585 & \cellcolor[rgb]{0.9, 0.54, 0.52} 47.661 \\
\hline mad\_normal & 58.347 & 57.798 & \bfseries 58.385 & \cellcolor[rgb]{0.9, 0.54, 0.52} 59.734 \\
\hline coef\_var & 0.001 & 0.001 & \bfseries 0.001 & \cellcolor[rgb]{0.9, 0.54, 0.52} 0.001 \\
\hline range & 198.000 & 198.000 & 198.000 & 198.000 \\
\hline max & 98199 & 98199 & 98199 & 98199 \\
\hline min & 98001 & 98001 & 98001 & 98001 \\
\hline skew & 0.402 & \cellcolor[rgb]{0.9, 0.54, 0.52} 0.423 & \bfseries 0.390 & 0.417 \\
\hline kurtosis & 2.153 & 2.182 & \bfseries 2.142 & \cellcolor[rgb]{0.9, 0.54, 0.52} 2.101 \\
\hline jarque\_bera & 983.027 & 1246.547 & \bfseries 1210.461 & \cellcolor[rgb]{0.9, 0.54, 0.52} 1354.816 \\
\hline jarque\_bera\_pval & 0.000 & 0.000 & 0.000 & 0.000 \\
\hline mode & 98103 & 98052 & \bfseries 98103 & \cellcolor[rgb]{0.9, 0.54, 0.52} 98034 \\
\hline mode\_freq & 0.028 & 0.033 & \bfseries 0.030 & \cellcolor[rgb]{0.9, 0.54, 0.52} 0.037 \\
\hline median & 98065 & \bfseries 98065 & \cellcolor[rgb]{0.9, 0.54, 0.52} 98070 & \cellcolor[rgb]{0.9, 0.54, 0.52} 98070 \\
\hline 0.1\% & 98001 & 98001 & 98001 & 98001 \\
\hline 1.0\% & 98001 & 98001 & 98001 & 98001 \\
\hline 5.0\% & 98004 & \bfseries 98004 & \bfseries 98004 & \cellcolor[rgb]{0.9, 0.54, 0.52} 98005 \\
\hline 25.0\% & 98033 & 98033 & 98033 & 98033 \\
\hline 75.0\% & 98117 & \bfseries 98117 & \bfseries 98117 & \cellcolor[rgb]{0.9, 0.54, 0.52} 98118 \\
\hline 95.0\% & 98177 & \bfseries 98177 & \bfseries 98177 & \cellcolor[rgb]{0.9, 0.54, 0.52} 98178 \\
\hline 99.0\% & 98199 & 98199 & 98199 & 98199 \\
\hline 99.9\% & 98199 & 98199 & 98199 & 98199 \\
\hline
\end{tabular}
\end{table}

\begin{table}[H]
\centering
\fontsize{8}{14}\selectfont
\caption{Propiedades  estadisticas de variable bathrooms, King county (A-2)}
\label{table-stats-king county-a-2-bathrooms}
\begin{tabular}{|l|m{10em}|m{10em}|m{10em}|m{10em}|}
\hline
 \rowcolor[gray]{0.8}
Variable/Modelo & Real & tddpm\_mlp & smote-enc & ctgan \\
\hline top5 & [2.5  1.   1.75 2.25 2.  ] & [2.5  1.   1.75 2.25 2.  ] & [2.5  1.   1.75 2.25 2.  ] & [2.5  1.75 1.   2.75 3.25] \\
\hline top5\_freq & [4333 3088 2425 1621 1526] & [6256 3998 3308 2019 1764] & [6751 4830 3303 1979 1320] & [4571 3423 2963 2032 1228] \\
\hline top5\_prob & [0.25060729 0.17860035 0.14025448 0.09375361 0.08825911] & [0.28945542 0.18498126 0.15305603 0.093416   0.08161754] & [0.31234385 0.22346627 0.15281762 0.09156103 0.06107153] & [0.21149308 0.1583769  0.13709342 0.09401749 0.05681766] \\
\hline nobs & 17290 & \bfseries 21613 & \cellcolor[rgb]{0.9, 0.54, 0.52} 21614 & \bfseries 21613 \\
\hline missing & 0.000 & 0.000 & 0.000 & 0.000 \\
\hline mean & 2.114 & \bfseries 2.080 & 2.011 & \cellcolor[rgb]{0.9, 0.54, 0.52} 2.291 \\
\hline std\_err & 0.006 & 0.005 & \cellcolor[rgb]{0.9, 0.54, 0.52} 0.005 & \bfseries 0.006 \\
\hline upper\_ci & 2.125 & \bfseries 2.089 & 2.020 & \cellcolor[rgb]{0.9, 0.54, 0.52} 2.304 \\
\hline lower\_ci & 2.102 & \bfseries 2.070 & 2.001 & \cellcolor[rgb]{0.9, 0.54, 0.52} 2.279 \\
\hline std & 0.767 & \bfseries 0.713 & 0.706 & \cellcolor[rgb]{0.9, 0.54, 0.52} 0.916 \\
\hline iqr & 1.000 & \cellcolor[rgb]{0.9, 0.54, 0.52} 0.750 & \bfseries 1.000 & \bfseries 1.000 \\
\hline iqr\_normal & 0.741 & \cellcolor[rgb]{0.9, 0.54, 0.52} 0.556 & \bfseries 0.741 & \bfseries 0.741 \\
\hline mad & 0.615 & 0.582 & \bfseries 0.592 & \cellcolor[rgb]{0.9, 0.54, 0.52} 0.728 \\
\hline mad\_normal & 0.771 & 0.730 & \bfseries 0.742 & \cellcolor[rgb]{0.9, 0.54, 0.52} 0.913 \\
\hline coef\_var & 0.363 & 0.343 & \bfseries 0.351 & \cellcolor[rgb]{0.9, 0.54, 0.52} 0.400 \\
\hline range & 8.000 & 7.750 & \cellcolor[rgb]{0.9, 0.54, 0.52} 4.750 & \bfseries 8.000 \\
\hline max & 8.000 & 7.750 & \cellcolor[rgb]{0.9, 0.54, 0.52} 5.500 & \bfseries 8.000 \\
\hline min & 0.000 & \bfseries 0.000 & \cellcolor[rgb]{0.9, 0.54, 0.52} 0.750 & \bfseries 0.000 \\
\hline skew & 0.464 & 0.234 & \cellcolor[rgb]{0.9, 0.54, 0.52} 0.146 & \bfseries 0.441 \\
\hline kurtosis & 3.989 & 3.439 & \cellcolor[rgb]{0.9, 0.54, 0.52} 2.843 & \bfseries 3.888 \\
\hline jarque\_bera & 1326 & 371 & \cellcolor[rgb]{0.9, 0.54, 0.52} 99 & \bfseries 1409 \\
\hline jarque\_bera\_pval & 0.000 & 0.000 & \cellcolor[rgb]{0.9, 0.54, 0.52} 0.000 & \bfseries 0.000 \\
\hline mode & 2.500 & 2.500 & 2.500 & 2.500 \\
\hline mode\_freq & 0.251 & \bfseries 0.289 & \cellcolor[rgb]{0.9, 0.54, 0.52} 0.312 & 0.211 \\
\hline median & 2.250 & \bfseries 2.250 & \bfseries 2.250 & \cellcolor[rgb]{0.9, 0.54, 0.52} 2.500 \\
\hline 0.1\% & 0.750 & \bfseries 1.000 & \bfseries 1.000 & \cellcolor[rgb]{0.9, 0.54, 0.52} 0.000 \\
\hline 1.0\% & 1.000 & \bfseries 1.000 & \bfseries 1.000 & \cellcolor[rgb]{0.9, 0.54, 0.52} 0.750 \\
\hline 5.0\% & 1.000 & 1.000 & 1.000 & 1.000 \\
\hline 25.0\% & 1.500 & \cellcolor[rgb]{0.9, 0.54, 0.52} 1.750 & \bfseries 1.500 & \cellcolor[rgb]{0.9, 0.54, 0.52} 1.750 \\
\hline 75.0\% & 2.500 & \bfseries 2.500 & \bfseries 2.500 & \cellcolor[rgb]{0.9, 0.54, 0.52} 2.750 \\
\hline 95.0\% & 3.500 & \bfseries 3.250 & \bfseries 3.250 & \cellcolor[rgb]{0.9, 0.54, 0.52} 4.000 \\
\hline 99.0\% & 4.250 & \bfseries 3.750 & \cellcolor[rgb]{0.9, 0.54, 0.52} 3.500 & \bfseries 4.750 \\
\hline 99.9\% & 5.428 & \bfseries 5.000 & 4.750 & \cellcolor[rgb]{0.9, 0.54, 0.52} 6.250 \\
\hline
\end{tabular}
\end{table}

\begin{table}[H]
\centering
\fontsize{8}{14}\selectfont
\caption{Propiedades  estadisticas de variable sqft\_living15, King county (A-2)}
\label{table-stats-king county-a-2-sqft_living15}
\begin{tabular}{|l|m{10em}|m{10em}|m{10em}|m{10em}|}
\hline
 \rowcolor[gray]{0.8}
Variable/Modelo & Real & tddpm\_mlp & smote-enc & ctgan \\
\hline top5 & [1440 1540 1560 1500 1610] & [1440. 1550. 1540. 1560. 1530.] & [1440. 1830. 1670. 1078. 2370.] & [1827 1696 1594 1790 1621] \\
\hline top5\_freq & [156 154 152 137 136] & [200 183 165 162 162] & [24 22 21 18 17] & [28 26 25 25 24] \\
\hline top5\_prob & [0.00902256 0.00890688 0.00879121 0.00792366 0.00786582] & [0.00925369 0.00846713 0.00763429 0.00749549 0.00749549] & [0.00111039 0.00101786 0.00097159 0.00083279 0.00078653] & [0.00129552 0.00120298 0.00115671 0.00115671 0.00111044] \\
\hline nobs & 17290 & \bfseries 21613 & \cellcolor[rgb]{0.9, 0.54, 0.52} 21614 & \bfseries 21613 \\
\hline missing & 0.000 & 0.000 & 0.000 & 0.000 \\
\hline mean & 1983 & 1971 & \bfseries 1974 & \cellcolor[rgb]{0.9, 0.54, 0.52} 1952 \\
\hline std\_err & 5.181 & \cellcolor[rgb]{0.9, 0.54, 0.52} 4.381 & 4.422 & \bfseries 5.044 \\
\hline upper\_ci & 1993 & 1980 & \bfseries 1982 & \cellcolor[rgb]{0.9, 0.54, 0.52} 1962 \\
\hline lower\_ci & 1973 & 1962 & \bfseries 1965 & \cellcolor[rgb]{0.9, 0.54, 0.52} 1942 \\
\hline std & 681.232 & 644.021 & \bfseries 650.115 & \cellcolor[rgb]{0.9, 0.54, 0.52} 741.571 \\
\hline iqr & 880.000 & \cellcolor[rgb]{0.9, 0.54, 0.52} 827.821 & 832.491 & \bfseries 853.000 \\
\hline iqr\_normal & 652.345 & \cellcolor[rgb]{0.9, 0.54, 0.52} 613.665 & 617.127 & \bfseries 632.330 \\
\hline mad & 533.237 & \cellcolor[rgb]{0.9, 0.54, 0.52} 504.400 & \bfseries 509.430 & 557.482 \\
\hline mad\_normal & 668.313 & \cellcolor[rgb]{0.9, 0.54, 0.52} 632.171 & \bfseries 638.476 & 698.700 \\
\hline coef\_var & 0.344 & 0.327 & \bfseries 0.329 & \cellcolor[rgb]{0.9, 0.54, 0.52} 0.380 \\
\hline range & 5811 & 5811 & \cellcolor[rgb]{0.9, 0.54, 0.52} 5320 & \bfseries 5811 \\
\hline max & 6210 & 6210 & \cellcolor[rgb]{0.9, 0.54, 0.52} 5993 & \bfseries 6210 \\
\hline min & 399.000 & \bfseries 399.000 & \cellcolor[rgb]{0.9, 0.54, 0.52} 672.953 & \bfseries 399.000 \\
\hline skew & 1.095 & \cellcolor[rgb]{0.9, 0.54, 0.52} 1.067 & \bfseries 1.083 & 1.083 \\
\hline kurtosis & 4.572 & \bfseries 4.604 & 4.443 & \cellcolor[rgb]{0.9, 0.54, 0.52} 5.613 \\
\hline jarque\_bera & 5237 & 6416 & \bfseries 6104 & \cellcolor[rgb]{0.9, 0.54, 0.52} 10375 \\
\hline jarque\_bera\_pval & 0.000 & 0.000 & 0.000 & 0.000 \\
\hline mode & 1440 & \bfseries 1440 & 1440 & \cellcolor[rgb]{0.9, 0.54, 0.52} 1827 \\
\hline mode\_freq & 0.009 & \bfseries 0.009 & \cellcolor[rgb]{0.9, 0.54, 0.52} 0.001 & 0.001 \\
\hline median & 1840 & 1830 & \bfseries 1835 & \cellcolor[rgb]{0.9, 0.54, 0.52} 1859 \\
\hline 0.1\% & 740.000 & \cellcolor[rgb]{0.9, 0.54, 0.52} 399.000 & \bfseries 791.697 & 468.060 \\
\hline 1.0\% & 950.000 & \bfseries 980.000 & 988.057 & \cellcolor[rgb]{0.9, 0.54, 0.52} 648.120 \\
\hline 5.0\% & 1140 & 1179 & \bfseries 1166 & \cellcolor[rgb]{0.9, 0.54, 0.52} 899 \\
\hline 25.0\% & 1480 & \cellcolor[rgb]{0.9, 0.54, 0.52} 1502 & 1495 & \bfseries 1469 \\
\hline 75.0\% & 2360 & \bfseries 2330 & 2328 & \cellcolor[rgb]{0.9, 0.54, 0.52} 2322 \\
\hline 95.0\% & 3280 & \cellcolor[rgb]{0.9, 0.54, 0.52} 3204 & 3222 & \bfseries 3307 \\
\hline 99.0\% & 4050 & 3868 & \bfseries 3939 & \cellcolor[rgb]{0.9, 0.54, 0.52} 4270 \\
\hline 99.9\% & 4986 & \bfseries 4889 & 4802 & \cellcolor[rgb]{0.9, 0.54, 0.52} 5989 \\
\hline
\end{tabular}
\end{table}

\begin{table}[H]
\centering
\fontsize{8}{14}\selectfont
\caption{Propiedades  estadisticas de variable price, King county (A-2)}
\label{table-stats-king county-a-2-price}
\begin{tabular}{|l|m{10em}|m{10em}|m{10em}|m{10em}|}
\hline
 \rowcolor[gray]{0.8}
Variable/Modelo & Real & tddpm\_mlp & smote-enc & ctgan \\
\hline top5 & [350000. 450000. 425000. 550000. 325000.] & [450000. 525000. 300000. 350000. 550000.] & [350000. 450000. 325000. 550000. 500000.] & [ 75000. 347056. 209754. 370839. 905913.] \\
\hline top5\_freq & [143 140 123 123 123] & [190 138 135 135 134] & [213 191 186 184 182] & [602   3   3   2   2] \\
\hline top5\_prob & [0.00827068 0.00809717 0.00711394 0.00711394 0.00711394] & [0.00879101 0.00638505 0.00624624 0.00624624 0.00619997] & [0.00985472 0.00883686 0.00860553 0.008513   0.00842047] & [2.78536066e-02 1.38805349e-04 1.38805349e-04 9.25368991e-05
 9.25368991e-05] \\
\hline nobs & 17290 & \bfseries 21613 & \cellcolor[rgb]{0.9, 0.54, 0.52} 21614 & \bfseries 21613 \\
\hline missing & 0.000 & 0.000 & 0.000 & 0.000 \\
\hline mean & 537768 & \bfseries 528328 & 519642 & \cellcolor[rgb]{0.9, 0.54, 0.52} 498555 \\
\hline std\_err & 2749 & 2828 & \cellcolor[rgb]{0.9, 0.54, 0.52} 2097 & \bfseries 2784 \\
\hline upper\_ci & 543156 & \bfseries 533872 & 523753 & \cellcolor[rgb]{0.9, 0.54, 0.52} 504011 \\
\hline lower\_ci & 532380 & \bfseries 522785 & 515532 & \cellcolor[rgb]{0.9, 0.54, 0.52} 493099 \\
\hline std & 361464 & \cellcolor[rgb]{0.9, 0.54, 0.52} 415801 & 308334 & \bfseries 409263 \\
\hline iqr & 319850 & 302163 & \bfseries 305500 & \cellcolor[rgb]{0.9, 0.54, 0.52} 359276 \\
\hline iqr\_normal & 237105 & 223994 & \bfseries 226467 & \cellcolor[rgb]{0.9, 0.54, 0.52} 266332 \\
\hline mad & 231680 & \bfseries 218397 & 210266 & \cellcolor[rgb]{0.9, 0.54, 0.52} 265501 \\
\hline mad\_normal & 290368 & \bfseries 273720 & 263529 & \cellcolor[rgb]{0.9, 0.54, 0.52} 332756 \\
\hline coef\_var & 0.672 & 0.787 & \bfseries 0.593 & \cellcolor[rgb]{0.9, 0.54, 0.52} 0.821 \\
\hline range & 7625000 & \bfseries 7625000 & \cellcolor[rgb]{0.9, 0.54, 0.52} 4130000 & 4404475 \\
\hline max & 7700000 & \bfseries 7700000 & \cellcolor[rgb]{0.9, 0.54, 0.52} 4208000 & 4479475 \\
\hline min & 75000 & \bfseries 75000 & \cellcolor[rgb]{0.9, 0.54, 0.52} 78000 & \bfseries 75000 \\
\hline skew & 4.032 & \cellcolor[rgb]{0.9, 0.54, 0.52} 9.124 & 2.762 & \bfseries 2.979 \\
\hline kurtosis & 39.678 & \cellcolor[rgb]{0.9, 0.54, 0.52} 141.899 & \bfseries 16.937 & 16.856 \\
\hline jarque\_bera & 1016020 & \cellcolor[rgb]{0.9, 0.54, 0.52} 17674040 & 202399 & \bfseries 204867 \\
\hline jarque\_bera\_pval & 0.000 & 0.000 & 0.000 & 0.000 \\
\hline mode & 350000 & 450000 & \bfseries 350000 & \cellcolor[rgb]{0.9, 0.54, 0.52} 75000 \\
\hline mode\_freq & 0.008 & \bfseries 0.009 & 0.010 & \cellcolor[rgb]{0.9, 0.54, 0.52} 0.028 \\
\hline median & 450000 & \bfseries 450000 & 450500 & \cellcolor[rgb]{0.9, 0.54, 0.52} 408171 \\
\hline 0.1\% & 95000 & 94193 & \bfseries 95000 & \cellcolor[rgb]{0.9, 0.54, 0.52} 75000 \\
\hline 1.0\% & 154467 & 161482 & \bfseries 158000 & \cellcolor[rgb]{0.9, 0.54, 0.52} 75000 \\
\hline 5.0\% & 210000 & 215000 & \bfseries 210000 & \cellcolor[rgb]{0.9, 0.54, 0.52} 100410 \\
\hline 25.0\% & 320150 & 322837 & \bfseries 319500 & \cellcolor[rgb]{0.9, 0.54, 0.52} 250152 \\
\hline 75.0\% & 640000 & \bfseries 625000 & \bfseries 625000 & \cellcolor[rgb]{0.9, 0.54, 0.52} 609428 \\
\hline 95.0\% & 1150000 & \cellcolor[rgb]{0.9, 0.54, 0.52} 1003061 & 1037350 & \bfseries 1209711 \\
\hline 99.0\% & 1950000 & \bfseries 1732519 & 1699856 & \cellcolor[rgb]{0.9, 0.54, 0.52} 2281066 \\
\hline 99.9\% & 3331995 & \cellcolor[rgb]{0.9, 0.54, 0.52} 7700000 & 2950000 & \bfseries 3517622 \\
\hline
\end{tabular}
\end{table}

\begin{table}[H]
\centering
\fontsize{8}{14}\selectfont
\caption{Propiedades  estadisticas de variable long, King county (A-2)}
\label{table-stats-king county-a-2-long}
\begin{tabular}{|l|m{10em}|m{10em}|m{10em}|m{10em}|}
\hline
 \rowcolor[gray]{0.8}
Variable/Modelo & Real & tddpm\_mlp & smote-enc & ctgan \\
\hline top5 & [-122.29  -122.362 -122.3   -122.372 -122.284] & [-122.3   -122.307 -122.301 -122.29  -122.299] & [-122.29  -122.34  -122.387 -122.375 -122.244] & [-122.519 -122.28  -122.332 -122.346 -122.323] \\
\hline top5\_freq & [100  88  81  81  81] & [137  84  84  83  81] & [25 21 21 20 18] & [108  98  86  83  81] \\
\hline top5\_prob & [0.00578369 0.00508965 0.00468479 0.00468479 0.00468479] & [0.00633878 0.00388655 0.00388655 0.00384028 0.00374774] & [0.00115666 0.00097159 0.00097159 0.00092533 0.00083279] & [0.00499699 0.00453431 0.00397909 0.00384028 0.00374774] \\
\hline nobs & 17290 & \bfseries 21613 & \cellcolor[rgb]{0.9, 0.54, 0.52} 21614 & \bfseries 21613 \\
\hline missing & 0.000 & 0.000 & 0.000 & 0.000 \\
\hline mean & -122.214 & \bfseries -122.213 & -122.216 & \cellcolor[rgb]{0.9, 0.54, 0.52} -122.211 \\
\hline std\_err & 0.001 & \cellcolor[rgb]{0.9, 0.54, 0.52} 0.001 & 0.001 & \bfseries 0.001 \\
\hline upper\_ci & -122.212 & \bfseries -122.211 & -122.214 & \cellcolor[rgb]{0.9, 0.54, 0.52} -122.209 \\
\hline lower\_ci & -122.216 & \bfseries -122.215 & -122.218 & \cellcolor[rgb]{0.9, 0.54, 0.52} -122.213 \\
\hline std & 0.140 & 0.136 & \bfseries 0.139 & \cellcolor[rgb]{0.9, 0.54, 0.52} 0.146 \\
\hline iqr & 0.204 & \cellcolor[rgb]{0.9, 0.54, 0.52} 0.196 & \bfseries 0.201 & 0.209 \\
\hline iqr\_normal & 0.151 & \cellcolor[rgb]{0.9, 0.54, 0.52} 0.145 & \bfseries 0.149 & 0.155 \\
\hline mad & 0.115 & \cellcolor[rgb]{0.9, 0.54, 0.52} 0.111 & \bfseries 0.115 & 0.119 \\
\hline mad\_normal & 0.144 & \cellcolor[rgb]{0.9, 0.54, 0.52} 0.140 & \bfseries 0.144 & 0.149 \\
\hline coef\_var & -0.001 & -0.001 & \bfseries -0.001 & \cellcolor[rgb]{0.9, 0.54, 0.52} -0.001 \\
\hline range & 1.204 & \bfseries 1.204 & 1.193 & \cellcolor[rgb]{0.9, 0.54, 0.52} 0.959 \\
\hline max & -121.315 & \bfseries -121.315 & -121.318 & \cellcolor[rgb]{0.9, 0.54, 0.52} -121.560 \\
\hline min & -122.519 & -122.519 & \cellcolor[rgb]{0.9, 0.54, 0.52} -122.511 & \bfseries -122.519 \\
\hline skew & 0.867 & 0.937 & \bfseries 0.844 & \cellcolor[rgb]{0.9, 0.54, 0.52} 0.754 \\
\hline kurtosis & 3.953 & \cellcolor[rgb]{0.9, 0.54, 0.52} 4.863 & \bfseries 3.780 & 3.753 \\
\hline jarque\_bera & 2819 & \cellcolor[rgb]{0.9, 0.54, 0.52} 6287 & 3116 & \bfseries 2558 \\
\hline jarque\_bera\_pval & 0.000 & 0.000 & 0.000 & 0.000 \\
\hline mode & -122.290 & -122.300 & \bfseries -122.290 & \cellcolor[rgb]{0.9, 0.54, 0.52} -122.519 \\
\hline mode\_freq & 0.006 & \bfseries 0.006 & \cellcolor[rgb]{0.9, 0.54, 0.52} 0.001 & 0.005 \\
\hline median & -122.231 & \cellcolor[rgb]{0.9, 0.54, 0.52} -122.221 & \bfseries -122.235 & -122.239 \\
\hline 0.1\% & -122.497 & \cellcolor[rgb]{0.9, 0.54, 0.52} -122.469 & \bfseries -122.483 & -122.519 \\
\hline 1.0\% & -122.408 & -122.399 & \bfseries -122.404 & \cellcolor[rgb]{0.9, 0.54, 0.52} -122.464 \\
\hline 5.0\% & -122.387 & -122.385 & \bfseries -122.386 & \cellcolor[rgb]{0.9, 0.54, 0.52} -122.406 \\
\hline 25.0\% & -122.329 & -122.322 & \bfseries -122.328 & \cellcolor[rgb]{0.9, 0.54, 0.52} -122.322 \\
\hline 75.0\% & -122.125 & \bfseries -122.126 & -122.128 & \cellcolor[rgb]{0.9, 0.54, 0.52} -122.113 \\
\hline 95.0\% & -121.979 & \cellcolor[rgb]{0.9, 0.54, 0.52} -121.996 & \bfseries -121.983 & -121.964 \\
\hline 99.0\% & -121.787 & \cellcolor[rgb]{0.9, 0.54, 0.52} -121.830 & -121.817 & \bfseries -121.758 \\
\hline 99.9\% & -121.699 & \cellcolor[rgb]{0.9, 0.54, 0.52} -121.469 & \bfseries -121.704 & -121.622 \\
\hline
\end{tabular}
\end{table}

\begin{table}[H]
\centering
\fontsize{8}{14}\selectfont
\caption{Propiedades  estadisticas de variable grade, King county (A-2)}
\label{table-stats-king county-a-2-grade}
\begin{tabular}{|l|m{10em}|m{10em}|m{10em}|m{10em}|}
\hline
 \rowcolor[gray]{0.8}
Variable/Modelo & Real & tddpm\_mlp & smote-enc & ctgan \\
\hline top5 & [ 7  8  9  6 10] & [ 7  8  9  6 10] & [ 7  8  9  6 10] & [ 8  7  9  6 10] \\
\hline top5\_freq & [7201 4879 2072 1620  915] & [9511 6272 2597 1701 1048] & [9815 6178 2584 1697  987] & [6536 3941 3726 3091 1408] \\
\hline top5\_prob & [0.41648352 0.28218623 0.11983806 0.09369578 0.05292076] & [0.44005922 0.29019572 0.12015916 0.07870263 0.04848934] & [0.45410382 0.28583326 0.11955214 0.07851393 0.04566485] & [0.30241059 0.18234396 0.17239624 0.14301578 0.06514598] \\
\hline nobs & 17290 & \bfseries 21613 & \cellcolor[rgb]{0.9, 0.54, 0.52} 21614 & \bfseries 21613 \\
\hline missing & 0.000 & 0.000 & 0.000 & 0.000 \\
\hline mean & 7.654 & \bfseries 7.650 & 7.621 & \cellcolor[rgb]{0.9, 0.54, 0.52} 7.929 \\
\hline std\_err & 0.009 & \bfseries 0.007 & 0.007 & \cellcolor[rgb]{0.9, 0.54, 0.52} 0.012 \\
\hline upper\_ci & 7.671 & \bfseries 7.664 & 7.635 & \cellcolor[rgb]{0.9, 0.54, 0.52} 7.952 \\
\hline lower\_ci & 7.636 & \bfseries 7.635 & 7.607 & \cellcolor[rgb]{0.9, 0.54, 0.52} 7.906 \\
\hline std & 1.170 & \bfseries 1.082 & 1.042 & \cellcolor[rgb]{0.9, 0.54, 0.52} 1.736 \\
\hline iqr & 1.000 & \bfseries 1.000 & \bfseries 1.000 & \cellcolor[rgb]{0.9, 0.54, 0.52} 2.000 \\
\hline iqr\_normal & 0.741 & \bfseries 0.741 & \bfseries 0.741 & \cellcolor[rgb]{0.9, 0.54, 0.52} 1.483 \\
\hline mad & 0.926 & \bfseries 0.867 & 0.844 & \cellcolor[rgb]{0.9, 0.54, 0.52} 1.275 \\
\hline mad\_normal & 1.160 & \bfseries 1.087 & 1.058 & \cellcolor[rgb]{0.9, 0.54, 0.52} 1.598 \\
\hline coef\_var & 0.153 & \bfseries 0.141 & 0.137 & \cellcolor[rgb]{0.9, 0.54, 0.52} 0.219 \\
\hline range & 12.000 & 9.000 & \cellcolor[rgb]{0.9, 0.54, 0.52} 8.000 & \bfseries 12.000 \\
\hline max & 13.000 & \bfseries 13.000 & \cellcolor[rgb]{0.9, 0.54, 0.52} 12.000 & \bfseries 13.000 \\
\hline min & 1.000 & \cellcolor[rgb]{0.9, 0.54, 0.52} 4.000 & \cellcolor[rgb]{0.9, 0.54, 0.52} 4.000 & \bfseries 1.000 \\
\hline skew & 0.758 & 0.820 & \bfseries 0.812 & \cellcolor[rgb]{0.9, 0.54, 0.52} 0.213 \\
\hline kurtosis & 4.209 & \bfseries 4.083 & 3.918 & \cellcolor[rgb]{0.9, 0.54, 0.52} 3.654 \\
\hline jarque\_bera & 2709 & 3479 & \bfseries 3133 & \cellcolor[rgb]{0.9, 0.54, 0.52} 549 \\
\hline jarque\_bera\_pval & 0.000 & \bfseries 0.000 & \bfseries 0.000 & \cellcolor[rgb]{0.9, 0.54, 0.52} 0.000 \\
\hline mode & 7.000 & \bfseries 7.000 & \bfseries 7.000 & \cellcolor[rgb]{0.9, 0.54, 0.52} 8.000 \\
\hline mode\_freq & 0.416 & \bfseries 0.440 & 0.454 & \cellcolor[rgb]{0.9, 0.54, 0.52} 0.302 \\
\hline median & 7.000 & \bfseries 7.000 & \bfseries 7.000 & \cellcolor[rgb]{0.9, 0.54, 0.52} 8.000 \\
\hline 0.1\% & 4.000 & 5.000 & 5.000 & 3.000 \\
\hline 1.0\% & 5.000 & 6.000 & 6.000 & 4.000 \\
\hline 5.0\% & 6.000 & \bfseries 6.000 & \bfseries 6.000 & \cellcolor[rgb]{0.9, 0.54, 0.52} 5.000 \\
\hline 25.0\% & 7.000 & 7.000 & 7.000 & 7.000 \\
\hline 75.0\% & 8.000 & \bfseries 8.000 & \bfseries 8.000 & \cellcolor[rgb]{0.9, 0.54, 0.52} 9.000 \\
\hline 95.0\% & 10.000 & \bfseries 10.000 & \bfseries 10.000 & \cellcolor[rgb]{0.9, 0.54, 0.52} 11.000 \\
\hline 99.0\% & 11.000 & \bfseries 11.000 & \bfseries 11.000 & \cellcolor[rgb]{0.9, 0.54, 0.52} 12.000 \\
\hline 99.9\% & 12.000 & \bfseries 12.000 & \bfseries 12.000 & \cellcolor[rgb]{0.9, 0.54, 0.52} 13.000 \\
\hline
\end{tabular}
\end{table}

\begin{table}[H]
\centering
\fontsize{8}{14}\selectfont
\caption{Propiedades  estadisticas de variable condition, King county (A-2)}
\label{table-stats-king county-a-2-condition}
\begin{tabular}{|l|m{10em}|m{10em}|m{10em}|m{10em}|}
\hline
 \rowcolor[gray]{0.8}
Variable/Modelo & Real & tddpm\_mlp & smote-enc & ctgan \\
\hline top5 & [3 4 5 2 1] & [3 4 5 2 1] & [3 4 5 2 1] & [3 4 5 2 1] \\
\hline top5\_freq & [11248  4512  1364   139    27] & [14872  5461  1228    49     3] & [15439  5203   945    26     1] & [12161  6512  2329   400   211] \\
\hline top5\_prob & [0.65054945 0.26096009 0.07888953 0.00803933 0.0015616 ] & [6.88104382e-01 2.52672003e-01 5.68176560e-02 2.26715403e-03
 1.38805349e-04] & [7.14305543e-01 2.40723605e-01 4.37216619e-02 1.20292403e-03
 4.62663089e-05] & [0.56267061 0.30130014 0.10775922 0.01850738 0.00976264] \\
\hline nobs & 17290 & \bfseries 21613 & \cellcolor[rgb]{0.9, 0.54, 0.52} 21614 & \bfseries 21613 \\
\hline missing & 0.000 & 0.000 & 0.000 & 0.000 \\
\hline mean & 3.408 & \bfseries 3.364 & \cellcolor[rgb]{0.9, 0.54, 0.52} 3.327 & 3.479 \\
\hline std\_err & 0.005 & 0.004 & \cellcolor[rgb]{0.9, 0.54, 0.52} 0.004 & \bfseries 0.005 \\
\hline upper\_ci & 3.417 & \bfseries 3.372 & \cellcolor[rgb]{0.9, 0.54, 0.52} 3.334 & 3.489 \\
\hline lower\_ci & 3.398 & \bfseries 3.356 & \cellcolor[rgb]{0.9, 0.54, 0.52} 3.319 & 3.469 \\
\hline std & 0.652 & \bfseries 0.592 & 0.557 & \cellcolor[rgb]{0.9, 0.54, 0.52} 0.749 \\
\hline iqr & 1.000 & 1.000 & 1.000 & 1.000 \\
\hline iqr\_normal & 0.741 & 0.741 & 0.741 & 0.741 \\
\hline mad & 0.560 & \bfseries 0.507 & \cellcolor[rgb]{0.9, 0.54, 0.52} 0.470 & 0.642 \\
\hline mad\_normal & 0.702 & \bfseries 0.636 & \cellcolor[rgb]{0.9, 0.54, 0.52} 0.590 & 0.805 \\
\hline coef\_var & 0.191 & \bfseries 0.176 & 0.167 & \cellcolor[rgb]{0.9, 0.54, 0.52} 0.215 \\
\hline range & 4.000 & 4.000 & 4.000 & 4.000 \\
\hline max & 5.000 & 5.000 & 5.000 & 5.000 \\
\hline min & 1.000 & 1.000 & 1.000 & 1.000 \\
\hline skew & 1.028 & \bfseries 1.317 & 1.447 & \cellcolor[rgb]{0.9, 0.54, 0.52} 0.361 \\
\hline kurtosis & 3.556 & 3.851 & \cellcolor[rgb]{0.9, 0.54, 0.52} 4.213 & \bfseries 3.455 \\
\hline jarque\_bera & 3269 & 6902 & \cellcolor[rgb]{0.9, 0.54, 0.52} 8863 & \bfseries 657 \\
\hline jarque\_bera\_pval & 0.000 & \bfseries 0.000 & \bfseries 0.000 & \cellcolor[rgb]{0.9, 0.54, 0.52} 0.000 \\
\hline mode & 3.000 & 3.000 & 3.000 & 3.000 \\
\hline mode\_freq & 0.651 & \bfseries 0.688 & 0.714 & \cellcolor[rgb]{0.9, 0.54, 0.52} 0.563 \\
\hline median & 3.000 & 3.000 & 3.000 & 3.000 \\
\hline 0.1\% & 1.000 & \cellcolor[rgb]{0.9, 0.54, 0.52} 2.000 & \cellcolor[rgb]{0.9, 0.54, 0.52} 2.000 & \bfseries 1.000 \\
\hline 1.0\% & 3.000 & \bfseries 3.000 & \bfseries 3.000 & \cellcolor[rgb]{0.9, 0.54, 0.52} 2.000 \\
\hline 5.0\% & 3.000 & 3.000 & 3.000 & 3.000 \\
\hline 25.0\% & 3.000 & 3.000 & 3.000 & 3.000 \\
\hline 75.0\% & 4.000 & 4.000 & 4.000 & 4.000 \\
\hline 95.0\% & 5.000 & \bfseries 5.000 & \cellcolor[rgb]{0.9, 0.54, 0.52} 4.000 & \bfseries 5.000 \\
\hline 99.0\% & 5.000 & 5.000 & 5.000 & 5.000 \\
\hline 99.9\% & 5.000 & 5.000 & 5.000 & 5.000 \\
\hline
\end{tabular}
\end{table}

\begin{table}[H]
\centering
\fontsize{8}{14}\selectfont
\caption{Propiedades  estadisticas de variable lat, King county (A-2)}
\label{table-stats-king county-a-2-lat}
\begin{tabular}{|l|m{10em}|m{10em}|m{10em}|m{10em}|}
\hline
 \rowcolor[gray]{0.8}
Variable/Modelo & Real & tddpm\_mlp & smote-enc & ctgan \\
\hline top5 & [47.5402 47.6914 47.6853 47.6624 47.6968] & [47.1559     47.7776     47.64939189 47.6493812  47.64937361] & [47.3363 47.6534 47.6647 47.6904 47.7438] & [47.1593 47.7776 47.6289 47.6879 47.643 ] \\
\hline top5\_freq & [14 13 13 13 13] & [16  5  1  1  1] & [7 6 6 5 4] & [430  56  18  16  16] \\
\hline top5\_prob & [0.00080972 0.00075188 0.00075188 0.00075188 0.00075188] & [7.40295193e-04 2.31342248e-04 4.62684495e-05 4.62684495e-05
 4.62684495e-05] & [0.00032386 0.0002776  0.0002776  0.00023133 0.00018507] & [0.01989543 0.00259103 0.00083283 0.0007403  0.0007403 ] \\
\hline nobs & 17290 & \bfseries 21613 & \cellcolor[rgb]{0.9, 0.54, 0.52} 21614 & \bfseries 21613 \\
\hline missing & 0.000 & 0.000 & 0.000 & 0.000 \\
\hline mean & 47.560 & 47.561 & \bfseries 47.561 & \cellcolor[rgb]{0.9, 0.54, 0.52} 47.527 \\
\hline std\_err & 0.001 & \cellcolor[rgb]{0.9, 0.54, 0.52} 0.001 & 0.001 & \bfseries 0.001 \\
\hline upper\_ci & 47.562 & 47.563 & \bfseries 47.563 & \cellcolor[rgb]{0.9, 0.54, 0.52} 47.529 \\
\hline lower\_ci & 47.558 & 47.559 & \bfseries 47.559 & \cellcolor[rgb]{0.9, 0.54, 0.52} 47.525 \\
\hline std & 0.138 & 0.137 & \bfseries 0.138 & \cellcolor[rgb]{0.9, 0.54, 0.52} 0.157 \\
\hline iqr & 0.206 & 0.208 & \bfseries 0.205 & \cellcolor[rgb]{0.9, 0.54, 0.52} 0.235 \\
\hline iqr\_normal & 0.153 & 0.154 & \bfseries 0.152 & \cellcolor[rgb]{0.9, 0.54, 0.52} 0.174 \\
\hline mad & 0.115 & 0.114 & \bfseries 0.114 & \cellcolor[rgb]{0.9, 0.54, 0.52} 0.131 \\
\hline mad\_normal & 0.144 & 0.143 & \bfseries 0.143 & \cellcolor[rgb]{0.9, 0.54, 0.52} 0.164 \\
\hline coef\_var & 0.003 & 0.003 & \bfseries 0.003 & \cellcolor[rgb]{0.9, 0.54, 0.52} 0.003 \\
\hline range & 0.618 & 0.622 & \cellcolor[rgb]{0.9, 0.54, 0.52} 0.606 & \bfseries 0.618 \\
\hline max & 47.778 & \bfseries 47.778 & \cellcolor[rgb]{0.9, 0.54, 0.52} 47.777 & \bfseries 47.778 \\
\hline min & 47.159 & 47.156 & \cellcolor[rgb]{0.9, 0.54, 0.52} 47.171 & \bfseries 47.159 \\
\hline skew & -0.487 & -0.459 & \bfseries -0.492 & \cellcolor[rgb]{0.9, 0.54, 0.52} -0.603 \\
\hline kurtosis & 2.328 & 2.267 & \bfseries 2.313 & \cellcolor[rgb]{0.9, 0.54, 0.52} 2.423 \\
\hline jarque\_bera & 1009 & \bfseries 1245 & 1298 & \cellcolor[rgb]{0.9, 0.54, 0.52} 1611 \\
\hline jarque\_bera\_pval & 0.000 & 0.000 & 0.000 & 0.000 \\
\hline mode & 47.540 & \cellcolor[rgb]{0.9, 0.54, 0.52} 47.156 & \bfseries 47.336 & 47.159 \\
\hline mode\_freq & 0.001 & \bfseries 0.001 & 0.000 & \cellcolor[rgb]{0.9, 0.54, 0.52} 0.020 \\
\hline median & 47.572 & 47.569 & \bfseries 47.572 & \cellcolor[rgb]{0.9, 0.54, 0.52} 47.547 \\
\hline 0.1\% & 47.193 & 47.171 & \bfseries 47.194 & \cellcolor[rgb]{0.9, 0.54, 0.52} 47.159 \\
\hline 1.0\% & 47.257 & 47.263 & \bfseries 47.258 & \cellcolor[rgb]{0.9, 0.54, 0.52} 47.159 \\
\hline 5.0\% & 47.311 & 47.313 & \bfseries 47.312 & \cellcolor[rgb]{0.9, 0.54, 0.52} 47.231 \\
\hline 25.0\% & 47.472 & \bfseries 47.472 & 47.473 & \cellcolor[rgb]{0.9, 0.54, 0.52} 47.424 \\
\hline 75.0\% & 47.678 & 47.680 & \bfseries 47.678 & \cellcolor[rgb]{0.9, 0.54, 0.52} 47.660 \\
\hline 95.0\% & 47.750 & 47.748 & \bfseries 47.750 & \cellcolor[rgb]{0.9, 0.54, 0.52} 47.724 \\
\hline 99.0\% & 47.773 & \bfseries 47.771 & 47.771 & \cellcolor[rgb]{0.9, 0.54, 0.52} 47.756 \\
\hline 99.9\% & 47.777 & \bfseries 47.777 & \cellcolor[rgb]{0.9, 0.54, 0.52} 47.776 & 47.778 \\
\hline
\end{tabular}
\end{table}

\begin{table}[H]
\centering
\fontsize{8}{14}\selectfont
\caption{Propiedades  estadisticas de variable sqft\_living, King county (A-2)}
\label{table-stats-king county-a-2-sqft_living}
\begin{tabular}{|l|m{10em}|m{10em}|m{10em}|m{10em}|}
\hline
 \rowcolor[gray]{0.8}
Variable/Modelo & Real & tddpm\_mlp & smote-enc & ctgan \\
\hline top5 & [1400 1300 1720 1250 1540] & [1440. 1300. 1800. 1320. 1820.] & [1250. 1280. 1160. 1690. 1800.] & [ 290 1925 2471 1406 2241] \\
\hline top5\_freq & [109 107 106 106 105] & [144 142 137 117 115] & [15 14 12 11 11] & [20 18 17 17 17] \\
\hline top5\_prob & [0.00630422 0.00618855 0.00613071 0.00613071 0.00607287] & [0.00666266 0.00657012 0.00633878 0.00541341 0.00532087] & [0.00069399 0.00064773 0.0005552  0.00050893 0.00050893] & [0.00092537 0.00083283 0.00078656 0.00078656 0.00078656] \\
\hline nobs & 17290 & \bfseries 21613 & \cellcolor[rgb]{0.9, 0.54, 0.52} 21614 & \bfseries 21613 \\
\hline missing & 0.000 & 0.000 & 0.000 & 0.000 \\
\hline mean & 2074 & 2032 & \bfseries 2049 & \cellcolor[rgb]{0.9, 0.54, 0.52} 2813 \\
\hline std\_err & 6.900 & \bfseries 6.968 & 5.741 & \cellcolor[rgb]{0.9, 0.54, 0.52} 10.591 \\
\hline upper\_ci & 2087 & 2046 & \bfseries 2060 & \cellcolor[rgb]{0.9, 0.54, 0.52} 2833 \\
\hline lower\_ci & 2060 & 2019 & \bfseries 2037 & \cellcolor[rgb]{0.9, 0.54, 0.52} 2792 \\
\hline std & 907.298 & 1024.446 & \bfseries 844.097 & \cellcolor[rgb]{0.9, 0.54, 0.52} 1556.997 \\
\hline iqr & 1110 & 1057 & \bfseries 1079 & \cellcolor[rgb]{0.9, 0.54, 0.52} 1810 \\
\hline iqr\_normal & 822.844 & 783.834 & \bfseries 799.642 & \cellcolor[rgb]{0.9, 0.54, 0.52} 1341.755 \\
\hline mad & 693.180 & \bfseries 679.548 & 657.789 & \cellcolor[rgb]{0.9, 0.54, 0.52} 1185.435 \\
\hline mad\_normal & 868.773 & \bfseries 851.687 & 824.416 & \cellcolor[rgb]{0.9, 0.54, 0.52} 1485.722 \\
\hline coef\_var & 0.437 & 0.504 & \bfseries 0.412 & \cellcolor[rgb]{0.9, 0.54, 0.52} 0.554 \\
\hline range & 11760 & \bfseries 13250 & \cellcolor[rgb]{0.9, 0.54, 0.52} 8998 & 10198 \\
\hline max & 12050 & \bfseries 13540 & \cellcolor[rgb]{0.9, 0.54, 0.52} 9404 & 10488 \\
\hline min & 290.000 & \bfseries 290.000 & \cellcolor[rgb]{0.9, 0.54, 0.52} 405.289 & \bfseries 290.000 \\
\hline skew & 1.371 & \cellcolor[rgb]{0.9, 0.54, 0.52} 4.333 & 1.117 & \bfseries 1.334 \\
\hline kurtosis & 7.167 & \cellcolor[rgb]{0.9, 0.54, 0.52} 44.030 & 5.084 & \bfseries 5.144 \\
\hline jarque\_bera & 17922 & \cellcolor[rgb]{0.9, 0.54, 0.52} 1583687 & 8402 & \bfseries 10545 \\
\hline jarque\_bera\_pval & 0.000 & 0.000 & 0.000 & 0.000 \\
\hline mode & 1400 & \bfseries 1440 & 1250 & \cellcolor[rgb]{0.9, 0.54, 0.52} 290 \\
\hline mode\_freq & 0.006 & \bfseries 0.007 & \cellcolor[rgb]{0.9, 0.54, 0.52} 0.001 & 0.001 \\
\hline median & 1910 & 1850 & \bfseries 1896 & \cellcolor[rgb]{0.9, 0.54, 0.52} 2442 \\
\hline 0.1\% & 522.890 & \bfseries 482.093 & 595.171 & \cellcolor[rgb]{0.9, 0.54, 0.52} 294.612 \\
\hline 1.0\% & 720.000 & \bfseries 726.418 & 757.374 & \cellcolor[rgb]{0.9, 0.54, 0.52} 592.120 \\
\hline 5.0\% & 940.000 & \bfseries 938.703 & 969.440 & \cellcolor[rgb]{0.9, 0.54, 0.52} 979.600 \\
\hline 25.0\% & 1430 & 1400 & \bfseries 1428 & \cellcolor[rgb]{0.9, 0.54, 0.52} 1715 \\
\hline 75.0\% & 2540 & 2457 & \bfseries 2507 & \cellcolor[rgb]{0.9, 0.54, 0.52} 3525 \\
\hline 95.0\% & 3740 & 3571 & \bfseries 3640 & \cellcolor[rgb]{0.9, 0.54, 0.52} 5727 \\
\hline 99.0\% & 4921 & \bfseries 4768 & 4590 & \cellcolor[rgb]{0.9, 0.54, 0.52} 8012 \\
\hline 99.9\% & 6966 & \cellcolor[rgb]{0.9, 0.54, 0.52} 13540 & \bfseries 6129 & 9755 \\
\hline
\end{tabular}
\end{table}

\begin{table}[H]
\centering
\fontsize{8}{14}\selectfont
\caption{Propiedades  estadisticas de variable waterfront, King county (A-2)}
\label{table-stats-king county-a-2-waterfront}
\begin{tabular}{|l|m{10em}|m{10em}|m{10em}|m{10em}|}
\hline
 \rowcolor[gray]{0.8}
Variable/Modelo & Real & tddpm\_mlp & smote-enc & ctgan \\
\hline top5 & [0 1] & [0 1] & [0 1] & [0 1] \\
\hline top5\_freq & [17166   124] & [21565    48] & [21582    32] & [20038  1575] \\
\hline top5\_prob & [0.99282822 0.00717178] & [0.99777911 0.00222089] & [0.99851948 0.00148052] & [0.92712719 0.07287281] \\
\hline nobs & 17290 & \bfseries 21613 & \cellcolor[rgb]{0.9, 0.54, 0.52} 21614 & \bfseries 21613 \\
\hline missing & 0.000 & 0.000 & 0.000 & 0.000 \\
\hline mean & 0.007 & \bfseries 0.002 & 0.001 & \cellcolor[rgb]{0.9, 0.54, 0.52} 0.073 \\
\hline std\_err & 0.001 & \bfseries 0.000 & 0.000 & \cellcolor[rgb]{0.9, 0.54, 0.52} 0.002 \\
\hline upper\_ci & 0.008 & \bfseries 0.003 & 0.002 & \cellcolor[rgb]{0.9, 0.54, 0.52} 0.076 \\
\hline lower\_ci & 0.006 & \bfseries 0.002 & 0.001 & \cellcolor[rgb]{0.9, 0.54, 0.52} 0.069 \\
\hline std & 0.084 & \bfseries 0.047 & 0.038 & \cellcolor[rgb]{0.9, 0.54, 0.52} 0.260 \\
\hline iqr & 0.000 & 0.000 & 0.000 & 0.000 \\
\hline iqr\_normal & 0.000 & 0.000 & 0.000 & 0.000 \\
\hline mad & 0.014 & \bfseries 0.004 & 0.003 & \cellcolor[rgb]{0.9, 0.54, 0.52} 0.135 \\
\hline mad\_normal & 0.018 & \bfseries 0.006 & 0.004 & \cellcolor[rgb]{0.9, 0.54, 0.52} 0.169 \\
\hline coef\_var & 11.766 & 21.197 & \cellcolor[rgb]{0.9, 0.54, 0.52} 25.971 & \bfseries 3.567 \\
\hline range & 1.000 & 1.000 & 1.000 & 1.000 \\
\hline max & 1.000 & 1.000 & 1.000 & 1.000 \\
\hline min & 0.000 & 0.000 & 0.000 & 0.000 \\
\hline skew & 11.681 & 21.149 & \cellcolor[rgb]{0.9, 0.54, 0.52} 25.931 & \bfseries 3.287 \\
\hline kurtosis & 137.443 & 448.273 & \cellcolor[rgb]{0.9, 0.54, 0.52} 673.439 & \bfseries 11.801 \\
\hline jarque\_bera & 13414600 & 180159835 & \cellcolor[rgb]{0.9, 0.54, 0.52} 407224138 & \bfseries 108664 \\
\hline jarque\_bera\_pval & 0.000 & 0.000 & 0.000 & 0.000 \\
\hline mode & 0.000 & 0.000 & 0.000 & 0.000 \\
\hline mode\_freq & 0.993 & \bfseries 0.998 & 0.999 & \cellcolor[rgb]{0.9, 0.54, 0.52} 0.927 \\
\hline median & 0.000 & 0.000 & 0.000 & 0.000 \\
\hline 0.1\% & 0.000 & 0.000 & 0.000 & 0.000 \\
\hline 1.0\% & 0.000 & 0.000 & 0.000 & 0.000 \\
\hline 5.0\% & 0.000 & 0.000 & 0.000 & 0.000 \\
\hline 25.0\% & 0.000 & 0.000 & 0.000 & 0.000 \\
\hline 75.0\% & 0.000 & 0.000 & 0.000 & 0.000 \\
\hline 95.0\% & 0.000 & \bfseries 0.000 & \bfseries 0.000 & \cellcolor[rgb]{0.9, 0.54, 0.52} 1.000 \\
\hline 99.0\% & 0.000 & \bfseries 0.000 & \bfseries 0.000 & \cellcolor[rgb]{0.9, 0.54, 0.52} 1.000 \\
\hline 99.9\% & 1.000 & 1.000 & 1.000 & 1.000 \\
\hline
\end{tabular}
\end{table}

\begin{table}[H]
\centering
\fontsize{8}{14}\selectfont
\caption{Propiedades  estadisticas de variable sqft\_basement, King county (A-2)}
\label{table-stats-king county-a-2-sqft_basement}
\begin{tabular}{|l|m{10em}|m{10em}|m{10em}|m{10em}|}
\hline
 \rowcolor[gray]{0.8}
Variable/Modelo & Real & tddpm\_mlp & smote-enc & ctgan \\
\hline top5 & [  0 600 700 500 800] & [  0. 500. 600. 700. 800.] & [  0. 800. 600. 850. 500.] & [ 0 10 11  9 12] \\
\hline top5\_freq & [10553   182   169   167   164] & [13604   239   196   187   177] & [11761    16    11     9     8] & [2222  863  815  804  804] \\
\hline top5\_prob & [0.61035281 0.01052632 0.00977444 0.00965876 0.00948525] & [0.62943599 0.01105816 0.00906862 0.0086522  0.00818952] & [5.44138059e-01 7.40260942e-04 5.08929398e-04 4.16396780e-04
 3.70130471e-04] & [0.10280849 0.03992967 0.03770879 0.03719983 0.03719983] \\
\hline nobs & 17290 & \bfseries 21613 & \cellcolor[rgb]{0.9, 0.54, 0.52} 21614 & \bfseries 21613 \\
\hline missing & 0.000 & 0.000 & 0.000 & 0.000 \\
\hline mean & 287.933 & \bfseries 286.843 & 274.142 & \cellcolor[rgb]{0.9, 0.54, 0.52} 457.012 \\
\hline std\_err & 3.337 & \bfseries 3.255 & 2.755 & \cellcolor[rgb]{0.9, 0.54, 0.52} 4.625 \\
\hline upper\_ci & 294.472 & \bfseries 293.223 & 279.541 & \cellcolor[rgb]{0.9, 0.54, 0.52} 466.076 \\
\hline lower\_ci & 281.393 & \bfseries 280.463 & 268.743 & \cellcolor[rgb]{0.9, 0.54, 0.52} 447.948 \\
\hline std & 438.727 & 478.554 & \bfseries 404.963 & \cellcolor[rgb]{0.9, 0.54, 0.52} 679.898 \\
\hline iqr & 550.000 & \bfseries 550.000 & 517.020 & \cellcolor[rgb]{0.9, 0.54, 0.52} 881.000 \\
\hline iqr\_normal & 407.716 & \bfseries 407.716 & 383.267 & \cellcolor[rgb]{0.9, 0.54, 0.52} 653.086 \\
\hline mad & 360.277 & \bfseries 365.019 & 332.362 & \cellcolor[rgb]{0.9, 0.54, 0.52} 562.364 \\
\hline mad\_normal & 451.541 & \bfseries 457.483 & 416.555 & \cellcolor[rgb]{0.9, 0.54, 0.52} 704.819 \\
\hline coef\_var & 1.524 & \cellcolor[rgb]{0.9, 0.54, 0.52} 1.668 & 1.477 & \bfseries 1.488 \\
\hline range & 4820 & \bfseries 4820 & \cellcolor[rgb]{0.9, 0.54, 0.52} 2713 & 3238 \\
\hline max & 4820 & \bfseries 4820 & \cellcolor[rgb]{0.9, 0.54, 0.52} 2713 & 3238 \\
\hline min & 0.000 & 0.000 & 0.000 & 0.000 \\
\hline skew & 1.571 & \cellcolor[rgb]{0.9, 0.54, 0.52} 3.129 & \bfseries 1.467 & 1.451 \\
\hline kurtosis & 5.639 & \cellcolor[rgb]{0.9, 0.54, 0.52} 23.527 & \bfseries 4.575 & 4.359 \\
\hline jarque\_bera & 12126 & \cellcolor[rgb]{0.9, 0.54, 0.52} 414719 & \bfseries 9983 & 9250 \\
\hline jarque\_bera\_pval & 0.000 & 0.000 & 0.000 & 0.000 \\
\hline mode & 0.000 & 0.000 & 0.000 & 0.000 \\
\hline mode\_freq & 0.610 & \bfseries 0.629 & 0.544 & \cellcolor[rgb]{0.9, 0.54, 0.52} 0.103 \\
\hline median & 0.000 & \bfseries 0.000 & \bfseries 0.000 & \cellcolor[rgb]{0.9, 0.54, 0.52} 13.000 \\
\hline 0.1\% & 0.000 & 0.000 & 0.000 & 0.000 \\
\hline 1.0\% & 0.000 & 0.000 & 0.000 & 0.000 \\
\hline 5.0\% & 0.000 & 0.000 & 0.000 & 0.000 \\
\hline 25.0\% & 0.000 & \bfseries 0.000 & \bfseries 0.000 & \cellcolor[rgb]{0.9, 0.54, 0.52} 6.000 \\
\hline 75.0\% & 550.000 & \bfseries 550.000 & 517.020 & \cellcolor[rgb]{0.9, 0.54, 0.52} 887.000 \\
\hline 95.0\% & 1180 & \bfseries 1134 & 1097 & \cellcolor[rgb]{0.9, 0.54, 0.52} 1868 \\
\hline 99.0\% & 1650 & \bfseries 1594 & 1539 & \cellcolor[rgb]{0.9, 0.54, 0.52} 2679 \\
\hline 99.9\% & 2324 & \cellcolor[rgb]{0.9, 0.54, 0.52} 4820 & \bfseries 2061 & 3082 \\
\hline
\end{tabular}
\end{table}

\begin{table}[H]
\centering
\fontsize{8}{14}\selectfont
\caption{Propiedades  estadisticas de variable sqft\_lot15, King county (A-2)}
\label{table-stats-king county-a-2-sqft_lot15}
\begin{tabular}{|l|m{10em}|m{10em}|m{10em}|m{10em}|}
\hline
 \rowcolor[gray]{0.8}
Variable/Modelo & Real & tddpm\_mlp & smote-enc & ctgan \\
\hline top5 & [5000 4000 6000 7200 4800] & [5000. 4000. 6000. 7200. 4800.] & [5000. 4000. 5200. 6000. 4080.] & [ 651 7137 9396 8240 4547] \\
\hline top5\_freq & [349 289 224 160 120] & [407 322 228 191 125] & [74 71 33 32 26] & [890   7   7   7   6] \\
\hline top5\_prob & [0.02018508 0.01671486 0.01295547 0.0092539  0.00694043] & [0.01883126 0.01489844 0.01054921 0.00883727 0.00578356] & [0.00342371 0.00328491 0.00152679 0.00148052 0.00120292] & [0.04117892 0.00032388 0.00032388 0.00032388 0.00027761] \\
\hline nobs & 17290 & \bfseries 21613 & \cellcolor[rgb]{0.9, 0.54, 0.52} 21614 & \bfseries 21613 \\
\hline missing & 0.000 & 0.000 & 0.000 & 0.000 \\
\hline mean & 12725 & \bfseries 12168 & 11832 & \cellcolor[rgb]{0.9, 0.54, 0.52} 14318 \\
\hline std\_err & 209.331 & \bfseries 221.883 & \cellcolor[rgb]{0.9, 0.54, 0.52} 155.914 & 163.365 \\
\hline upper\_ci & 13135 & \bfseries 12603 & 12137 & \cellcolor[rgb]{0.9, 0.54, 0.52} 14638 \\
\hline lower\_ci & 12315 & \bfseries 11733 & 11526 & \cellcolor[rgb]{0.9, 0.54, 0.52} 13997 \\
\hline std & 27525 & \cellcolor[rgb]{0.9, 0.54, 0.52} 32620 & 22922 & \bfseries 24017 \\
\hline iqr & 4963 & 4663 & \bfseries 4930 & \cellcolor[rgb]{0.9, 0.54, 0.52} 8395 \\
\hline iqr\_normal & 3679 & 3457 & \bfseries 3654 & \cellcolor[rgb]{0.9, 0.54, 0.52} 6223 \\
\hline mad & 10095 & 9042 & \cellcolor[rgb]{0.9, 0.54, 0.52} 8618 & \bfseries 10910 \\
\hline mad\_normal & 12652 & 11332 & \cellcolor[rgb]{0.9, 0.54, 0.52} 10800 & \bfseries 13674 \\
\hline coef\_var & 2.163 & \cellcolor[rgb]{0.9, 0.54, 0.52} 2.681 & \bfseries 1.937 & 1.677 \\
\hline range & 870549 & \bfseries 870549 & 680511 & \cellcolor[rgb]{0.9, 0.54, 0.52} 309873 \\
\hline max & 871200 & \bfseries 871200 & 681231 & \cellcolor[rgb]{0.9, 0.54, 0.52} 310524 \\
\hline min & 651.000 & \bfseries 651.000 & \cellcolor[rgb]{0.9, 0.54, 0.52} 719.360 & \bfseries 651.000 \\
\hline skew & 9.701 & \cellcolor[rgb]{0.9, 0.54, 0.52} 16.907 & \bfseries 8.316 & 6.341 \\
\hline kurtosis & 163.253 & \cellcolor[rgb]{0.9, 0.54, 0.52} 383.753 & \bfseries 101.830 & 54.342 \\
\hline jarque\_bera & 18772189 & \cellcolor[rgb]{0.9, 0.54, 0.52} 131583876 & \bfseries 9045446 & 2518709 \\
\hline jarque\_bera\_pval & 0.000 & 0.000 & 0.000 & 0.000 \\
\hline mode & 5000 & \bfseries 5000 & 5000 & \cellcolor[rgb]{0.9, 0.54, 0.52} 651 \\
\hline mode\_freq & 0.020 & \bfseries 0.019 & 0.003 & \cellcolor[rgb]{0.9, 0.54, 0.52} 0.041 \\
\hline median & 7615 & 7723 & \bfseries 7644 & \cellcolor[rgb]{0.9, 0.54, 0.52} 9083 \\
\hline 0.1\% & 886.289 & \bfseries 878.456 & 913.044 & \cellcolor[rgb]{0.9, 0.54, 0.52} 651.000 \\
\hline 1.0\% & 1189 & \bfseries 1215 & 1230 & \cellcolor[rgb]{0.9, 0.54, 0.52} 651 \\
\hline 5.0\% & 1965 & 2545 & \bfseries 2106 & \cellcolor[rgb]{0.9, 0.54, 0.52} 970 \\
\hline 25.0\% & 5083 & 5183 & \bfseries 5100 & \cellcolor[rgb]{0.9, 0.54, 0.52} 5235 \\
\hline 75.0\% & 10046 & 9846 & \bfseries 10030 & \cellcolor[rgb]{0.9, 0.54, 0.52} 13630 \\
\hline 95.0\% & 36822 & 35077 & \bfseries 35134 & \cellcolor[rgb]{0.9, 0.54, 0.52} 45254 \\
\hline 99.0\% & 168296 & \cellcolor[rgb]{0.9, 0.54, 0.52} 107518 & 130809 & \bfseries 138474 \\
\hline 99.9\% & 306998 & \cellcolor[rgb]{0.9, 0.54, 0.52} 532713 & 263407 & \bfseries 269421 \\
\hline
\end{tabular}
\end{table}

\begin{table}[H]
\centering
\fontsize{8}{14}\selectfont
\caption{Propiedades  estadisticas de variable date, King county (A-2)}
\label{table-stats-king county-a-2-date}
\begin{tabular}{|l|m{10em}|m{10em}|m{10em}|m{10em}|}
\hline
 \rowcolor[gray]{0.8}
Variable/Modelo & Real & tddpm\_mlp & smote-enc & ctgan \\
\hline top5 & ['20140623T000000' '20140625T000000' '20140626T000000' '20150421T000000'
 '20150325T000000'] & ['20140623T000000' '20140625T000000' '20150427T000000' '20140520T000000'
 '20150421T000000'] & ['20140623T000000' '20140625T000000' '20150414T000000' '20140520T000000'
 '20150421T000000'] & ['20150310T000000' '20150327T000000' '20140603T000000' '20150226T000000'
 '20150329T000000'] \\
\hline top5\_freq & [123 105 101 101 101] & [195 167 163 161 151] & [151 146 139 136 135] & [489 421 337 288 283] \\
\hline top5\_prob & [0.00711394 0.00607287 0.00584153 0.00584153 0.00584153] & [0.00902235 0.00772683 0.00754176 0.00744922 0.00698654] & [0.00698621 0.00675488 0.00643102 0.00629222 0.00624595] & [0.02262527 0.01947902 0.01559247 0.01332531 0.01309397] \\
\hline nobs & 17290 & \bfseries 21613 & \cellcolor[rgb]{0.9, 0.54, 0.52} 21614 & \bfseries 21613 \\
\hline missing & 17290 & 0 & 0 & 0 \\
\hline
\end{tabular}
\end{table}




\section{Estadísticos Económicos - Conjunto B}
\label{propiedades-estadisticas-economicos-B}
\begin{table}[H]
\centering
\fontsize{8}{14}\selectfont
\caption{Propiedades  estadisticas de variable yr\_built, King county (A-2)}
\label{table-stats-king county-a-2-yr_built}
\begin{tabular}{|l|m{10em}|m{10em}|m{10em}|m{10em}|}
\hline
 \rowcolor[gray]{0.8}
Variable/Modelo & Real & tddpm\_mlp & smote-enc & ctgan \\
\hline top5 & [2014 2005 2006 2004 2007] & [2006. 2005. 2004. 1977. 2003.] & [2014. 2006. 2005. 2007. 2003.] & [1900 1976 1977 1975 1974] \\
\hline top5\_freq & [449 371 366 350 347] & [463 451 442 439 415] & [247 113 102  97  93] & [639 355 343 330 329] \\
\hline top5\_prob & [0.02596877 0.02145749 0.02116831 0.02024291 0.0200694 ] & [0.02142229 0.02086707 0.02045065 0.02031185 0.01920141] & [0.01142778 0.00522809 0.00471916 0.00448783 0.00430277] & [0.02956554 0.0164253  0.01587008 0.01526859 0.01522232] \\
\hline nobs & 17290 & \bfseries 21613 & \cellcolor[rgb]{0.9, 0.54, 0.52} 21614 & \bfseries 21613 \\
\hline missing & 0.000 & 0.000 & 0.000 & 0.000 \\
\hline mean & 1971 & 1972 & \bfseries 1971 & \cellcolor[rgb]{0.9, 0.54, 0.52} 1961 \\
\hline std\_err & 0.224 & \cellcolor[rgb]{0.9, 0.54, 0.52} 0.190 & 0.198 & \bfseries 0.204 \\
\hline upper\_ci & 1972 & 1972 & \bfseries 1972 & \cellcolor[rgb]{0.9, 0.54, 0.52} 1961 \\
\hline lower\_ci & 1971 & 1972 & \bfseries 1971 & \cellcolor[rgb]{0.9, 0.54, 0.52} 1960 \\
\hline std & 29.436 & \cellcolor[rgb]{0.9, 0.54, 0.52} 27.911 & \bfseries 29.169 & 29.969 \\
\hline iqr & 46.000 & \cellcolor[rgb]{0.9, 0.54, 0.52} 42.578 & \bfseries 45.832 & 45.000 \\
\hline iqr\_normal & 34.100 & \cellcolor[rgb]{0.9, 0.54, 0.52} 31.563 & \bfseries 33.975 & 33.359 \\
\hline mad & 24.632 & \cellcolor[rgb]{0.9, 0.54, 0.52} 23.153 & \bfseries 24.459 & 24.966 \\
\hline mad\_normal & 30.872 & \cellcolor[rgb]{0.9, 0.54, 0.52} 29.018 & \bfseries 30.655 & 31.290 \\
\hline coef\_var & 0.015 & \cellcolor[rgb]{0.9, 0.54, 0.52} 0.014 & \bfseries 0.015 & 0.015 \\
\hline range & 115.000 & 115.000 & 115.000 & 115.000 \\
\hline max & 2015 & \bfseries 2015 & \cellcolor[rgb]{0.9, 0.54, 0.52} 2015 & \bfseries 2015 \\
\hline min & 1900 & \bfseries 1900 & \cellcolor[rgb]{0.9, 0.54, 0.52} 1900 & \bfseries 1900 \\
\hline skew & -0.472 & \bfseries -0.475 & -0.464 & \cellcolor[rgb]{0.9, 0.54, 0.52} -0.305 \\
\hline kurtosis & 2.337 & 2.438 & \bfseries 2.309 & \cellcolor[rgb]{0.9, 0.54, 0.52} 2.155 \\
\hline jarque\_bera & 957.631 & 1095.917 & \cellcolor[rgb]{0.9, 0.54, 0.52} 1205.886 & \bfseries 978.229 \\
\hline jarque\_bera\_pval & 0.000 & \cellcolor[rgb]{0.9, 0.54, 0.52} 0.000 & \cellcolor[rgb]{0.9, 0.54, 0.52} 0.000 & \bfseries 0.000 \\
\hline mode & 2014 & 2006 & \bfseries 2014 & \cellcolor[rgb]{0.9, 0.54, 0.52} 1900 \\
\hline mode\_freq & 0.026 & 0.021 & \cellcolor[rgb]{0.9, 0.54, 0.52} 0.011 & \bfseries 0.030 \\
\hline median & 1975 & \bfseries 1975 & 1974 & \cellcolor[rgb]{0.9, 0.54, 0.52} 1963 \\
\hline 0.1\% & 1900 & \bfseries 1900 & \cellcolor[rgb]{0.9, 0.54, 0.52} 1901 & \bfseries 1900 \\
\hline 1.0\% & 1904 & 1906 & \bfseries 1905 & \cellcolor[rgb]{0.9, 0.54, 0.52} 1900 \\
\hline 5.0\% & 1915 & 1919 & \bfseries 1916 & \cellcolor[rgb]{0.9, 0.54, 0.52} 1908 \\
\hline 25.0\% & 1951 & 1954 & \bfseries 1952 & \cellcolor[rgb]{0.9, 0.54, 0.52} 1939 \\
\hline 75.0\% & 1997 & \bfseries 1997 & 1998 & \cellcolor[rgb]{0.9, 0.54, 0.52} 1984 \\
\hline 95.0\% & 2011 & 2009 & \bfseries 2010 & \cellcolor[rgb]{0.9, 0.54, 0.52} 2005 \\
\hline 99.0\% & 2014 & \bfseries 2014 & \bfseries 2014 & \cellcolor[rgb]{0.9, 0.54, 0.52} 2013 \\
\hline 99.9\% & 2015 & \bfseries 2015 & \cellcolor[rgb]{0.9, 0.54, 0.52} 2014 & \bfseries 2015 \\
\hline
\end{tabular}
\end{table}

\begin{table}[H]
\centering
\fontsize{8}{14}\selectfont
\caption{Propiedades  estadisticas de variable sqft\_above, King county (A-2)}
\label{table-stats-king county-a-2-sqft_above}
\begin{tabular}{|l|m{10em}|m{10em}|m{10em}|m{10em}|}
\hline
 \rowcolor[gray]{0.8}
Variable/Modelo & Real & tddpm\_mlp & smote-enc & ctgan \\
\hline top5 & [1300 1010 1200 1220 1140] & [1300. 1010. 1220. 1140. 1340.] & [1290. 1160. 1830. 1010. 1320.] & [1563 1241 1203 1170 1197] \\
\hline top5\_freq & [166 165 160 152 148] & [203 185 180 171 162] & [15 15 14 13 12] & [20 20 20 19 19] \\
\hline top5\_prob & [0.00960093 0.00954309 0.0092539  0.00879121 0.00855986] & [0.0093925  0.00855966 0.00832832 0.0079119  0.00749549] & [0.00069399 0.00069399 0.00064773 0.00060146 0.0005552 ] & [0.00092537 0.00092537 0.00092537 0.0008791  0.0008791 ] \\
\hline nobs & 17290 & \bfseries 21613 & \cellcolor[rgb]{0.9, 0.54, 0.52} 21614 & \bfseries 21613 \\
\hline missing & 0.000 & 0.000 & 0.000 & 0.000 \\
\hline mean & 1786 & \bfseries 1778 & 1769 & \cellcolor[rgb]{0.9, 0.54, 0.52} 2018 \\
\hline std\_err & 6.249 & \bfseries 5.620 & 5.283 & \cellcolor[rgb]{0.9, 0.54, 0.52} 7.466 \\
\hline upper\_ci & 1798 & \bfseries 1789 & 1780 & \cellcolor[rgb]{0.9, 0.54, 0.52} 2033 \\
\hline lower\_ci & 1774 & \bfseries 1767 & 1759 & \cellcolor[rgb]{0.9, 0.54, 0.52} 2004 \\
\hline std & 821.626 & \bfseries 826.164 & 776.654 & \cellcolor[rgb]{0.9, 0.54, 0.52} 1097.579 \\
\hline iqr & 1000.000 & \bfseries 975.555 & 972.791 & \cellcolor[rgb]{0.9, 0.54, 0.52} 1434.000 \\
\hline iqr\_normal & 741.301 & \bfseries 723.180 & 721.131 & \cellcolor[rgb]{0.9, 0.54, 0.52} 1063.026 \\
\hline mad & 635.012 & \bfseries 615.572 & 608.256 & \cellcolor[rgb]{0.9, 0.54, 0.52} 872.298 \\
\hline mad\_normal & 795.870 & \bfseries 771.505 & 762.336 & \cellcolor[rgb]{0.9, 0.54, 0.52} 1093.263 \\
\hline coef\_var & 0.460 & \bfseries 0.465 & 0.439 & \cellcolor[rgb]{0.9, 0.54, 0.52} 0.544 \\
\hline range & 8570 & 9120 & \bfseries 8368 & \cellcolor[rgb]{0.9, 0.54, 0.52} 7269 \\
\hline max & 8860 & 9410 & \bfseries 8670 & \cellcolor[rgb]{0.9, 0.54, 0.52} 7559 \\
\hline min & 290.000 & \bfseries 290.000 & \cellcolor[rgb]{0.9, 0.54, 0.52} 301.856 & \bfseries 290.000 \\
\hline skew & 1.428 & \cellcolor[rgb]{0.9, 0.54, 0.52} 2.236 & \bfseries 1.320 & 1.097 \\
\hline kurtosis & 6.260 & \cellcolor[rgb]{0.9, 0.54, 0.52} 15.164 & \bfseries 5.306 & 3.945 \\
\hline jarque\_bera & 13530 & \cellcolor[rgb]{0.9, 0.54, 0.52} 151241 & \bfseries 11064 & 5139 \\
\hline jarque\_bera\_pval & 0.000 & 0.000 & 0.000 & 0.000 \\
\hline mode & 1300 & \bfseries 1300 & \cellcolor[rgb]{0.9, 0.54, 0.52} 1160 & 1203 \\
\hline mode\_freq & 0.010 & \bfseries 0.009 & \cellcolor[rgb]{0.9, 0.54, 0.52} 0.001 & 0.001 \\
\hline median & 1560 & \bfseries 1560 & 1540 & \cellcolor[rgb]{0.9, 0.54, 0.52} 1721 \\
\hline 0.1\% & 520.000 & \bfseries 544.640 & 603.805 & \cellcolor[rgb]{0.9, 0.54, 0.52} 360.224 \\
\hline 1.0\% & 700.000 & \bfseries 720.516 & 740.191 & \cellcolor[rgb]{0.9, 0.54, 0.52} 512.120 \\
\hline 5.0\% & 850.000 & \bfseries 878.073 & 888.091 & \cellcolor[rgb]{0.9, 0.54, 0.52} 731.000 \\
\hline 25.0\% & 1200 & 1207 & \cellcolor[rgb]{0.9, 0.54, 0.52} 1207 & \bfseries 1198 \\
\hline 75.0\% & 2200 & \bfseries 2182 & 2180 & \cellcolor[rgb]{0.9, 0.54, 0.52} 2632 \\
\hline 95.0\% & 3380 & 3269 & \bfseries 3295 & \cellcolor[rgb]{0.9, 0.54, 0.52} 4226 \\
\hline 99.0\% & 4371 & 4194 & \bfseries 4199 & \cellcolor[rgb]{0.9, 0.54, 0.52} 5324 \\
\hline 99.9\% & 6070 & \cellcolor[rgb]{0.9, 0.54, 0.52} 9410 & 5258 & \bfseries 6386 \\
\hline
\end{tabular}
\end{table}

\begin{table}[H]
\centering
\fontsize{8}{14}\selectfont
\caption{Propiedades  estadisticas de variable yr\_renovated, King county (A-2)}
\label{table-stats-king county-a-2-yr_renovated}
\begin{tabular}{|l|m{10em}|m{10em}|m{10em}|m{10em}|}
\hline
 \rowcolor[gray]{0.8}
Variable/Modelo & Real & tddpm\_mlp & smote-enc & ctgan \\
\hline top5 & [   0 2014 2005 2000 2003] & [   0.         2014.         2015.          557.16404191 2005.93527463] & [   0. 2014. 2005. 2006. 2013.] & [   0 2015    2    3    4] \\
\hline top5\_freq & [16571    76    32    30    29] & [20892    62    53     1     1] & [20753    23     4     3     2] & [7471 1690 1337 1320 1309] \\
\hline top5\_prob & [0.95841527 0.0043956  0.00185078 0.00173511 0.00167727] & [9.66640448e-01 2.86864387e-03 2.45222783e-03 4.62684495e-05
 4.62684495e-05] & [9.60164708e-01 1.06412510e-03 1.85065235e-04 1.38798927e-04
 9.25326177e-05] & [0.34567159 0.07819368 0.06186092 0.06107435 0.0605654 ] \\
\hline nobs & 17290 & \bfseries 21613 & \cellcolor[rgb]{0.9, 0.54, 0.52} 21614 & \bfseries 21613 \\
\hline missing & 0.000 & 0.000 & 0.000 & 0.000 \\
\hline mean & 83.003 & 66.202 & \bfseries 74.187 & \cellcolor[rgb]{0.9, 0.54, 0.52} 280.061 \\
\hline std\_err & 3.031 & 2.431 & \bfseries 2.533 & \cellcolor[rgb]{0.9, 0.54, 0.52} 4.664 \\
\hline upper\_ci & 88.943 & 70.967 & \bfseries 79.151 & \cellcolor[rgb]{0.9, 0.54, 0.52} 289.203 \\
\hline lower\_ci & 77.063 & 61.438 & \bfseries 69.222 & \cellcolor[rgb]{0.9, 0.54, 0.52} 270.920 \\
\hline std & 398.503 & 357.413 & \bfseries 372.384 & \cellcolor[rgb]{0.9, 0.54, 0.52} 685.680 \\
\hline iqr & 0.000 & \bfseries 0.000 & \bfseries 0.000 & \cellcolor[rgb]{0.9, 0.54, 0.52} 7.000 \\
\hline iqr\_normal & 0.000 & \bfseries 0.000 & \bfseries 0.000 & \cellcolor[rgb]{0.9, 0.54, 0.52} 5.189 \\
\hline mad & 159.103 & 127.988 & \bfseries 142.469 & \cellcolor[rgb]{0.9, 0.54, 0.52} 476.288 \\
\hline mad\_normal & 199.407 & 160.409 & \bfseries 178.559 & \cellcolor[rgb]{0.9, 0.54, 0.52} 596.939 \\
\hline coef\_var & 4.801 & 5.399 & \bfseries 5.020 & \cellcolor[rgb]{0.9, 0.54, 0.52} 2.448 \\
\hline range & 2015 & \bfseries 2015 & \cellcolor[rgb]{0.9, 0.54, 0.52} 2015 & \bfseries 2015 \\
\hline max & 2015 & \bfseries 2015 & \cellcolor[rgb]{0.9, 0.54, 0.52} 2015 & \bfseries 2015 \\
\hline min & 0.000 & 0.000 & 0.000 & 0.000 \\
\hline skew & 4.593 & 5.224 & \bfseries 4.886 & \cellcolor[rgb]{0.9, 0.54, 0.52} 2.074 \\
\hline kurtosis & 22.096 & 28.309 & \bfseries 25.024 & \cellcolor[rgb]{0.9, 0.54, 0.52} 5.308 \\
\hline jarque\_bera & 323506 & \cellcolor[rgb]{0.9, 0.54, 0.52} 675119 & \bfseries 522838 & 20287 \\
\hline jarque\_bera\_pval & 0.000 & 0.000 & 0.000 & 0.000 \\
\hline mode & 0.000 & 0.000 & 0.000 & 0.000 \\
\hline mode\_freq & 0.958 & 0.967 & \bfseries 0.960 & \cellcolor[rgb]{0.9, 0.54, 0.52} 0.346 \\
\hline median & 0.000 & \bfseries 0.000 & \bfseries 0.000 & \cellcolor[rgb]{0.9, 0.54, 0.52} 3.000 \\
\hline 0.1\% & 0.000 & 0.000 & 0.000 & 0.000 \\
\hline 1.0\% & 0.000 & 0.000 & 0.000 & 0.000 \\
\hline 5.0\% & 0.000 & 0.000 & 0.000 & 0.000 \\
\hline 25.0\% & 0.000 & 0.000 & 0.000 & 0.000 \\
\hline 75.0\% & 0.000 & \bfseries 0.000 & \bfseries 0.000 & \cellcolor[rgb]{0.9, 0.54, 0.52} 7.000 \\
\hline 95.0\% & 0.000 & \bfseries 0.000 & \bfseries 0.000 & \cellcolor[rgb]{0.9, 0.54, 0.52} 2015.000 \\
\hline 99.0\% & 2008 & \bfseries 2010 & 2005 & \cellcolor[rgb]{0.9, 0.54, 0.52} 2015 \\
\hline 99.9\% & 2014 & \cellcolor[rgb]{0.9, 0.54, 0.52} 2015 & \bfseries 2014 & \cellcolor[rgb]{0.9, 0.54, 0.52} 2015 \\
\hline
\end{tabular}
\end{table}

\begin{table}[H]
\centering
\fontsize{8}{14}\selectfont
\caption{Propiedades  estadisticas de variable sqft\_lot, King county (A-2)}
\label{table-stats-king county-a-2-sqft_lot}
\begin{tabular}{|l|m{10em}|m{10em}|m{10em}|m{10em}|}
\hline
 \rowcolor[gray]{0.8}
Variable/Modelo & Real & tddpm\_mlp & smote-enc & ctgan \\
\hline top5 & [5000 4000 6000 7200 4800] & [5000. 4000. 6000. 7200. 4500.] & [5000. 4000. 4080. 6000. 8000.] & [  520 12941 13597 12637 12255] \\
\hline top5\_freq & [301 209 208 179  98] & [356 248 230 195 109] & [56 40 22 17 16] & [321   9   9   7   7] \\
\hline top5\_prob & [0.01740891 0.01208791 0.01203008 0.01035281 0.00566802] & [0.01647157 0.01147458 0.01064174 0.00902235 0.00504326] & [0.00259091 0.00185065 0.00101786 0.00078653 0.00074026] & [0.01485217 0.00041642 0.00041642 0.00032388 0.00032388] \\
\hline nobs & 17290 & \bfseries 21613 & \cellcolor[rgb]{0.9, 0.54, 0.52} 21614 & \bfseries 21613 \\
\hline missing & 0.000 & 0.000 & 0.000 & 0.000 \\
\hline mean & 14799 & 16628 & \bfseries 14059 & \cellcolor[rgb]{0.9, 0.54, 0.52} 18126 \\
\hline std\_err & 295.375 & \cellcolor[rgb]{0.9, 0.54, 0.52} 611.901 & \bfseries 216.074 & 206.230 \\
\hline upper\_ci & 15378 & 17827 & \bfseries 14482 & \cellcolor[rgb]{0.9, 0.54, 0.52} 18530 \\
\hline lower\_ci & 14220 & 15428 & \bfseries 13635 & \cellcolor[rgb]{0.9, 0.54, 0.52} 17722 \\
\hline std & 38839 & \cellcolor[rgb]{0.9, 0.54, 0.52} 89958 & \bfseries 31766 & 30319 \\
\hline iqr & 5606 & 4905 & \bfseries 5584 & \cellcolor[rgb]{0.9, 0.54, 0.52} 7113 \\
\hline iqr\_normal & 4155 & 3636 & \bfseries 4139 & \cellcolor[rgb]{0.9, 0.54, 0.52} 5273 \\
\hline mad & 13382 & \cellcolor[rgb]{0.9, 0.54, 0.52} 17162 & 11949 & \bfseries 13481 \\
\hline mad\_normal & 16772 & \cellcolor[rgb]{0.9, 0.54, 0.52} 21510 & 14976 & \bfseries 16896 \\
\hline coef\_var & 2.624 & \cellcolor[rgb]{0.9, 0.54, 0.52} 5.410 & \bfseries 2.260 & 1.673 \\
\hline range & 1164274 & 1650839 & \bfseries 962591 & \cellcolor[rgb]{0.9, 0.54, 0.52} 333705 \\
\hline max & 1164794 & 1651359 & \bfseries 963282 & \cellcolor[rgb]{0.9, 0.54, 0.52} 334225 \\
\hline min & 520.000 & \bfseries 520.000 & \cellcolor[rgb]{0.9, 0.54, 0.52} 690.676 & \bfseries 520.000 \\
\hline skew & 11.588 & 16.678 & \bfseries 8.762 & \cellcolor[rgb]{0.9, 0.54, 0.52} 6.046 \\
\hline kurtosis & 215.591 & \bfseries 297.135 & 123.595 & \cellcolor[rgb]{0.9, 0.54, 0.52} 46.341 \\
\hline jarque\_bera & 32946220 & \cellcolor[rgb]{0.9, 0.54, 0.52} 78912817 & \bfseries 13373770 & 1823319 \\
\hline jarque\_bera\_pval & 0.000 & 0.000 & 0.000 & 0.000 \\
\hline mode & 5000 & \bfseries 5000 & \bfseries 5000 & \cellcolor[rgb]{0.9, 0.54, 0.52} 520 \\
\hline mode\_freq & 0.017 & \bfseries 0.016 & \cellcolor[rgb]{0.9, 0.54, 0.52} 0.003 & 0.015 \\
\hline median & 7600 & \bfseries 7491 & 7769 & \cellcolor[rgb]{0.9, 0.54, 0.52} 11577 \\
\hline 0.1\% & 737.156 & \bfseries 812.333 & 833.904 & \cellcolor[rgb]{0.9, 0.54, 0.52} 520.000 \\
\hline 1.0\% & 1005 & \bfseries 1019 & 1067 & \cellcolor[rgb]{0.9, 0.54, 0.52} 520 \\
\hline 5.0\% & 1756 & 1589 & \bfseries 1773 & \cellcolor[rgb]{0.9, 0.54, 0.52} 2927 \\
\hline 25.0\% & 5001 & \bfseries 5000 & 5136 & \cellcolor[rgb]{0.9, 0.54, 0.52} 7973 \\
\hline 75.0\% & 10607 & 9905 & \bfseries 10720 & \cellcolor[rgb]{0.9, 0.54, 0.52} 15086 \\
\hline 95.0\% & 42999 & 36194 & \bfseries 40503 & \cellcolor[rgb]{0.9, 0.54, 0.52} 56973 \\
\hline 99.0\% & 212192 & \cellcolor[rgb]{0.9, 0.54, 0.52} 181619 & 191877 & \bfseries 196506 \\
\hline 99.9\% & 435600 & \cellcolor[rgb]{0.9, 0.54, 0.52} 1651359 & \bfseries 377395 & 300056 \\
\hline
\end{tabular}
\end{table}

\begin{table}[H]
\centering
\fontsize{8}{14}\selectfont
\caption{Propiedades  estadisticas de variable view, King county (A-2)}
\label{table-stats-king county-a-2-view}
\begin{tabular}{|l|m{10em}|m{10em}|m{10em}|m{10em}|}
\hline
 \rowcolor[gray]{0.8}
Variable/Modelo & Real & tddpm\_mlp & smote-enc & ctgan \\
\hline top5 & [0 2 3 1 4] & [0 2 3 1 4] & [0 2 3 4 1] & [0 2 4 3 1] \\
\hline top5\_freq & [15586   783   396   275   250] & [20619   514   224   147   109] & [20891   336   176   140    71] & [18103  1617   748   604   541] \\
\hline top5\_prob & [0.90144592 0.04528629 0.02290341 0.01590515 0.01445922] & [0.95400916 0.02378198 0.01036413 0.00680146 0.00504326] & [0.96654946 0.01554548 0.00814287 0.00647728 0.00328491] & [0.83759774 0.07481608 0.0346088  0.02794614 0.02503123] \\
\hline nobs & 17290 & \bfseries 21613 & \cellcolor[rgb]{0.9, 0.54, 0.52} 21614 & \bfseries 21613 \\
\hline missing & 0.000 & 0.000 & 0.000 & 0.000 \\
\hline mean & 0.233 & \bfseries 0.106 & 0.085 & \cellcolor[rgb]{0.9, 0.54, 0.52} 0.397 \\
\hline std\_err & 0.006 & 0.003 & \cellcolor[rgb]{0.9, 0.54, 0.52} 0.003 & \bfseries 0.007 \\
\hline upper\_ci & 0.244 & \bfseries 0.112 & 0.091 & \cellcolor[rgb]{0.9, 0.54, 0.52} 0.410 \\
\hline lower\_ci & 0.222 & \bfseries 0.099 & 0.078 & \cellcolor[rgb]{0.9, 0.54, 0.52} 0.384 \\
\hline std & 0.762 & 0.515 & \cellcolor[rgb]{0.9, 0.54, 0.52} 0.485 & \bfseries 0.986 \\
\hline iqr & 0.000 & 0.000 & 0.000 & 0.000 \\
\hline iqr\_normal & 0.000 & 0.000 & 0.000 & 0.000 \\
\hline mad & 0.420 & \bfseries 0.202 & \cellcolor[rgb]{0.9, 0.54, 0.52} 0.164 & 0.665 \\
\hline mad\_normal & 0.527 & \bfseries 0.253 & \cellcolor[rgb]{0.9, 0.54, 0.52} 0.205 & 0.833 \\
\hline coef\_var & 3.269 & 4.871 & \cellcolor[rgb]{0.9, 0.54, 0.52} 5.725 & \bfseries 2.484 \\
\hline range & 4.000 & 4.000 & 4.000 & 4.000 \\
\hline max & 4.000 & 4.000 & 4.000 & 4.000 \\
\hline min & 0.000 & 0.000 & 0.000 & 0.000 \\
\hline skew & 3.402 & 5.246 & \cellcolor[rgb]{0.9, 0.54, 0.52} 6.151 & \bfseries 2.476 \\
\hline kurtosis & 13.971 & 31.362 & \cellcolor[rgb]{0.9, 0.54, 0.52} 41.967 & \bfseries 8.080 \\
\hline jarque\_bera & 120072 & 823540 & \cellcolor[rgb]{0.9, 0.54, 0.52} 1503760 & \bfseries 45333 \\
\hline jarque\_bera\_pval & 0.000 & 0.000 & 0.000 & 0.000 \\
\hline mode & 0.000 & 0.000 & 0.000 & 0.000 \\
\hline mode\_freq & 0.901 & \bfseries 0.954 & \cellcolor[rgb]{0.9, 0.54, 0.52} 0.967 & 0.838 \\
\hline median & 0.000 & 0.000 & 0.000 & 0.000 \\
\hline 0.1\% & 0.000 & 0.000 & 0.000 & 0.000 \\
\hline 1.0\% & 0.000 & 0.000 & 0.000 & 0.000 \\
\hline 5.0\% & 0.000 & 0.000 & 0.000 & 0.000 \\
\hline 25.0\% & 0.000 & 0.000 & 0.000 & 0.000 \\
\hline 75.0\% & 0.000 & 0.000 & 0.000 & 0.000 \\
\hline 95.0\% & 2.000 & \cellcolor[rgb]{0.9, 0.54, 0.52} 0.000 & \cellcolor[rgb]{0.9, 0.54, 0.52} 0.000 & \bfseries 3.000 \\
\hline 99.0\% & 4.000 & \cellcolor[rgb]{0.9, 0.54, 0.52} 3.000 & \cellcolor[rgb]{0.9, 0.54, 0.52} 3.000 & \bfseries 4.000 \\
\hline 99.9\% & 4.000 & 4.000 & 4.000 & 4.000 \\
\hline
\end{tabular}
\end{table}

\begin{table}[H]
\centering
\fontsize{8}{14}\selectfont
\caption{Propiedades  estadisticas de variable floors, King county (A-2)}
\label{table-stats-king county-a-2-floors}
\begin{tabular}{|l|m{10em}|m{10em}|m{10em}|m{10em}|}
\hline
 \rowcolor[gray]{0.8}
Variable/Modelo & Real & tddpm\_mlp & smote-enc & ctgan \\
\hline top5 & [1.  2.  1.5 3.  2.5] & [1.  2.  1.5 3.  2.5] & [1.  2.  1.5 3.  2.5] & [1.  2.  1.5 3.  2.5] \\
\hline top5\_freq & [8488 6628 1523  517  128] & [11201  8302  1488   584    36] & [11267  8351  1329   623    44] & [12847  4573  2944   693   398] \\
\hline top5\_prob & [0.49091961 0.38334297 0.0880856  0.02990168 0.00740312] & [0.5182529  0.38412067 0.06884745 0.02702077 0.00166566] & [0.5212825  0.38636995 0.06148792 0.02882391 0.00203572] & [0.59441077 0.21158562 0.13621432 0.03206404 0.01841484] \\
\hline nobs & 17290 & \bfseries 21613 & \cellcolor[rgb]{0.9, 0.54, 0.52} 21614 & \bfseries 21613 \\
\hline missing & 0.000 & 0.000 & 0.000 & 0.000 \\
\hline mean & 1.499 & 1.475 & \bfseries 1.478 & \cellcolor[rgb]{0.9, 0.54, 0.52} 1.390 \\
\hline std\_err & 0.004 & \cellcolor[rgb]{0.9, 0.54, 0.52} 0.004 & 0.004 & \bfseries 0.004 \\
\hline upper\_ci & 1.507 & 1.482 & \bfseries 1.485 & \cellcolor[rgb]{0.9, 0.54, 0.52} 1.397 \\
\hline lower\_ci & 1.491 & 1.468 & \bfseries 1.471 & \cellcolor[rgb]{0.9, 0.54, 0.52} 1.382 \\
\hline std & 0.543 & 0.536 & \bfseries 0.542 & \cellcolor[rgb]{0.9, 0.54, 0.52} 0.556 \\
\hline iqr & 1.000 & 1.000 & 1.000 & 1.000 \\
\hline iqr\_normal & 0.741 & 0.741 & 0.741 & 0.741 \\
\hline mad & 0.490 & \bfseries 0.493 & 0.498 & \cellcolor[rgb]{0.9, 0.54, 0.52} 0.463 \\
\hline mad\_normal & 0.614 & \bfseries 0.617 & 0.624 & \cellcolor[rgb]{0.9, 0.54, 0.52} 0.581 \\
\hline coef\_var & 0.362 & \bfseries 0.364 & 0.366 & \cellcolor[rgb]{0.9, 0.54, 0.52} 0.400 \\
\hline range & 2.500 & \bfseries 2.500 & \cellcolor[rgb]{0.9, 0.54, 0.52} 2.000 & \bfseries 2.500 \\
\hline max & 3.500 & \bfseries 3.500 & \cellcolor[rgb]{0.9, 0.54, 0.52} 3.000 & \bfseries 3.500 \\
\hline min & 1.000 & 1.000 & 1.000 & 1.000 \\
\hline skew & 0.615 & \bfseries 0.636 & 0.642 & \cellcolor[rgb]{0.9, 0.54, 0.52} 1.401 \\
\hline kurtosis & 2.526 & \bfseries 2.474 & 2.474 & \cellcolor[rgb]{0.9, 0.54, 0.52} 4.517 \\
\hline jarque\_bera & 1252 & \bfseries 1705 & 1734 & \cellcolor[rgb]{0.9, 0.54, 0.52} 9146 \\
\hline jarque\_bera\_pval & 0.000 & 0.000 & 0.000 & 0.000 \\
\hline mode & 1.000 & 1.000 & 1.000 & 1.000 \\
\hline mode\_freq & 0.491 & \bfseries 0.518 & 0.521 & \cellcolor[rgb]{0.9, 0.54, 0.52} 0.594 \\
\hline median & 1.500 & 1.000 & 1.000 & 1.000 \\
\hline 0.1\% & 1.000 & 1.000 & 1.000 & 1.000 \\
\hline 1.0\% & 1.000 & 1.000 & 1.000 & 1.000 \\
\hline 5.0\% & 1.000 & 1.000 & 1.000 & 1.000 \\
\hline 25.0\% & 1.000 & 1.000 & 1.000 & 1.000 \\
\hline 75.0\% & 2.000 & 2.000 & 2.000 & 2.000 \\
\hline 95.0\% & 2.000 & \bfseries 2.000 & \bfseries 2.000 & \cellcolor[rgb]{0.9, 0.54, 0.52} 2.500 \\
\hline 99.0\% & 3.000 & 3.000 & 3.000 & 3.000 \\
\hline 99.9\% & 3.000 & \bfseries 3.000 & \bfseries 3.000 & \cellcolor[rgb]{0.9, 0.54, 0.52} 3.500 \\
\hline
\end{tabular}
\end{table}

\begin{table}[H]
\centering
\fontsize{8}{14}\selectfont
\caption{Propiedades  estadisticas de variable bedrooms, King county (A-2)}
\label{table-stats-king county-a-2-bedrooms}
\begin{tabular}{|l|m{10em}|m{10em}|m{10em}|m{10em}|}
\hline
 \rowcolor[gray]{0.8}
Variable/Modelo & Real & tddpm\_mlp & smote-enc & ctgan \\
\hline top5 & [3 4 2 5 6] & [3 4 2 5 6] & [3 4 2 5 1] & [4 3 2 5 6] \\
\hline top5\_freq & [7865 5477 2237 1292  212] & [10728  6929  2646  1132    86] & [11430  7061  2430   621    47] & [8624 4895 4079 1578  679] \\
\hline top5\_prob & [0.45488722 0.3167727  0.12938115 0.07472527 0.01226142] & [0.49636793 0.32059409 0.12242632 0.05237588 0.00397909] & [0.52882391 0.32668641 0.11242713 0.02873138 0.00217452] & [0.39901911 0.22648406 0.18872901 0.07301161 0.03141628] \\
\hline nobs & 17290 & \bfseries 21613 & \cellcolor[rgb]{0.9, 0.54, 0.52} 21614 & \bfseries 21613 \\
\hline missing & 0.000 & 0.000 & 0.000 & 0.000 \\
\hline mean & 3.368 & \bfseries 3.307 & 3.271 & \cellcolor[rgb]{0.9, 0.54, 0.52} 3.975 \\
\hline std\_err & 0.007 & \bfseries 0.005 & 0.005 & \cellcolor[rgb]{0.9, 0.54, 0.52} 0.025 \\
\hline upper\_ci & 3.382 & \bfseries 3.318 & 3.280 & \cellcolor[rgb]{0.9, 0.54, 0.52} 4.025 \\
\hline lower\_ci & 3.354 & \bfseries 3.297 & 3.261 & \cellcolor[rgb]{0.9, 0.54, 0.52} 3.925 \\
\hline std & 0.931 & \bfseries 0.785 & 0.707 & \cellcolor[rgb]{0.9, 0.54, 0.52} 3.732 \\
\hline iqr & 1.000 & 1.000 & 1.000 & 1.000 \\
\hline iqr\_normal & 0.741 & 0.741 & 0.741 & 0.741 \\
\hline mad & 0.734 & \bfseries 0.645 & 0.582 & \cellcolor[rgb]{0.9, 0.54, 0.52} 1.395 \\
\hline mad\_normal & 0.920 & \bfseries 0.808 & 0.730 & \cellcolor[rgb]{0.9, 0.54, 0.52} 1.748 \\
\hline coef\_var & 0.277 & \bfseries 0.237 & 0.216 & \cellcolor[rgb]{0.9, 0.54, 0.52} 0.939 \\
\hline range & 33.000 & 9.000 & \cellcolor[rgb]{0.9, 0.54, 0.52} 5.000 & \bfseries 33.000 \\
\hline max & 33.000 & 9.000 & \cellcolor[rgb]{0.9, 0.54, 0.52} 6.000 & \bfseries 33.000 \\
\hline min & 0.000 & \bfseries 0.000 & \cellcolor[rgb]{0.9, 0.54, 0.52} 1.000 & \bfseries 0.000 \\
\hline skew & 2.304 & \bfseries 0.243 & 0.091 & \cellcolor[rgb]{0.9, 0.54, 0.52} 6.612 \\
\hline kurtosis & 63.268 & 3.485 & \cellcolor[rgb]{0.9, 0.54, 0.52} 3.072 & \bfseries 51.539 \\
\hline jarque\_bera & 2631992 & 424 & \cellcolor[rgb]{0.9, 0.54, 0.52} 34 & \bfseries 2279187 \\
\hline jarque\_bera\_pval & 0.000 & 0.000 & \cellcolor[rgb]{0.9, 0.54, 0.52} 0.000 & \bfseries 0.000 \\
\hline mode & 3.000 & \bfseries 3.000 & \bfseries 3.000 & \cellcolor[rgb]{0.9, 0.54, 0.52} 4.000 \\
\hline mode\_freq & 0.455 & \bfseries 0.496 & \cellcolor[rgb]{0.9, 0.54, 0.52} 0.529 & 0.399 \\
\hline median & 3.000 & \bfseries 3.000 & \bfseries 3.000 & \cellcolor[rgb]{0.9, 0.54, 0.52} 4.000 \\
\hline 0.1\% & 1.000 & \bfseries 1.000 & \bfseries 1.000 & \cellcolor[rgb]{0.9, 0.54, 0.52} 0.000 \\
\hline 1.0\% & 2.000 & \bfseries 2.000 & \bfseries 2.000 & \cellcolor[rgb]{0.9, 0.54, 0.52} 0.000 \\
\hline 5.0\% & 2.000 & 2.000 & 2.000 & 2.000 \\
\hline 25.0\% & 3.000 & 3.000 & 3.000 & 3.000 \\
\hline 75.0\% & 4.000 & 4.000 & 4.000 & 4.000 \\
\hline 95.0\% & 5.000 & \bfseries 5.000 & 4.000 & \cellcolor[rgb]{0.9, 0.54, 0.52} 7.000 \\
\hline 99.0\% & 6.000 & \bfseries 5.000 & \bfseries 5.000 & \cellcolor[rgb]{0.9, 0.54, 0.52} 33.000 \\
\hline 99.9\% & 7.000 & \bfseries 6.000 & \bfseries 6.000 & \cellcolor[rgb]{0.9, 0.54, 0.52} 33.000 \\
\hline
\end{tabular}
\end{table}

\begin{table}[H]
\centering
\fontsize{8}{14}\selectfont
\caption{Propiedades  estadisticas de variable zipcode, King county (A-2)}
\label{table-stats-king county-a-2-zipcode}
\begin{tabular}{|l|m{10em}|m{10em}|m{10em}|m{10em}|}
\hline
 \rowcolor[gray]{0.8}
Variable/Modelo & Real & tddpm\_mlp & smote-enc & ctgan \\
\hline top5 & [98103 98038 98115 98052 98117] & [98052 98103 98038 98115 98042] & [98103 98115 98052 98038 98117] & [98034 98118 98006 98023 98103] \\
\hline top5\_freq & [489 473 462 459 455] & [714 658 645 634 601] & [650 623 574 570 557] & [804 711 638 630 612] \\
\hline top5\_prob & [0.02828224 0.02735685 0.02672065 0.02654714 0.02631579] & [0.03303567 0.03044464 0.02984315 0.0293342  0.02780734] & [0.0300731  0.02882391 0.02655686 0.0263718  0.02577033] & [0.03719983 0.03289687 0.02951927 0.02914912 0.02831629] \\
\hline nobs & 17290 & \bfseries 21613 & \cellcolor[rgb]{0.9, 0.54, 0.52} 21614 & \bfseries 21613 \\
\hline missing & 0.000 & 0.000 & 0.000 & 0.000 \\
\hline mean & 98078 & 98077 & \bfseries 98078 & \cellcolor[rgb]{0.9, 0.54, 0.52} 98079 \\
\hline std\_err & 0.406 & \cellcolor[rgb]{0.9, 0.54, 0.52} 0.360 & 0.363 & \bfseries 0.370 \\
\hline upper\_ci & 98079 & 98078 & \bfseries 98079 & \cellcolor[rgb]{0.9, 0.54, 0.52} 98080 \\
\hline lower\_ci & 98077 & 98077 & \bfseries 98077 & \cellcolor[rgb]{0.9, 0.54, 0.52} 98078 \\
\hline std & 53.326 & 52.955 & \bfseries 53.304 & \cellcolor[rgb]{0.9, 0.54, 0.52} 54.353 \\
\hline iqr & 84.000 & \bfseries 84.000 & \bfseries 84.000 & \cellcolor[rgb]{0.9, 0.54, 0.52} 85.000 \\
\hline iqr\_normal & 62.269 & \bfseries 62.269 & \bfseries 62.269 & \cellcolor[rgb]{0.9, 0.54, 0.52} 63.011 \\
\hline mad & 46.554 & 46.116 & \bfseries 46.585 & \cellcolor[rgb]{0.9, 0.54, 0.52} 47.661 \\
\hline mad\_normal & 58.347 & 57.798 & \bfseries 58.385 & \cellcolor[rgb]{0.9, 0.54, 0.52} 59.734 \\
\hline coef\_var & 0.001 & 0.001 & \bfseries 0.001 & \cellcolor[rgb]{0.9, 0.54, 0.52} 0.001 \\
\hline range & 198.000 & 198.000 & 198.000 & 198.000 \\
\hline max & 98199 & 98199 & 98199 & 98199 \\
\hline min & 98001 & 98001 & 98001 & 98001 \\
\hline skew & 0.402 & \cellcolor[rgb]{0.9, 0.54, 0.52} 0.423 & \bfseries 0.390 & 0.417 \\
\hline kurtosis & 2.153 & 2.182 & \bfseries 2.142 & \cellcolor[rgb]{0.9, 0.54, 0.52} 2.101 \\
\hline jarque\_bera & 983.027 & 1246.547 & \bfseries 1210.461 & \cellcolor[rgb]{0.9, 0.54, 0.52} 1354.816 \\
\hline jarque\_bera\_pval & 0.000 & 0.000 & 0.000 & 0.000 \\
\hline mode & 98103 & 98052 & \bfseries 98103 & \cellcolor[rgb]{0.9, 0.54, 0.52} 98034 \\
\hline mode\_freq & 0.028 & 0.033 & \bfseries 0.030 & \cellcolor[rgb]{0.9, 0.54, 0.52} 0.037 \\
\hline median & 98065 & \bfseries 98065 & \cellcolor[rgb]{0.9, 0.54, 0.52} 98070 & \cellcolor[rgb]{0.9, 0.54, 0.52} 98070 \\
\hline 0.1\% & 98001 & 98001 & 98001 & 98001 \\
\hline 1.0\% & 98001 & 98001 & 98001 & 98001 \\
\hline 5.0\% & 98004 & \bfseries 98004 & \bfseries 98004 & \cellcolor[rgb]{0.9, 0.54, 0.52} 98005 \\
\hline 25.0\% & 98033 & 98033 & 98033 & 98033 \\
\hline 75.0\% & 98117 & \bfseries 98117 & \bfseries 98117 & \cellcolor[rgb]{0.9, 0.54, 0.52} 98118 \\
\hline 95.0\% & 98177 & \bfseries 98177 & \bfseries 98177 & \cellcolor[rgb]{0.9, 0.54, 0.52} 98178 \\
\hline 99.0\% & 98199 & 98199 & 98199 & 98199 \\
\hline 99.9\% & 98199 & 98199 & 98199 & 98199 \\
\hline
\end{tabular}
\end{table}

\begin{table}[H]
\centering
\fontsize{8}{14}\selectfont
\caption{Propiedades  estadisticas de variable bathrooms, King county (A-2)}
\label{table-stats-king county-a-2-bathrooms}
\begin{tabular}{|l|m{10em}|m{10em}|m{10em}|m{10em}|}
\hline
 \rowcolor[gray]{0.8}
Variable/Modelo & Real & tddpm\_mlp & smote-enc & ctgan \\
\hline top5 & [2.5  1.   1.75 2.25 2.  ] & [2.5  1.   1.75 2.25 2.  ] & [2.5  1.   1.75 2.25 2.  ] & [2.5  1.75 1.   2.75 3.25] \\
\hline top5\_freq & [4333 3088 2425 1621 1526] & [6256 3998 3308 2019 1764] & [6751 4830 3303 1979 1320] & [4571 3423 2963 2032 1228] \\
\hline top5\_prob & [0.25060729 0.17860035 0.14025448 0.09375361 0.08825911] & [0.28945542 0.18498126 0.15305603 0.093416   0.08161754] & [0.31234385 0.22346627 0.15281762 0.09156103 0.06107153] & [0.21149308 0.1583769  0.13709342 0.09401749 0.05681766] \\
\hline nobs & 17290 & \bfseries 21613 & \cellcolor[rgb]{0.9, 0.54, 0.52} 21614 & \bfseries 21613 \\
\hline missing & 0.000 & 0.000 & 0.000 & 0.000 \\
\hline mean & 2.114 & \bfseries 2.080 & 2.011 & \cellcolor[rgb]{0.9, 0.54, 0.52} 2.291 \\
\hline std\_err & 0.006 & 0.005 & \cellcolor[rgb]{0.9, 0.54, 0.52} 0.005 & \bfseries 0.006 \\
\hline upper\_ci & 2.125 & \bfseries 2.089 & 2.020 & \cellcolor[rgb]{0.9, 0.54, 0.52} 2.304 \\
\hline lower\_ci & 2.102 & \bfseries 2.070 & 2.001 & \cellcolor[rgb]{0.9, 0.54, 0.52} 2.279 \\
\hline std & 0.767 & \bfseries 0.713 & 0.706 & \cellcolor[rgb]{0.9, 0.54, 0.52} 0.916 \\
\hline iqr & 1.000 & \cellcolor[rgb]{0.9, 0.54, 0.52} 0.750 & \bfseries 1.000 & \bfseries 1.000 \\
\hline iqr\_normal & 0.741 & \cellcolor[rgb]{0.9, 0.54, 0.52} 0.556 & \bfseries 0.741 & \bfseries 0.741 \\
\hline mad & 0.615 & 0.582 & \bfseries 0.592 & \cellcolor[rgb]{0.9, 0.54, 0.52} 0.728 \\
\hline mad\_normal & 0.771 & 0.730 & \bfseries 0.742 & \cellcolor[rgb]{0.9, 0.54, 0.52} 0.913 \\
\hline coef\_var & 0.363 & 0.343 & \bfseries 0.351 & \cellcolor[rgb]{0.9, 0.54, 0.52} 0.400 \\
\hline range & 8.000 & 7.750 & \cellcolor[rgb]{0.9, 0.54, 0.52} 4.750 & \bfseries 8.000 \\
\hline max & 8.000 & 7.750 & \cellcolor[rgb]{0.9, 0.54, 0.52} 5.500 & \bfseries 8.000 \\
\hline min & 0.000 & \bfseries 0.000 & \cellcolor[rgb]{0.9, 0.54, 0.52} 0.750 & \bfseries 0.000 \\
\hline skew & 0.464 & 0.234 & \cellcolor[rgb]{0.9, 0.54, 0.52} 0.146 & \bfseries 0.441 \\
\hline kurtosis & 3.989 & 3.439 & \cellcolor[rgb]{0.9, 0.54, 0.52} 2.843 & \bfseries 3.888 \\
\hline jarque\_bera & 1326 & 371 & \cellcolor[rgb]{0.9, 0.54, 0.52} 99 & \bfseries 1409 \\
\hline jarque\_bera\_pval & 0.000 & 0.000 & \cellcolor[rgb]{0.9, 0.54, 0.52} 0.000 & \bfseries 0.000 \\
\hline mode & 2.500 & 2.500 & 2.500 & 2.500 \\
\hline mode\_freq & 0.251 & \bfseries 0.289 & \cellcolor[rgb]{0.9, 0.54, 0.52} 0.312 & 0.211 \\
\hline median & 2.250 & \bfseries 2.250 & \bfseries 2.250 & \cellcolor[rgb]{0.9, 0.54, 0.52} 2.500 \\
\hline 0.1\% & 0.750 & \bfseries 1.000 & \bfseries 1.000 & \cellcolor[rgb]{0.9, 0.54, 0.52} 0.000 \\
\hline 1.0\% & 1.000 & \bfseries 1.000 & \bfseries 1.000 & \cellcolor[rgb]{0.9, 0.54, 0.52} 0.750 \\
\hline 5.0\% & 1.000 & 1.000 & 1.000 & 1.000 \\
\hline 25.0\% & 1.500 & \cellcolor[rgb]{0.9, 0.54, 0.52} 1.750 & \bfseries 1.500 & \cellcolor[rgb]{0.9, 0.54, 0.52} 1.750 \\
\hline 75.0\% & 2.500 & \bfseries 2.500 & \bfseries 2.500 & \cellcolor[rgb]{0.9, 0.54, 0.52} 2.750 \\
\hline 95.0\% & 3.500 & \bfseries 3.250 & \bfseries 3.250 & \cellcolor[rgb]{0.9, 0.54, 0.52} 4.000 \\
\hline 99.0\% & 4.250 & \bfseries 3.750 & \cellcolor[rgb]{0.9, 0.54, 0.52} 3.500 & \bfseries 4.750 \\
\hline 99.9\% & 5.428 & \bfseries 5.000 & 4.750 & \cellcolor[rgb]{0.9, 0.54, 0.52} 6.250 \\
\hline
\end{tabular}
\end{table}

\begin{table}[H]
\centering
\fontsize{8}{14}\selectfont
\caption{Propiedades  estadisticas de variable sqft\_living15, King county (A-2)}
\label{table-stats-king county-a-2-sqft_living15}
\begin{tabular}{|l|m{10em}|m{10em}|m{10em}|m{10em}|}
\hline
 \rowcolor[gray]{0.8}
Variable/Modelo & Real & tddpm\_mlp & smote-enc & ctgan \\
\hline top5 & [1440 1540 1560 1500 1610] & [1440. 1550. 1540. 1560. 1530.] & [1440. 1830. 1670. 1078. 2370.] & [1827 1696 1594 1790 1621] \\
\hline top5\_freq & [156 154 152 137 136] & [200 183 165 162 162] & [24 22 21 18 17] & [28 26 25 25 24] \\
\hline top5\_prob & [0.00902256 0.00890688 0.00879121 0.00792366 0.00786582] & [0.00925369 0.00846713 0.00763429 0.00749549 0.00749549] & [0.00111039 0.00101786 0.00097159 0.00083279 0.00078653] & [0.00129552 0.00120298 0.00115671 0.00115671 0.00111044] \\
\hline nobs & 17290 & \bfseries 21613 & \cellcolor[rgb]{0.9, 0.54, 0.52} 21614 & \bfseries 21613 \\
\hline missing & 0.000 & 0.000 & 0.000 & 0.000 \\
\hline mean & 1983 & 1971 & \bfseries 1974 & \cellcolor[rgb]{0.9, 0.54, 0.52} 1952 \\
\hline std\_err & 5.181 & \cellcolor[rgb]{0.9, 0.54, 0.52} 4.381 & 4.422 & \bfseries 5.044 \\
\hline upper\_ci & 1993 & 1980 & \bfseries 1982 & \cellcolor[rgb]{0.9, 0.54, 0.52} 1962 \\
\hline lower\_ci & 1973 & 1962 & \bfseries 1965 & \cellcolor[rgb]{0.9, 0.54, 0.52} 1942 \\
\hline std & 681.232 & 644.021 & \bfseries 650.115 & \cellcolor[rgb]{0.9, 0.54, 0.52} 741.571 \\
\hline iqr & 880.000 & \cellcolor[rgb]{0.9, 0.54, 0.52} 827.821 & 832.491 & \bfseries 853.000 \\
\hline iqr\_normal & 652.345 & \cellcolor[rgb]{0.9, 0.54, 0.52} 613.665 & 617.127 & \bfseries 632.330 \\
\hline mad & 533.237 & \cellcolor[rgb]{0.9, 0.54, 0.52} 504.400 & \bfseries 509.430 & 557.482 \\
\hline mad\_normal & 668.313 & \cellcolor[rgb]{0.9, 0.54, 0.52} 632.171 & \bfseries 638.476 & 698.700 \\
\hline coef\_var & 0.344 & 0.327 & \bfseries 0.329 & \cellcolor[rgb]{0.9, 0.54, 0.52} 0.380 \\
\hline range & 5811 & 5811 & \cellcolor[rgb]{0.9, 0.54, 0.52} 5320 & \bfseries 5811 \\
\hline max & 6210 & 6210 & \cellcolor[rgb]{0.9, 0.54, 0.52} 5993 & \bfseries 6210 \\
\hline min & 399.000 & \bfseries 399.000 & \cellcolor[rgb]{0.9, 0.54, 0.52} 672.953 & \bfseries 399.000 \\
\hline skew & 1.095 & \cellcolor[rgb]{0.9, 0.54, 0.52} 1.067 & \bfseries 1.083 & 1.083 \\
\hline kurtosis & 4.572 & \bfseries 4.604 & 4.443 & \cellcolor[rgb]{0.9, 0.54, 0.52} 5.613 \\
\hline jarque\_bera & 5237 & 6416 & \bfseries 6104 & \cellcolor[rgb]{0.9, 0.54, 0.52} 10375 \\
\hline jarque\_bera\_pval & 0.000 & 0.000 & 0.000 & 0.000 \\
\hline mode & 1440 & \bfseries 1440 & 1440 & \cellcolor[rgb]{0.9, 0.54, 0.52} 1827 \\
\hline mode\_freq & 0.009 & \bfseries 0.009 & \cellcolor[rgb]{0.9, 0.54, 0.52} 0.001 & 0.001 \\
\hline median & 1840 & 1830 & \bfseries 1835 & \cellcolor[rgb]{0.9, 0.54, 0.52} 1859 \\
\hline 0.1\% & 740.000 & \cellcolor[rgb]{0.9, 0.54, 0.52} 399.000 & \bfseries 791.697 & 468.060 \\
\hline 1.0\% & 950.000 & \bfseries 980.000 & 988.057 & \cellcolor[rgb]{0.9, 0.54, 0.52} 648.120 \\
\hline 5.0\% & 1140 & 1179 & \bfseries 1166 & \cellcolor[rgb]{0.9, 0.54, 0.52} 899 \\
\hline 25.0\% & 1480 & \cellcolor[rgb]{0.9, 0.54, 0.52} 1502 & 1495 & \bfseries 1469 \\
\hline 75.0\% & 2360 & \bfseries 2330 & 2328 & \cellcolor[rgb]{0.9, 0.54, 0.52} 2322 \\
\hline 95.0\% & 3280 & \cellcolor[rgb]{0.9, 0.54, 0.52} 3204 & 3222 & \bfseries 3307 \\
\hline 99.0\% & 4050 & 3868 & \bfseries 3939 & \cellcolor[rgb]{0.9, 0.54, 0.52} 4270 \\
\hline 99.9\% & 4986 & \bfseries 4889 & 4802 & \cellcolor[rgb]{0.9, 0.54, 0.52} 5989 \\
\hline
\end{tabular}
\end{table}

\begin{table}[H]
\centering
\fontsize{8}{14}\selectfont
\caption{Propiedades  estadisticas de variable price, King county (A-2)}
\label{table-stats-king county-a-2-price}
\begin{tabular}{|l|m{10em}|m{10em}|m{10em}|m{10em}|}
\hline
 \rowcolor[gray]{0.8}
Variable/Modelo & Real & tddpm\_mlp & smote-enc & ctgan \\
\hline top5 & [350000. 450000. 425000. 550000. 325000.] & [450000. 525000. 300000. 350000. 550000.] & [350000. 450000. 325000. 550000. 500000.] & [ 75000. 347056. 209754. 370839. 905913.] \\
\hline top5\_freq & [143 140 123 123 123] & [190 138 135 135 134] & [213 191 186 184 182] & [602   3   3   2   2] \\
\hline top5\_prob & [0.00827068 0.00809717 0.00711394 0.00711394 0.00711394] & [0.00879101 0.00638505 0.00624624 0.00624624 0.00619997] & [0.00985472 0.00883686 0.00860553 0.008513   0.00842047] & [2.78536066e-02 1.38805349e-04 1.38805349e-04 9.25368991e-05
 9.25368991e-05] \\
\hline nobs & 17290 & \bfseries 21613 & \cellcolor[rgb]{0.9, 0.54, 0.52} 21614 & \bfseries 21613 \\
\hline missing & 0.000 & 0.000 & 0.000 & 0.000 \\
\hline mean & 537768 & \bfseries 528328 & 519642 & \cellcolor[rgb]{0.9, 0.54, 0.52} 498555 \\
\hline std\_err & 2749 & 2828 & \cellcolor[rgb]{0.9, 0.54, 0.52} 2097 & \bfseries 2784 \\
\hline upper\_ci & 543156 & \bfseries 533872 & 523753 & \cellcolor[rgb]{0.9, 0.54, 0.52} 504011 \\
\hline lower\_ci & 532380 & \bfseries 522785 & 515532 & \cellcolor[rgb]{0.9, 0.54, 0.52} 493099 \\
\hline std & 361464 & \cellcolor[rgb]{0.9, 0.54, 0.52} 415801 & 308334 & \bfseries 409263 \\
\hline iqr & 319850 & 302163 & \bfseries 305500 & \cellcolor[rgb]{0.9, 0.54, 0.52} 359276 \\
\hline iqr\_normal & 237105 & 223994 & \bfseries 226467 & \cellcolor[rgb]{0.9, 0.54, 0.52} 266332 \\
\hline mad & 231680 & \bfseries 218397 & 210266 & \cellcolor[rgb]{0.9, 0.54, 0.52} 265501 \\
\hline mad\_normal & 290368 & \bfseries 273720 & 263529 & \cellcolor[rgb]{0.9, 0.54, 0.52} 332756 \\
\hline coef\_var & 0.672 & 0.787 & \bfseries 0.593 & \cellcolor[rgb]{0.9, 0.54, 0.52} 0.821 \\
\hline range & 7625000 & \bfseries 7625000 & \cellcolor[rgb]{0.9, 0.54, 0.52} 4130000 & 4404475 \\
\hline max & 7700000 & \bfseries 7700000 & \cellcolor[rgb]{0.9, 0.54, 0.52} 4208000 & 4479475 \\
\hline min & 75000 & \bfseries 75000 & \cellcolor[rgb]{0.9, 0.54, 0.52} 78000 & \bfseries 75000 \\
\hline skew & 4.032 & \cellcolor[rgb]{0.9, 0.54, 0.52} 9.124 & 2.762 & \bfseries 2.979 \\
\hline kurtosis & 39.678 & \cellcolor[rgb]{0.9, 0.54, 0.52} 141.899 & \bfseries 16.937 & 16.856 \\
\hline jarque\_bera & 1016020 & \cellcolor[rgb]{0.9, 0.54, 0.52} 17674040 & 202399 & \bfseries 204867 \\
\hline jarque\_bera\_pval & 0.000 & 0.000 & 0.000 & 0.000 \\
\hline mode & 350000 & 450000 & \bfseries 350000 & \cellcolor[rgb]{0.9, 0.54, 0.52} 75000 \\
\hline mode\_freq & 0.008 & \bfseries 0.009 & 0.010 & \cellcolor[rgb]{0.9, 0.54, 0.52} 0.028 \\
\hline median & 450000 & \bfseries 450000 & 450500 & \cellcolor[rgb]{0.9, 0.54, 0.52} 408171 \\
\hline 0.1\% & 95000 & 94193 & \bfseries 95000 & \cellcolor[rgb]{0.9, 0.54, 0.52} 75000 \\
\hline 1.0\% & 154467 & 161482 & \bfseries 158000 & \cellcolor[rgb]{0.9, 0.54, 0.52} 75000 \\
\hline 5.0\% & 210000 & 215000 & \bfseries 210000 & \cellcolor[rgb]{0.9, 0.54, 0.52} 100410 \\
\hline 25.0\% & 320150 & 322837 & \bfseries 319500 & \cellcolor[rgb]{0.9, 0.54, 0.52} 250152 \\
\hline 75.0\% & 640000 & \bfseries 625000 & \bfseries 625000 & \cellcolor[rgb]{0.9, 0.54, 0.52} 609428 \\
\hline 95.0\% & 1150000 & \cellcolor[rgb]{0.9, 0.54, 0.52} 1003061 & 1037350 & \bfseries 1209711 \\
\hline 99.0\% & 1950000 & \bfseries 1732519 & 1699856 & \cellcolor[rgb]{0.9, 0.54, 0.52} 2281066 \\
\hline 99.9\% & 3331995 & \cellcolor[rgb]{0.9, 0.54, 0.52} 7700000 & 2950000 & \bfseries 3517622 \\
\hline
\end{tabular}
\end{table}

\begin{table}[H]
\centering
\fontsize{8}{14}\selectfont
\caption{Propiedades  estadisticas de variable long, King county (A-2)}
\label{table-stats-king county-a-2-long}
\begin{tabular}{|l|m{10em}|m{10em}|m{10em}|m{10em}|}
\hline
 \rowcolor[gray]{0.8}
Variable/Modelo & Real & tddpm\_mlp & smote-enc & ctgan \\
\hline top5 & [-122.29  -122.362 -122.3   -122.372 -122.284] & [-122.3   -122.307 -122.301 -122.29  -122.299] & [-122.29  -122.34  -122.387 -122.375 -122.244] & [-122.519 -122.28  -122.332 -122.346 -122.323] \\
\hline top5\_freq & [100  88  81  81  81] & [137  84  84  83  81] & [25 21 21 20 18] & [108  98  86  83  81] \\
\hline top5\_prob & [0.00578369 0.00508965 0.00468479 0.00468479 0.00468479] & [0.00633878 0.00388655 0.00388655 0.00384028 0.00374774] & [0.00115666 0.00097159 0.00097159 0.00092533 0.00083279] & [0.00499699 0.00453431 0.00397909 0.00384028 0.00374774] \\
\hline nobs & 17290 & \bfseries 21613 & \cellcolor[rgb]{0.9, 0.54, 0.52} 21614 & \bfseries 21613 \\
\hline missing & 0.000 & 0.000 & 0.000 & 0.000 \\
\hline mean & -122.214 & \bfseries -122.213 & -122.216 & \cellcolor[rgb]{0.9, 0.54, 0.52} -122.211 \\
\hline std\_err & 0.001 & \cellcolor[rgb]{0.9, 0.54, 0.52} 0.001 & 0.001 & \bfseries 0.001 \\
\hline upper\_ci & -122.212 & \bfseries -122.211 & -122.214 & \cellcolor[rgb]{0.9, 0.54, 0.52} -122.209 \\
\hline lower\_ci & -122.216 & \bfseries -122.215 & -122.218 & \cellcolor[rgb]{0.9, 0.54, 0.52} -122.213 \\
\hline std & 0.140 & 0.136 & \bfseries 0.139 & \cellcolor[rgb]{0.9, 0.54, 0.52} 0.146 \\
\hline iqr & 0.204 & \cellcolor[rgb]{0.9, 0.54, 0.52} 0.196 & \bfseries 0.201 & 0.209 \\
\hline iqr\_normal & 0.151 & \cellcolor[rgb]{0.9, 0.54, 0.52} 0.145 & \bfseries 0.149 & 0.155 \\
\hline mad & 0.115 & \cellcolor[rgb]{0.9, 0.54, 0.52} 0.111 & \bfseries 0.115 & 0.119 \\
\hline mad\_normal & 0.144 & \cellcolor[rgb]{0.9, 0.54, 0.52} 0.140 & \bfseries 0.144 & 0.149 \\
\hline coef\_var & -0.001 & -0.001 & \bfseries -0.001 & \cellcolor[rgb]{0.9, 0.54, 0.52} -0.001 \\
\hline range & 1.204 & \bfseries 1.204 & 1.193 & \cellcolor[rgb]{0.9, 0.54, 0.52} 0.959 \\
\hline max & -121.315 & \bfseries -121.315 & -121.318 & \cellcolor[rgb]{0.9, 0.54, 0.52} -121.560 \\
\hline min & -122.519 & -122.519 & \cellcolor[rgb]{0.9, 0.54, 0.52} -122.511 & \bfseries -122.519 \\
\hline skew & 0.867 & 0.937 & \bfseries 0.844 & \cellcolor[rgb]{0.9, 0.54, 0.52} 0.754 \\
\hline kurtosis & 3.953 & \cellcolor[rgb]{0.9, 0.54, 0.52} 4.863 & \bfseries 3.780 & 3.753 \\
\hline jarque\_bera & 2819 & \cellcolor[rgb]{0.9, 0.54, 0.52} 6287 & 3116 & \bfseries 2558 \\
\hline jarque\_bera\_pval & 0.000 & 0.000 & 0.000 & 0.000 \\
\hline mode & -122.290 & -122.300 & \bfseries -122.290 & \cellcolor[rgb]{0.9, 0.54, 0.52} -122.519 \\
\hline mode\_freq & 0.006 & \bfseries 0.006 & \cellcolor[rgb]{0.9, 0.54, 0.52} 0.001 & 0.005 \\
\hline median & -122.231 & \cellcolor[rgb]{0.9, 0.54, 0.52} -122.221 & \bfseries -122.235 & -122.239 \\
\hline 0.1\% & -122.497 & \cellcolor[rgb]{0.9, 0.54, 0.52} -122.469 & \bfseries -122.483 & -122.519 \\
\hline 1.0\% & -122.408 & -122.399 & \bfseries -122.404 & \cellcolor[rgb]{0.9, 0.54, 0.52} -122.464 \\
\hline 5.0\% & -122.387 & -122.385 & \bfseries -122.386 & \cellcolor[rgb]{0.9, 0.54, 0.52} -122.406 \\
\hline 25.0\% & -122.329 & -122.322 & \bfseries -122.328 & \cellcolor[rgb]{0.9, 0.54, 0.52} -122.322 \\
\hline 75.0\% & -122.125 & \bfseries -122.126 & -122.128 & \cellcolor[rgb]{0.9, 0.54, 0.52} -122.113 \\
\hline 95.0\% & -121.979 & \cellcolor[rgb]{0.9, 0.54, 0.52} -121.996 & \bfseries -121.983 & -121.964 \\
\hline 99.0\% & -121.787 & \cellcolor[rgb]{0.9, 0.54, 0.52} -121.830 & -121.817 & \bfseries -121.758 \\
\hline 99.9\% & -121.699 & \cellcolor[rgb]{0.9, 0.54, 0.52} -121.469 & \bfseries -121.704 & -121.622 \\
\hline
\end{tabular}
\end{table}

\begin{table}[H]
\centering
\fontsize{8}{14}\selectfont
\caption{Propiedades  estadisticas de variable grade, King county (A-2)}
\label{table-stats-king county-a-2-grade}
\begin{tabular}{|l|m{10em}|m{10em}|m{10em}|m{10em}|}
\hline
 \rowcolor[gray]{0.8}
Variable/Modelo & Real & tddpm\_mlp & smote-enc & ctgan \\
\hline top5 & [ 7  8  9  6 10] & [ 7  8  9  6 10] & [ 7  8  9  6 10] & [ 8  7  9  6 10] \\
\hline top5\_freq & [7201 4879 2072 1620  915] & [9511 6272 2597 1701 1048] & [9815 6178 2584 1697  987] & [6536 3941 3726 3091 1408] \\
\hline top5\_prob & [0.41648352 0.28218623 0.11983806 0.09369578 0.05292076] & [0.44005922 0.29019572 0.12015916 0.07870263 0.04848934] & [0.45410382 0.28583326 0.11955214 0.07851393 0.04566485] & [0.30241059 0.18234396 0.17239624 0.14301578 0.06514598] \\
\hline nobs & 17290 & \bfseries 21613 & \cellcolor[rgb]{0.9, 0.54, 0.52} 21614 & \bfseries 21613 \\
\hline missing & 0.000 & 0.000 & 0.000 & 0.000 \\
\hline mean & 7.654 & \bfseries 7.650 & 7.621 & \cellcolor[rgb]{0.9, 0.54, 0.52} 7.929 \\
\hline std\_err & 0.009 & \bfseries 0.007 & 0.007 & \cellcolor[rgb]{0.9, 0.54, 0.52} 0.012 \\
\hline upper\_ci & 7.671 & \bfseries 7.664 & 7.635 & \cellcolor[rgb]{0.9, 0.54, 0.52} 7.952 \\
\hline lower\_ci & 7.636 & \bfseries 7.635 & 7.607 & \cellcolor[rgb]{0.9, 0.54, 0.52} 7.906 \\
\hline std & 1.170 & \bfseries 1.082 & 1.042 & \cellcolor[rgb]{0.9, 0.54, 0.52} 1.736 \\
\hline iqr & 1.000 & \bfseries 1.000 & \bfseries 1.000 & \cellcolor[rgb]{0.9, 0.54, 0.52} 2.000 \\
\hline iqr\_normal & 0.741 & \bfseries 0.741 & \bfseries 0.741 & \cellcolor[rgb]{0.9, 0.54, 0.52} 1.483 \\
\hline mad & 0.926 & \bfseries 0.867 & 0.844 & \cellcolor[rgb]{0.9, 0.54, 0.52} 1.275 \\
\hline mad\_normal & 1.160 & \bfseries 1.087 & 1.058 & \cellcolor[rgb]{0.9, 0.54, 0.52} 1.598 \\
\hline coef\_var & 0.153 & \bfseries 0.141 & 0.137 & \cellcolor[rgb]{0.9, 0.54, 0.52} 0.219 \\
\hline range & 12.000 & 9.000 & \cellcolor[rgb]{0.9, 0.54, 0.52} 8.000 & \bfseries 12.000 \\
\hline max & 13.000 & \bfseries 13.000 & \cellcolor[rgb]{0.9, 0.54, 0.52} 12.000 & \bfseries 13.000 \\
\hline min & 1.000 & \cellcolor[rgb]{0.9, 0.54, 0.52} 4.000 & \cellcolor[rgb]{0.9, 0.54, 0.52} 4.000 & \bfseries 1.000 \\
\hline skew & 0.758 & 0.820 & \bfseries 0.812 & \cellcolor[rgb]{0.9, 0.54, 0.52} 0.213 \\
\hline kurtosis & 4.209 & \bfseries 4.083 & 3.918 & \cellcolor[rgb]{0.9, 0.54, 0.52} 3.654 \\
\hline jarque\_bera & 2709 & 3479 & \bfseries 3133 & \cellcolor[rgb]{0.9, 0.54, 0.52} 549 \\
\hline jarque\_bera\_pval & 0.000 & \bfseries 0.000 & \bfseries 0.000 & \cellcolor[rgb]{0.9, 0.54, 0.52} 0.000 \\
\hline mode & 7.000 & \bfseries 7.000 & \bfseries 7.000 & \cellcolor[rgb]{0.9, 0.54, 0.52} 8.000 \\
\hline mode\_freq & 0.416 & \bfseries 0.440 & 0.454 & \cellcolor[rgb]{0.9, 0.54, 0.52} 0.302 \\
\hline median & 7.000 & \bfseries 7.000 & \bfseries 7.000 & \cellcolor[rgb]{0.9, 0.54, 0.52} 8.000 \\
\hline 0.1\% & 4.000 & 5.000 & 5.000 & 3.000 \\
\hline 1.0\% & 5.000 & 6.000 & 6.000 & 4.000 \\
\hline 5.0\% & 6.000 & \bfseries 6.000 & \bfseries 6.000 & \cellcolor[rgb]{0.9, 0.54, 0.52} 5.000 \\
\hline 25.0\% & 7.000 & 7.000 & 7.000 & 7.000 \\
\hline 75.0\% & 8.000 & \bfseries 8.000 & \bfseries 8.000 & \cellcolor[rgb]{0.9, 0.54, 0.52} 9.000 \\
\hline 95.0\% & 10.000 & \bfseries 10.000 & \bfseries 10.000 & \cellcolor[rgb]{0.9, 0.54, 0.52} 11.000 \\
\hline 99.0\% & 11.000 & \bfseries 11.000 & \bfseries 11.000 & \cellcolor[rgb]{0.9, 0.54, 0.52} 12.000 \\
\hline 99.9\% & 12.000 & \bfseries 12.000 & \bfseries 12.000 & \cellcolor[rgb]{0.9, 0.54, 0.52} 13.000 \\
\hline
\end{tabular}
\end{table}

\begin{table}[H]
\centering
\fontsize{8}{14}\selectfont
\caption{Propiedades  estadisticas de variable condition, King county (A-2)}
\label{table-stats-king county-a-2-condition}
\begin{tabular}{|l|m{10em}|m{10em}|m{10em}|m{10em}|}
\hline
 \rowcolor[gray]{0.8}
Variable/Modelo & Real & tddpm\_mlp & smote-enc & ctgan \\
\hline top5 & [3 4 5 2 1] & [3 4 5 2 1] & [3 4 5 2 1] & [3 4 5 2 1] \\
\hline top5\_freq & [11248  4512  1364   139    27] & [14872  5461  1228    49     3] & [15439  5203   945    26     1] & [12161  6512  2329   400   211] \\
\hline top5\_prob & [0.65054945 0.26096009 0.07888953 0.00803933 0.0015616 ] & [6.88104382e-01 2.52672003e-01 5.68176560e-02 2.26715403e-03
 1.38805349e-04] & [7.14305543e-01 2.40723605e-01 4.37216619e-02 1.20292403e-03
 4.62663089e-05] & [0.56267061 0.30130014 0.10775922 0.01850738 0.00976264] \\
\hline nobs & 17290 & \bfseries 21613 & \cellcolor[rgb]{0.9, 0.54, 0.52} 21614 & \bfseries 21613 \\
\hline missing & 0.000 & 0.000 & 0.000 & 0.000 \\
\hline mean & 3.408 & \bfseries 3.364 & \cellcolor[rgb]{0.9, 0.54, 0.52} 3.327 & 3.479 \\
\hline std\_err & 0.005 & 0.004 & \cellcolor[rgb]{0.9, 0.54, 0.52} 0.004 & \bfseries 0.005 \\
\hline upper\_ci & 3.417 & \bfseries 3.372 & \cellcolor[rgb]{0.9, 0.54, 0.52} 3.334 & 3.489 \\
\hline lower\_ci & 3.398 & \bfseries 3.356 & \cellcolor[rgb]{0.9, 0.54, 0.52} 3.319 & 3.469 \\
\hline std & 0.652 & \bfseries 0.592 & 0.557 & \cellcolor[rgb]{0.9, 0.54, 0.52} 0.749 \\
\hline iqr & 1.000 & 1.000 & 1.000 & 1.000 \\
\hline iqr\_normal & 0.741 & 0.741 & 0.741 & 0.741 \\
\hline mad & 0.560 & \bfseries 0.507 & \cellcolor[rgb]{0.9, 0.54, 0.52} 0.470 & 0.642 \\
\hline mad\_normal & 0.702 & \bfseries 0.636 & \cellcolor[rgb]{0.9, 0.54, 0.52} 0.590 & 0.805 \\
\hline coef\_var & 0.191 & \bfseries 0.176 & 0.167 & \cellcolor[rgb]{0.9, 0.54, 0.52} 0.215 \\
\hline range & 4.000 & 4.000 & 4.000 & 4.000 \\
\hline max & 5.000 & 5.000 & 5.000 & 5.000 \\
\hline min & 1.000 & 1.000 & 1.000 & 1.000 \\
\hline skew & 1.028 & \bfseries 1.317 & 1.447 & \cellcolor[rgb]{0.9, 0.54, 0.52} 0.361 \\
\hline kurtosis & 3.556 & 3.851 & \cellcolor[rgb]{0.9, 0.54, 0.52} 4.213 & \bfseries 3.455 \\
\hline jarque\_bera & 3269 & 6902 & \cellcolor[rgb]{0.9, 0.54, 0.52} 8863 & \bfseries 657 \\
\hline jarque\_bera\_pval & 0.000 & \bfseries 0.000 & \bfseries 0.000 & \cellcolor[rgb]{0.9, 0.54, 0.52} 0.000 \\
\hline mode & 3.000 & 3.000 & 3.000 & 3.000 \\
\hline mode\_freq & 0.651 & \bfseries 0.688 & 0.714 & \cellcolor[rgb]{0.9, 0.54, 0.52} 0.563 \\
\hline median & 3.000 & 3.000 & 3.000 & 3.000 \\
\hline 0.1\% & 1.000 & \cellcolor[rgb]{0.9, 0.54, 0.52} 2.000 & \cellcolor[rgb]{0.9, 0.54, 0.52} 2.000 & \bfseries 1.000 \\
\hline 1.0\% & 3.000 & \bfseries 3.000 & \bfseries 3.000 & \cellcolor[rgb]{0.9, 0.54, 0.52} 2.000 \\
\hline 5.0\% & 3.000 & 3.000 & 3.000 & 3.000 \\
\hline 25.0\% & 3.000 & 3.000 & 3.000 & 3.000 \\
\hline 75.0\% & 4.000 & 4.000 & 4.000 & 4.000 \\
\hline 95.0\% & 5.000 & \bfseries 5.000 & \cellcolor[rgb]{0.9, 0.54, 0.52} 4.000 & \bfseries 5.000 \\
\hline 99.0\% & 5.000 & 5.000 & 5.000 & 5.000 \\
\hline 99.9\% & 5.000 & 5.000 & 5.000 & 5.000 \\
\hline
\end{tabular}
\end{table}

\begin{table}[H]
\centering
\fontsize{8}{14}\selectfont
\caption{Propiedades  estadisticas de variable lat, King county (A-2)}
\label{table-stats-king county-a-2-lat}
\begin{tabular}{|l|m{10em}|m{10em}|m{10em}|m{10em}|}
\hline
 \rowcolor[gray]{0.8}
Variable/Modelo & Real & tddpm\_mlp & smote-enc & ctgan \\
\hline top5 & [47.5402 47.6914 47.6853 47.6624 47.6968] & [47.1559     47.7776     47.64939189 47.6493812  47.64937361] & [47.3363 47.6534 47.6647 47.6904 47.7438] & [47.1593 47.7776 47.6289 47.6879 47.643 ] \\
\hline top5\_freq & [14 13 13 13 13] & [16  5  1  1  1] & [7 6 6 5 4] & [430  56  18  16  16] \\
\hline top5\_prob & [0.00080972 0.00075188 0.00075188 0.00075188 0.00075188] & [7.40295193e-04 2.31342248e-04 4.62684495e-05 4.62684495e-05
 4.62684495e-05] & [0.00032386 0.0002776  0.0002776  0.00023133 0.00018507] & [0.01989543 0.00259103 0.00083283 0.0007403  0.0007403 ] \\
\hline nobs & 17290 & \bfseries 21613 & \cellcolor[rgb]{0.9, 0.54, 0.52} 21614 & \bfseries 21613 \\
\hline missing & 0.000 & 0.000 & 0.000 & 0.000 \\
\hline mean & 47.560 & 47.561 & \bfseries 47.561 & \cellcolor[rgb]{0.9, 0.54, 0.52} 47.527 \\
\hline std\_err & 0.001 & \cellcolor[rgb]{0.9, 0.54, 0.52} 0.001 & 0.001 & \bfseries 0.001 \\
\hline upper\_ci & 47.562 & 47.563 & \bfseries 47.563 & \cellcolor[rgb]{0.9, 0.54, 0.52} 47.529 \\
\hline lower\_ci & 47.558 & 47.559 & \bfseries 47.559 & \cellcolor[rgb]{0.9, 0.54, 0.52} 47.525 \\
\hline std & 0.138 & 0.137 & \bfseries 0.138 & \cellcolor[rgb]{0.9, 0.54, 0.52} 0.157 \\
\hline iqr & 0.206 & 0.208 & \bfseries 0.205 & \cellcolor[rgb]{0.9, 0.54, 0.52} 0.235 \\
\hline iqr\_normal & 0.153 & 0.154 & \bfseries 0.152 & \cellcolor[rgb]{0.9, 0.54, 0.52} 0.174 \\
\hline mad & 0.115 & 0.114 & \bfseries 0.114 & \cellcolor[rgb]{0.9, 0.54, 0.52} 0.131 \\
\hline mad\_normal & 0.144 & 0.143 & \bfseries 0.143 & \cellcolor[rgb]{0.9, 0.54, 0.52} 0.164 \\
\hline coef\_var & 0.003 & 0.003 & \bfseries 0.003 & \cellcolor[rgb]{0.9, 0.54, 0.52} 0.003 \\
\hline range & 0.618 & 0.622 & \cellcolor[rgb]{0.9, 0.54, 0.52} 0.606 & \bfseries 0.618 \\
\hline max & 47.778 & \bfseries 47.778 & \cellcolor[rgb]{0.9, 0.54, 0.52} 47.777 & \bfseries 47.778 \\
\hline min & 47.159 & 47.156 & \cellcolor[rgb]{0.9, 0.54, 0.52} 47.171 & \bfseries 47.159 \\
\hline skew & -0.487 & -0.459 & \bfseries -0.492 & \cellcolor[rgb]{0.9, 0.54, 0.52} -0.603 \\
\hline kurtosis & 2.328 & 2.267 & \bfseries 2.313 & \cellcolor[rgb]{0.9, 0.54, 0.52} 2.423 \\
\hline jarque\_bera & 1009 & \bfseries 1245 & 1298 & \cellcolor[rgb]{0.9, 0.54, 0.52} 1611 \\
\hline jarque\_bera\_pval & 0.000 & 0.000 & 0.000 & 0.000 \\
\hline mode & 47.540 & \cellcolor[rgb]{0.9, 0.54, 0.52} 47.156 & \bfseries 47.336 & 47.159 \\
\hline mode\_freq & 0.001 & \bfseries 0.001 & 0.000 & \cellcolor[rgb]{0.9, 0.54, 0.52} 0.020 \\
\hline median & 47.572 & 47.569 & \bfseries 47.572 & \cellcolor[rgb]{0.9, 0.54, 0.52} 47.547 \\
\hline 0.1\% & 47.193 & 47.171 & \bfseries 47.194 & \cellcolor[rgb]{0.9, 0.54, 0.52} 47.159 \\
\hline 1.0\% & 47.257 & 47.263 & \bfseries 47.258 & \cellcolor[rgb]{0.9, 0.54, 0.52} 47.159 \\
\hline 5.0\% & 47.311 & 47.313 & \bfseries 47.312 & \cellcolor[rgb]{0.9, 0.54, 0.52} 47.231 \\
\hline 25.0\% & 47.472 & \bfseries 47.472 & 47.473 & \cellcolor[rgb]{0.9, 0.54, 0.52} 47.424 \\
\hline 75.0\% & 47.678 & 47.680 & \bfseries 47.678 & \cellcolor[rgb]{0.9, 0.54, 0.52} 47.660 \\
\hline 95.0\% & 47.750 & 47.748 & \bfseries 47.750 & \cellcolor[rgb]{0.9, 0.54, 0.52} 47.724 \\
\hline 99.0\% & 47.773 & \bfseries 47.771 & 47.771 & \cellcolor[rgb]{0.9, 0.54, 0.52} 47.756 \\
\hline 99.9\% & 47.777 & \bfseries 47.777 & \cellcolor[rgb]{0.9, 0.54, 0.52} 47.776 & 47.778 \\
\hline
\end{tabular}
\end{table}

\begin{table}[H]
\centering
\fontsize{8}{14}\selectfont
\caption{Propiedades  estadisticas de variable sqft\_living, King county (A-2)}
\label{table-stats-king county-a-2-sqft_living}
\begin{tabular}{|l|m{10em}|m{10em}|m{10em}|m{10em}|}
\hline
 \rowcolor[gray]{0.8}
Variable/Modelo & Real & tddpm\_mlp & smote-enc & ctgan \\
\hline top5 & [1400 1300 1720 1250 1540] & [1440. 1300. 1800. 1320. 1820.] & [1250. 1280. 1160. 1690. 1800.] & [ 290 1925 2471 1406 2241] \\
\hline top5\_freq & [109 107 106 106 105] & [144 142 137 117 115] & [15 14 12 11 11] & [20 18 17 17 17] \\
\hline top5\_prob & [0.00630422 0.00618855 0.00613071 0.00613071 0.00607287] & [0.00666266 0.00657012 0.00633878 0.00541341 0.00532087] & [0.00069399 0.00064773 0.0005552  0.00050893 0.00050893] & [0.00092537 0.00083283 0.00078656 0.00078656 0.00078656] \\
\hline nobs & 17290 & \bfseries 21613 & \cellcolor[rgb]{0.9, 0.54, 0.52} 21614 & \bfseries 21613 \\
\hline missing & 0.000 & 0.000 & 0.000 & 0.000 \\
\hline mean & 2074 & 2032 & \bfseries 2049 & \cellcolor[rgb]{0.9, 0.54, 0.52} 2813 \\
\hline std\_err & 6.900 & \bfseries 6.968 & 5.741 & \cellcolor[rgb]{0.9, 0.54, 0.52} 10.591 \\
\hline upper\_ci & 2087 & 2046 & \bfseries 2060 & \cellcolor[rgb]{0.9, 0.54, 0.52} 2833 \\
\hline lower\_ci & 2060 & 2019 & \bfseries 2037 & \cellcolor[rgb]{0.9, 0.54, 0.52} 2792 \\
\hline std & 907.298 & 1024.446 & \bfseries 844.097 & \cellcolor[rgb]{0.9, 0.54, 0.52} 1556.997 \\
\hline iqr & 1110 & 1057 & \bfseries 1079 & \cellcolor[rgb]{0.9, 0.54, 0.52} 1810 \\
\hline iqr\_normal & 822.844 & 783.834 & \bfseries 799.642 & \cellcolor[rgb]{0.9, 0.54, 0.52} 1341.755 \\
\hline mad & 693.180 & \bfseries 679.548 & 657.789 & \cellcolor[rgb]{0.9, 0.54, 0.52} 1185.435 \\
\hline mad\_normal & 868.773 & \bfseries 851.687 & 824.416 & \cellcolor[rgb]{0.9, 0.54, 0.52} 1485.722 \\
\hline coef\_var & 0.437 & 0.504 & \bfseries 0.412 & \cellcolor[rgb]{0.9, 0.54, 0.52} 0.554 \\
\hline range & 11760 & \bfseries 13250 & \cellcolor[rgb]{0.9, 0.54, 0.52} 8998 & 10198 \\
\hline max & 12050 & \bfseries 13540 & \cellcolor[rgb]{0.9, 0.54, 0.52} 9404 & 10488 \\
\hline min & 290.000 & \bfseries 290.000 & \cellcolor[rgb]{0.9, 0.54, 0.52} 405.289 & \bfseries 290.000 \\
\hline skew & 1.371 & \cellcolor[rgb]{0.9, 0.54, 0.52} 4.333 & 1.117 & \bfseries 1.334 \\
\hline kurtosis & 7.167 & \cellcolor[rgb]{0.9, 0.54, 0.52} 44.030 & 5.084 & \bfseries 5.144 \\
\hline jarque\_bera & 17922 & \cellcolor[rgb]{0.9, 0.54, 0.52} 1583687 & 8402 & \bfseries 10545 \\
\hline jarque\_bera\_pval & 0.000 & 0.000 & 0.000 & 0.000 \\
\hline mode & 1400 & \bfseries 1440 & 1250 & \cellcolor[rgb]{0.9, 0.54, 0.52} 290 \\
\hline mode\_freq & 0.006 & \bfseries 0.007 & \cellcolor[rgb]{0.9, 0.54, 0.52} 0.001 & 0.001 \\
\hline median & 1910 & 1850 & \bfseries 1896 & \cellcolor[rgb]{0.9, 0.54, 0.52} 2442 \\
\hline 0.1\% & 522.890 & \bfseries 482.093 & 595.171 & \cellcolor[rgb]{0.9, 0.54, 0.52} 294.612 \\
\hline 1.0\% & 720.000 & \bfseries 726.418 & 757.374 & \cellcolor[rgb]{0.9, 0.54, 0.52} 592.120 \\
\hline 5.0\% & 940.000 & \bfseries 938.703 & 969.440 & \cellcolor[rgb]{0.9, 0.54, 0.52} 979.600 \\
\hline 25.0\% & 1430 & 1400 & \bfseries 1428 & \cellcolor[rgb]{0.9, 0.54, 0.52} 1715 \\
\hline 75.0\% & 2540 & 2457 & \bfseries 2507 & \cellcolor[rgb]{0.9, 0.54, 0.52} 3525 \\
\hline 95.0\% & 3740 & 3571 & \bfseries 3640 & \cellcolor[rgb]{0.9, 0.54, 0.52} 5727 \\
\hline 99.0\% & 4921 & \bfseries 4768 & 4590 & \cellcolor[rgb]{0.9, 0.54, 0.52} 8012 \\
\hline 99.9\% & 6966 & \cellcolor[rgb]{0.9, 0.54, 0.52} 13540 & \bfseries 6129 & 9755 \\
\hline
\end{tabular}
\end{table}

\begin{table}[H]
\centering
\fontsize{8}{14}\selectfont
\caption{Propiedades  estadisticas de variable waterfront, King county (A-2)}
\label{table-stats-king county-a-2-waterfront}
\begin{tabular}{|l|m{10em}|m{10em}|m{10em}|m{10em}|}
\hline
 \rowcolor[gray]{0.8}
Variable/Modelo & Real & tddpm\_mlp & smote-enc & ctgan \\
\hline top5 & [0 1] & [0 1] & [0 1] & [0 1] \\
\hline top5\_freq & [17166   124] & [21565    48] & [21582    32] & [20038  1575] \\
\hline top5\_prob & [0.99282822 0.00717178] & [0.99777911 0.00222089] & [0.99851948 0.00148052] & [0.92712719 0.07287281] \\
\hline nobs & 17290 & \bfseries 21613 & \cellcolor[rgb]{0.9, 0.54, 0.52} 21614 & \bfseries 21613 \\
\hline missing & 0.000 & 0.000 & 0.000 & 0.000 \\
\hline mean & 0.007 & \bfseries 0.002 & 0.001 & \cellcolor[rgb]{0.9, 0.54, 0.52} 0.073 \\
\hline std\_err & 0.001 & \bfseries 0.000 & 0.000 & \cellcolor[rgb]{0.9, 0.54, 0.52} 0.002 \\
\hline upper\_ci & 0.008 & \bfseries 0.003 & 0.002 & \cellcolor[rgb]{0.9, 0.54, 0.52} 0.076 \\
\hline lower\_ci & 0.006 & \bfseries 0.002 & 0.001 & \cellcolor[rgb]{0.9, 0.54, 0.52} 0.069 \\
\hline std & 0.084 & \bfseries 0.047 & 0.038 & \cellcolor[rgb]{0.9, 0.54, 0.52} 0.260 \\
\hline iqr & 0.000 & 0.000 & 0.000 & 0.000 \\
\hline iqr\_normal & 0.000 & 0.000 & 0.000 & 0.000 \\
\hline mad & 0.014 & \bfseries 0.004 & 0.003 & \cellcolor[rgb]{0.9, 0.54, 0.52} 0.135 \\
\hline mad\_normal & 0.018 & \bfseries 0.006 & 0.004 & \cellcolor[rgb]{0.9, 0.54, 0.52} 0.169 \\
\hline coef\_var & 11.766 & 21.197 & \cellcolor[rgb]{0.9, 0.54, 0.52} 25.971 & \bfseries 3.567 \\
\hline range & 1.000 & 1.000 & 1.000 & 1.000 \\
\hline max & 1.000 & 1.000 & 1.000 & 1.000 \\
\hline min & 0.000 & 0.000 & 0.000 & 0.000 \\
\hline skew & 11.681 & 21.149 & \cellcolor[rgb]{0.9, 0.54, 0.52} 25.931 & \bfseries 3.287 \\
\hline kurtosis & 137.443 & 448.273 & \cellcolor[rgb]{0.9, 0.54, 0.52} 673.439 & \bfseries 11.801 \\
\hline jarque\_bera & 13414600 & 180159835 & \cellcolor[rgb]{0.9, 0.54, 0.52} 407224138 & \bfseries 108664 \\
\hline jarque\_bera\_pval & 0.000 & 0.000 & 0.000 & 0.000 \\
\hline mode & 0.000 & 0.000 & 0.000 & 0.000 \\
\hline mode\_freq & 0.993 & \bfseries 0.998 & 0.999 & \cellcolor[rgb]{0.9, 0.54, 0.52} 0.927 \\
\hline median & 0.000 & 0.000 & 0.000 & 0.000 \\
\hline 0.1\% & 0.000 & 0.000 & 0.000 & 0.000 \\
\hline 1.0\% & 0.000 & 0.000 & 0.000 & 0.000 \\
\hline 5.0\% & 0.000 & 0.000 & 0.000 & 0.000 \\
\hline 25.0\% & 0.000 & 0.000 & 0.000 & 0.000 \\
\hline 75.0\% & 0.000 & 0.000 & 0.000 & 0.000 \\
\hline 95.0\% & 0.000 & \bfseries 0.000 & \bfseries 0.000 & \cellcolor[rgb]{0.9, 0.54, 0.52} 1.000 \\
\hline 99.0\% & 0.000 & \bfseries 0.000 & \bfseries 0.000 & \cellcolor[rgb]{0.9, 0.54, 0.52} 1.000 \\
\hline 99.9\% & 1.000 & 1.000 & 1.000 & 1.000 \\
\hline
\end{tabular}
\end{table}

\begin{table}[H]
\centering
\fontsize{8}{14}\selectfont
\caption{Propiedades  estadisticas de variable sqft\_basement, King county (A-2)}
\label{table-stats-king county-a-2-sqft_basement}
\begin{tabular}{|l|m{10em}|m{10em}|m{10em}|m{10em}|}
\hline
 \rowcolor[gray]{0.8}
Variable/Modelo & Real & tddpm\_mlp & smote-enc & ctgan \\
\hline top5 & [  0 600 700 500 800] & [  0. 500. 600. 700. 800.] & [  0. 800. 600. 850. 500.] & [ 0 10 11  9 12] \\
\hline top5\_freq & [10553   182   169   167   164] & [13604   239   196   187   177] & [11761    16    11     9     8] & [2222  863  815  804  804] \\
\hline top5\_prob & [0.61035281 0.01052632 0.00977444 0.00965876 0.00948525] & [0.62943599 0.01105816 0.00906862 0.0086522  0.00818952] & [5.44138059e-01 7.40260942e-04 5.08929398e-04 4.16396780e-04
 3.70130471e-04] & [0.10280849 0.03992967 0.03770879 0.03719983 0.03719983] \\
\hline nobs & 17290 & \bfseries 21613 & \cellcolor[rgb]{0.9, 0.54, 0.52} 21614 & \bfseries 21613 \\
\hline missing & 0.000 & 0.000 & 0.000 & 0.000 \\
\hline mean & 287.933 & \bfseries 286.843 & 274.142 & \cellcolor[rgb]{0.9, 0.54, 0.52} 457.012 \\
\hline std\_err & 3.337 & \bfseries 3.255 & 2.755 & \cellcolor[rgb]{0.9, 0.54, 0.52} 4.625 \\
\hline upper\_ci & 294.472 & \bfseries 293.223 & 279.541 & \cellcolor[rgb]{0.9, 0.54, 0.52} 466.076 \\
\hline lower\_ci & 281.393 & \bfseries 280.463 & 268.743 & \cellcolor[rgb]{0.9, 0.54, 0.52} 447.948 \\
\hline std & 438.727 & 478.554 & \bfseries 404.963 & \cellcolor[rgb]{0.9, 0.54, 0.52} 679.898 \\
\hline iqr & 550.000 & \bfseries 550.000 & 517.020 & \cellcolor[rgb]{0.9, 0.54, 0.52} 881.000 \\
\hline iqr\_normal & 407.716 & \bfseries 407.716 & 383.267 & \cellcolor[rgb]{0.9, 0.54, 0.52} 653.086 \\
\hline mad & 360.277 & \bfseries 365.019 & 332.362 & \cellcolor[rgb]{0.9, 0.54, 0.52} 562.364 \\
\hline mad\_normal & 451.541 & \bfseries 457.483 & 416.555 & \cellcolor[rgb]{0.9, 0.54, 0.52} 704.819 \\
\hline coef\_var & 1.524 & \cellcolor[rgb]{0.9, 0.54, 0.52} 1.668 & 1.477 & \bfseries 1.488 \\
\hline range & 4820 & \bfseries 4820 & \cellcolor[rgb]{0.9, 0.54, 0.52} 2713 & 3238 \\
\hline max & 4820 & \bfseries 4820 & \cellcolor[rgb]{0.9, 0.54, 0.52} 2713 & 3238 \\
\hline min & 0.000 & 0.000 & 0.000 & 0.000 \\
\hline skew & 1.571 & \cellcolor[rgb]{0.9, 0.54, 0.52} 3.129 & \bfseries 1.467 & 1.451 \\
\hline kurtosis & 5.639 & \cellcolor[rgb]{0.9, 0.54, 0.52} 23.527 & \bfseries 4.575 & 4.359 \\
\hline jarque\_bera & 12126 & \cellcolor[rgb]{0.9, 0.54, 0.52} 414719 & \bfseries 9983 & 9250 \\
\hline jarque\_bera\_pval & 0.000 & 0.000 & 0.000 & 0.000 \\
\hline mode & 0.000 & 0.000 & 0.000 & 0.000 \\
\hline mode\_freq & 0.610 & \bfseries 0.629 & 0.544 & \cellcolor[rgb]{0.9, 0.54, 0.52} 0.103 \\
\hline median & 0.000 & \bfseries 0.000 & \bfseries 0.000 & \cellcolor[rgb]{0.9, 0.54, 0.52} 13.000 \\
\hline 0.1\% & 0.000 & 0.000 & 0.000 & 0.000 \\
\hline 1.0\% & 0.000 & 0.000 & 0.000 & 0.000 \\
\hline 5.0\% & 0.000 & 0.000 & 0.000 & 0.000 \\
\hline 25.0\% & 0.000 & \bfseries 0.000 & \bfseries 0.000 & \cellcolor[rgb]{0.9, 0.54, 0.52} 6.000 \\
\hline 75.0\% & 550.000 & \bfseries 550.000 & 517.020 & \cellcolor[rgb]{0.9, 0.54, 0.52} 887.000 \\
\hline 95.0\% & 1180 & \bfseries 1134 & 1097 & \cellcolor[rgb]{0.9, 0.54, 0.52} 1868 \\
\hline 99.0\% & 1650 & \bfseries 1594 & 1539 & \cellcolor[rgb]{0.9, 0.54, 0.52} 2679 \\
\hline 99.9\% & 2324 & \cellcolor[rgb]{0.9, 0.54, 0.52} 4820 & \bfseries 2061 & 3082 \\
\hline
\end{tabular}
\end{table}

\begin{table}[H]
\centering
\fontsize{8}{14}\selectfont
\caption{Propiedades  estadisticas de variable sqft\_lot15, King county (A-2)}
\label{table-stats-king county-a-2-sqft_lot15}
\begin{tabular}{|l|m{10em}|m{10em}|m{10em}|m{10em}|}
\hline
 \rowcolor[gray]{0.8}
Variable/Modelo & Real & tddpm\_mlp & smote-enc & ctgan \\
\hline top5 & [5000 4000 6000 7200 4800] & [5000. 4000. 6000. 7200. 4800.] & [5000. 4000. 5200. 6000. 4080.] & [ 651 7137 9396 8240 4547] \\
\hline top5\_freq & [349 289 224 160 120] & [407 322 228 191 125] & [74 71 33 32 26] & [890   7   7   7   6] \\
\hline top5\_prob & [0.02018508 0.01671486 0.01295547 0.0092539  0.00694043] & [0.01883126 0.01489844 0.01054921 0.00883727 0.00578356] & [0.00342371 0.00328491 0.00152679 0.00148052 0.00120292] & [0.04117892 0.00032388 0.00032388 0.00032388 0.00027761] \\
\hline nobs & 17290 & \bfseries 21613 & \cellcolor[rgb]{0.9, 0.54, 0.52} 21614 & \bfseries 21613 \\
\hline missing & 0.000 & 0.000 & 0.000 & 0.000 \\
\hline mean & 12725 & \bfseries 12168 & 11832 & \cellcolor[rgb]{0.9, 0.54, 0.52} 14318 \\
\hline std\_err & 209.331 & \bfseries 221.883 & \cellcolor[rgb]{0.9, 0.54, 0.52} 155.914 & 163.365 \\
\hline upper\_ci & 13135 & \bfseries 12603 & 12137 & \cellcolor[rgb]{0.9, 0.54, 0.52} 14638 \\
\hline lower\_ci & 12315 & \bfseries 11733 & 11526 & \cellcolor[rgb]{0.9, 0.54, 0.52} 13997 \\
\hline std & 27525 & \cellcolor[rgb]{0.9, 0.54, 0.52} 32620 & 22922 & \bfseries 24017 \\
\hline iqr & 4963 & 4663 & \bfseries 4930 & \cellcolor[rgb]{0.9, 0.54, 0.52} 8395 \\
\hline iqr\_normal & 3679 & 3457 & \bfseries 3654 & \cellcolor[rgb]{0.9, 0.54, 0.52} 6223 \\
\hline mad & 10095 & 9042 & \cellcolor[rgb]{0.9, 0.54, 0.52} 8618 & \bfseries 10910 \\
\hline mad\_normal & 12652 & 11332 & \cellcolor[rgb]{0.9, 0.54, 0.52} 10800 & \bfseries 13674 \\
\hline coef\_var & 2.163 & \cellcolor[rgb]{0.9, 0.54, 0.52} 2.681 & \bfseries 1.937 & 1.677 \\
\hline range & 870549 & \bfseries 870549 & 680511 & \cellcolor[rgb]{0.9, 0.54, 0.52} 309873 \\
\hline max & 871200 & \bfseries 871200 & 681231 & \cellcolor[rgb]{0.9, 0.54, 0.52} 310524 \\
\hline min & 651.000 & \bfseries 651.000 & \cellcolor[rgb]{0.9, 0.54, 0.52} 719.360 & \bfseries 651.000 \\
\hline skew & 9.701 & \cellcolor[rgb]{0.9, 0.54, 0.52} 16.907 & \bfseries 8.316 & 6.341 \\
\hline kurtosis & 163.253 & \cellcolor[rgb]{0.9, 0.54, 0.52} 383.753 & \bfseries 101.830 & 54.342 \\
\hline jarque\_bera & 18772189 & \cellcolor[rgb]{0.9, 0.54, 0.52} 131583876 & \bfseries 9045446 & 2518709 \\
\hline jarque\_bera\_pval & 0.000 & 0.000 & 0.000 & 0.000 \\
\hline mode & 5000 & \bfseries 5000 & 5000 & \cellcolor[rgb]{0.9, 0.54, 0.52} 651 \\
\hline mode\_freq & 0.020 & \bfseries 0.019 & 0.003 & \cellcolor[rgb]{0.9, 0.54, 0.52} 0.041 \\
\hline median & 7615 & 7723 & \bfseries 7644 & \cellcolor[rgb]{0.9, 0.54, 0.52} 9083 \\
\hline 0.1\% & 886.289 & \bfseries 878.456 & 913.044 & \cellcolor[rgb]{0.9, 0.54, 0.52} 651.000 \\
\hline 1.0\% & 1189 & \bfseries 1215 & 1230 & \cellcolor[rgb]{0.9, 0.54, 0.52} 651 \\
\hline 5.0\% & 1965 & 2545 & \bfseries 2106 & \cellcolor[rgb]{0.9, 0.54, 0.52} 970 \\
\hline 25.0\% & 5083 & 5183 & \bfseries 5100 & \cellcolor[rgb]{0.9, 0.54, 0.52} 5235 \\
\hline 75.0\% & 10046 & 9846 & \bfseries 10030 & \cellcolor[rgb]{0.9, 0.54, 0.52} 13630 \\
\hline 95.0\% & 36822 & 35077 & \bfseries 35134 & \cellcolor[rgb]{0.9, 0.54, 0.52} 45254 \\
\hline 99.0\% & 168296 & \cellcolor[rgb]{0.9, 0.54, 0.52} 107518 & 130809 & \bfseries 138474 \\
\hline 99.9\% & 306998 & \cellcolor[rgb]{0.9, 0.54, 0.52} 532713 & 263407 & \bfseries 269421 \\
\hline
\end{tabular}
\end{table}

\begin{table}[H]
\centering
\fontsize{8}{14}\selectfont
\caption{Propiedades  estadisticas de variable date, King county (A-2)}
\label{table-stats-king county-a-2-date}
\begin{tabular}{|l|m{10em}|m{10em}|m{10em}|m{10em}|}
\hline
 \rowcolor[gray]{0.8}
Variable/Modelo & Real & tddpm\_mlp & smote-enc & ctgan \\
\hline top5 & ['20140623T000000' '20140625T000000' '20140626T000000' '20150421T000000'
 '20150325T000000'] & ['20140623T000000' '20140625T000000' '20150427T000000' '20140520T000000'
 '20150421T000000'] & ['20140623T000000' '20140625T000000' '20150414T000000' '20140520T000000'
 '20150421T000000'] & ['20150310T000000' '20150327T000000' '20140603T000000' '20150226T000000'
 '20150329T000000'] \\
\hline top5\_freq & [123 105 101 101 101] & [195 167 163 161 151] & [151 146 139 136 135] & [489 421 337 288 283] \\
\hline top5\_prob & [0.00711394 0.00607287 0.00584153 0.00584153 0.00584153] & [0.00902235 0.00772683 0.00754176 0.00744922 0.00698654] & [0.00698621 0.00675488 0.00643102 0.00629222 0.00624595] & [0.02262527 0.01947902 0.01559247 0.01332531 0.01309397] \\
\hline nobs & 17290 & \bfseries 21613 & \cellcolor[rgb]{0.9, 0.54, 0.52} 21614 & \bfseries 21613 \\
\hline missing & 17290 & 0 & 0 & 0 \\
\hline
\end{tabular}
\end{table}



\section{Ejemplos de código con fines de reproducibilidad}
\label{anexo:reproducibilidad}

En el Código \ref{codigo-show-score}, se muestra cómo se calcula y se muestra el puntaje promedio para una selección específica de modelos. El código utiliza la función "sort\_values" para ordenar los resultados en orden descendente según el puntaje. Luego, se filtran los resultados para incluir solo los modelos seleccionados y las columnas que muestran el puntaje y la Distancia al registro más cercano (DCR) en los tres umbrales \emph{Synthetic vs Train} (ST), \emph{Synthetic vs Hold} (SH) y \emph{Train vs Hold} TH.
\begin{listing}[H]
    \begin{minted}[linenos=true,frame=lines,framesep=2mm,baselinestretch=1.2]{python}
avg = syn.scores[syn.scores["type"] == "avg"]
avg.sort_values("score", ascending=False).loc[ ["tddpm_mlp","smote-enc","gaussiancopula","tvae","gaussiancopula", "copulagan","ctgan"], ["score", "DCR ST 5th", "DCR SH 5th", "DCR TH 5th"]]
    \end{minted}
\caption{Mostrando Puntajes Promedios Calculados}
\label{codigo-show-score}
\end{listing}

En el ejemplo presentado en el Código \ref{code-economicos-synthetic}, se crea una instancia de la clase \emph{Synthetic} utilizando un pandas dataframe previamente pre procesado. Se especifican las columnas que se considerarán como categorías, las que se considerarán como texto y las que se excluirán del análisis. Además, se indica el directorio donde se almacenarán los archivos temporales, se seleccionan los modelos a utilizar, se establece el número de registros sintéticos deseados y se define una columna objetiva para realizar pruebas con aprendizaje automático y estratificar los conjuntos parciales de datos que se utilizarán. De esta manera, se configura de forma flexible el proceso de generación de datos sintéticos según las necesidades específicas del usuario.

\begin{listing}[H]
\inputminted[
    firstline=45, lastline=54
    ]{python}{../../notebooks/economicos_train.py}
\caption{Instanciando clase Synthetic}
\label{code-economicos-synthetic}
\end{listing}