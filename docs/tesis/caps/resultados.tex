\chapter{Resultados}
Este proyecto se ha centrado en la generación de datos sintéticos a través de diversos métodos de preprocesamiento y modelos de aprendizaje automático. Los resultados se examinan en base al rendimiento de los modelos, los cuales fueron entrenados con los datos sintéticos, y se valoran respecto a la similitud, privacidad y utilidad de los datos generados.

Cabe mencionar que los resultados son inherentes a cada conjunto de datos y modelo utilizado. Por lo tanto, se proporciona un análisis exhaustivo de los resultados en cada escenario específico. Esto permite una mejor comprensión de la eficacia de los métodos utilizados en la generación de datos sintéticos y su comparación con los datos originales.

A continuación, se presentan los conjuntos de datos de King County y Económicos. El conjunto Económicos se subdivide en dos subconjuntos, que difieren en su tratamiento de los datos nulos durante el preprocesamiento.

\newpage
\section{King County}
\subsection{Reportes}
La Tabla \ref{table-score-king county-a} muestra los puntajes obtenidos por los distintos patrones utilizados en este estudio. Es notorio que los patrones con puntajes más altos, como tddpm\_mlp y smote-enc, presentan una mayor similitud con el conjunto de datos original. En contraposición, los patrones con puntajes más bajos, como ctgan, exhiben una correspondencia considerablemente menor con el conjunto original. Se muestra el promedio ± desviación estándar basado en las 3 ejecuciones realizadas.

\begin{table}[H]
\centering
\fontsize{9}{14}\selectfont
\caption{Evaluaci\'on de M\'etricas de Rendimiento para Diversos Modelos de Aprendizaje Autom\'atico, King County}
\label{table-score-king county-a}
\begin{tabular}{|l|r|r|r|r|r|}
\hline
\rowcolor[gray]{0.8}
Model Name & Column Pair Trends & Column Shapes & Coverage & Boundaries & \textbf{Score} \\
\hline tddpm\_mlp & 9.37e-01±3.80e-03 & \bfseries 9.67e-01±1.48e-03 & \bfseries 9.66e-01±4.96e-03 & 1.00e+00±0.00e+00 & 9.52e-01±2.36e-03 \\
\hline smote-enc & \bfseries 9.41e-01±2.60e-04 & 9.65e-01±3.06e-04 & 8.42e-01±8.31e-03 & \bfseries 1.00e+00±1.02e-05 & \bfseries 9.53e-01±2.45e-04 \\
\hline ctgan & 8.10e-01±1.40e-02 & 8.38e-01±2.67e-02 & 8.56e-01±2.25e-03 & 1.00e+00±0.00e+00 & 8.24e-01±2.02e-02 \\
\hline tablepreset & 8.37e-01±0.00e+00 & 8.37e-01±1.36e-16 & 7.53e-01±0.00e+00 & 1.00e+00±0.00e+00 & 8.37e-01±7.85e-17 \\
\hline copulagan & 7.64e-01±4.93e-03 & 8.14e-01±4.70e-03 & 8.40e-01±1.74e-02 & 1.00e+00±0.00e+00 & 7.89e-01±2.92e-03 \\
\hline gaussiancopula & 7.65e-01±0.00e+00 & 8.11e-01±0.00e+00 & 7.51e-01±7.85e-17 & 1.00e+00±0.00e+00 & 7.88e-01±0.00e+00 \\
\hline tvae & 7.07e-01±1.19e-02 & 7.68e-01±1.22e-02 & 4.53e-01±1.63e-02 & 1.00e+00±0.00e+00 & 7.38e-01±1.18e-02 \\
\hline
\end{tabular}
\end{table}
    
A pesar de que los patrones TDDPM y SMOTE alcanzan calificaciones prometedoras en general, se observa una diferencia significativa entre ambos en términos de cobertura (\emph{Coverage}). Específicamente, SMOTE no logra capturar la diversidad del conjunto de datos, reflejándose en una calificación de cobertura marcadamente inferior a la de TDDPM.

\newpage

\subsubsection{Correlación pairwise}

En el Anexo \ref{A-pairwise-kingcounty-top2-a-1}, se contrasta la lista completa de cada modelo. Se observa que, en general, los modelos con puntajes más altos exhiben una mayor similitud visual con los datos reales. A modo de ilustración, las imágenes \fullref{pairwise-king county-a-1-copulagan} y \fullref{pairwise-king county-a-1-gaussiancopula} contrastan los datos reales con los generados por los modelos gaussiancopula y copulagan. A pesar de que estos modelos presentan puntajes similares, el modelo gaussiancopula muestra una mayor similitud visual con los datos reales en comparación con el modelo copulagan.

\begin{figure}[H]
    \centering
    \includesvg[scale=.5,inkscapelatex=false]{datasets/kingcounty-a-1/pairwise/copulagan.svg}
    \caption{Correlación de conjunto Real y Modelo: copulagan}
    \label{pairwise-king county-a-1-copulagan}
\end{figure}

\begin{figure}[H]
    \centering
    \includesvg[scale=.6,inkscapelatex=false]{datasets/kingcounty-a-1/pairwise/gaussiancopula.svg}
    \caption{Correlación de conjunto original de entrenamiento y Gaussiancopula, King county (A-1)}
    \label{pairwise-king county-a-1-gaussiancopula}
\end{figure}

\newpage 
Cabe resaltar que, entre los modelos con puntajes superiores al 90\%, la evaluación visual para determinar cuál es superior puede ser un desafío. Esta dificultad surge debido a que, a medida que el puntaje se incrementa, la similitud visual entre los datos reales y los generados se intensifica. Esto se aprecia en las figuras \fullref{pairwise-king county-a-1-smote-enc} y \fullref{pairwise-king county-a-1-tddpm_mlp}, donde se contrastan los datos reales con los generados por los modelos smote-enc y tddpm\_mlp, respectivamente. Ambos modelos ostentan puntajes por encima del 90\%, y la correspondencia visual entre los datos reales y los generados es notoriamente alta en ambos casos.

\begin{figure}[H]
    \centering
    \includesvg[scale=.6,inkscapelatex=false]{datasets/kingcounty-a-1/pairwise/smote-enc.svg}
    \caption{Correlación de conjunto original de entrenamiento y Smote, King county (A-1)}
    \label{pairwise-king county-a-1-smote-enc}
\end{figure}

\begin{figure}[H]
    \centering
    \includesvg[scale=.5,inkscapelatex=false]{datasets/kingcounty-a-1/pairwise/tddpm_mlp.svg}
    \caption{Correlación de conjunto Real y Modelo: tddpm\_mlp}
    \label{pairwise-king county-a-1-tddpm_mlp}
\end{figure}
En la evaluación mediante SDMetrics y en la comparación visual a través de la correlación de Wise, los modelos más sobresalientes resultan ser TDDPM y SMOTE. Dichos modelos han logrado los puntajes más elevados en ambas métricas y han demostrado una notable similitud visual con los datos reales. Por ende, se puede inferir que estos modelos resultan ser los más eficaces para la generación de datos sintéticos beneficiosos para este conjunto de datos en particular.

\newpage
\subsubsection{Revisión de Columnas}
La Tabla \ref{table-coverage-king county-a} evidencia la predominancia del modelo TDDPM en términos de cobertura de los diferentes valores en general, aunque se presentan casos en los que ambos modelos no logran una cobertura completa. Por ejemplo, en la columna correspondientes a \emph{bedrooms}, TDDPM solo logra una cobertura que supera el 71.8\%, sin embargo, aún así supera a SMOTE. Por su parte, SMOTE solo alcanzan una cobertura del 51\% en la misma columna.


\begin{table}[H]
\centering
\fontsize{10}{14}\selectfont
\caption{Cobertura Categor{\'\i}a/Rango para Modelos Smote y Tddpm, King County}
\label{table-coverage-king county-a}
\begin{tabular}{|l|l|r|r|}
\hline
\rowcolor[gray]{0.8}
Columna & Metrica & smote-enc & tddpm\_mlp \\
\hline bathrooms & CategoryCoverage & 6.56e-01±3.85e-02 & \bfseries 8.11e-01±3.85e-02 \\
\hline bedrooms & CategoryCoverage & \cellcolor[rgb]{0.9, 0.54, 0.52} 5.13e-01±4.44e-02 & \cellcolor[rgb]{0.9, 0.54, 0.52} \bfseries 7.18e-01±4.44e-02 \\
\hline condition & CategoryCoverage & 9.33e-01±1.15e-01 & \bfseries 1.00e+00±0.00e+00 \\
\hline date & CategoryCoverage & \bfseries 9.64e-01±6.77e-03 & 9.44e-01±9.69e-03 \\
\hline floors & CategoryCoverage & 8.33e-01±0.00e+00 & \bfseries 9.44e-01±9.62e-02 \\
\hline grade & CategoryCoverage & 7.50e-01±0.00e+00 & \bfseries 8.61e-01±4.81e-02 \\
\hline id & RangeCoverage & 9.93e-01±4.54e-04 & \bfseries 1.00e+00±7.79e-04 \\
\hline lat & RangeCoverage & 9.65e-01±8.31e-03 & \bfseries 1.00e+00±0.00e+00 \\
\hline long & RangeCoverage & 9.91e-01±5.54e-03 & \bfseries 1.00e+00±2.45e-04 \\
\hline price & RangeCoverage & 5.72e-01±1.02e-01 & \bfseries 1.00e+00±1.25e-05 \\
\hline sqft\_above & RangeCoverage & 7.88e-01±2.98e-02 & \bfseries 1.00e+00±1.45e-05 \\
\hline sqft\_basement & RangeCoverage & 7.47e-01±2.02e-01 & \bfseries 1.00e+00±0.00e+00 \\
\hline sqft\_living & RangeCoverage & 7.03e-01±4.89e-02 & \bfseries 1.00e+00±1.33e-05 \\
\hline sqft\_living15 & RangeCoverage & 8.49e-01±5.19e-02 & \bfseries 1.00e+00±5.14e-05 \\
\hline sqft\_lot & RangeCoverage & 5.86e-01±7.02e-03 & \bfseries 1.00e+00±3.49e-06 \\
\hline sqft\_lot15 & RangeCoverage & 8.30e-01±2.80e-01 & \bfseries 1.00e+00±4.72e-05 \\
\hline view & CategoryCoverage & \bfseries 1.00e+00±0.00e+00 & \bfseries 1.00e+00±0.00e+00 \\
\hline waterfront & CategoryCoverage & \bfseries 1.00e+00±0.00e+00 & \bfseries 1.00e+00±0.00e+00 \\
\hline yr\_built & RangeCoverage & \bfseries 1.00e+00±4.11e-05 & 1.00e+00±0.00e+00 \\
\hline yr\_renovated & RangeCoverage & \bfseries 1.00e+00±9.76e-05 & 1.00e+00±0.00e+00 \\
\hline zipcode & CategoryCoverage & \bfseries 1.00e+00±0.00e+00 & \bfseries 1.00e+00±0.00e+00 \\
\hline
\end{tabular}
\end{table}


\newpage
En términos generales, la distribución en ambos modelos se aproxima a la real, en casi todos los casos superando el 90\%. La única excepción es el modelo SMOTE en la variable \emph{bathrooms}.

\begin{table}[H]
\centering
\fontsize{10}{14}\selectfont
\caption{Evaluaci\'on de Similitud de Distribuci\'on para Modelos SMOTE-ENC y TDDPM\_MLP, King County}
\label{table-shape-king county-a}
\begin{tabular}{|l|l|r|r|}
\hline
\rowcolor[gray]{0.8}
Columna & Metrica & smote-enc & tddpm\_mlp \\
\hline bathrooms & TVComplement & 8.84e-01±5.09e-03 & \bfseries 9.46e-01±6.18e-03 \\
\hline bedrooms & TVComplement & 9.18e-01±7.87e-04 & \bfseries 9.50e-01±5.73e-03 \\
\hline condition & TVComplement & 9.33e-01±1.23e-03 & \bfseries 9.61e-01±5.43e-03 \\
\hline date & TVComplement & \bfseries 9.38e-01±1.73e-03 & 9.26e-01±2.29e-03 \\
\hline floors & TVComplement & 9.66e-01±1.12e-03 & \bfseries 9.68e-01±4.38e-03 \\
\hline grade & TVComplement & 9.58e-01±6.82e-04 & \bfseries 9.64e-01±1.19e-03 \\
\hline id & KSComplement & \bfseries 9.86e-01±6.51e-04 & 9.75e-01±2.95e-03 \\
\hline lat & KSComplement & \bfseries 9.89e-01±1.69e-03 & 9.83e-01±8.10e-04 \\
\hline long & KSComplement & \bfseries 9.88e-01±2.22e-03 & 9.78e-01±1.98e-03 \\
\hline price & KSComplement & \bfseries 9.81e-01±6.63e-04 & 9.72e-01±7.86e-03 \\
\hline sqft\_above & KSComplement & 9.72e-01±1.42e-03 & \bfseries 9.77e-01±8.75e-03 \\
\hline sqft\_basement & KSComplement & 9.35e-01±3.60e-03 & \bfseries 9.75e-01±3.87e-03 \\
\hline sqft\_living & KSComplement & \bfseries 9.81e-01±2.50e-03 & 9.73e-01±5.59e-03 \\
\hline sqft\_living15 & KSComplement & \bfseries 9.81e-01±1.63e-03 & 9.76e-01±4.34e-03 \\
\hline sqft\_lot & KSComplement & \bfseries 9.83e-01±4.81e-03 & 9.58e-01±8.34e-03 \\
\hline sqft\_lot15 & KSComplement & \bfseries 9.84e-01±3.16e-03 & 9.62e-01±8.15e-03 \\
\hline view & TVComplement & 9.36e-01±9.73e-04 & \bfseries 9.52e-01±4.70e-03 \\
\hline waterfront & TVComplement & 9.94e-01±1.22e-04 & \bfseries 9.95e-01±6.04e-04 \\
\hline yr\_built & KSComplement & \bfseries 9.83e-01±4.71e-04 & 9.76e-01±6.80e-03 \\
\hline yr\_renovated & KSComplement & \bfseries 9.92e-01±4.17e-04 & 9.91e-01±1.00e-03 \\
\hline zipcode & TVComplement & \bfseries 9.74e-01±1.57e-03 & 9.50e-01±4.11e-04 \\
\hline
\end{tabular}
\end{table}


En la revisión de las columnas de los conjuntos de datos completos, se puede observar en la lista \fullref{A-pairwise-kingcounty-top2-a-1} una similitud entre los tres conjuntos analizados: Real, SMOTE y TDDPM. Sin embargo, también se pueden notar diferencias significativas. Es importante mencionar que los conjuntos de datos generados son aproximadamente un 20\% más grandes que el conjunto real.

En varias columnas, la distribución de datos en los tres conjuntos es similar, tal como ocurre en los casos de bathrooms, sqft\_lot, sqft\_above, price, sqft\_living, sqft\_basement, yr\_built, sqft\_living15 y grade. Un ejemplo de esto se puede apreciar en la Figura \ref{frecuency-top2-grade}.


\begin{figure}[H]
    \centering
    \includesvg[scale=.7,inkscapelatex=false]{datasets/kingcounty-a-1/top2/grade.svg}
    \caption{Frecuencia del campo Grade en el modelo real y Top 2, King county (A-1)}
    \label{frecuency-top2-grade}
\end{figure}

La distribución de los atributos bedrooms, condition, view y floors en el conjunto de datos generado por el modelo TDDPM presenta una característica notable: contiene un mayor número de elementos menos frecuentes en comparación con los otros conjuntos. Tomando como ejemplo la columna \emph{bedrooms} (véase \fullref{frecuency-top2-bedrooms}), la distribución de valores en el conjunto TDDPM difiere de la del conjunto SMOTE. En particular, se observa un incremento en la cantidad de registros correspondientes a los valores 6 y 1.



\begin{figure}[H]
    \centering
    \includesvg[scale=.5,inkscapelatex=false]{datasets/kingcounty-a-2/top2/bedrooms.svg}
    \caption{Frecuencia del campo bedrooms en el modelo real y top2}
    \label{frecuency-top2-bedrooms}
\end{figure}

\newpage
Por otro lado, en el caso de la columna \emph{sqft\_lot15}, la distribución generada por el modelo SMOTE resulta ser más similar a la del conjunto de datos real, como se puede apreciar en \fullref{frecuency-top2-sqft lot15}.


\begin{figure}[H]
    \centering
    \includesvg[scale=.7,inkscapelatex=false]{datasets/kingcounty-a-1/top2/sqft_lot15.svg}
    \caption{Frecuencia del campo sqft lot15 en el modelo real y top2}
    \label{frecuency-top2-sqft lot15}
\end{figure}

\newpage
\subsubsection{Privacidad}
Al analizar los registros más cercanos entre los conjuntos de datos reales utilizados para el entrenamiento, los generados por los modelos, y el conjunto de datos reales almacenados, encontramos que las distancias entre ellos se presentan en las siguientes tablas. Es importante destacar que la distancia mínima para el modelo TDDPM es de 0.0134, indicando que cada registro tiene al menos esa distancia respecto al conjunto real. La determinación del epsilon requerido para asegurar la privacidad de los datos depende del análisis específico de los datos a proteger y sus probabilidades asociadas. Sin embargo, si el objetivo es proteger el 95\% de los datos, el modelo TDDPM alcanza una distancia de 0.0579, mientras que el modelo SMOTE tiene una distancia de 0.00704.


\begin{table}[H]
\centering
\fontsize{10}{14}\selectfont
\caption{Proporción entre el más cercano y el segundo más cercano, percentil 5, datos king county}
\label{table-dcr-king county-a}
\begin{tabular}{|l|l|r|r|r|r|r|r|r|}
\hline
\rowcolor[gray]{0.8}
Modelo & NNDR ST & NNDR SH & NNDR TH & \textbf{Score} \\
\hline tddpm\_mlp & 6.12e-01±2.28e-03 & 6.03e-01±4.17e-03 & 3.76e-01±0.00e+00 & 9.52e-01±2.36e-03 \\
\hline smote-enc & \cellcolor[rgb]{0.9, 0.54, 0.52} 1.98e-01±3.57e-03 & \cellcolor[rgb]{0.9, 0.54, 0.52} 4.09e-01±6.38e-03 & 3.76e-01±0.00e+00 & \bfseries 9.53e-01±2.45e-04 \\
\hline ctgan & 8.09e-01±8.59e-03 & 8.15e-01±4.89e-03 & 3.76e-01±0.00e+00 & 8.24e-01±2.02e-02 \\
\hline tablepreset & 8.25e-01±0.00e+00 & 8.18e-01±7.85e-17 & 3.76e-01±0.00e+00 & 8.37e-01±7.85e-17 \\
\hline copulagan & \bfseries 8.30e-01±5.92e-03 & \bfseries 8.24e-01±3.49e-03 & 3.76e-01±0.00e+00 & 7.89e-01±2.92e-03 \\
\hline gaussiancopula & 7.53e-01±0.00e+00 & 7.52e-01±1.36e-16 & 3.76e-01±0.00e+00 & 7.88e-01±0.00e+00 \\
\hline tvae & 7.32e-01±5.74e-03 & 7.04e-01±4.22e-03 & 3.76e-01±0.00e+00 & \cellcolor[rgb]{0.9, 0.54, 0.52} 7.38e-01±1.18e-02 \\
\hline
\end{tabular}
\end{table}

\begin{table}[H]
\centering
\fontsize{10}{14}\selectfont
\caption{Proporción entre el más cercano y el segundo más cercano, percentil 1, datos king county}
\label{table-dcr-king county-a}
\begin{tabular}{|l|l|r|r|r|r|r|r|r|}
\hline
\rowcolor[gray]{0.8}
Modelo & NNDR ST & NNDR SH & NNDR TH & \textbf{Score} \\
\hline tddpm\_mlp & 4.49e-01±7.33e-03 & 4.49e-01±7.70e-03 & 7.98e-02±0.00e+00 & 9.52e-01±2.36e-03 \\
\hline smote-enc & \cellcolor[rgb]{0.9, 0.54, 0.52} 6.74e-02±4.35e-03 & \cellcolor[rgb]{0.9, 0.54, 0.52} 1.88e-01±1.05e-02 & 7.98e-02±0.00e+00 & \bfseries 9.53e-01±2.45e-04 \\
\hline ctgan & 7.14e-01±7.52e-03 & 7.19e-01±7.48e-03 & 7.98e-02±0.00e+00 & 8.24e-01±2.02e-02 \\
\hline tablepreset & 7.20e-01±1.11e-16 & 7.19e-01±1.11e-16 & 7.98e-02±0.00e+00 & 8.37e-01±7.85e-17 \\
\hline copulagan & \bfseries 7.56e-01±7.65e-03 & \bfseries 7.47e-01±2.64e-03 & 7.98e-02±0.00e+00 & 7.89e-01±2.92e-03 \\
\hline gaussiancopula & 6.48e-01±7.85e-17 & 6.43e-01±7.85e-17 & 7.98e-02±0.00e+00 & 7.88e-01±0.00e+00 \\
\hline tvae & 6.06e-01±1.27e-02 & 5.87e-01±4.93e-03 & 7.98e-02±0.00e+00 & \cellcolor[rgb]{0.9, 0.54, 0.52} 7.38e-01±1.18e-02 \\
\hline
\end{tabular}
\end{table}

\begin{table}[H]
\centering
\fontsize{10}{14}\selectfont
\caption{Proporción entre el más cercano y el segundo más cercano, mínimo, datos king county}
\label{table-dcr-king county-a}
\begin{tabular}{|l|l|r|r|r|r|r|r|r|}
\hline
\rowcolor[gray]{0.8}
Modelo & NNDR ST & NNDR SH & NNDR TH & \textbf{Score} \\
\hline tddpm\_mlp & 1.23e-01±1.54e-02 & 1.57e-01±3.36e-02 & 0.00e+00±0.00e+00 & 9.52e-01±2.36e-03 \\
\hline smote-enc & \cellcolor[rgb]{0.9, 0.54, 0.52} 0.00e+00±0.00e+00 & \cellcolor[rgb]{0.9, 0.54, 0.52} 1.10e-02±5.41e-03 & 0.00e+00±0.00e+00 & \bfseries 9.53e-01±2.45e-04 \\
\hline ctgan & 4.25e-01±3.23e-02 & 3.91e-01±4.06e-02 & 0.00e+00±0.00e+00 & 8.24e-01±2.02e-02 \\
\hline tablepreset & 4.51e-01±6.80e-17 & 3.58e-01±0.00e+00 & 0.00e+00±0.00e+00 & 8.37e-01±7.85e-17 \\
\hline copulagan & \bfseries 5.48e-01±1.58e-02 & \bfseries 5.32e-01±3.85e-02 & 0.00e+00±0.00e+00 & 7.89e-01±2.92e-03 \\
\hline gaussiancopula & 3.90e-01±5.55e-17 & 4.08e-01±0.00e+00 & 0.00e+00±0.00e+00 & 7.88e-01±0.00e+00 \\
\hline tvae & 3.44e-01±1.91e-02 & 3.43e-01±1.63e-02 & 0.00e+00±0.00e+00 & \cellcolor[rgb]{0.9, 0.54, 0.52} 7.38e-01±1.18e-02 \\
\hline
\end{tabular}
\end{table}

\newpage
Al analizar los ratios entre la distancia al primer vecino más cercano y la distancia al segundo para el modelo TDDPM, se evidencia que para el percentil 5, la distancia al vecino más cercano es solo 2/3 de la distancia al segundo más cercano. Sin embargo, para el percentil 1, esta distancia se reduce a la mitad. En contraposición, para el modelo SMOTE, en el percentil 5, la distancia al vecino más cercano es solo un 20\% de la distancia al segundo más cercano, y disminuye rápidamente a un 6\% para el percentil 1.

\begin{table}[H]
\centering
\fontsize{10}{14}\selectfont
\caption{Proporción entre el más cercano y el segundo más cercano, percentil 5, datos king county}
\label{table-dcr-king county-a}
\begin{tabular}{|l|l|r|r|r|r|r|r|r|}
\hline
\rowcolor[gray]{0.8}
Modelo & NNDR ST & NNDR SH & NNDR TH & \textbf{Score} \\
\hline tddpm\_mlp & 6.12e-01±2.28e-03 & 6.03e-01±4.17e-03 & 3.76e-01±0.00e+00 & \bfseries 9.52e-01±2.36e-03 \\
\hline smote-enc & \cellcolor[rgb]{0.9, 0.54, 0.52} 1.98e-01±3.57e-03 & \cellcolor[rgb]{0.9, 0.54, 0.52} 4.09e-01±6.38e-03 & 3.76e-01±0.00e+00 & 9.53e-01±2.45e-04 \\
\hline ctgan & 8.09e-01±8.59e-03 & 8.15e-01±4.89e-03 & 3.76e-01±0.00e+00 & 8.24e-01±2.02e-02 \\
\hline tablepreset & 8.25e-01±1.11e-16 & 8.18e-01±0.00e+00 & 3.76e-01±0.00e+00 & 8.37e-01±7.85e-17 \\
\hline copulagan & \bfseries 8.30e-01±5.92e-03 & \bfseries 8.24e-01±3.49e-03 & 3.76e-01±0.00e+00 & 7.89e-01±2.92e-03 \\
\hline gaussiancopula & 7.53e-01±1.11e-16 & 7.52e-01±1.36e-16 & 3.76e-01±0.00e+00 & 7.88e-01±0.00e+00 \\
\hline tvae & 7.32e-01±5.74e-03 & 7.04e-01±4.22e-03 & 3.76e-01±0.00e+00 & \cellcolor[rgb]{0.9, 0.54, 0.52} 7.38e-01±1.18e-02 \\
\hline
\end{tabular}
\end{table}

\begin{table}[H]
\centering
\fontsize{10}{14}\selectfont
\caption{Proporción entre el más cercano y el segundo más cercano, percentil 1, datos king county}
\label{table-dcr-king county-a}
\begin{tabular}{|l|l|r|r|r|r|r|r|r|}
\hline
\rowcolor[gray]{0.8}
Modelo & NNDR ST & NNDR SH & NNDR TH & \textbf{Score} \\
\hline tddpm\_mlp & 4.49e-01±7.33e-03 & 4.49e-01±7.70e-03 & 7.98e-02±0.00e+00 & \bfseries 9.52e-01±2.36e-03 \\
\hline smote-enc & \cellcolor[rgb]{0.9, 0.54, 0.52} 6.74e-02±4.35e-03 & \cellcolor[rgb]{0.9, 0.54, 0.52} 1.88e-01±1.05e-02 & 7.98e-02±0.00e+00 & 9.53e-01±2.45e-04 \\
\hline ctgan & 7.14e-01±7.52e-03 & 7.19e-01±7.48e-03 & 7.98e-02±0.00e+00 & 8.24e-01±2.02e-02 \\
\hline tablepreset & 7.20e-01±0.00e+00 & 7.19e-01±1.11e-16 & 7.98e-02±0.00e+00 & 8.37e-01±7.85e-17 \\
\hline copulagan & \bfseries 7.56e-01±7.65e-03 & \bfseries 7.47e-01±2.64e-03 & 7.98e-02±0.00e+00 & 7.89e-01±2.92e-03 \\
\hline gaussiancopula & 6.48e-01±7.85e-17 & 6.43e-01±1.36e-16 & 7.98e-02±0.00e+00 & 7.88e-01±0.00e+00 \\
\hline tvae & 6.06e-01±1.27e-02 & 5.87e-01±4.93e-03 & 7.98e-02±0.00e+00 & \cellcolor[rgb]{0.9, 0.54, 0.52} 7.38e-01±1.18e-02 \\
\hline
\end{tabular}
\end{table}

\begin{table}[H]
\centering
\fontsize{10}{14}\selectfont
\caption{Proporción entre el más cercano y el segundo más cercano, minimo, datos king county}
\label{table-dcr-king county-a}
\begin{tabular}{|l|l|r|r|r|r|r|r|r|}
\hline
\rowcolor[gray]{0.8}
Modelo & NNDR ST & NNDR SH & NNDR TH & \textbf{Score} \\
\hline tddpm\_mlp & 1.23e-01±1.54e-02 & 1.57e-01±3.36e-02 & 0.00e+00±0.00e+00 & \bfseries 9.52e-01±2.36e-03 \\
\hline smote-enc & \cellcolor[rgb]{0.9, 0.54, 0.52} 0.00e+00±0.00e+00 & \cellcolor[rgb]{0.9, 0.54, 0.52} 1.10e-02±5.41e-03 & 0.00e+00±0.00e+00 & 9.53e-01±2.45e-04 \\
\hline ctgan & 4.25e-01±3.23e-02 & 3.91e-01±4.06e-02 & 0.00e+00±0.00e+00 & 8.24e-01±2.02e-02 \\
\hline tablepreset & 4.51e-01±6.80e-17 & 3.58e-01±5.55e-17 & 0.00e+00±0.00e+00 & 8.37e-01±7.85e-17 \\
\hline copulagan & \bfseries 5.48e-01±1.58e-02 & \bfseries 5.32e-01±3.85e-02 & 0.00e+00±0.00e+00 & 7.89e-01±2.92e-03 \\
\hline gaussiancopula & 3.90e-01±3.93e-17 & 4.08e-01±0.00e+00 & 0.00e+00±0.00e+00 & 7.88e-01±0.00e+00 \\
\hline tvae & 3.44e-01±1.91e-02 & 3.43e-01±1.63e-02 & 0.00e+00±0.00e+00 & \cellcolor[rgb]{0.9, 0.54, 0.52} 7.38e-01±1.18e-02 \\
\hline
\end{tabular}
\end{table}

\newpage
%\begin{table}[H]
\centering
\caption{Distancia de registros más cercanos entre conjuntos Sinteticos, \emph{Train} y \emph{Hold}}
\label{table-dcr-king county-a-1}
\begin{tabular}{|l|l|r|r|r|r|}
\hline
\rowcolor[gray]{0.8}
Modelo & DCR ST & DCR SH & DCR TH & 	extbf\{Score\} \\
\hline tddpm\_mlp & 0.058 & 0.077 & \bfseries 0.036 & \bfseries 0.953 \\
\hline smote-enc & \cellcolor{red} 0.007 & \cellcolor{red} 0.037 & \bfseries 0.036 & 0.953 \\
\hline tablepreset & 0.180 & 0.200 & \bfseries 0.036 & 0.837 \\
\hline ctgan & 0.217 & 0.240 & \bfseries 0.036 & 0.832 \\
\hline gaussiancopula & 0.263 & 0.306 & \bfseries 0.036 & 0.788 \\
\hline copulagan & \bfseries 0.364 & \bfseries 0.404 & \bfseries 0.036 & 0.786 \\
\hline tvae & 0.081 & 0.099 & \bfseries 0.036 & 0.735 \\
\hline
\end{tabular}
\end{table}

%\begin{table}[H]
\centering
\caption{Distancia de registros más cercanos entre conjuntos Sinteticos, \emph{Train} y \emph{Hold}}
\label{table-dcr-king county-a-2}
\begin{tabular}{|l|l|r|r|r|r|r|r|r|}
\hline
\rowcolor[gray]{0.8}
Modelo & DCR ST & DCR SH & DCR TH & NNDR ST & NNDR SH & NNDR TH & 	extbf\{Score\} \\
\hline smote-enc & \cellcolor{red} 0.007 & \cellcolor{red} 0.037 & \bfseries 0.036 & nan & 0.407 & \bfseries 0.376 & \bfseries 0.953 \\
\hline tddpm\_mlp & 0.057 & 0.075 & \bfseries 0.036 & 0.610 & 0.601 & \bfseries 0.376 & 0.949 \\
\hline tablepreset & 0.180 & 0.200 & \bfseries 0.036 & 0.825 & 0.818 & \bfseries 0.376 & 0.837 \\
\hline ctgan & 0.227 & 0.250 & \bfseries 0.036 & 0.806 & \bfseries 0.819 & \bfseries 0.376 & 0.801 \\
\hline copulagan & \bfseries 0.381 & \bfseries 0.416 & \bfseries 0.036 & \bfseries 0.831 & 0.416 & \bfseries 0.376 & 0.791 \\
\hline gaussiancopula & 0.263 & 0.306 & \bfseries 0.036 & 0.753 & 0.752 & \bfseries 0.376 & 0.788 \\
\hline tvae & 0.081 & 0.099 & \bfseries 0.036 & 0.736 & 0.705 & \bfseries 0.376 & 0.751 \\
\hline
\end{tabular}
\end{table}

%\begin{table}[H]
\centering
\caption{Distancia de registros más cercanos entre conjuntos Sinteticos, \emph{Train} y \emph{Hold}}
\label{table-dcr-king county-a-3}
\begin{tabular}{|l|l|r|r|r|r|r|r|r|}
\hline
\rowcolor[gray]{0.8}
Modelo & DCR ST & DCR SH & DCR TH & NNDR ST & NNDR SH & NNDR TH & 	extbf\{Score\} \\
\hline tddpm\_mlp & 0.058 & 0.077 & \bfseries 0.036 & 0.614 & 0.601 & \bfseries 0.376 & \bfseries 0.954 \\
\hline smote-enc & \cellcolor{red} 0.007 & \cellcolor{red} 0.036 & \bfseries 0.036 & nan & 0.404 & \bfseries 0.376 & 0.953 \\
\hline ctgan & 0.201 & 0.224 & \bfseries 0.036 & 0.801 & 0.809 & \bfseries 0.376 & 0.839 \\
\hline tablepreset & 0.180 & 0.200 & \bfseries 0.036 & 0.825 & 0.818 & \bfseries 0.376 & 0.837 \\
\hline copulagan & \bfseries 0.380 & \bfseries 0.416 & \bfseries 0.036 & \bfseries 0.835 & \bfseries 0.820 & \bfseries 0.376 & 0.790 \\
\hline gaussiancopula & 0.263 & 0.306 & \bfseries 0.036 & 0.753 & 0.752 & \bfseries 0.376 & 0.788 \\
\hline tvae & 0.081 & 0.098 & \bfseries 0.036 & 0.725 & 0.699 & \bfseries 0.376 & 0.728 \\
\hline
\end{tabular}
\end{table}



En la Figura \ref{frecuency-top2-privacy} solo se consideran los modelos TDDPM y SMOTE para su comparación. En ambos casos, existe una distancia mayor a cero. Sin embargo, esta distancia es mayor en el caso de TDDPM, lo que sugiere que este conjunto puede ser considerado superior en términos de privacidad.


\begin{figure}[H]
    \centering
    \includesvg[scale=.7,inkscapelatex=false]{datasets/kingcounty-a-2/top2/privacy.svg}
    \caption{Frecuencia del campo Privacy en el modelo real y Top 2}
    \label{frecuency-top2-privacy}
\end{figure}


\newpage
\section{Economicos}
El conjunto de Económicos, a diferencia del conjunto de datos de King County que fue filtrado y preprocesado para evitar valores nulos, contiene elementos nulos. A continuación se describen dos tratamientos de estos elementos nulos. El primer enfoque simplemente elimina todos los registros que contienen un registro vacío utilizando el método `dropna`, como se muestra en el Código \ref{codigo-remove-nan}; este será considerado como el Conjunto A. En el segundo enfoque, los valores nulos son reemplazados por algún valor predeterminado o calculado, como se muestra en el Código \ref{codigo-replace-nan}; este será considerado como el Conjunto B.

\begin{listing}[H]
    \begin{minted}[linenos=true,frame=lines,framesep=2mm,baselinestretch=1.2]{python}
df_converted = df.dropna().astype({k: 'str' for k in ("description", "price", "title", "address", "owner",)})
basedate = pd.Timestamp('2017-12-01')
dtime = df_converted.pop("publication_date")
df_converted["publication_date"] = dtime.apply(lambda x: (x - basedate).days)
    \end{minted}
\caption{Eliminación de valores nulos en el conjunto de datos de Económicos}
\label{codigo-remove-nan}
\end{listing}

\begin{listing}[H]
    \begin{minted}[linenos=true,frame=lines,framesep=2mm,baselinestretch=1.2]{python}
df_converted = df.fillna(dict(
            property_type = "None",
            transaction_type = "None",
            state = "None",
            county = "None",
            rooms = -1,
            bathrooms = -1,
            m_built = -1,
            m_size = -1,
            source = "None"
    )).fillna(-1).astype({k: 'str' for k in ("description", "price", "title", "address", "owner",)})
basedate = pd.Timestamp('2017-12-01')
dtime = df_converted.pop("publication_date")
df_converted["publication_date"] = dtime.apply(lambda x: (x - basedate).days)
    \end{minted}
\caption{Reemplazo de valores nulos en el conjunto de datos de Económicos}
\label{codigo-replace-nan}
\end{listing}

\newpage
\subsection{Reportes - Conjunto A}
\label{ds-conjunto-a}
Para el conjunto A, como se muestra en la Tabla \ref{table-score-economicos-a}, TDDPM es un punto superior a SMOTE y ambos superan en más de 10 puntos al siguiente modelo. Sin embargo, un punto crucial es que SMOTE tiene una cobertura (\emph{Coverage}) que es 12 puntos inferior a TDDPM.

\begin{table}[H]
\centering
\fontsize{7}{14}\selectfont
\caption{Evaluación de Métricas de Rendimiento para Diversos Modelos de Aprendizaje Automático, Economicos}
\label{table-score-economicos-a}
\begin{tabular}{|l|r|r|r|r|r|r|}
\hline
 \rowcolor[gray]{0.8}
Model Name & Column Pair Trends & Column Shapes & Coverage & Boundaries & Synthesis & \textbf{Score} \\
\hline tddpm\_mlp & \bfseries 9.72e-01±1.50e-03 & \bfseries 9.83e-01±1.09e-03 & \bfseries 8.12e-01±1.89e-02 & \bfseries 1.00e+00±0.00e+00 & 9.90e-01±8.52e-04 & \bfseries 9.77e-01±6.88e-04 \\
\hline smote-enc & 9.59e-01±1.20e-03 & 9.76e-01±4.34e-04 & 6.27e-01±1.31e-02 & \bfseries 1.00e+00±0.00e+00 & 9.24e-01±1.97e-03 & 9.67e-01±8.19e-04 \\
\hline copulagan & 7.60e-01±1.58e-02 & 8.02e-01±2.69e-02 & 6.80e-01±6.95e-03 & \bfseries 1.00e+00±0.00e+00 & \bfseries 1.00e+00±0.00e+00 & 7.81e-01±2.03e-02 \\
\hline ctgan & 7.43e-01±1.27e-02 & 6.49e-01±1.69e-02 & 6.76e-01±7.85e-04 & \bfseries 1.00e+00±0.00e+00 & \bfseries 1.00e+00±0.00e+00 & 6.96e-01±1.00e-02 \\
\hline gaussiancopula & 6.94e-01±0.00e+00 & 6.87e-01±6.41e-17 & 5.52e-01±9.06e-17 & \bfseries 1.00e+00±0.00e+00 & \bfseries 1.00e+00±0.00e+00 & 6.91e-01±6.41e-17 \\
\hline tvae & 5.95e-01±4.07e-03 & 6.85e-01±2.67e-03 & 9.53e-02±1.48e-03 & \bfseries 1.00e+00±0.00e+00 & \bfseries 1.00e+00±0.00e+00 & 6.40e-01±3.35e-03 \\
\hline
\end{tabular}
\end{table}


\newpage
Es interesante observar que, para el percentil 1 y el 5, el modelo TDDPM muestra que la cercanía de los registros más cercanos es predominante al comparar el conjunto sintético y el conjunto de retención (\emph{Hold}). Este comportamiento no se observa con ninguna otra comparación. También se nota que las diferencias mínimas alcanzan cero en los dos mejores modelos (TDDPM y SMOTE), y que los valores de distancia son muy pequeños. En el percentil 5, TDDPM registra una distancia de $4.48 \times 10^{-9}$.


\begin{table}[H]
\centering
\fontsize{10}{14}\selectfont
\caption{Distancia de registros más cercanos entre conjuntos Sinteticos, percentil 5, Economicos}
\label{table-dcr-economicos-a-5th}
\begin{tabular}{|l|l|r|r|r|r|}
\hline
\rowcolor[gray]{0.8}
Modelo & DCR ST & DCR SH & DCR TH & \textbf{Score} \\
\hline tddpm\_mlp & 4.29e-09±2.16e-10 & \cellcolor[rgb]{0.9, 0.54, 0.52} 3.50e-08±1.92e-09 & \bfseries \cellcolor[rgb]{0.9, 0.54, 0.52} 1.28e-08±0.00e+00 & \bfseries 9.77e-01±6.88e-04 \\
\hline smote-enc & \cellcolor[rgb]{0.9, 0.54, 0.52} 2.90e-11±1.13e-12 & 4.41e-08±2.36e-09 & \bfseries \cellcolor[rgb]{0.9, 0.54, 0.52} 1.28e-08±0.00e+00 & 9.67e-01±8.19e-04 \\
\hline ctgan & \bfseries 7.59e-06±5.75e-06 & \bfseries 1.91e-05±2.01e-05 & \bfseries \cellcolor[rgb]{0.9, 0.54, 0.52} 1.28e-08±0.00e+00 & 6.96e-01±1.00e-02 \\
\hline copulagan & 1.27e-06±3.04e-07 & 2.73e-06±5.89e-07 & \bfseries \cellcolor[rgb]{0.9, 0.54, 0.52} 1.28e-08±0.00e+00 & 7.81e-01±2.03e-02 \\
\hline gaussiancopula & 5.11e-06±0.00e+00 & 8.25e-06±0.00e+00 & \bfseries \cellcolor[rgb]{0.9, 0.54, 0.52} 1.28e-08±0.00e+00 & 6.91e-01±6.41e-17 \\
\hline tvae & 2.19e-07±1.60e-09 & 4.15e-07±5.43e-09 & \bfseries \cellcolor[rgb]{0.9, 0.54, 0.52} 1.28e-08±0.00e+00 & \cellcolor[rgb]{0.9, 0.54, 0.52} 6.40e-01±3.35e-03 \\
\hline
\end{tabular}
\end{table}

\begin{table}[H]
\centering
\fontsize{10}{14}\selectfont
\caption{Distancia de registros más cercanos, percentil 1, datos economicos}
\label{table-dcr-economicos-a-1th}
\begin{tabular}{|l|l|r|r|r|r|r|r|r|}
\hline
\rowcolor[gray]{0.8}
Modelo & DCR ST & DCR SH & DCR TH & \textbf{Score} \\
\hline tddpm\_mlp & 1.46e-10±3.86e-12 & \cellcolor[rgb]{0.9, 0.54, 0.52} 1.44e-09±1.01e-10 & 0.00e+00±0.00e+00 & \bfseries 9.79e-01±1.27e-03 \\
\hline smote-enc & \cellcolor[rgb]{0.9, 0.54, 0.52} 0.00e+00±0.00e+00 & 1.54e-09±5.32e-13 & 0.00e+00±0.00e+00 & 9.69e-01±6.71e-04 \\
\hline copulagan & 1.97e-07±4.64e-08 & 4.53e-07±9.95e-08 & 0.00e+00±0.00e+00 & 7.68e-01±2.96e-02 \\
\hline ctgan & \bfseries 3.18e-06±4.34e-07 & \bfseries 5.23e-06±1.44e-06 & 0.00e+00±0.00e+00 & 6.98e-01±2.63e-02 \\
\hline gaussiancopula & 7.84e-07±7.49e-23 & 1.75e-06±0.00e+00 & 0.00e+00±0.00e+00 & 6.92e-01±0.00e+00 \\
\hline tvae & 1.48e-07±9.24e-08 & 2.35e-07±1.18e-07 & 0.00e+00±0.00e+00 & \cellcolor[rgb]{0.9, 0.54, 0.52} 6.12e-01±2.50e-02 \\
\hline
\end{tabular}
\end{table}

\begin{table}[H]
\centering
\fontsize{10}{14}\selectfont
\caption{Distancia de registros más cercanos, minimo, datos economicos}
\label{table-dcr-economicos-a}
\begin{tabular}{|l|l|r|r|r|r|r|r|r|}
\hline
\rowcolor[gray]{0.8}
Modelo & DCR ST & DCR SH & DCR TH & \textbf{Score} \\
\hline tddpm\_mlp & \cellcolor[rgb]{0.9, 0.54, 0.52} 0.00e+00±0.00e+00 & \cellcolor[rgb]{0.9, 0.54, 0.52} 0.00e+00±0.00e+00 & 0.00e+00±0.00e+00 & \bfseries 9.79e-01±1.27e-03 \\
\hline smote-enc & \cellcolor[rgb]{0.9, 0.54, 0.52} 0.00e+00±0.00e+00 & \cellcolor[rgb]{0.9, 0.54, 0.52} 0.00e+00±0.00e+00 & 0.00e+00±0.00e+00 & 9.69e-01±6.71e-04 \\
\hline copulagan & 5.88e-09±2.05e-09 & 1.21e-08±3.19e-09 & 0.00e+00±0.00e+00 & 7.68e-01±2.96e-02 \\
\hline ctgan & \bfseries 2.83e-08±3.88e-08 & \bfseries 6.05e-08±2.56e-08 & 0.00e+00±0.00e+00 & 6.98e-01±2.63e-02 \\
\hline gaussiancopula & 1.13e-08±0.00e+00 & 1.75e-08±0.00e+00 & 0.00e+00±0.00e+00 & 6.92e-01±0.00e+00 \\
\hline tvae & 5.65e-09±3.07e-09 & 2.56e-08±3.04e-08 & 0.00e+00±0.00e+00 & \cellcolor[rgb]{0.9, 0.54, 0.52} 6.12e-01±2.50e-02 \\
\hline
\end{tabular}
\end{table}


\newpage
También se puede observar una disminución en la relación entre el registro más cercano y el segundo más cercano en comparación con el conjunto de datos de King County. En el percentil 5, el segundo registro más cercano está a 15 veces la distancia del primero. Esta relación se reduce a 10 veces cuando se compara con el conjunto \emph{Hold}.

\begin{table}[H]
\centering
\fontsize{10}{14}\selectfont
\caption{Proporción entre el más cercano y el segundo más cercano, percentil 5, Economicos}
\label{table-nndr-economicos-a-5th}
\begin{tabular}{|l|l|r|r|r|r|}
\hline
\rowcolor[gray]{0.8}
Modelo & NNDR ST & NNDR SH & NNDR TH & \textbf{Score} \\
\hline tddpm\_mlp & 6.79e-02±7.37e-04 & \bfseries 1.00e-01±2.26e-03 & \cellcolor[rgb]{0.9, 0.54, 0.52} \bfseries 1.31e-02±0.00e+00 & \cellcolor[rgb]{0.9, 0.54, 0.52} 9.77e-01±6.88e-04 \\
\hline smote-enc & \bfseries 7.15e-04±7.49e-06 & 1.14e-01±4.79e-03 & 1.31e-02±0.00e+00 & 9.67e-01±8.19e-04 \\
\hline ctgan & 2.57e-01±8.81e-03 & 3.27e-01±4.72e-02 & 1.31e-02±0.00e+00 & 6.96e-01±1.00e-02 \\
\hline copulagan & 2.01e-01±1.27e-02 & 2.23e-01±5.47e-02 & 1.31e-02±0.00e+00 & 7.81e-01±2.03e-02 \\
\hline gaussiancopula & \cellcolor[rgb]{0.9, 0.54, 0.52} 3.07e-01±0.00e+00 & 2.76e-01±0.00e+00 & 1.31e-02±0.00e+00 & 6.91e-01±6.41e-17 \\
\hline tvae & 3.02e-01±6.15e-03 & \cellcolor[rgb]{0.9, 0.54, 0.52} 3.49e-01±1.26e-03 & 1.31e-02±0.00e+00 & \bfseries 6.40e-01±3.35e-03 \\
\hline
\end{tabular}
\end{table}

\begin{table}[H]
\centering
\fontsize{10}{14}\selectfont
\caption{Proporción entre el más cercano y el segundo más cercano, percentil 1, Economicos}
\label{table-nndr-economicos-a-1th}
\begin{tabular}{|l|l|r|r|r|r|}
\hline
\rowcolor[gray]{0.8}
Modelo & NNDR ST & NNDR SH & NNDR TH & \textbf{Score} \\
\hline tddpm\_mlp & 2.52e-03±1.71e-04 & 9.61e-03±1.17e-04 & \cellcolor[rgb]{0.9, 0.54, 0.52} \bfseries 0.00e+00±0.00e+00 & \cellcolor[rgb]{0.9, 0.54, 0.52} 9.77e-01±6.88e-04 \\
\hline smote-enc & \bfseries 0.00e+00±0.00e+00 & \bfseries 3.00e-03±1.28e-03 & 0.00e+00±0.00e+00 & 9.67e-01±8.19e-04 \\
\hline ctgan & 1.35e-02±2.69e-03 & 2.77e-02±1.69e-02 & 0.00e+00±0.00e+00 & 6.96e-01±1.00e-02 \\
\hline copulagan & 1.03e-02±1.29e-03 & 1.11e-02±2.64e-03 & 0.00e+00±0.00e+00 & 7.81e-01±2.03e-02 \\
\hline gaussiancopula & 2.90e-02±0.00e+00 & 2.94e-02±3.47e-18 & 0.00e+00±0.00e+00 & 6.91e-01±6.41e-17 \\
\hline tvae & \cellcolor[rgb]{0.9, 0.54, 0.52} 3.26e-02±1.29e-02 & \cellcolor[rgb]{0.9, 0.54, 0.52} 1.45e-01±1.91e-03 & 0.00e+00±0.00e+00 & \bfseries 6.40e-01±3.35e-03 \\
\hline
\end{tabular}
\end{table}

\begin{table}[H]
\centering
\fontsize{10}{14}\selectfont
\caption{Proporción entre el más cercano y el segundo más cercano, minimo, datos economicos}
\label{table-dcr-economicos-a}
\begin{tabular}{|l|l|r|r|r|r|r|r|r|}
\hline
\rowcolor[gray]{0.8}
Modelo & NNDR ST & NNDR SH & NNDR TH & \textbf{Score} \\
\hline tddpm\_mlp & \cellcolor[rgb]{0.9, 0.54, 0.52} 0.00e+00±0.00e+00 & \cellcolor[rgb]{0.9, 0.54, 0.52} 0.00e+00±0.00e+00 & 0.00e+00±0.00e+00 & \bfseries 9.79e-01±1.27e-03 \\
\hline smote-enc & \cellcolor[rgb]{0.9, 0.54, 0.52} 0.00e+00±0.00e+00 & \cellcolor[rgb]{0.9, 0.54, 0.52} 0.00e+00±0.00e+00 & 0.00e+00±0.00e+00 & 9.69e-01±6.71e-04 \\
\hline copulagan & 1.22e-04±7.03e-05 & 1.84e-04±1.12e-04 & 0.00e+00±0.00e+00 & 7.68e-01±2.96e-02 \\
\hline ctgan & 4.21e-04±2.19e-04 & 1.32e-03±1.54e-03 & 0.00e+00±0.00e+00 & 6.98e-01±2.63e-02 \\
\hline gaussiancopula & 4.99e-05±0.00e+00 & 7.59e-06±8.47e-22 & 0.00e+00±0.00e+00 & 6.92e-01±0.00e+00 \\
\hline tvae & \bfseries 8.11e-04±1.77e-04 & \bfseries 7.24e-03±3.14e-03 & 0.00e+00±0.00e+00 & \cellcolor[rgb]{0.9, 0.54, 0.52} 6.12e-01±2.50e-02 \\
\hline
\end{tabular}
\end{table}

\newpage
En las tablas detalles de cobertura \ref{table-coverage-economicos-a} se puede ver el porqué ambos tenían una puntuación tan baja. Existen elementos con una cobertura menor al 40\%, por ejemplo, la variable \texttt{m\_size}. Aun así, se puede ver que TDDPM es ligeramente mejor en la mayoría de las columnas.

%\begin{table}[H]
\centering
\caption{Distancia de registros más cercanos entre conjuntos Sinteticos, \emph{Train} y \emph{Hold}}
\label{table-dcr-economicos-a-1}
\begin{tabular}{|l|l|r|r|r|r|r|r|r|}
\hline
\rowcolor[gray]{0.8}
Modelo & DCR ST & DCR SH & DCR TH & NNDR ST & NNDR SH & NNDR TH & \textbf{Score} \\
\hline tddpm\_mlp & 4.605e-09 & \cellcolor[rgb]{0.9, 0.54, 0.52} 3.719e-08 & \bfseries 1.282e-08 & 0.069 & 3.719e-08 & \bfseries 0.013 & \bfseries 0.977 \\
\hline smote-enc & \cellcolor[rgb]{0.9, 0.54, 0.52} 2.858e-11 & 4.348e-08 & \bfseries 1.282e-08 & 0.001 & 4.348e-08 & \bfseries 0.013 & 0.968 \\
\hline copulagan & 1.566e-06 & 3.272e-06 & \bfseries 1.282e-08 & 0.311 & 3.272e-06 & \bfseries 0.013 & 0.778 \\
\hline gaussiancopula & 5.276e-06 & 8.215e-06 & \bfseries 1.282e-08 & 0.293 & 8.215e-06 & \bfseries 0.013 & 0.692 \\
\hline ctgan & \bfseries 1.914e-05 & \bfseries 3.186e-05 & \bfseries 1.282e-08 & 0.250 & \bfseries 3.186e-05 & \bfseries 0.013 & 0.674 \\
\hline tvae & 2.813e-07 & 4.954e-07 & \bfseries 1.282e-08 & \bfseries 0.320 & 4.954e-07 & \bfseries 0.013 & 0.623 \\
\hline
\end{tabular}
\end{table}

\begin{table}[H]
\centering
\fontsize{10}{14}\selectfont
\caption{Evaluaci\'on de Cobertura Categor{\'\i}a-Rango para Modelos SMOTE-ENC y TDDPM\_MLP, Economicos}
\label{table-coverage-economicos-a}
\begin{tabular}{|l|l|r|r|}
\hline
\rowcolor[gray]{0.8}
Columna & Metrica & smote-enc & tddpm\_mlp \\
\hline \_price & RangeCoverage & \bfseries 9.68e-01±5.48e-02 & 9.66e-01±3.30e-02 \\
\hline bathrooms & CategoryCoverage & \bfseries 8.63e-01±3.40e-02 & 6.76e-01±2.94e-02 \\
\hline county & CategoryCoverage & 5.97e-01±3.73e-03 & \bfseries 7.87e-01±2.27e-02 \\
\hline m\_built & RangeCoverage & 5.52e-01±3.16e-01 & \bfseries 7.71e-01±3.97e-01 \\
\hline m\_size & RangeCoverage & \cellcolor[rgb]{0.9, 0.54, 0.52} 1.79e-02±8.52e-03 & \cellcolor[rgb]{0.9, 0.54, 0.52} \bfseries 3.36e-01±4.53e-02 \\
\hline property\_type & CategoryCoverage & 6.67e-01±5.56e-02 & \bfseries 9.07e-01±3.21e-02 \\
\hline publication\_date & RangeCoverage & 9.70e-01±5.80e-03 & \bfseries 9.81e-01±2.86e-03 \\
\hline rooms & CategoryCoverage & 7.40e-01±1.41e-02 & \bfseries 7.80e-01±6.45e-02 \\
\hline state & CategoryCoverage & 7.92e-01±3.61e-02 & \bfseries 9.58e-01±3.61e-02 \\
\hline transaction\_type & CategoryCoverage & 5.00e-01±0.00e+00 & \bfseries 7.50e-01±2.50e-01 \\
\hline
\end{tabular}
\end{table}


Ambos modelos muestran un buen rendimiento en cuanto a la forma y la distribución de los datos, como se evidencia en la Tabla \ref{table-shape-economicos-a}.
%\begin{table}[H]
\centering
\caption{Scores Economicos}
\label{table-coverage-economicos-a-1}
\begin{tabular}{|l|l|r|r|}
\hline
\rowcolor[gray]{0.8}
Columna & Metrica & smote-enc & tddpm\_mlp \\
\hline \_price & RangeCoverage & \bfseries 1.000 & 0.964 \\
\hline bathrooms & CategoryCoverage & \bfseries 0.824 & 0.647 \\
\hline county & CategoryCoverage & 0.593 & \bfseries 0.799 \\
\hline m\_built & RangeCoverage & 0.255 & \bfseries 1.000 \\
\hline m\_size & RangeCoverage & 0.022 & \bfseries 0.287 \\
\hline property\_type & CategoryCoverage & 0.667 & \bfseries 0.944 \\
\hline publication\_date & RangeCoverage & 0.971 & \bfseries 0.983 \\
\hline rooms & CategoryCoverage & \bfseries 0.732 & 0.707 \\
\hline state & CategoryCoverage & 0.812 & \bfseries 0.938 \\
\hline transaction\_type & CategoryCoverage & 0.500 & \bfseries 0.750 \\
\hline
\end{tabular}
\end{table}

\begin{table}[H]
\centering
\caption{Evaluación de Similitud de Distribución para Modelos Smote y Tddpm, Economicos}
\label{table-shape-economicos-a}
\begin{tabular}{|l|l|r|r|}
\hline
\rowcolor[gray]{0.8}
Columna & Metrica & smote-enc & tddpm\_mlp \\
\hline \_price & KSComplement & \bfseries 9.91e-01±3.85e-04 & 9.84e-01±3.53e-03 \\
\hline bathrooms & TVComplement & \bfseries 9.94e-01±6.66e-04 & 9.87e-01±2.15e-03 \\
\hline county & TVComplement & \cellcolor[rgb]{0.9, 0.54, 0.52} 9.22e-01±9.28e-04 & \bfseries \cellcolor[rgb]{0.9, 0.54, 0.52} 9.66e-01±2.10e-03 \\
\hline m\_built & KSComplement & \bfseries 9.87e-01±2.14e-03 & \bfseries 9.87e-01±1.11e-03 \\
\hline m\_size & KSComplement & 9.72e-01±7.43e-04 & \bfseries 9.84e-01±3.22e-03 \\
\hline property\_type & TVComplement & 9.67e-01±1.33e-03 & \bfseries 9.82e-01±9.49e-04 \\
\hline publication\_date & KSComplement & 9.80e-01±1.61e-03 & \bfseries 9.85e-01±1.61e-03 \\
\hline rooms & TVComplement & 9.77e-01±2.28e-03 & \bfseries 9.81e-01±3.18e-03 \\
\hline state & TVComplement & 9.69e-01±4.29e-04 & \bfseries 9.83e-01±1.05e-03 \\
\hline transaction\_type & TVComplement & \bfseries 9.98e-01±1.07e-03 & 9.93e-01±3.54e-03 \\
\hline
\end{tabular}
\end{table}

%\begin{table}[H]
\centering
\caption{Distribución Economicos (A-1)}
\label{table-shape-economicos-a-1}
\begin{tabular}{|l|l|r|r|}
\hline
\rowcolor[gray]{0.8}
Columna & Metrica & smote-enc & tddpm\_mlp \\
\hline \_price & KSComplement & 0.989 & \bfseries 0.992 \\
\hline bathrooms & TVComplement & \bfseries 0.996 & 0.986 \\
\hline county & TVComplement & 0.920 & \bfseries 0.965 \\
\hline m\_built & KSComplement & 0.987 & \bfseries 0.987 \\
\hline m\_size & KSComplement & 0.975 & \bfseries 0.984 \\
\hline property\_type & TVComplement & 0.970 & \bfseries 0.978 \\
\hline publication\_date & KSComplement & 0.977 & \bfseries 0.983 \\
\hline rooms & TVComplement & \bfseries 0.979 & 0.979 \\
\hline state & TVComplement & 0.972 & \bfseries 0.984 \\
\hline transaction\_type & TVComplement & 0.999 & \bfseries 0.999 \\
\hline
\end{tabular}
\end{table}

\newpage
Aunque la diferencia es pequeña, se puede apreciar al comparar visualmente las Figuras \ref{pairwise-economicos-a-2-smote-enc} y \ref{pairwise-economicos-a-2-tddpm_mlp} que el segundo modelo, TDDPM, presenta una mayor similitud en las variables \emph{rooms} y \emph{bathrooms}.
\begin{figure}[H]
    \centering
    \includesvg[scale=.5,inkscapelatex=false]{datasets/economicos-a-1/pairwise/smote-enc.svg}
    \caption{Correlación de conjunto Real y Modelo: smote-enc}
    \label{pairwise-smote-enc}
\end{figure}
\begin{figure}[H]
    \centering
    \includesvg[scale=.7,inkscapelatex=false]{datasets/kingcounty-a-1/tddpm_mlp/privacy.svg}
    \caption{Frecuencia del campo Privacy en el modelo real y tddpm}
    \label{frecuency-tddpm-privacy}
\end{figure}
\begin{figure}[H]
    \centering
    \includesvg[scale=.7,inkscapelatex=false]{datasets/kingcounty-a-1/tddpm_mlp/bedrooms.svg}
    \caption{Frecuencia del campo Bedrooms en el modelo real y tddpm}
    \label{frecuency-tddpm-bedrooms}
\end{figure}
\begin{figure}[H]
    \centering
    \includesvg[scale=.7,inkscapelatex=false]{datasets/kingcounty-a-1/tddpm_mlp/grade.svg}
    \caption{Frecuencia del campo Grade en el modelo real y tddpm}
    \label{frecuency-tddpm-grade}
\end{figure}
\begin{figure}[H]
    \centering
    \includesvg[scale=.7,inkscapelatex=false]{datasets/kingcounty-a-1/tddpm_mlp/floors.svg}
    \caption{Frecuencia del campo Floors en el modelo real y tddpm, King county (A-1)}
    \label{frecuency-tddpm-floors}
\end{figure}
\begin{figure}[H]
    \centering
    \includesvg[scale=.7,inkscapelatex=false]{datasets/kingcounty-a-2/tddpm_mlp/bathrooms.svg}
    \caption{Frecuencia del campo Bathrooms en el modelo real y tddpm}
    \label{frecuency-tddpm-bathrooms}
\end{figure}
\begin{figure}[H]
    \centering
    \includesvg[scale=.7,inkscapelatex=false]{datasets/kingcounty-a-3/tddpm_mlp/sqft_basement.svg}
    \caption{Frecuencia del campo Sqft basement en el modelo real y tddpm, King county (A-3)}
    \label{frecuency-tddpm-sqft basement}
\end{figure}
\begin{figure}[H]
    \centering
    \includesvg[scale=.7,inkscapelatex=false]{datasets/kingcounty-a-3/tddpm_mlp/sqft_living.svg}
    \caption{Frecuencia del campo Sqft living en el modelo real y tddpm}
    \label{frecuency-tddpm-sqft living}
\end{figure}
\begin{figure}[H]
    \centering
    \includesvg[scale=.7,inkscapelatex=false]{datasets/kingcounty-a-2/tddpm_mlp/waterfront.svg}
    \caption{Frecuencia del campo Waterfront en el modelo real y tddpm}
    \label{frecuency-tddpm-waterfront}
\end{figure}
\begin{figure}[H]
    \centering
    \includesvg[scale=.7,inkscapelatex=false]{datasets/kingcounty-a-3/tddpm_mlp/sqft_lot.svg}
    \caption{Frecuencia del campo Sqft lot en el modelo real y tddpm, King county (A-3)}
    \label{frecuency-tddpm-sqft lot}
\end{figure}
\begin{figure}[H]
    \centering
    \includesvg[scale=.7,inkscapelatex=false]{datasets/kingcounty-a-2/tddpm_mlp/sqft_living15.svg}
    \caption{Frecuencia del campo Sqft living15 en el modelo real y tddpm}
    \label{frecuency-tddpm-sqft living15}
\end{figure}
\begin{figure}[H]
    \centering
    \includesvg[scale=.7,inkscapelatex=false]{datasets/kingcounty-a-1/tddpm_mlp/yr_built.svg}
    \caption{Frecuencia del campo Yr built en el modelo real y tddpm, King county (A-1)}
    \label{frecuency-tddpm-yr built}
\end{figure}
\begin{figure}[H]
    \centering
    \includesvg[scale=.7,inkscapelatex=false]{datasets/kingcounty-a-2/tddpm_mlp/condition.svg}
    \caption{Frecuencia del campo Condition en el modelo real y tddpm, King county (A-2)}
    \label{frecuency-tddpm-condition}
\end{figure}
\begin{figure}[H]
    \centering
    \includesvg[scale=.7,inkscapelatex=false]{datasets/kingcounty-a-3/tddpm_mlp/sqft_above.svg}
    \caption{Frecuencia del campo Sqft above en el modelo real y tddpm, King county (A-3)}
    \label{frecuency-tddpm-sqft above}
\end{figure}
\begin{figure}[H]
    \centering
    \includesvg[scale=.7,inkscapelatex=false]{datasets/kingcounty-a-2/tddpm_mlp/sqft_lot15.svg}
    \caption{Frecuencia del campo Sqft lot15 en el modelo real y tddpm}
    \label{frecuency-tddpm-sqft lot15}
\end{figure}
\begin{figure}[H]
    \centering
    \includesvg[scale=.7,inkscapelatex=false]{datasets/kingcounty-a-1/tddpm_mlp/view.svg}
    \caption{Frecuencia del campo View en el modelo real y tddpm}
    \label{frecuency-tddpm-view}
\end{figure}
\begin{figure}[H]
    \centering
    \includesvg[scale=.7,inkscapelatex=false]{datasets/kingcounty-a-2/tddpm_mlp/price.svg}
    \caption{Frecuencia del campo Price en el modelo real y tddpm, King county (A-2)}
    \label{frecuency-tddpm-price}
\end{figure}

\newpage
\subsection{Reportes - Conjunto B}
\label{ds-conjunto-b}
Iniciaría contrastando los resultados entre ambos conjuntos para el modelo TDDPM La Tabla \ref{table-score-economicos-b} muestra mejores \emph{Score}, \emph{Coverage}, \emph{Column Shape} y \emph{Column Pair Trends} comparadas con la Tabla \ref{table-score-economicos-a}. Puede deverse que al ser una cantidad de datos mayor, pudo tener más tiempo de aprender la distribución. Mejoria no notoria en los demás modelos, lo que podría indicar una mayor capacidad de TDDPM. Se puede ver que la cobertura es el indicador más bajo, solo alcanzando el 87\% en el mejor de los casos.

\begin{table}[H]
\centering
\fontsize{10}{14}\selectfont
\caption{Evaluaci\'on de M\'etricas de Rendimiento para Diversos Modelos de Aprendizaje Autom\'atico, Economicos}
\label{table-score-economicos-b}
\begin{tabular}{|l|r|r|r|r|r|}
\hline
\rowcolor[gray]{0.8}
Model Name & Column Pair Trends & Column Shapes & Coverage & Boundaries & \textbf{Score} \\
\hline tddpm\_mlp & \bfseries 0.98±2.79e-03 & \bfseries 0.99±1.71e-03 & \bfseries 0.87±3.37e-03 & \bfseries 1.00±0.00e+00 & \bfseries 0.98±1.85e-03 \\
\hline smote-enc & 0.97±1.01e-03 & 0.92±1.07e-04 & 0.70±3.34e-02 & \bfseries 1.00±0.00e+00 & 0.94±4.67e-04 \\
\hline copulagan & 0.77±2.32e-02 & 0.78±1.75e-02 & 0.63±5.61e-04 & \bfseries 1.00±0.00e+00 & 0.77±2.02e-02 \\
\hline tvae & 0.78±1.68e-02 & 0.70±1.76e-02 & 0.28±3.70e-03 & \bfseries 1.00±0.00e+00 & 0.74±1.48e-02 \\
\hline ctgan & 0.77±1.35e-02 & 0.70±8.58e-03 & 0.63±9.52e-04 & \bfseries 1.00±0.00e+00 & 0.73±5.42e-03 \\
\hline gaussiancopula & 0.63±0.00e+00 & 0.63±7.85e-17 & 0.56±0.00e+00 & \bfseries 1.00±0.00e+00 & 0.63±0.00e+00 \\
\hline
\end{tabular}
\end{table}

\newpage
Las distancias mínimas para los percentiles 5 y 1 son varias magnitudes menores en el Conjunto B que en el Conjunto A, pasando de $\times 10^{-9}$ en el Conjunto A a $\times 10^{-15}$ en el Conjunto B, como se puede ver al comparar la Tabla \ref{table-dcr-economicos-b} con la Tabla \ref{table-dcr-economicos-a}. Se puede afirmar que el 95\% de los registros tiene al menos una distancia de $9.12 \times 10^{-15}$.

\begin{table}[H]
\centering
\fontsize{10}{14}\selectfont
\caption{Distancia de registros más cercanos, percentil 5, datos economicos}
\label{table-dcr-economicos-b-5th}
\begin{tabular}{|l|l|r|r|r|r|r|r|r|}
\hline
\rowcolor[gray]{0.8}
Modelo & DCR ST & DCR SH & DCR TH & \textbf{Score} \\
\hline tddpm\_mlp & 9.12e-15±1.09e-15 & 9.99e-15±8.14e-16 & 9.00e-17±0.00e+00 & \bfseries 9.84e-01±1.85e-03 \\
\hline smote-enc & 9.19e-15±6.41e-16 & 1.17e-14±6.96e-16 & 9.00e-17±0.00e+00 & 9.43e-01±4.67e-04 \\
\hline copulagan & \cellcolor[rgb]{0.9, 0.54, 0.52} 2.65e-16±1.60e-16 & \cellcolor[rgb]{0.9, 0.54, 0.52} 2.84e-16±1.73e-16 & 9.00e-17±0.00e+00 & 7.74e-01±2.02e-02 \\
\hline tvae & 1.00e-09±1.74e-09 & 1.00e-09±1.74e-09 & 9.00e-17±0.00e+00 & 7.38e-01±1.48e-02 \\
\hline ctgan & \bfseries 7.29e-09±8.52e-09 & \bfseries 7.35e-09±8.45e-09 & 9.00e-17±0.00e+00 & 7.34e-01±5.42e-03 \\
\hline gaussiancopula & 9.23e-13±0.00e+00 & 1.02e-12±0.00e+00 & 9.00e-17±0.00e+00 & \cellcolor[rgb]{0.9, 0.54, 0.52} 6.31e-01±0.00e+00 \\
\hline
\end{tabular}
\end{table}

\begin{table}[H]
\centering
\fontsize{10}{14}\selectfont
\caption{Distancia de registros más cercanos entre conjuntos Sinteticos, percentil 1, Economicos}
\label{table-dcr-economicos-b-1th}
\begin{tabular}{|l|l|r|r|r|r|}
\hline
\rowcolor[gray]{0.8}
Modelo & DCR ST & DCR SH & DCR TH & \textbf{Score} \\
\hline tddpm\_mlp & 1.44e-10±6.01e-12 & \cellcolor[rgb]{0.9, 0.54, 0.52} 1.40e-09±1.05e-10 & \bfseries \cellcolor[rgb]{0.9, 0.54, 0.52} 0.00e+00±0.00e+00 & \bfseries 9.77e-01±6.88e-04 \\
\hline smote-enc & \cellcolor[rgb]{0.9, 0.54, 0.52} 0.00e+00±0.00e+00 & 1.41e-09±4.21e-10 & \bfseries \cellcolor[rgb]{0.9, 0.54, 0.52} 0.00e+00±0.00e+00 & 9.67e-01±8.19e-04 \\
\hline ctgan & \bfseries 2.20e-06±1.50e-06 & \bfseries 3.24e-06±1.58e-06 & \bfseries \cellcolor[rgb]{0.9, 0.54, 0.52} 0.00e+00±0.00e+00 & 6.96e-01±1.00e-02 \\
\hline copulagan & 2.04e-07±2.85e-08 & 4.37e-07±6.38e-08 & \bfseries \cellcolor[rgb]{0.9, 0.54, 0.52} 0.00e+00±0.00e+00 & 7.81e-01±2.03e-02 \\
\hline gaussiancopula & 8.04e-07±8.64e-23 & 1.93e-06±0.00e+00 & \bfseries \cellcolor[rgb]{0.9, 0.54, 0.52} 0.00e+00±0.00e+00 & 6.91e-01±6.41e-17 \\
\hline tvae & 7.41e-08±1.95e-09 & 1.16e-07±3.43e-09 & \bfseries \cellcolor[rgb]{0.9, 0.54, 0.52} 0.00e+00±0.00e+00 & \cellcolor[rgb]{0.9, 0.54, 0.52} 6.40e-01±3.35e-03 \\
\hline
\end{tabular}
\end{table}

\begin{table}[H]
\centering
\fontsize{10}{14}\selectfont
\caption{Distancia de registros más cercanos, minimo, datos economicos}
\label{table-dcr-economicos-b}
\begin{tabular}{|l|l|r|r|r|r|r|r|r|}
\hline
\rowcolor[gray]{0.8}
Modelo & DCR ST & DCR SH & DCR TH & \textbf{Score} \\
\hline tddpm\_mlp & \cellcolor[rgb]{0.9, 0.54, 0.52} 0.00e+00±0.00e+00 & \cellcolor[rgb]{0.9, 0.54, 0.52} 0.00e+00±0.00e+00 & 0.00e+00±0.00e+00 & \bfseries 9.84e-01±1.85e-03 \\
\hline smote-enc & \cellcolor[rgb]{0.9, 0.54, 0.52} 0.00e+00±0.00e+00 & \cellcolor[rgb]{0.9, 0.54, 0.52} 0.00e+00±0.00e+00 & 0.00e+00±0.00e+00 & 9.43e-01±4.67e-04 \\
\hline copulagan & 4.57e-19±3.77e-21 & \bfseries 5.21e-19±1.82e-22 & 0.00e+00±0.00e+00 & 7.74e-01±2.02e-02 \\
\hline tvae & 8.99e-20±0.00e+00 & 8.99e-20±0.00e+00 & 0.00e+00±0.00e+00 & 7.38e-01±1.48e-02 \\
\hline ctgan & 8.99e-20±0.00e+00 & 8.99e-20±0.00e+00 & 0.00e+00±0.00e+00 & 7.34e-01±5.42e-03 \\
\hline gaussiancopula & \bfseries 5.23e-19±0.00e+00 & 5.09e-19±0.00e+00 & 0.00e+00±0.00e+00 & \cellcolor[rgb]{0.9, 0.54, 0.52} 6.31e-01±0.00e+00 \\
\hline
\end{tabular}
\end{table}

\newpage
Se puede observar que la distancia del segundo registro es al menos tres veces la distancia del primer registro.

\begin{table}[H]
\centering
\fontsize{10}{14}\selectfont
\caption{Proporción entre el más cercano y el segundo más cercano, percentil 5, datos economicos}
\label{table-nndr-economicos-b-5th}
\begin{tabular}{|l|l|r|r|r|r|r|r|r|}
\hline
\rowcolor[gray]{0.8}
Modelo & NNDR ST & NNDR SH & NNDR TH & \textbf{Score} \\
\hline tddpm\_mlp & \bfseries 3.03e-01±4.42e-03 & \bfseries 2.96e-01±1.27e-02 & 1.15e-07±0.00e+00 & \bfseries 9.84e-01±1.85e-03 \\
\hline smote-enc & 2.47e-01±3.63e-03 & 2.60e-01±6.24e-03 & 1.15e-07±0.00e+00 & 9.43e-01±4.67e-04 \\
\hline copulagan & \cellcolor[rgb]{0.9, 0.54, 0.52} 1.07e-05±4.91e-06 & \cellcolor[rgb]{0.9, 0.54, 0.52} 2.27e-05±1.82e-05 & 1.15e-07±0.00e+00 & 7.74e-01±2.02e-02 \\
\hline tvae & 4.28e-04±2.75e-04 & 4.49e-04±2.88e-04 & 1.15e-07±0.00e+00 & 7.38e-01±1.48e-02 \\
\hline ctgan & 2.10e-03±7.18e-04 & 7.23e-03±1.01e-02 & 1.15e-07±0.00e+00 & 7.34e-01±5.42e-03 \\
\hline gaussiancopula & 1.52e-02±0.00e+00 & 1.38e-02±0.00e+00 & 1.15e-07±0.00e+00 & \cellcolor[rgb]{0.9, 0.54, 0.52} 6.31e-01±0.00e+00 \\
\hline
\end{tabular}
\end{table}

\begin{table}[H]
\centering
\fontsize{10}{14}\selectfont
\caption{Proporción entre el más cercano y el segundo más cercano, percentil 1, datos economicos}
\label{table-nndr-economicos-b-1th}
\begin{tabular}{|l|l|r|r|r|r|r|r|r|}
\hline
\rowcolor[gray]{0.8}
Modelo & NNDR ST & NNDR SH & NNDR TH & \textbf{Score} \\
\hline tddpm\_mlp & \bfseries 3.14e-02±4.92e-03 & \bfseries 3.08e-02±3.94e-03 & 0.00e+00±0.00e+00 & \bfseries 9.84e-01±1.85e-03 \\
\hline smote-enc & 2.52e-03±1.07e-03 & 3.47e-03±2.68e-04 & 0.00e+00±0.00e+00 & 9.43e-01±4.67e-04 \\
\hline copulagan & \cellcolor[rgb]{0.9, 0.54, 0.52} 5.33e-09±1.38e-09 & \cellcolor[rgb]{0.9, 0.54, 0.52} 1.15e-07±1.65e-07 & 0.00e+00±0.00e+00 & 7.74e-01±2.02e-02 \\
\hline tvae & 3.02e-05±4.14e-05 & 3.04e-05±4.15e-05 & 0.00e+00±0.00e+00 & 7.38e-01±1.48e-02 \\
\hline ctgan & 1.21e-04±1.18e-04 & 1.35e-04±1.66e-04 & 0.00e+00±0.00e+00 & 7.34e-01±5.42e-03 \\
\hline gaussiancopula & 6.43e-06±0.00e+00 & 6.43e-06±0.00e+00 & 0.00e+00±0.00e+00 & \cellcolor[rgb]{0.9, 0.54, 0.52} 6.31e-01±0.00e+00 \\
\hline
\end{tabular}
\end{table}

\begin{table}[H]
\centering
\fontsize{10}{14}\selectfont
\caption{Proporción entre el más cercano y el segundo más cercano, mínimo, Economicos}
\label{table-nndr-economicos-b-min}
\begin{tabular}{|l|l|r|r|r|r|}
\hline
\rowcolor[gray]{0.8}
Modelo & NNDR ST & NNDR SH & NNDR TH & \textbf{Score} \\
\hline tddpm\_mlp & \bfseries 0.00e+00±0.00e+00 & \bfseries 0.00e+00±0.00e+00 & \cellcolor[rgb]{0.9, 0.54, 0.52} \bfseries 0.00e+00±0.00e+00 & \cellcolor[rgb]{0.9, 0.54, 0.52} 9.77e-01±6.88e-04 \\
\hline smote-enc & 0.00e+00±0.00e+00 & 0.00e+00±0.00e+00 & 0.00e+00±0.00e+00 & 9.67e-01±8.19e-04 \\
\hline ctgan & 4.47e-04±2.18e-04 & 1.88e-04±2.25e-04 & 0.00e+00±0.00e+00 & 6.96e-01±1.00e-02 \\
\hline copulagan & 1.79e-04±4.03e-05 & 2.19e-04±3.23e-05 & 0.00e+00±0.00e+00 & 7.81e-01±2.03e-02 \\
\hline gaussiancopula & 1.99e-04±0.00e+00 & 1.04e-04±1.36e-20 & 0.00e+00±0.00e+00 & 6.91e-01±6.41e-17 \\
\hline tvae & \cellcolor[rgb]{0.9, 0.54, 0.52} 8.02e-04±2.62e-04 & \cellcolor[rgb]{0.9, 0.54, 0.52} 8.94e-03±1.12e-03 & 0.00e+00±0.00e+00 & \bfseries 6.40e-01±3.35e-03 \\
\hline
\end{tabular}
\end{table}

\newpage
La cobertura es notablemente baja en las variables \emph{rooms} y \emph{m\_size} en SMOTE, y en \emph{bathrooms} y \emph{rooms} en el caso de TDDPM. En general, el modelo TDDPM es ligeramente superior a SMOTE.

\begin{table}[H]
\centering
\caption{Cobertura Categoría/Rango para Modelos Smote y Tddpm, Economicos}
\label{table-coverage-economicos-b}
\begin{tabular}{|l|l|r|r|}
\hline
\rowcolor[gray]{0.8}
Columna & Metrica & smote-enc & tddpm\_mlp \\
\hline \_price & RangeCoverage & 8.10e-01±1.34e-01 & \bfseries 9.11e-01±1.37e-02 \\
\hline bathrooms & CategoryCoverage & \bfseries 8.63e-01±5.00e-02 & 6.67e-01±1.39e-02 \\
\hline county & CategoryCoverage & 5.90e-01±3.05e-03 & \bfseries 7.99e-01±2.20e-02 \\
\hline m\_built & RangeCoverage & 3.18e-01±1.01e-01 & \bfseries 7.54e-01±1.77e-01 \\
\hline m\_size & RangeCoverage & \cellcolor[rgb]{0.9, 0.54, 0.52} 3.45e-02±1.98e-03 & \bfseries \cellcolor[rgb]{0.9, 0.54, 0.52} 4.00e-01±1.51e-01 \\
\hline property\_type & CategoryCoverage & 6.30e-01±5.24e-02 & \bfseries 9.07e-01±5.24e-02 \\
\hline publication\_date & RangeCoverage & 9.77e-01±6.18e-03 & \bfseries 9.88e-01±4.44e-03 \\
\hline rooms & CategoryCoverage & 7.56e-01±3.98e-02 & \bfseries 7.97e-01±3.04e-02 \\
\hline state & CategoryCoverage & 7.92e-01±2.95e-02 & \bfseries 9.79e-01±2.95e-02 \\
\hline transaction\_type & CategoryCoverage & 5.00e-01±0.00e+00 & \bfseries 9.17e-01±1.18e-01 \\
\hline
\end{tabular}
\end{table}

Ambos modelos presentan buenas métricas, superando el 91\% en términos de distribución y forma. Sin embargo, se observan excepciones en los casos de \emph{m\_built} (85\%) y \emph{m\_size} (55\%).
\begin{table}[H]
\centering
\fontsize{10}{14}\selectfont
\caption{Evaluaci\'on de Similitud de Distribuci\'on para Modelos SMOTE-ENC y TDDPM\_MLP, Economicos}
\label{table-shape-economicos-b}
\begin{tabular}{|l|l|r|r|}
\hline
\rowcolor[gray]{0.8}
Columna & Metrica & smote-enc & tddpm\_mlp \\
\hline \_price & KSComplement & 9.85e-01±1.94e-04 & \bfseries 9.93e-01±8.05e-04 \\
\hline bathrooms & TVComplement & \bfseries 9.98e-01±3.13e-04 & 9.95e-01±4.98e-04 \\
\hline county & TVComplement & 9.10e-01±5.37e-04 & \bfseries 9.84e-01±2.56e-03 \\
\hline m\_built & KSComplement & 8.56e-01±1.32e-03 & \bfseries 9.91e-01±1.44e-03 \\
\hline m\_size & KSComplement & 5.51e-01±8.46e-07 & \bfseries 9.90e-01±2.65e-03 \\
\hline property\_type & TVComplement & 9.79e-01±7.12e-04 & \bfseries 9.89e-01±3.27e-03 \\
\hline publication\_date & KSComplement & 9.66e-01±9.67e-05 & \bfseries 9.91e-01±5.41e-03 \\
\hline rooms & TVComplement & 9.87e-01±9.57e-04 & \bfseries 9.95e-01±7.29e-04 \\
\hline state & TVComplement & 9.78e-01±4.57e-04 & \bfseries 9.90e-01±1.06e-03 \\
\hline transaction\_type & TVComplement & 9.94e-01±1.97e-04 & \bfseries 9.97e-01±1.53e-03 \\
\hline
\end{tabular}
\end{table}


\newpage
Los modelos SMOTE y TDDPM, al ser comparados con el conjunto original, presentan diferencias marcadas. Los conjuntos sintéticos han creado correlaciones que no se ven presentes en los datos originales. En el caso del modelo SMOTE, se presentan correlaciones en las variables \emph{bathrooms}-\emph{rooms}, \emph{m\_size}-\emph{m\_built}; mientras que TDDPM adicionalmente genera una correlación entre \emph{\_price}-\emph{m\_size} y \emph{\_price}-\emph{m\_built}.


\begin{figure}[H]
    \centering
    \includesvg[scale=.5,inkscapelatex=false]{datasets/economicos-a-1/pairwise/smote-enc.svg}
    \caption{Correlación de conjunto Real y Modelo: smote-enc}
    \label{pairwise-smote-enc}
\end{figure}
\begin{figure}[H]
    \centering
    \includesvg[scale=.7,inkscapelatex=false]{datasets/kingcounty-a-1/tddpm_mlp/privacy.svg}
    \caption{Frecuencia del campo Privacy en el modelo real y tddpm}
    \label{frecuency-tddpm-privacy}
\end{figure}
\begin{figure}[H]
    \centering
    \includesvg[scale=.7,inkscapelatex=false]{datasets/kingcounty-a-1/tddpm_mlp/bedrooms.svg}
    \caption{Frecuencia del campo Bedrooms en el modelo real y tddpm}
    \label{frecuency-tddpm-bedrooms}
\end{figure}
\begin{figure}[H]
    \centering
    \includesvg[scale=.7,inkscapelatex=false]{datasets/kingcounty-a-1/tddpm_mlp/grade.svg}
    \caption{Frecuencia del campo Grade en el modelo real y tddpm}
    \label{frecuency-tddpm-grade}
\end{figure}
\begin{figure}[H]
    \centering
    \includesvg[scale=.7,inkscapelatex=false]{datasets/kingcounty-a-1/tddpm_mlp/floors.svg}
    \caption{Frecuencia del campo Floors en el modelo real y tddpm, King county (A-1)}
    \label{frecuency-tddpm-floors}
\end{figure}
\begin{figure}[H]
    \centering
    \includesvg[scale=.7,inkscapelatex=false]{datasets/kingcounty-a-2/tddpm_mlp/bathrooms.svg}
    \caption{Frecuencia del campo Bathrooms en el modelo real y tddpm}
    \label{frecuency-tddpm-bathrooms}
\end{figure}
\begin{figure}[H]
    \centering
    \includesvg[scale=.7,inkscapelatex=false]{datasets/kingcounty-a-3/tddpm_mlp/sqft_basement.svg}
    \caption{Frecuencia del campo Sqft basement en el modelo real y tddpm, King county (A-3)}
    \label{frecuency-tddpm-sqft basement}
\end{figure}
\begin{figure}[H]
    \centering
    \includesvg[scale=.7,inkscapelatex=false]{datasets/kingcounty-a-3/tddpm_mlp/sqft_living.svg}
    \caption{Frecuencia del campo Sqft living en el modelo real y tddpm}
    \label{frecuency-tddpm-sqft living}
\end{figure}
\begin{figure}[H]
    \centering
    \includesvg[scale=.7,inkscapelatex=false]{datasets/kingcounty-a-2/tddpm_mlp/waterfront.svg}
    \caption{Frecuencia del campo Waterfront en el modelo real y tddpm}
    \label{frecuency-tddpm-waterfront}
\end{figure}
\begin{figure}[H]
    \centering
    \includesvg[scale=.7,inkscapelatex=false]{datasets/kingcounty-a-3/tddpm_mlp/sqft_lot.svg}
    \caption{Frecuencia del campo Sqft lot en el modelo real y tddpm, King county (A-3)}
    \label{frecuency-tddpm-sqft lot}
\end{figure}
\begin{figure}[H]
    \centering
    \includesvg[scale=.7,inkscapelatex=false]{datasets/kingcounty-a-2/tddpm_mlp/sqft_living15.svg}
    \caption{Frecuencia del campo Sqft living15 en el modelo real y tddpm}
    \label{frecuency-tddpm-sqft living15}
\end{figure}
\begin{figure}[H]
    \centering
    \includesvg[scale=.7,inkscapelatex=false]{datasets/kingcounty-a-1/tddpm_mlp/yr_built.svg}
    \caption{Frecuencia del campo Yr built en el modelo real y tddpm, King county (A-1)}
    \label{frecuency-tddpm-yr built}
\end{figure}
\begin{figure}[H]
    \centering
    \includesvg[scale=.7,inkscapelatex=false]{datasets/kingcounty-a-2/tddpm_mlp/condition.svg}
    \caption{Frecuencia del campo Condition en el modelo real y tddpm, King county (A-2)}
    \label{frecuency-tddpm-condition}
\end{figure}
\begin{figure}[H]
    \centering
    \includesvg[scale=.7,inkscapelatex=false]{datasets/kingcounty-a-3/tddpm_mlp/sqft_above.svg}
    \caption{Frecuencia del campo Sqft above en el modelo real y tddpm, King county (A-3)}
    \label{frecuency-tddpm-sqft above}
\end{figure}
\begin{figure}[H]
    \centering
    \includesvg[scale=.7,inkscapelatex=false]{datasets/kingcounty-a-2/tddpm_mlp/sqft_lot15.svg}
    \caption{Frecuencia del campo Sqft lot15 en el modelo real y tddpm}
    \label{frecuency-tddpm-sqft lot15}
\end{figure}
\begin{figure}[H]
    \centering
    \includesvg[scale=.7,inkscapelatex=false]{datasets/kingcounty-a-1/tddpm_mlp/view.svg}
    \caption{Frecuencia del campo View en el modelo real y tddpm}
    \label{frecuency-tddpm-view}
\end{figure}
\begin{figure}[H]
    \centering
    \includesvg[scale=.7,inkscapelatex=false]{datasets/kingcounty-a-2/tddpm_mlp/price.svg}
    \caption{Frecuencia del campo Price en el modelo real y tddpm, King county (A-2)}
    \label{frecuency-tddpm-price}
\end{figure}

