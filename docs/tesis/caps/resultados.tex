\chapter{Resultados}
\label{chap:resultados}
Este capítulo aborda los resultados obtenidos en el actual trabajo, donde se emplearon diversas técnicas de preprocesamiento y modelos de aprendizaje automático. Aquí se presentan los resultados en función del desempeño de los modelos, la similitud con los datos originales y la tensión entre privacidad y utilidad de los datos generados.

Se enfocará en la evaluación de los conjuntos de datos de King County y Económicos, resaltando los logros de los modelos Tddpm y Smote en términos de similitud con los datos originales y cobertura. Se explorará además el análisis de privacidad, destacando el rendimiento superior del modelo Tddpm en términos de privacidad.

Finalmente, se hará un resumen de los hallazgos más relevantes, destacando la eficacia de los modelos Tddpm y Smote en la generación de datos sintéticos útiles, y se abordarán las diferencias significativas observadas en la cobertura, distribución y privacidad entre los conjuntos de datos.

\newpage
\section{King County}
\subsection{SDMetrics Score}
La Tabla \ref{table-score-king county-a} muestra los puntajes obtenidos por los distintos patrones utilizados en este estudio. Es notorio que los patrones con puntajes más altos, como Tddpm y Smote, presentan una mayor similitud con el conjunto de datos original. En contraposición, los patrones con puntajes más bajos, como ctgan, exhiben una correspondencia considerablemente menor con el conjunto original. Se muestra el promedio ± desviación estándar basado en las 3 ejecuciones realizadas.

\begin{table}[H]
\centering
\fontsize{9}{14}\selectfont
\caption{Evaluaci\'on de M\'etricas de Rendimiento para Diversos Modelos de Aprendizaje Autom\'atico, King County}
\label{table-score-king county-a}
\begin{tabular}{|l|r|r|r|r|r|}
\hline
\rowcolor[gray]{0.8}
Model Name & Column Pair Trends & Column Shapes & Coverage & Boundaries & \textbf{Score} \\
\hline tddpm\_mlp & 9.37e-01±3.80e-03 & \bfseries 9.67e-01±1.48e-03 & \bfseries 9.66e-01±4.96e-03 & 1.00e+00±0.00e+00 & 9.52e-01±2.36e-03 \\
\hline smote-enc & \bfseries 9.41e-01±2.60e-04 & 9.65e-01±3.06e-04 & 8.42e-01±8.31e-03 & \bfseries 1.00e+00±1.02e-05 & \bfseries 9.53e-01±2.45e-04 \\
\hline ctgan & 8.10e-01±1.40e-02 & 8.38e-01±2.67e-02 & 8.56e-01±2.25e-03 & 1.00e+00±0.00e+00 & 8.24e-01±2.02e-02 \\
\hline tablepreset & 8.37e-01±0.00e+00 & 8.37e-01±1.36e-16 & 7.53e-01±0.00e+00 & 1.00e+00±0.00e+00 & 8.37e-01±7.85e-17 \\
\hline copulagan & 7.64e-01±4.93e-03 & 8.14e-01±4.70e-03 & 8.40e-01±1.74e-02 & 1.00e+00±0.00e+00 & 7.89e-01±2.92e-03 \\
\hline gaussiancopula & 7.65e-01±0.00e+00 & 8.11e-01±0.00e+00 & 7.51e-01±7.85e-17 & 1.00e+00±0.00e+00 & 7.88e-01±0.00e+00 \\
\hline tvae & 7.07e-01±1.19e-02 & 7.68e-01±1.22e-02 & 4.53e-01±1.63e-02 & 1.00e+00±0.00e+00 & 7.38e-01±1.18e-02 \\
\hline
\end{tabular}
\end{table}
    
A pesar de que los patrones Tddpm y Smote alcanzan calificaciones prometedoras en general, se observa una diferencia significativa entre ambos en términos de cobertura (\emph{Coverage}). Específicamente, Smote no logra capturar la diversidad del conjunto de datos, reflejándose en una calificación de cobertura marcadamente inferior a la de Tddpm.

\newpage

\subsection{Correlación}
En el Anexo \ref{A-pairwise-kingcounty-top2-a-1}, se contrasta la lista completa de cada modelo. Se observa que, en general, los modelos con puntajes más altos exhiben una mayor similitud visual con los datos reales. A modo de ilustración, las imágenes \ref{pairwise-king county-a-2-copulagan} y \ref{pairwise-king county-a-2-gaussiancopula} contrastan los datos reales con los generados por los modelos gaussiancopula y copulagan. A pesar de que estos modelos presentan puntajes similares, el modelo gaussiancopula muestra una mayor similitud visual con los datos reales en comparación con el modelo copulagan.
\begin{figure}[H]
    \centering
    \includesvg[scale=.7,inkscapelatex=false]{datasets/kingcounty-a-2/pairwise/copulagan.svg}
    \caption{Correlación de conjunto Real y Modelo: copulagan}
    \label{pairwise-king county-a-2-copulagan}
\end{figure}

\begin{figure}[H]
    \centering
    \includesvg[scale=.6,inkscapelatex=false]{datasets/kingcounty-a-2/pairwise/gaussiancopula.svg}
    \caption{Correlación de conjunto original de entrenamiento y Gaussiancopula}
    \label{pairwise-king county-a-2-gaussiancopula}
\end{figure}
\newpage
Es especialmente relevante que, entre los modelos con puntajes superiores al 90\%, la evaluación visual para determinar cuál es superior puede ser un desafío. Esta dificultad surge debido a que, a medida que el puntaje se incrementa, la similitud visual entre los datos reales y los generados se intensifica. Este fenómeno se ilustra en las figuras \ref{pairwise-king county-a-2-smote-enc} y \ref{pairwise-king county-a-2-tddpm_mlp}, donde se contrastan los datos reales con los generados por los modelos Smote y Tddpm, respectivamente. Ambos modelos ostentan puntajes por encima del 90\%, y la correspondencia visual entre los datos reales y los generados es notablemente alta en ambos casos.

\begin{figure}[H]
    \centering
    \includesvg[scale=.7,inkscapelatex=false]{datasets/kingcounty-a-2/pairwise/smote-enc.svg}
    \caption{Correlación de conjunto Real y Modelo: smote-enc}
    \label{pairwise-king county-a-2-smote-enc}
\end{figure}

\begin{figure}[H]
    \centering
    \includesvg[scale=.7,inkscapelatex=false]{datasets/kingcounty-a-2/pairwise/tddpm_mlp.svg}
    \caption{Correlación de conjunto Real y Modelo: tddpm\_mlp}
    \label{pairwise-king county-a-2-tddpm_mlp}
\end{figure}
\newpage
En la evaluación mediante SDMetrics y en la comparación visual a través de la correlación de en parejas, los modelos más sobresalientes resultan ser Tddpm y Smote. Dichos modelos han logrado los puntajes más elevados en ambas métricas y han demostrado una notable similitud visual con los datos reales. Por ende, se puede inferir que estos modelos resultan ser los más eficaces para la generación de datos sintéticos beneficiosos para este conjunto de datos en particular.

\subsection{Reporte diagnóstico}
La Tabla \ref{table-coverage-king county-a} evidencia la superioridad del modelo Tddpm en términos de cobertura de valores distintos, aunque hay casos donde ningún modelo alcanza una cobertura completa. Un caso notable es la variable \emph{bedrooms}, en la que Tddpm solo logra un 71.8\% de cobertura, pero aún así supera al modelo Smote, que apenas alcanza el 51\% para la misma variable.

\begin{table}[H]
\centering
\fontsize{10}{14}\selectfont
\caption{Cobertura Categor{\'\i}a/Rango para Modelos Smote y Tddpm, King County}
\label{table-coverage-king county-a}
\begin{tabular}{|l|l|r|r|}
\hline
\rowcolor[gray]{0.8}
Columna & Metrica & smote-enc & tddpm\_mlp \\
\hline bathrooms & CategoryCoverage & 6.56e-01±3.85e-02 & \bfseries 8.11e-01±3.85e-02 \\
\hline bedrooms & CategoryCoverage & \cellcolor[rgb]{0.9, 0.54, 0.52} 5.13e-01±4.44e-02 & \cellcolor[rgb]{0.9, 0.54, 0.52} \bfseries 7.18e-01±4.44e-02 \\
\hline condition & CategoryCoverage & 9.33e-01±1.15e-01 & \bfseries 1.00e+00±0.00e+00 \\
\hline date & CategoryCoverage & \bfseries 9.64e-01±6.77e-03 & 9.44e-01±9.69e-03 \\
\hline floors & CategoryCoverage & 8.33e-01±0.00e+00 & \bfseries 9.44e-01±9.62e-02 \\
\hline grade & CategoryCoverage & 7.50e-01±0.00e+00 & \bfseries 8.61e-01±4.81e-02 \\
\hline id & RangeCoverage & 9.93e-01±4.54e-04 & \bfseries 1.00e+00±7.79e-04 \\
\hline lat & RangeCoverage & 9.65e-01±8.31e-03 & \bfseries 1.00e+00±0.00e+00 \\
\hline long & RangeCoverage & 9.91e-01±5.54e-03 & \bfseries 1.00e+00±2.45e-04 \\
\hline price & RangeCoverage & 5.72e-01±1.02e-01 & \bfseries 1.00e+00±1.25e-05 \\
\hline sqft\_above & RangeCoverage & 7.88e-01±2.98e-02 & \bfseries 1.00e+00±1.45e-05 \\
\hline sqft\_basement & RangeCoverage & 7.47e-01±2.02e-01 & \bfseries 1.00e+00±0.00e+00 \\
\hline sqft\_living & RangeCoverage & 7.03e-01±4.89e-02 & \bfseries 1.00e+00±1.33e-05 \\
\hline sqft\_living15 & RangeCoverage & 8.49e-01±5.19e-02 & \bfseries 1.00e+00±5.14e-05 \\
\hline sqft\_lot & RangeCoverage & 5.86e-01±7.02e-03 & \bfseries 1.00e+00±3.49e-06 \\
\hline sqft\_lot15 & RangeCoverage & 8.30e-01±2.80e-01 & \bfseries 1.00e+00±4.72e-05 \\
\hline view & CategoryCoverage & \bfseries 1.00e+00±0.00e+00 & \bfseries 1.00e+00±0.00e+00 \\
\hline waterfront & CategoryCoverage & \bfseries 1.00e+00±0.00e+00 & \bfseries 1.00e+00±0.00e+00 \\
\hline yr\_built & RangeCoverage & \bfseries 1.00e+00±4.11e-05 & 1.00e+00±0.00e+00 \\
\hline yr\_renovated & RangeCoverage & \bfseries 1.00e+00±9.76e-05 & 1.00e+00±0.00e+00 \\
\hline zipcode & CategoryCoverage & \bfseries 1.00e+00±0.00e+00 & \bfseries 1.00e+00±0.00e+00 \\
\hline
\end{tabular}
\end{table}

\newpage
\subsection{Reporte de calidad}
En términos generales, la distribución en ambos modelos se aproxima a la real, en casi todos los casos superando el 90\%. La única excepción es el modelo Smote en la variable \emph{bathrooms}.

\begin{table}[H]
\centering
\fontsize{10}{14}\selectfont
\caption{Evaluaci\'on de Similitud de Distribuci\'on para Modelos SMOTE-ENC y TDDPM\_MLP, King County}
\label{table-shape-king county-a}
\begin{tabular}{|l|l|r|r|}
\hline
\rowcolor[gray]{0.8}
Columna & Metrica & smote-enc & tddpm\_mlp \\
\hline bathrooms & TVComplement & 8.84e-01±5.09e-03 & \bfseries 9.46e-01±6.18e-03 \\
\hline bedrooms & TVComplement & 9.18e-01±7.87e-04 & \bfseries 9.50e-01±5.73e-03 \\
\hline condition & TVComplement & 9.33e-01±1.23e-03 & \bfseries 9.61e-01±5.43e-03 \\
\hline date & TVComplement & \bfseries 9.38e-01±1.73e-03 & 9.26e-01±2.29e-03 \\
\hline floors & TVComplement & 9.66e-01±1.12e-03 & \bfseries 9.68e-01±4.38e-03 \\
\hline grade & TVComplement & 9.58e-01±6.82e-04 & \bfseries 9.64e-01±1.19e-03 \\
\hline id & KSComplement & \bfseries 9.86e-01±6.51e-04 & 9.75e-01±2.95e-03 \\
\hline lat & KSComplement & \bfseries 9.89e-01±1.69e-03 & 9.83e-01±8.10e-04 \\
\hline long & KSComplement & \bfseries 9.88e-01±2.22e-03 & 9.78e-01±1.98e-03 \\
\hline price & KSComplement & \bfseries 9.81e-01±6.63e-04 & 9.72e-01±7.86e-03 \\
\hline sqft\_above & KSComplement & 9.72e-01±1.42e-03 & \bfseries 9.77e-01±8.75e-03 \\
\hline sqft\_basement & KSComplement & 9.35e-01±3.60e-03 & \bfseries 9.75e-01±3.87e-03 \\
\hline sqft\_living & KSComplement & \bfseries 9.81e-01±2.50e-03 & 9.73e-01±5.59e-03 \\
\hline sqft\_living15 & KSComplement & \bfseries 9.81e-01±1.63e-03 & 9.76e-01±4.34e-03 \\
\hline sqft\_lot & KSComplement & \bfseries 9.83e-01±4.81e-03 & 9.58e-01±8.34e-03 \\
\hline sqft\_lot15 & KSComplement & \bfseries 9.84e-01±3.16e-03 & 9.62e-01±8.15e-03 \\
\hline view & TVComplement & 9.36e-01±9.73e-04 & \bfseries 9.52e-01±4.70e-03 \\
\hline waterfront & TVComplement & 9.94e-01±1.22e-04 & \bfseries 9.95e-01±6.04e-04 \\
\hline yr\_built & KSComplement & \bfseries 9.83e-01±4.71e-04 & 9.76e-01±6.80e-03 \\
\hline yr\_renovated & KSComplement & \bfseries 9.92e-01±4.17e-04 & 9.91e-01±1.00e-03 \\
\hline zipcode & TVComplement & \bfseries 9.74e-01±1.57e-03 & 9.50e-01±4.11e-04 \\
\hline
\end{tabular}
\end{table}

\newpage
Al examinar las variables de los conjuntos de datos completos, como se ilustra en la lista Anexa \ref{A-pairwise-kingcounty-top2-a-1}, se observa una similitud entre los tres conjuntos analizados: Real, Smote y Tddpm. Sin embargo, también surgen diferencias significativas. Es relevante mencionar que los conjuntos de datos generados son aproximadamente un 20\% más grandes que el conjunto real. En varias columnas, la distribución de datos en los tres conjuntos es similar, como se evidencia en los casos de bathrooms, sqft\_lot, sqft\_above, price, sqft\_living, sqft\_basement, yr\_built, sqft\_living15 y grade. Este patrón se puede apreciar en la Figura \ref{frecuency-top2-grade}.


\begin{figure}[H]
    \centering
    \includesvg[scale=.7,inkscapelatex=false]{datasets/kingcounty-a-1/top2/grade.svg}
    \caption{Frecuencia del campo Grade en el modelo real y Top 2, King county (A-1)}
    \label{frecuency-top2-grade}
\end{figure}
\newpage
Por otra parte, la distribución de los atributos bedrooms, condition, view y floors en el conjunto de datos generado por el modelo Tddpm presenta una particularidad: contiene un mayor número de elementos menos frecuentes comparado con los demás conjuntos. Al considerar la columna \emph{bedrooms} como ejemplo (refiérase a Figura \ref{frecuency-top2-bedrooms}), la distribución de valores en el conjunto Tddpm se desvía de la del conjunto Smote. En específico, se registra un aumento en la cantidad de registros correspondientes a los valores 6 y 1.

\begin{figure}[H]
    \centering
    \includesvg[scale=.5,inkscapelatex=false]{datasets/kingcounty-a-2/top2/bedrooms.svg}
    \caption{Frecuencia del campo bedrooms en el modelo real y top2}
    \label{frecuency-top2-bedrooms}
\end{figure}

\newpage
En el caso de la variable \emph{sqft\_lot15}, la distribución generada por el modelo Smote resulta ser más similar a la del conjunto de datos real, como se puede apreciar en la figura \ref{frecuency-top2-sqft lot15}.
\begin{figure}[H]
    \centering
    \includesvg[scale=.7,inkscapelatex=false]{datasets/kingcounty-a-1/top2/sqft_lot15.svg}
    \caption{Frecuencia del campo sqft lot15 en el modelo real y top2}
    \label{frecuency-top2-sqft lot15}
\end{figure}

\newpage
\subsection{Privacidad}
Al analizar los registros más cercanos entre los conjuntos de datos reales utilizados para el entrenamiento, los generados por los modelos, y el conjunto de datos reales almacenados, encontramos que las distancias entre ellos se presentan en las siguientes tablas. Es importante destacar que la distancia mínima para el modelo Tddpm es de 0.0134, indicando que cada registro tiene al menos esa distancia respecto al conjunto real. La determinación del epsilon requerido para asegurar la privacidad de los datos depende del análisis específico de los datos a proteger y sus probabilidades asociadas. Sin embargo, si el objetivo es proteger el 95\% de los datos, el modelo Tddpm alcanza una distancia de 0.0579, mientras que el modelo Smote tiene una distancia de 0.00704.


\begin{table}[H]
\centering
\fontsize{10}{14}\selectfont
\caption{Proporción entre el más cercano y el segundo más cercano, percentil 5, datos king county}
\label{table-dcr-king county-a}
\begin{tabular}{|l|l|r|r|r|r|r|r|r|}
\hline
\rowcolor[gray]{0.8}
Modelo & NNDR ST & NNDR SH & NNDR TH & \textbf{Score} \\
\hline tddpm\_mlp & 6.12e-01±2.28e-03 & 6.03e-01±4.17e-03 & 3.76e-01±0.00e+00 & 9.52e-01±2.36e-03 \\
\hline smote-enc & \cellcolor[rgb]{0.9, 0.54, 0.52} 1.98e-01±3.57e-03 & \cellcolor[rgb]{0.9, 0.54, 0.52} 4.09e-01±6.38e-03 & 3.76e-01±0.00e+00 & \bfseries 9.53e-01±2.45e-04 \\
\hline ctgan & 8.09e-01±8.59e-03 & 8.15e-01±4.89e-03 & 3.76e-01±0.00e+00 & 8.24e-01±2.02e-02 \\
\hline tablepreset & 8.25e-01±0.00e+00 & 8.18e-01±7.85e-17 & 3.76e-01±0.00e+00 & 8.37e-01±7.85e-17 \\
\hline copulagan & \bfseries 8.30e-01±5.92e-03 & \bfseries 8.24e-01±3.49e-03 & 3.76e-01±0.00e+00 & 7.89e-01±2.92e-03 \\
\hline gaussiancopula & 7.53e-01±0.00e+00 & 7.52e-01±1.36e-16 & 3.76e-01±0.00e+00 & 7.88e-01±0.00e+00 \\
\hline tvae & 7.32e-01±5.74e-03 & 7.04e-01±4.22e-03 & 3.76e-01±0.00e+00 & \cellcolor[rgb]{0.9, 0.54, 0.52} 7.38e-01±1.18e-02 \\
\hline
\end{tabular}
\end{table}

\begin{table}[H]
\centering
\fontsize{10}{14}\selectfont
\caption{Proporción entre el más cercano y el segundo más cercano, percentil 1, datos king county}
\label{table-dcr-king county-a}
\begin{tabular}{|l|l|r|r|r|r|r|r|r|}
\hline
\rowcolor[gray]{0.8}
Modelo & NNDR ST & NNDR SH & NNDR TH & \textbf{Score} \\
\hline tddpm\_mlp & 4.49e-01±7.33e-03 & 4.49e-01±7.70e-03 & 7.98e-02±0.00e+00 & 9.52e-01±2.36e-03 \\
\hline smote-enc & \cellcolor[rgb]{0.9, 0.54, 0.52} 6.74e-02±4.35e-03 & \cellcolor[rgb]{0.9, 0.54, 0.52} 1.88e-01±1.05e-02 & 7.98e-02±0.00e+00 & \bfseries 9.53e-01±2.45e-04 \\
\hline ctgan & 7.14e-01±7.52e-03 & 7.19e-01±7.48e-03 & 7.98e-02±0.00e+00 & 8.24e-01±2.02e-02 \\
\hline tablepreset & 7.20e-01±1.11e-16 & 7.19e-01±1.11e-16 & 7.98e-02±0.00e+00 & 8.37e-01±7.85e-17 \\
\hline copulagan & \bfseries 7.56e-01±7.65e-03 & \bfseries 7.47e-01±2.64e-03 & 7.98e-02±0.00e+00 & 7.89e-01±2.92e-03 \\
\hline gaussiancopula & 6.48e-01±7.85e-17 & 6.43e-01±7.85e-17 & 7.98e-02±0.00e+00 & 7.88e-01±0.00e+00 \\
\hline tvae & 6.06e-01±1.27e-02 & 5.87e-01±4.93e-03 & 7.98e-02±0.00e+00 & \cellcolor[rgb]{0.9, 0.54, 0.52} 7.38e-01±1.18e-02 \\
\hline
\end{tabular}
\end{table}

\begin{table}[H]
\centering
\fontsize{10}{14}\selectfont
\caption{Proporción entre el más cercano y el segundo más cercano, mínimo, datos king county}
\label{table-dcr-king county-a}
\begin{tabular}{|l|l|r|r|r|r|r|r|r|}
\hline
\rowcolor[gray]{0.8}
Modelo & NNDR ST & NNDR SH & NNDR TH & \textbf{Score} \\
\hline tddpm\_mlp & 1.23e-01±1.54e-02 & 1.57e-01±3.36e-02 & 0.00e+00±0.00e+00 & 9.52e-01±2.36e-03 \\
\hline smote-enc & \cellcolor[rgb]{0.9, 0.54, 0.52} 0.00e+00±0.00e+00 & \cellcolor[rgb]{0.9, 0.54, 0.52} 1.10e-02±5.41e-03 & 0.00e+00±0.00e+00 & \bfseries 9.53e-01±2.45e-04 \\
\hline ctgan & 4.25e-01±3.23e-02 & 3.91e-01±4.06e-02 & 0.00e+00±0.00e+00 & 8.24e-01±2.02e-02 \\
\hline tablepreset & 4.51e-01±6.80e-17 & 3.58e-01±0.00e+00 & 0.00e+00±0.00e+00 & 8.37e-01±7.85e-17 \\
\hline copulagan & \bfseries 5.48e-01±1.58e-02 & \bfseries 5.32e-01±3.85e-02 & 0.00e+00±0.00e+00 & 7.89e-01±2.92e-03 \\
\hline gaussiancopula & 3.90e-01±5.55e-17 & 4.08e-01±0.00e+00 & 0.00e+00±0.00e+00 & 7.88e-01±0.00e+00 \\
\hline tvae & 3.44e-01±1.91e-02 & 3.43e-01±1.63e-02 & 0.00e+00±0.00e+00 & \cellcolor[rgb]{0.9, 0.54, 0.52} 7.38e-01±1.18e-02 \\
\hline
\end{tabular}
\end{table}

\newpage
Al analizar los ratios entre la distancia al primer vecino más cercano y la distancia al segundo para el modelo Tddpm, se evidencia que para el percentil 5, la distancia al vecino más cercano es solo 2/3 de la distancia al segundo más cercano. Sin embargo, para el percentil 1, esta distancia se reduce a la mitad. En contraposición, para el modelo Smote, en el percentil 5, la distancia al vecino más cercano es solo un 20\% de la distancia al segundo más cercano, y disminuye rápidamente a un 6\% para el percentil 1.

\begin{table}[H]
\centering
\fontsize{10}{14}\selectfont
\caption{Proporción entre el más cercano y el segundo más cercano, percentil 5, datos king county}
\label{table-dcr-king county-a}
\begin{tabular}{|l|l|r|r|r|r|r|r|r|}
\hline
\rowcolor[gray]{0.8}
Modelo & NNDR ST & NNDR SH & NNDR TH & \textbf{Score} \\
\hline tddpm\_mlp & 6.12e-01±2.28e-03 & 6.03e-01±4.17e-03 & 3.76e-01±0.00e+00 & \bfseries 9.52e-01±2.36e-03 \\
\hline smote-enc & \cellcolor[rgb]{0.9, 0.54, 0.52} 1.98e-01±3.57e-03 & \cellcolor[rgb]{0.9, 0.54, 0.52} 4.09e-01±6.38e-03 & 3.76e-01±0.00e+00 & 9.53e-01±2.45e-04 \\
\hline ctgan & 8.09e-01±8.59e-03 & 8.15e-01±4.89e-03 & 3.76e-01±0.00e+00 & 8.24e-01±2.02e-02 \\
\hline tablepreset & 8.25e-01±1.11e-16 & 8.18e-01±0.00e+00 & 3.76e-01±0.00e+00 & 8.37e-01±7.85e-17 \\
\hline copulagan & \bfseries 8.30e-01±5.92e-03 & \bfseries 8.24e-01±3.49e-03 & 3.76e-01±0.00e+00 & 7.89e-01±2.92e-03 \\
\hline gaussiancopula & 7.53e-01±1.11e-16 & 7.52e-01±1.36e-16 & 3.76e-01±0.00e+00 & 7.88e-01±0.00e+00 \\
\hline tvae & 7.32e-01±5.74e-03 & 7.04e-01±4.22e-03 & 3.76e-01±0.00e+00 & \cellcolor[rgb]{0.9, 0.54, 0.52} 7.38e-01±1.18e-02 \\
\hline
\end{tabular}
\end{table}

\begin{table}[H]
\centering
\fontsize{10}{14}\selectfont
\caption{Proporción entre el más cercano y el segundo más cercano, percentil 1, datos king county}
\label{table-dcr-king county-a}
\begin{tabular}{|l|l|r|r|r|r|r|r|r|}
\hline
\rowcolor[gray]{0.8}
Modelo & NNDR ST & NNDR SH & NNDR TH & \textbf{Score} \\
\hline tddpm\_mlp & 4.49e-01±7.33e-03 & 4.49e-01±7.70e-03 & 7.98e-02±0.00e+00 & \bfseries 9.52e-01±2.36e-03 \\
\hline smote-enc & \cellcolor[rgb]{0.9, 0.54, 0.52} 6.74e-02±4.35e-03 & \cellcolor[rgb]{0.9, 0.54, 0.52} 1.88e-01±1.05e-02 & 7.98e-02±0.00e+00 & 9.53e-01±2.45e-04 \\
\hline ctgan & 7.14e-01±7.52e-03 & 7.19e-01±7.48e-03 & 7.98e-02±0.00e+00 & 8.24e-01±2.02e-02 \\
\hline tablepreset & 7.20e-01±0.00e+00 & 7.19e-01±1.11e-16 & 7.98e-02±0.00e+00 & 8.37e-01±7.85e-17 \\
\hline copulagan & \bfseries 7.56e-01±7.65e-03 & \bfseries 7.47e-01±2.64e-03 & 7.98e-02±0.00e+00 & 7.89e-01±2.92e-03 \\
\hline gaussiancopula & 6.48e-01±7.85e-17 & 6.43e-01±1.36e-16 & 7.98e-02±0.00e+00 & 7.88e-01±0.00e+00 \\
\hline tvae & 6.06e-01±1.27e-02 & 5.87e-01±4.93e-03 & 7.98e-02±0.00e+00 & \cellcolor[rgb]{0.9, 0.54, 0.52} 7.38e-01±1.18e-02 \\
\hline
\end{tabular}
\end{table}

\begin{table}[H]
\centering
\fontsize{10}{14}\selectfont
\caption{Proporción entre el más cercano y el segundo más cercano, minimo, datos king county}
\label{table-dcr-king county-a}
\begin{tabular}{|l|l|r|r|r|r|r|r|r|}
\hline
\rowcolor[gray]{0.8}
Modelo & NNDR ST & NNDR SH & NNDR TH & \textbf{Score} \\
\hline tddpm\_mlp & 1.23e-01±1.54e-02 & 1.57e-01±3.36e-02 & 0.00e+00±0.00e+00 & \bfseries 9.52e-01±2.36e-03 \\
\hline smote-enc & \cellcolor[rgb]{0.9, 0.54, 0.52} 0.00e+00±0.00e+00 & \cellcolor[rgb]{0.9, 0.54, 0.52} 1.10e-02±5.41e-03 & 0.00e+00±0.00e+00 & 9.53e-01±2.45e-04 \\
\hline ctgan & 4.25e-01±3.23e-02 & 3.91e-01±4.06e-02 & 0.00e+00±0.00e+00 & 8.24e-01±2.02e-02 \\
\hline tablepreset & 4.51e-01±6.80e-17 & 3.58e-01±5.55e-17 & 0.00e+00±0.00e+00 & 8.37e-01±7.85e-17 \\
\hline copulagan & \bfseries 5.48e-01±1.58e-02 & \bfseries 5.32e-01±3.85e-02 & 0.00e+00±0.00e+00 & 7.89e-01±2.92e-03 \\
\hline gaussiancopula & 3.90e-01±3.93e-17 & 4.08e-01±0.00e+00 & 0.00e+00±0.00e+00 & 7.88e-01±0.00e+00 \\
\hline tvae & 3.44e-01±1.91e-02 & 3.43e-01±1.63e-02 & 0.00e+00±0.00e+00 & \cellcolor[rgb]{0.9, 0.54, 0.52} 7.38e-01±1.18e-02 \\
\hline
\end{tabular}
\end{table}


\newpage
En la Figura \ref{frecuency-top2-privacy} solo se consideran los modelos Tddpm y Smote para su comparación. En ambos casos, existe una distancia mayor a cero. Sin embargo, esta distancia es mayor en el caso de Tddpm, lo que sugiere que este conjunto puede ser considerado superior en términos de privacidad.


\begin{figure}[H]
    \centering
    \includesvg[scale=.7,inkscapelatex=false]{datasets/kingcounty-a-2/top2/privacy.svg}
    \caption{Frecuencia del campo Privacy en el modelo real y Top 2}
    \label{frecuency-top2-privacy}
\end{figure}
\newpage
\subsection{Ejemplo de registros}
Las Tablas \ref{table-example-king county-a-1-smote-enc-min} y \ref{table-example-king county-a-1-tddpm_mlp-min} presentan un ejemplo de la mínima distancia en los modelos Smote y Tddpm, respectivamente. Los nombres de las columnas representan la distancia de Minkowski al registro Sintético, indicado de esta manera en la columna correspondiente. Las celdas coloreadas en rojo señalan que el valor de la característica para una propiedad específica es idéntico al valor correspondiente de la propiedad de referencia. Así, la tabla proporciona una comparación detallada de las propiedades que son similares en términos de las características seleccionadas.

En la Tabla \ref{table-example-king county-a-1-smote-enc-min}, se puede observar claramente que, excepto por las variables de precio y fecha en el segundo registro más cercano, son idénticas a las del original. Esto significa que ese registro fue transferido en su totalidad al conjunto sintético. 


\begin{table}[H]
\centering
\fontsize{10}{14}\selectfont
\caption{Ejemplos para el modelo smote-enc, minimo}
\label{table-example-king county-a-1-smote-enc-min}
\begin{tabular}{|l|r|r|r|}
\hline
\rowcolor[gray]{0.8}
Variable/Distancia & Sintético & DCR1 d(0.00e+00) & DCR2 d(1.01e-02) \\
\hline sqft\_living & \cellcolor[rgb]{0.9, 0.54, 0.52} 1790.000000 & \cellcolor[rgb]{0.9, 0.54, 0.52} 1790.000000 & \cellcolor[rgb]{0.9, 0.54, 0.52} 1790.000000 \\
\hline sqft\_basement & \cellcolor[rgb]{0.9, 0.54, 0.52} 0.000000 & \cellcolor[rgb]{0.9, 0.54, 0.52} 0.000000 & \cellcolor[rgb]{0.9, 0.54, 0.52} 0.000000 \\
\hline id & \cellcolor[rgb]{0.9, 0.54, 0.52} 1721801010.000000 & \cellcolor[rgb]{0.9, 0.54, 0.52} 1721801010.000000 & \cellcolor[rgb]{0.9, 0.54, 0.52} 1721801010.000000 \\
\hline sqft\_above & \cellcolor[rgb]{0.9, 0.54, 0.52} 1790.000000 & \cellcolor[rgb]{0.9, 0.54, 0.52} 1790.000000 & \cellcolor[rgb]{0.9, 0.54, 0.52} 1790.000000 \\
\hline price & \cellcolor[rgb]{0.9, 0.54, 0.52} 225000.000000 & \cellcolor[rgb]{0.9, 0.54, 0.52} 225000.000000 & 302100.000000 \\
\hline view & \cellcolor[rgb]{0.9, 0.54, 0.52} 0 & \cellcolor[rgb]{0.9, 0.54, 0.52} 0 & \cellcolor[rgb]{0.9, 0.54, 0.52} 0 \\
\hline waterfront & \cellcolor[rgb]{0.9, 0.54, 0.52} 0 & \cellcolor[rgb]{0.9, 0.54, 0.52} 0 & \cellcolor[rgb]{0.9, 0.54, 0.52} 0 \\
\hline sqft\_lot & \cellcolor[rgb]{0.9, 0.54, 0.52} 6120.000000 & \cellcolor[rgb]{0.9, 0.54, 0.52} 6120.000000 & \cellcolor[rgb]{0.9, 0.54, 0.52} 6120.000000 \\
\hline sqft\_living15 & \cellcolor[rgb]{0.9, 0.54, 0.52} 830.000000 & \cellcolor[rgb]{0.9, 0.54, 0.52} 830.000000 & \cellcolor[rgb]{0.9, 0.54, 0.52} 830.000000 \\
\hline grade & \cellcolor[rgb]{0.9, 0.54, 0.52} 6 & \cellcolor[rgb]{0.9, 0.54, 0.52} 6 & \cellcolor[rgb]{0.9, 0.54, 0.52} 6 \\
\hline bathrooms & \cellcolor[rgb]{0.9, 0.54, 0.52} 1.000000 & \cellcolor[rgb]{0.9, 0.54, 0.52} 1.000000 & \cellcolor[rgb]{0.9, 0.54, 0.52} 1.000000 \\
\hline long & \cellcolor[rgb]{0.9, 0.54, 0.52} -122.337000 & \cellcolor[rgb]{0.9, 0.54, 0.52} -122.337000 & \cellcolor[rgb]{0.9, 0.54, 0.52} -122.337000 \\
\hline yr\_renovated & \cellcolor[rgb]{0.9, 0.54, 0.52} 1964.000000 & \cellcolor[rgb]{0.9, 0.54, 0.52} 1964.000000 & \cellcolor[rgb]{0.9, 0.54, 0.52} 1964.000000 \\
\hline zipcode & \cellcolor[rgb]{0.9, 0.54, 0.52} 98146 & \cellcolor[rgb]{0.9, 0.54, 0.52} 98146 & \cellcolor[rgb]{0.9, 0.54, 0.52} 98146 \\
\hline condition & \cellcolor[rgb]{0.9, 0.54, 0.52} 3 & \cellcolor[rgb]{0.9, 0.54, 0.52} 3 & \cellcolor[rgb]{0.9, 0.54, 0.52} 3 \\
\hline bedrooms & \cellcolor[rgb]{0.9, 0.54, 0.52} 3 & \cellcolor[rgb]{0.9, 0.54, 0.52} 3 & \cellcolor[rgb]{0.9, 0.54, 0.52} 3 \\
\hline date & \cellcolor[rgb]{0.9, 0.54, 0.52} 20140903T000000 & \cellcolor[rgb]{0.9, 0.54, 0.52} 20140903T000000 & 20150424T000000 \\
\hline sqft\_lot15 & \cellcolor[rgb]{0.9, 0.54, 0.52} 6120.000000 & \cellcolor[rgb]{0.9, 0.54, 0.52} 6120.000000 & \cellcolor[rgb]{0.9, 0.54, 0.52} 6120.000000 \\
\hline lat & \cellcolor[rgb]{0.9, 0.54, 0.52} 47.508000 & \cellcolor[rgb]{0.9, 0.54, 0.52} 47.508000 & \cellcolor[rgb]{0.9, 0.54, 0.52} 47.508000 \\
\hline yr\_built & \cellcolor[rgb]{0.9, 0.54, 0.52} 1937.000000 & \cellcolor[rgb]{0.9, 0.54, 0.52} 1937.000000 & \cellcolor[rgb]{0.9, 0.54, 0.52} 1937.000000 \\
\hline floors & \cellcolor[rgb]{0.9, 0.54, 0.52} 1.000000 & \cellcolor[rgb]{0.9, 0.54, 0.52} 1.000000 & \cellcolor[rgb]{0.9, 0.54, 0.52} 1.000000 \\
\hline
\end{tabular}
\end{table}

\newpage
La Tabla \ref{table-example-king county-a-1-tddpm_mlp-min} presenta valores de distancia mayores que los obtenidos en la tabla correspondiente a Smote (\ref{table-example-king county-a-1-smote-enc-min}). Se pueden observar diferencias en las variables \emph{sqft\_living}, \emph{sqft\_lot}, \emph{sqft\_above}, \emph{yr\_built} y \emph{lat}, entre otras. Esta es la mínima distancia encontrada por la métrica.
\begin{table}[H]
\centering
\caption{Ejemplos para el modelo tddpm\_mlp, minimo}
\label{table-example-king county-a-1}
\begin{tabular}{|l|r|r|r|}
\hline
\rowcolor[gray]{0.8}
Variable/Distancia & 0 & 9.90e-03 & 1.03e-02 \\
\hline id & \cellcolor[rgb]{0.9, 0.54, 0.52} 7202202136.850033 & 7202330530.000000 & 7202330030.000000 \\
\hline sqft\_living & \cellcolor[rgb]{0.9, 0.54, 0.52} 1700.000000 & 1690.000000 & 1650.000000 \\
\hline sqft\_lot & \cellcolor[rgb]{0.9, 0.54, 0.52} 3524.360953 & 3322.000000 & 5683.000000 \\
\hline sqft\_above & \cellcolor[rgb]{0.9, 0.54, 0.52} 1654.928371 & 1690.000000 & 1650.000000 \\
\hline sqft\_basement & \cellcolor[rgb]{0.9, 0.54, 0.52} 0.000000 & \cellcolor[rgb]{0.9, 0.54, 0.52} 0.000000 & \cellcolor[rgb]{0.9, 0.54, 0.52} 0.000000 \\
\hline yr\_built & \cellcolor[rgb]{0.9, 0.54, 0.52} 2004.000000 & 2003.000000 & 2003.000000 \\
\hline yr\_renovated & \cellcolor[rgb]{0.9, 0.54, 0.52} 0.000000 & \cellcolor[rgb]{0.9, 0.54, 0.52} 0.000000 & \cellcolor[rgb]{0.9, 0.54, 0.52} 0.000000 \\
\hline lat & \cellcolor[rgb]{0.9, 0.54, 0.52} 47.683989 & 47.682400 & 47.683000 \\
\hline long & \cellcolor[rgb]{0.9, 0.54, 0.52} -122.036195 & \cellcolor[rgb]{0.9, 0.54, 0.52} -122.036000 & \cellcolor[rgb]{0.9, 0.54, 0.52} -122.035000 \\
\hline sqft\_living15 & \cellcolor[rgb]{0.9, 0.54, 0.52} 1650.000000 & \cellcolor[rgb]{0.9, 0.54, 0.52} 1650.000000 & \cellcolor[rgb]{0.9, 0.54, 0.52} 1650.000000 \\
\hline sqft\_lot15 & \cellcolor[rgb]{0.9, 0.54, 0.52} 3796.678538 & 3446.000000 & 4193.000000 \\
\hline price & \cellcolor[rgb]{0.9, 0.54, 0.52} 475000.000000 & 479000.000000 & 500000.000000 \\
\hline date & \cellcolor[rgb]{0.9, 0.54, 0.52} 20140908T000000 & 20150116T000000 & 20140822T000000 \\
\hline bedrooms & \cellcolor[rgb]{0.9, 0.54, 0.52} 3 & \cellcolor[rgb]{0.9, 0.54, 0.52} 3 & \cellcolor[rgb]{0.9, 0.54, 0.52} 3 \\
\hline bathrooms & \cellcolor[rgb]{0.9, 0.54, 0.52} 2.500000 & \cellcolor[rgb]{0.9, 0.54, 0.52} 2.500000 & \cellcolor[rgb]{0.9, 0.54, 0.52} 2.500000 \\
\hline floors & \cellcolor[rgb]{0.9, 0.54, 0.52} 2.000000 & \cellcolor[rgb]{0.9, 0.54, 0.52} 2.000000 & \cellcolor[rgb]{0.9, 0.54, 0.52} 2.000000 \\
\hline waterfront & \cellcolor[rgb]{0.9, 0.54, 0.52} 0 & \cellcolor[rgb]{0.9, 0.54, 0.52} 0 & \cellcolor[rgb]{0.9, 0.54, 0.52} 0 \\
\hline view & \cellcolor[rgb]{0.9, 0.54, 0.52} 0 & \cellcolor[rgb]{0.9, 0.54, 0.52} 0 & \cellcolor[rgb]{0.9, 0.54, 0.52} 0 \\
\hline condition & \cellcolor[rgb]{0.9, 0.54, 0.52} 3 & \cellcolor[rgb]{0.9, 0.54, 0.52} 3 & \cellcolor[rgb]{0.9, 0.54, 0.52} 3 \\
\hline grade & \cellcolor[rgb]{0.9, 0.54, 0.52} 7 & \cellcolor[rgb]{0.9, 0.54, 0.52} 7 & \cellcolor[rgb]{0.9, 0.54, 0.52} 7 \\
\hline zipcode & \cellcolor[rgb]{0.9, 0.54, 0.52} 98053 & \cellcolor[rgb]{0.9, 0.54, 0.52} 98053 & \cellcolor[rgb]{0.9, 0.54, 0.52} 98053 \\
\hline
\end{tabular}
\end{table}

\newpage
En la Tabla \ref{table-example-king county-a-1-smote-enc-1p}, se puede observar una notable mejoría en el modelo Smote. Esta tabla presenta un registro cercano con múltiples diferencias, entre las cuales se pueden destacar \emph{sqft\_lot} y \emph{price}.
\begin{table}[H]
\centering
\fontsize{10}{14}\selectfont
\caption{Ejemplos para el modelo smote-enc, percentil 1}
\label{table-example-king county-a-1-smote-enc-1p}
\begin{tabular}{|l|r|r|r|}
\hline
\rowcolor[gray]{0.8}
Variable/Distancia & Sintético & DCR1 d(2.07e-03) & DCR2 d(5.44e-03) \\
\hline sqft\_living & \cellcolor[rgb]{0.9, 0.54, 0.52} 1798.822360 & 1800.000000 & 1800.000000 \\
\hline sqft\_basement & \cellcolor[rgb]{0.9, 0.54, 0.52} 0.000000 & \cellcolor[rgb]{0.9, 0.54, 0.52} 0.000000 & \cellcolor[rgb]{0.9, 0.54, 0.52} 0.000000 \\
\hline id & \cellcolor[rgb]{0.9, 0.54, 0.52} 3862710182.355279 & 3862710050.000000 & 3862710210.000000 \\
\hline sqft\_above & \cellcolor[rgb]{0.9, 0.54, 0.52} 1798.822360 & 1800.000000 & 1800.000000 \\
\hline price & \cellcolor[rgb]{0.9, 0.54, 0.52} 450000.000000 & 437718.000000 & 409316.000000 \\
\hline view & \cellcolor[rgb]{0.9, 0.54, 0.52} 0 & \cellcolor[rgb]{0.9, 0.54, 0.52} 0 & \cellcolor[rgb]{0.9, 0.54, 0.52} 0 \\
\hline waterfront & \cellcolor[rgb]{0.9, 0.54, 0.52} 0 & \cellcolor[rgb]{0.9, 0.54, 0.52} 0 & \cellcolor[rgb]{0.9, 0.54, 0.52} 0 \\
\hline sqft\_lot & \cellcolor[rgb]{0.9, 0.54, 0.52} 2851.472564 & 3265.000000 & 3168.000000 \\
\hline sqft\_living15 & \cellcolor[rgb]{0.9, 0.54, 0.52} 1800.000000 & \cellcolor[rgb]{0.9, 0.54, 0.52} 1800.000000 & \cellcolor[rgb]{0.9, 0.54, 0.52} 1800.000000 \\
\hline grade & \cellcolor[rgb]{0.9, 0.54, 0.52} 8 & \cellcolor[rgb]{0.9, 0.54, 0.52} 8 & \cellcolor[rgb]{0.9, 0.54, 0.52} 8 \\
\hline bathrooms & \cellcolor[rgb]{0.9, 0.54, 0.52} 2.500000 & \cellcolor[rgb]{0.9, 0.54, 0.52} 2.500000 & \cellcolor[rgb]{0.9, 0.54, 0.52} 2.500000 \\
\hline long & \cellcolor[rgb]{0.9, 0.54, 0.52} -121.841000 & \cellcolor[rgb]{0.9, 0.54, 0.52} -121.841000 & \cellcolor[rgb]{0.9, 0.54, 0.52} -121.841000 \\
\hline yr\_renovated & \cellcolor[rgb]{0.9, 0.54, 0.52} 0.000000 & \cellcolor[rgb]{0.9, 0.54, 0.52} 0.000000 & \cellcolor[rgb]{0.9, 0.54, 0.52} 0.000000 \\
\hline zipcode & \cellcolor[rgb]{0.9, 0.54, 0.52} 98065 & \cellcolor[rgb]{0.9, 0.54, 0.52} 98065 & \cellcolor[rgb]{0.9, 0.54, 0.52} 98065 \\
\hline condition & \cellcolor[rgb]{0.9, 0.54, 0.52} 3 & \cellcolor[rgb]{0.9, 0.54, 0.52} 3 & \cellcolor[rgb]{0.9, 0.54, 0.52} 3 \\
\hline bedrooms & \cellcolor[rgb]{0.9, 0.54, 0.52} 3 & \cellcolor[rgb]{0.9, 0.54, 0.52} 3 & \cellcolor[rgb]{0.9, 0.54, 0.52} 3 \\
\hline date & \cellcolor[rgb]{0.9, 0.54, 0.52} 20140520T000000 & 20141113T000000 & \cellcolor[rgb]{0.9, 0.54, 0.52} 20140520T000000 \\
\hline sqft\_lot15 & \cellcolor[rgb]{0.9, 0.54, 0.52} 3280.073791 & 3663.000000 & 3393.000000 \\
\hline lat & \cellcolor[rgb]{0.9, 0.54, 0.52} 47.534176 & 47.533800 & 47.534200 \\
\hline yr\_built & \cellcolor[rgb]{0.9, 0.54, 0.52} 2013.882236 & 2014.000000 & 2014.000000 \\
\hline floors & \cellcolor[rgb]{0.9, 0.54, 0.52} 2.000000 & \cellcolor[rgb]{0.9, 0.54, 0.52} 2.000000 & \cellcolor[rgb]{0.9, 0.54, 0.52} 2.000000 \\
\hline
\end{tabular}
\end{table}

\newpage
\subsection{Propiedades estadísticas}
El listado completo de las propiedades estadísticas se encuentra en el Anexo \ref{propiedades-estadisticas-kingCounty}. A continuación, se procede a mostrar las propiedades estadisticas que entre el modelo Tddpm y Smote consigan una diferencia mayor al 5\% con respecto al conjunto original de entrenamiento. Se agrega el modelo Ctgan como referencia. las variables fueron seleccionadas por se 1) El peor resultado en la cobertura y 2) El peor resultado en la distribución respectivamente.

Como se puede apreciar en la Tabla \ref{table-stats-king county-a-1-bedrooms-short}, en general, el modelo Tddpm muestra propiedades estadísticas más cercanas al conjunto original, con excepciones notables en las métricas de máximo, kurtosis y Jarque-Bera. La diferencia en la métrica de \emph{máximo} podría contribuir a la baja puntuación en la métrica de cobertura mostrada en la Tabla \ref{table-coverage-king county-a}. Por otro lado, las diferencias en las métricas de kurtosis, skew y Jarque-Bera podrían explicar las desviaciones observadas en la métrica de distribución de la Tabla \ref{table-shape-king county-a}.
\begin{table}[H]
\centering
\fontsize{8}{14}\selectfont
\caption{Propiedades estadisticas de variable bedrooms con cambio\ensuremath{>}5\%, King county (A-1)}
\label{table-stats-king county-a-1-bedrooms-short}
\begin{tabular}{|l|m{10em}|m{10em}|m{10em}|m{10em}|}
\hline
 \rowcolor[gray]{0.8}
Variable/Modelo & Real & tddpm\_mlp & smote-enc & ctgan \\
\hline nobs & 17290 & \bfseries 21613 & \cellcolor[rgb]{0.9, 0.54, 0.52} 21614 & \bfseries 21613 \\
\hline mean & 3.368 & \bfseries 3.337 & 3.279 & \cellcolor[rgb]{0.9, 0.54, 0.52} 4.075 \\
\hline std\_err & 0.007 & \bfseries 0.005 & 0.005 & \cellcolor[rgb]{0.9, 0.54, 0.52} 0.025 \\
\hline upper\_ci & 3.382 & \bfseries 3.348 & 3.289 & \cellcolor[rgb]{0.9, 0.54, 0.52} 4.124 \\
\hline lower\_ci & 3.354 & \bfseries 3.327 & 3.270 & \cellcolor[rgb]{0.9, 0.54, 0.52} 4.026 \\
\hline std & 0.931 & \bfseries 0.777 & 0.710 & \cellcolor[rgb]{0.9, 0.54, 0.52} 3.682 \\
\hline mad & 0.734 & \bfseries 0.643 & 0.587 & \cellcolor[rgb]{0.9, 0.54, 0.52} 1.323 \\
\hline mad\_normal & 0.920 & \bfseries 0.806 & 0.735 & \cellcolor[rgb]{0.9, 0.54, 0.52} 1.658 \\
\hline coef\_var & 0.277 & \bfseries 0.233 & 0.217 & \cellcolor[rgb]{0.9, 0.54, 0.52} 0.904 \\
\hline range & 33.000 & \cellcolor[rgb]{0.9, 0.54, 0.52} 8.000 & \cellcolor[rgb]{0.9, 0.54, 0.52} 8.000 & \bfseries 33.000 \\
\hline max & 33.000 & \cellcolor[rgb]{0.9, 0.54, 0.52} 8.000 & 9.000 & \bfseries 33.000 \\
\hline min & 0.000 & \bfseries 0.000 & \cellcolor[rgb]{0.9, 0.54, 0.52} 1.000 & \bfseries 0.000 \\
\hline skew & 2.304 & \bfseries 0.273 & 0.140 & \cellcolor[rgb]{0.9, 0.54, 0.52} 6.792 \\
\hline kurtosis & 63.268 & 3.444 & \cellcolor[rgb]{0.9, 0.54, 0.52} 3.196 & \bfseries 53.464 \\
\hline jarque\_bera & 2631992 & 446 & \cellcolor[rgb]{0.9, 0.54, 0.52} 105 & \bfseries 2459502 \\
\hline jarque\_bera\_pval & 0.000 & 0.000 & \cellcolor[rgb]{0.9, 0.54, 0.52} 0.000 & \bfseries 0.000 \\
\hline mode\_freq & 0.455 & 0.504 & \cellcolor[rgb]{0.9, 0.54, 0.52} 0.528 & \bfseries 0.420 \\
\hline 95.0\% & 5.000 & \bfseries 5.000 & \cellcolor[rgb]{0.9, 0.54, 0.52} 4.000 & \cellcolor[rgb]{0.9, 0.54, 0.52} 6.000 \\
\hline 99.0\% & 6.000 & \bfseries 5.000 & \bfseries 5.000 & \cellcolor[rgb]{0.9, 0.54, 0.52} 33.000 \\
\hline 99.9\% & 7.000 & \bfseries 6.000 & 5.000 & \cellcolor[rgb]{0.9, 0.54, 0.52} 33.000 \\
\hline
\end{tabular}
\end{table}

\newpage
Es evidente que Smote presenta varias métricas inferiores a las de Tddpm. Entre estas destacan el mínimo, el máximo, la asimetría (skew) y los percentiles 0.1, 95, 99 y 99.9.
\begin{table}[H]
\centering
\fontsize{8}{14}\selectfont
\caption{Propiedades estadisticas de variable bathrooms con cambio\ensuremath{>}5\%, King county (A-1)}
\label{table-stats-king county-a-1-bathrooms-short}
\begin{tabular}{|l|m{10em}|m{10em}|m{10em}|m{10em}|}
\hline
 \rowcolor[gray]{0.8}
Variable/Modelo & Real & tddpm\_mlp & smote-enc & ctgan \\
\hline nobs & 17290 & \bfseries 21613 & \cellcolor[rgb]{0.9, 0.54, 0.52} 21614 & \bfseries 21613 \\
\hline mean & 2.114 & \bfseries 2.073 & 2.007 & \cellcolor[rgb]{0.9, 0.54, 0.52} 2.364 \\
\hline std\_err & 0.006 & 0.005 & \cellcolor[rgb]{0.9, 0.54, 0.52} 0.005 & \bfseries 0.006 \\
\hline upper\_ci & 2.125 & \bfseries 2.083 & 2.016 & \cellcolor[rgb]{0.9, 0.54, 0.52} 2.376 \\
\hline lower\_ci & 2.102 & \bfseries 2.064 & 1.997 & \cellcolor[rgb]{0.9, 0.54, 0.52} 2.351 \\
\hline std & 0.767 & \bfseries 0.720 & 0.703 & \cellcolor[rgb]{0.9, 0.54, 0.52} 0.935 \\
\hline mad & 0.615 & 0.586 & \bfseries 0.588 & \cellcolor[rgb]{0.9, 0.54, 0.52} 0.732 \\
\hline mad\_normal & 0.771 & 0.734 & \bfseries 0.737 & \cellcolor[rgb]{0.9, 0.54, 0.52} 0.918 \\
\hline coef\_var & 0.363 & 0.347 & \bfseries 0.351 & \cellcolor[rgb]{0.9, 0.54, 0.52} 0.396 \\
\hline range & 8.000 & \bfseries 8.000 & \cellcolor[rgb]{0.9, 0.54, 0.52} 5.250 & \bfseries 8.000 \\
\hline max & 8.000 & \bfseries 8.000 & \cellcolor[rgb]{0.9, 0.54, 0.52} 6.000 & \bfseries 8.000 \\
\hline min & 0.000 & \bfseries 0.000 & \cellcolor[rgb]{0.9, 0.54, 0.52} 0.750 & \bfseries 0.000 \\
\hline skew & 0.464 & 0.335 & \cellcolor[rgb]{0.9, 0.54, 0.52} 0.142 & \bfseries 0.381 \\
\hline kurtosis & 3.989 & \bfseries 3.881 & \cellcolor[rgb]{0.9, 0.54, 0.52} 2.798 & 3.556 \\
\hline jarque\_bera & 1326 & \bfseries 1103 & \cellcolor[rgb]{0.9, 0.54, 0.52} 110 & 801 \\
\hline jarque\_bera\_pval & 0.000 & \bfseries 0.000 & \cellcolor[rgb]{0.9, 0.54, 0.52} 0.000 & 0.000 \\
\hline mode\_freq & 0.251 & \bfseries 0.282 & \cellcolor[rgb]{0.9, 0.54, 0.52} 0.302 & 0.204 \\
\hline median & 2.250 & \bfseries 2.250 & \cellcolor[rgb]{0.9, 0.54, 0.52} 2.000 & \cellcolor[rgb]{0.9, 0.54, 0.52} 2.500 \\
\hline 0.1\% & 0.750 & \bfseries 0.750 & 1.000 & \cellcolor[rgb]{0.9, 0.54, 0.52} 0.000 \\
\hline 95.0\% & 3.500 & \bfseries 3.250 & \bfseries 3.250 & \cellcolor[rgb]{0.9, 0.54, 0.52} 4.000 \\
\hline 99.0\% & 4.250 & \bfseries 4.000 & \cellcolor[rgb]{0.9, 0.54, 0.52} 3.500 & 4.750 \\
\hline 99.9\% & 5.428 & 5.097 & \cellcolor[rgb]{0.9, 0.54, 0.52} 4.500 & \bfseries 5.750 \\
\hline
\end{tabular}
\end{table}

\newpage
\section{Conjunto de datos proveniente de Economicos}
\subsection{Tratamiento de nulos en conjunto A y B}
El conjunto de Económicos, a diferencia del conjunto de datos de King County que fue filtrado y preprocesado para evitar valores nulos, contiene elementos nulos. A continuación se describen dos tratamientos de estos elementos nulos. El primer enfoque simplemente elimina todos los registros que contienen un registro vacío utilizando el método `dropna`, como se muestra en el Código \ref{codigo-remove-nan}; este será considerado como el Conjunto A. En el segundo enfoque, los valores nulos son reemplazados por algún valor predeterminado o calculado, como se muestra en el Código \ref{codigo-replace-nan}; este será considerado como el Conjunto B.

\begin{listing}[H]
    \begin{minted}[linenos=true,frame=lines,framesep=2mm,baselinestretch=1.2]{python}
df_converted = df.dropna().astype({k: 'str' for k in ("description", "price", "title", "address", "owner",)})
basedate = pd.Timestamp('2017-12-01')
dtime = df_converted.pop("publication_date")
df_converted["publication_date"] = dtime.apply(lambda x: (x - basedate).days)
    \end{minted}
\caption{Eliminación de valores nulos en el conjunto de datos de Económicos}
\label{codigo-remove-nan}
\end{listing}

\begin{listing}[H]
    \begin{minted}[linenos=true,frame=lines,framesep=2mm,baselinestretch=1.2]{python}
df_converted = df.fillna(dict(
            property_type = "None",
            transaction_type = "None",
            state = "None",
            county = "None",
            rooms = -1,
            bathrooms = -1,
            m_built = -1,
            m_size = -1,
            source = "None"
    )).fillna(-1).astype({k: 'str' for k in ("description", "price", "title", "address", "owner",)})
basedate = pd.Timestamp('2017-12-01')
dtime = df_converted.pop("publication_date")
df_converted["publication_date"] = dtime.apply(lambda x: (x - basedate).days)
    \end{minted}
\caption{Reemplazo de valores nulos en el conjunto de datos de Económicos}
\label{codigo-replace-nan}
\end{listing}

\newpage
\subsection{SDMetrics Score - Conjunto A}
\label{ds-conjunto-a}
Para el conjunto A, como se muestra en la Tabla \ref{table-score-economicos-a}, Tddpm es un punto superior a Smote y ambos superan en más de 10 puntos al siguiente modelo. Sin embargo, un punto crucial es que Smote tiene una cobertura (\emph{Coverage}) que es 12 puntos inferior a Tddpm.

\begin{table}[H]
\centering
\fontsize{10}{14}\selectfont
\caption{Evaluaci\'on de M\'etricas de Rendimiento para Diversos Modelos de Aprendizaje Autom\'atico, Economicos}
\label{table-score-economicos-a}
\begin{tabular}{|l|r|r|r|r|r|}
\hline
\rowcolor[gray]{0.8}
Model Name & Column Pair Trends & Column Shapes & Coverage & Boundaries & \textbf{Score} \\
\hline tddpm\_mlp & \bfseries 9.73e-01±2.21e-03 & \bfseries 9.84e-01±3.63e-04 & 7.91e-01±5.31e-02 & \bfseries 1.00e+00±0.00e+00 & \bfseries 9.79e-01±1.27e-03 \\
\hline smote-enc & 9.62e-01±1.52e-03 & 9.76e-01±4.01e-04 & 6.67e-01±2.79e-02 & \bfseries 1.00e+00±0.00e+00 & 9.69e-01±6.71e-04 \\
\hline copulagan & 7.46e-01±3.30e-02 & 7.90e-01±2.63e-02 & 6.80e-01±2.57e-03 & \bfseries 1.00e+00±0.00e+00 & 7.68e-01±2.96e-02 \\
\hline ctgan & 7.44e-01±1.96e-02 & 6.53e-01±4.72e-02 & 6.75e-01±1.75e-03 & \bfseries 1.00e+00±0.00e+00 & 6.98e-01±2.63e-02 \\
\hline gaussiancopula & 6.96e-01±0.00e+00 & 6.88e-01±0.00e+00 & 5.65e-01±0.00e+00 & \bfseries 1.00e+00±0.00e+00 & 6.92e-01±0.00e+00 \\
\hline tvae & 5.83e-01±1.02e-02 & 6.41e-01±4.66e-02 & \bfseries 8.59e-02±1.28e-02 & \bfseries 1.00e+00±0.00e+00 & 6.12e-01±2.50e-02 \\
\hline
\end{tabular}
\end{table}


\newpage
\subsection{Correlación - Conjunto A}
Aunque la diferencia es pequeña, se puede apreciar al comparar visualmente las Figuras \ref{pairwise-economicos-a-2-smote-enc} y \ref{pairwise-economicos-a-2-tddpm_mlp} que el segundo modelo, Tddpm, presenta una mayor similitud en las variables \emph{rooms} y \emph{bathrooms}.
\begin{figure}[H]
    \centering
    \includesvg[scale=.5,inkscapelatex=false]{datasets/economicos-a-1/pairwise/smote-enc.svg}
    \caption{Correlación de conjunto Real y Modelo: smote-enc}
    \label{pairwise-smote-enc}
\end{figure}
\begin{figure}[H]
    \centering
    \includesvg[scale=.7,inkscapelatex=false]{datasets/kingcounty-a-1/tddpm_mlp/privacy.svg}
    \caption{Frecuencia del campo Privacy en el modelo real y tddpm}
    \label{frecuency-tddpm-privacy}
\end{figure}
\begin{figure}[H]
    \centering
    \includesvg[scale=.7,inkscapelatex=false]{datasets/kingcounty-a-1/tddpm_mlp/bedrooms.svg}
    \caption{Frecuencia del campo Bedrooms en el modelo real y tddpm}
    \label{frecuency-tddpm-bedrooms}
\end{figure}
\begin{figure}[H]
    \centering
    \includesvg[scale=.7,inkscapelatex=false]{datasets/kingcounty-a-1/tddpm_mlp/grade.svg}
    \caption{Frecuencia del campo Grade en el modelo real y tddpm}
    \label{frecuency-tddpm-grade}
\end{figure}
\begin{figure}[H]
    \centering
    \includesvg[scale=.7,inkscapelatex=false]{datasets/kingcounty-a-1/tddpm_mlp/floors.svg}
    \caption{Frecuencia del campo Floors en el modelo real y tddpm, King county (A-1)}
    \label{frecuency-tddpm-floors}
\end{figure}
\begin{figure}[H]
    \centering
    \includesvg[scale=.7,inkscapelatex=false]{datasets/kingcounty-a-2/tddpm_mlp/bathrooms.svg}
    \caption{Frecuencia del campo Bathrooms en el modelo real y tddpm}
    \label{frecuency-tddpm-bathrooms}
\end{figure}
\begin{figure}[H]
    \centering
    \includesvg[scale=.7,inkscapelatex=false]{datasets/kingcounty-a-3/tddpm_mlp/sqft_basement.svg}
    \caption{Frecuencia del campo Sqft basement en el modelo real y tddpm, King county (A-3)}
    \label{frecuency-tddpm-sqft basement}
\end{figure}
\begin{figure}[H]
    \centering
    \includesvg[scale=.7,inkscapelatex=false]{datasets/kingcounty-a-3/tddpm_mlp/sqft_living.svg}
    \caption{Frecuencia del campo Sqft living en el modelo real y tddpm}
    \label{frecuency-tddpm-sqft living}
\end{figure}
\begin{figure}[H]
    \centering
    \includesvg[scale=.7,inkscapelatex=false]{datasets/kingcounty-a-2/tddpm_mlp/waterfront.svg}
    \caption{Frecuencia del campo Waterfront en el modelo real y tddpm}
    \label{frecuency-tddpm-waterfront}
\end{figure}
\begin{figure}[H]
    \centering
    \includesvg[scale=.7,inkscapelatex=false]{datasets/kingcounty-a-3/tddpm_mlp/sqft_lot.svg}
    \caption{Frecuencia del campo Sqft lot en el modelo real y tddpm, King county (A-3)}
    \label{frecuency-tddpm-sqft lot}
\end{figure}
\begin{figure}[H]
    \centering
    \includesvg[scale=.7,inkscapelatex=false]{datasets/kingcounty-a-2/tddpm_mlp/sqft_living15.svg}
    \caption{Frecuencia del campo Sqft living15 en el modelo real y tddpm}
    \label{frecuency-tddpm-sqft living15}
\end{figure}
\begin{figure}[H]
    \centering
    \includesvg[scale=.7,inkscapelatex=false]{datasets/kingcounty-a-1/tddpm_mlp/yr_built.svg}
    \caption{Frecuencia del campo Yr built en el modelo real y tddpm, King county (A-1)}
    \label{frecuency-tddpm-yr built}
\end{figure}
\begin{figure}[H]
    \centering
    \includesvg[scale=.7,inkscapelatex=false]{datasets/kingcounty-a-2/tddpm_mlp/condition.svg}
    \caption{Frecuencia del campo Condition en el modelo real y tddpm, King county (A-2)}
    \label{frecuency-tddpm-condition}
\end{figure}
\begin{figure}[H]
    \centering
    \includesvg[scale=.7,inkscapelatex=false]{datasets/kingcounty-a-3/tddpm_mlp/sqft_above.svg}
    \caption{Frecuencia del campo Sqft above en el modelo real y tddpm, King county (A-3)}
    \label{frecuency-tddpm-sqft above}
\end{figure}
\begin{figure}[H]
    \centering
    \includesvg[scale=.7,inkscapelatex=false]{datasets/kingcounty-a-2/tddpm_mlp/sqft_lot15.svg}
    \caption{Frecuencia del campo Sqft lot15 en el modelo real y tddpm}
    \label{frecuency-tddpm-sqft lot15}
\end{figure}
\begin{figure}[H]
    \centering
    \includesvg[scale=.7,inkscapelatex=false]{datasets/kingcounty-a-1/tddpm_mlp/view.svg}
    \caption{Frecuencia del campo View en el modelo real y tddpm}
    \label{frecuency-tddpm-view}
\end{figure}
\begin{figure}[H]
    \centering
    \includesvg[scale=.7,inkscapelatex=false]{datasets/kingcounty-a-2/tddpm_mlp/price.svg}
    \caption{Frecuencia del campo Price en el modelo real y tddpm, King county (A-2)}
    \label{frecuency-tddpm-price}
\end{figure}



\newpage
\subsection{Reporte diagnóstico - Conjunto A}
En las tablas detalles de cobertura \ref{table-coverage-economicos-a} se puede ver el porqué ambos tenían una puntuación tan baja. Existen elementos con una cobertura menor al 40\%, por ejemplo, la variable \texttt{m\_size}. Aun así, se puede ver que Tddpm es ligeramente mejor en la mayoría de las columnas.
\begin{table}[H]
\centering
\fontsize{10}{14}\selectfont
\caption{Evaluaci\'on de Cobertura Categor{\'\i}a-Rango para Modelos SMOTE-ENC y TDDPM\_MLP, Economicos}
\label{table-coverage-economicos-a}
\begin{tabular}{|l|l|r|r|}
\hline
\rowcolor[gray]{0.8}
Columna & Metrica & smote-enc & tddpm\_mlp \\
\hline \_price & RangeCoverage & \bfseries 9.68e-01±5.48e-02 & 9.66e-01±3.30e-02 \\
\hline bathrooms & CategoryCoverage & \bfseries 8.63e-01±3.40e-02 & 6.76e-01±2.94e-02 \\
\hline county & CategoryCoverage & 5.97e-01±3.73e-03 & \bfseries 7.87e-01±2.27e-02 \\
\hline m\_built & RangeCoverage & 5.52e-01±3.16e-01 & \bfseries 7.71e-01±3.97e-01 \\
\hline m\_size & RangeCoverage & \cellcolor[rgb]{0.9, 0.54, 0.52} 1.79e-02±8.52e-03 & \cellcolor[rgb]{0.9, 0.54, 0.52} \bfseries 3.36e-01±4.53e-02 \\
\hline property\_type & CategoryCoverage & 6.67e-01±5.56e-02 & \bfseries 9.07e-01±3.21e-02 \\
\hline publication\_date & RangeCoverage & 9.70e-01±5.80e-03 & \bfseries 9.81e-01±2.86e-03 \\
\hline rooms & CategoryCoverage & 7.40e-01±1.41e-02 & \bfseries 7.80e-01±6.45e-02 \\
\hline state & CategoryCoverage & 7.92e-01±3.61e-02 & \bfseries 9.58e-01±3.61e-02 \\
\hline transaction\_type & CategoryCoverage & 5.00e-01±0.00e+00 & \bfseries 7.50e-01±2.50e-01 \\
\hline
\end{tabular}
\end{table}

La escasa cobertura en m\_size podría atribuirse a su distribución. Como se ilustra en la figura \ref{frecuency-M Size-top2}, esta presenta una larga cola, caracterizada por valores altos pero infrecuentes.
\begin{figure}[H]
    \centering
    \includesvg[scale=.7,inkscapelatex=false]{datasets/economicos-b-1/smote-enc/m_size.svg}
    \caption{Frecuencia del campo M size en el modelo real y smote}
    \label{frecuency-M Size-smote-enc}
\end{figure}
\newpage
\subsection{Reporte de calidad - Conjunto A}
Ambos modelos muestran un buen rendimiento en cuanto a la forma y la distribución de los datos, como se evidencia en la Tabla \ref{table-shape-economicos-a}. Como se vio en la Figura  \ref{frecuency-M Size-top2} una buena distribución no asegura una cobertura completa.
\begin{table}[H]
\centering
\caption{Evaluación de Similitud de Distribución para Modelos Smote y Tddpm, Economicos}
\label{table-shape-economicos-a}
\begin{tabular}{|l|l|r|r|}
\hline
\rowcolor[gray]{0.8}
Columna & Metrica & smote-enc & tddpm\_mlp \\
\hline \_price & KSComplement & \bfseries 9.91e-01±3.85e-04 & 9.84e-01±3.53e-03 \\
\hline bathrooms & TVComplement & \bfseries 9.94e-01±6.66e-04 & 9.87e-01±2.15e-03 \\
\hline county & TVComplement & \cellcolor[rgb]{0.9, 0.54, 0.52} 9.22e-01±9.28e-04 & \bfseries \cellcolor[rgb]{0.9, 0.54, 0.52} 9.66e-01±2.10e-03 \\
\hline m\_built & KSComplement & \bfseries 9.87e-01±2.14e-03 & \bfseries 9.87e-01±1.11e-03 \\
\hline m\_size & KSComplement & 9.72e-01±7.43e-04 & \bfseries 9.84e-01±3.22e-03 \\
\hline property\_type & TVComplement & 9.67e-01±1.33e-03 & \bfseries 9.82e-01±9.49e-04 \\
\hline publication\_date & KSComplement & 9.80e-01±1.61e-03 & \bfseries 9.85e-01±1.61e-03 \\
\hline rooms & TVComplement & 9.77e-01±2.28e-03 & \bfseries 9.81e-01±3.18e-03 \\
\hline state & TVComplement & 9.69e-01±4.29e-04 & \bfseries 9.83e-01±1.05e-03 \\
\hline transaction\_type & TVComplement & \bfseries 9.98e-01±1.07e-03 & 9.93e-01±3.54e-03 \\
\hline
\end{tabular}
\end{table}



\newpage
\subsection{Privacidad - Conjunto A}
Resulta interesante notar que, para el percentil 1 y el 5, en las Tablas \ref{table-dcr-economicos-a-1th} y \ref{table-dcr-economicos-a-5th} respectivamente, el modelo Tddpm demuestra que la cercanía de los registros más próximos es predominante al comparar el conjunto sintético con el conjunto de retención (\emph{Hold}). Este fenómeno no se evidencia en ninguna otra comparación. Asimismo, se destaca que las diferencias mínimas llegan a cero en los dos modelos más efectivos (Tddpm y Smote), y que los valores de distancia son extremadamente reducidos. Para el percentil 5, Tddpm registra una distancia de $4.48 \times 10^{-9}$.
\begin{table}[H]
\centering
\fontsize{10}{14}\selectfont
\caption{Distancia de registros más cercanos entre conjuntos Sinteticos, percentil 5, Economicos}
\label{table-dcr-economicos-a-5th}
\begin{tabular}{|l|l|r|r|r|r|}
\hline
\rowcolor[gray]{0.8}
Modelo & DCR ST & DCR SH & DCR TH & \textbf{Score} \\
\hline tddpm\_mlp & 4.29e-09±2.16e-10 & \cellcolor[rgb]{0.9, 0.54, 0.52} 3.50e-08±1.92e-09 & \bfseries \cellcolor[rgb]{0.9, 0.54, 0.52} 1.28e-08±0.00e+00 & \bfseries 9.77e-01±6.88e-04 \\
\hline smote-enc & \cellcolor[rgb]{0.9, 0.54, 0.52} 2.90e-11±1.13e-12 & 4.41e-08±2.36e-09 & \bfseries \cellcolor[rgb]{0.9, 0.54, 0.52} 1.28e-08±0.00e+00 & 9.67e-01±8.19e-04 \\
\hline ctgan & \bfseries 7.59e-06±5.75e-06 & \bfseries 1.91e-05±2.01e-05 & \bfseries \cellcolor[rgb]{0.9, 0.54, 0.52} 1.28e-08±0.00e+00 & 6.96e-01±1.00e-02 \\
\hline copulagan & 1.27e-06±3.04e-07 & 2.73e-06±5.89e-07 & \bfseries \cellcolor[rgb]{0.9, 0.54, 0.52} 1.28e-08±0.00e+00 & 7.81e-01±2.03e-02 \\
\hline gaussiancopula & 5.11e-06±0.00e+00 & 8.25e-06±0.00e+00 & \bfseries \cellcolor[rgb]{0.9, 0.54, 0.52} 1.28e-08±0.00e+00 & 6.91e-01±6.41e-17 \\
\hline tvae & 2.19e-07±1.60e-09 & 4.15e-07±5.43e-09 & \bfseries \cellcolor[rgb]{0.9, 0.54, 0.52} 1.28e-08±0.00e+00 & \cellcolor[rgb]{0.9, 0.54, 0.52} 6.40e-01±3.35e-03 \\
\hline
\end{tabular}
\end{table}

\begin{table}[H]
\centering
\fontsize{10}{14}\selectfont
\caption{Distancia de registros más cercanos, percentil 1, datos economicos}
\label{table-dcr-economicos-a-1th}
\begin{tabular}{|l|l|r|r|r|r|r|r|r|}
\hline
\rowcolor[gray]{0.8}
Modelo & DCR ST & DCR SH & DCR TH & \textbf{Score} \\
\hline tddpm\_mlp & 1.46e-10±3.86e-12 & \cellcolor[rgb]{0.9, 0.54, 0.52} 1.44e-09±1.01e-10 & 0.00e+00±0.00e+00 & \bfseries 9.79e-01±1.27e-03 \\
\hline smote-enc & \cellcolor[rgb]{0.9, 0.54, 0.52} 0.00e+00±0.00e+00 & 1.54e-09±5.32e-13 & 0.00e+00±0.00e+00 & 9.69e-01±6.71e-04 \\
\hline copulagan & 1.97e-07±4.64e-08 & 4.53e-07±9.95e-08 & 0.00e+00±0.00e+00 & 7.68e-01±2.96e-02 \\
\hline ctgan & \bfseries 3.18e-06±4.34e-07 & \bfseries 5.23e-06±1.44e-06 & 0.00e+00±0.00e+00 & 6.98e-01±2.63e-02 \\
\hline gaussiancopula & 7.84e-07±7.49e-23 & 1.75e-06±0.00e+00 & 0.00e+00±0.00e+00 & 6.92e-01±0.00e+00 \\
\hline tvae & 1.48e-07±9.24e-08 & 2.35e-07±1.18e-07 & 0.00e+00±0.00e+00 & \cellcolor[rgb]{0.9, 0.54, 0.52} 6.12e-01±2.50e-02 \\
\hline
\end{tabular}
\end{table}

\begin{table}[H]
\centering
\fontsize{10}{14}\selectfont
\caption{Distancia de registros más cercanos, minimo, datos economicos}
\label{table-dcr-economicos-a}
\begin{tabular}{|l|l|r|r|r|r|r|r|r|}
\hline
\rowcolor[gray]{0.8}
Modelo & DCR ST & DCR SH & DCR TH & \textbf{Score} \\
\hline tddpm\_mlp & \cellcolor[rgb]{0.9, 0.54, 0.52} 0.00e+00±0.00e+00 & \cellcolor[rgb]{0.9, 0.54, 0.52} 0.00e+00±0.00e+00 & 0.00e+00±0.00e+00 & \bfseries 9.79e-01±1.27e-03 \\
\hline smote-enc & \cellcolor[rgb]{0.9, 0.54, 0.52} 0.00e+00±0.00e+00 & \cellcolor[rgb]{0.9, 0.54, 0.52} 0.00e+00±0.00e+00 & 0.00e+00±0.00e+00 & 9.69e-01±6.71e-04 \\
\hline copulagan & 5.88e-09±2.05e-09 & 1.21e-08±3.19e-09 & 0.00e+00±0.00e+00 & 7.68e-01±2.96e-02 \\
\hline ctgan & \bfseries 2.83e-08±3.88e-08 & \bfseries 6.05e-08±2.56e-08 & 0.00e+00±0.00e+00 & 6.98e-01±2.63e-02 \\
\hline gaussiancopula & 1.13e-08±0.00e+00 & 1.75e-08±0.00e+00 & 0.00e+00±0.00e+00 & 6.92e-01±0.00e+00 \\
\hline tvae & 5.65e-09±3.07e-09 & 2.56e-08±3.04e-08 & 0.00e+00±0.00e+00 & \cellcolor[rgb]{0.9, 0.54, 0.52} 6.12e-01±2.50e-02 \\
\hline
\end{tabular}
\end{table}


\newpage
También se puede observar una disminución en la relación entre el registro más cercano y el segundo más cercano en comparación con el conjunto de datos de King County. En el percentil 5, el segundo registro más cercano está a 15 veces la distancia del primero. Esta relación se reduce a 10 veces cuando se compara con el conjunto \emph{Hold}.

\begin{table}[H]
\centering
\fontsize{10}{14}\selectfont
\caption{Proporción entre el más cercano y el segundo más cercano, percentil 5, Economicos}
\label{table-nndr-economicos-a-5th}
\begin{tabular}{|l|l|r|r|r|r|}
\hline
\rowcolor[gray]{0.8}
Modelo & NNDR ST & NNDR SH & NNDR TH & \textbf{Score} \\
\hline tddpm\_mlp & 6.79e-02±7.37e-04 & \bfseries 1.00e-01±2.26e-03 & \cellcolor[rgb]{0.9, 0.54, 0.52} \bfseries 1.31e-02±0.00e+00 & \cellcolor[rgb]{0.9, 0.54, 0.52} 9.77e-01±6.88e-04 \\
\hline smote-enc & \bfseries 7.15e-04±7.49e-06 & 1.14e-01±4.79e-03 & 1.31e-02±0.00e+00 & 9.67e-01±8.19e-04 \\
\hline ctgan & 2.57e-01±8.81e-03 & 3.27e-01±4.72e-02 & 1.31e-02±0.00e+00 & 6.96e-01±1.00e-02 \\
\hline copulagan & 2.01e-01±1.27e-02 & 2.23e-01±5.47e-02 & 1.31e-02±0.00e+00 & 7.81e-01±2.03e-02 \\
\hline gaussiancopula & \cellcolor[rgb]{0.9, 0.54, 0.52} 3.07e-01±0.00e+00 & 2.76e-01±0.00e+00 & 1.31e-02±0.00e+00 & 6.91e-01±6.41e-17 \\
\hline tvae & 3.02e-01±6.15e-03 & \cellcolor[rgb]{0.9, 0.54, 0.52} 3.49e-01±1.26e-03 & 1.31e-02±0.00e+00 & \bfseries 6.40e-01±3.35e-03 \\
\hline
\end{tabular}
\end{table}

\begin{table}[H]
\centering
\fontsize{10}{14}\selectfont
\caption{Proporción entre el más cercano y el segundo más cercano, percentil 1, Economicos}
\label{table-nndr-economicos-a-1th}
\begin{tabular}{|l|l|r|r|r|r|}
\hline
\rowcolor[gray]{0.8}
Modelo & NNDR ST & NNDR SH & NNDR TH & \textbf{Score} \\
\hline tddpm\_mlp & 2.52e-03±1.71e-04 & 9.61e-03±1.17e-04 & \cellcolor[rgb]{0.9, 0.54, 0.52} \bfseries 0.00e+00±0.00e+00 & \cellcolor[rgb]{0.9, 0.54, 0.52} 9.77e-01±6.88e-04 \\
\hline smote-enc & \bfseries 0.00e+00±0.00e+00 & \bfseries 3.00e-03±1.28e-03 & 0.00e+00±0.00e+00 & 9.67e-01±8.19e-04 \\
\hline ctgan & 1.35e-02±2.69e-03 & 2.77e-02±1.69e-02 & 0.00e+00±0.00e+00 & 6.96e-01±1.00e-02 \\
\hline copulagan & 1.03e-02±1.29e-03 & 1.11e-02±2.64e-03 & 0.00e+00±0.00e+00 & 7.81e-01±2.03e-02 \\
\hline gaussiancopula & 2.90e-02±0.00e+00 & 2.94e-02±3.47e-18 & 0.00e+00±0.00e+00 & 6.91e-01±6.41e-17 \\
\hline tvae & \cellcolor[rgb]{0.9, 0.54, 0.52} 3.26e-02±1.29e-02 & \cellcolor[rgb]{0.9, 0.54, 0.52} 1.45e-01±1.91e-03 & 0.00e+00±0.00e+00 & \bfseries 6.40e-01±3.35e-03 \\
\hline
\end{tabular}
\end{table}

\begin{table}[H]
\centering
\fontsize{10}{14}\selectfont
\caption{Proporción entre el más cercano y el segundo más cercano, minimo, datos economicos}
\label{table-dcr-economicos-a}
\begin{tabular}{|l|l|r|r|r|r|r|r|r|}
\hline
\rowcolor[gray]{0.8}
Modelo & NNDR ST & NNDR SH & NNDR TH & \textbf{Score} \\
\hline tddpm\_mlp & \cellcolor[rgb]{0.9, 0.54, 0.52} 0.00e+00±0.00e+00 & \cellcolor[rgb]{0.9, 0.54, 0.52} 0.00e+00±0.00e+00 & 0.00e+00±0.00e+00 & \bfseries 9.79e-01±1.27e-03 \\
\hline smote-enc & \cellcolor[rgb]{0.9, 0.54, 0.52} 0.00e+00±0.00e+00 & \cellcolor[rgb]{0.9, 0.54, 0.52} 0.00e+00±0.00e+00 & 0.00e+00±0.00e+00 & 9.69e-01±6.71e-04 \\
\hline copulagan & 1.22e-04±7.03e-05 & 1.84e-04±1.12e-04 & 0.00e+00±0.00e+00 & 7.68e-01±2.96e-02 \\
\hline ctgan & 4.21e-04±2.19e-04 & 1.32e-03±1.54e-03 & 0.00e+00±0.00e+00 & 6.98e-01±2.63e-02 \\
\hline gaussiancopula & 4.99e-05±0.00e+00 & 7.59e-06±8.47e-22 & 0.00e+00±0.00e+00 & 6.92e-01±0.00e+00 \\
\hline tvae & \bfseries 8.11e-04±1.77e-04 & \bfseries 7.24e-03±3.14e-03 & 0.00e+00±0.00e+00 & \cellcolor[rgb]{0.9, 0.54, 0.52} 6.12e-01±2.50e-02 \\
\hline
\end{tabular}
\end{table}

\newpage


\subsection{Ejemplos de registros - Conjunto A}
Es fácil entender que la implicancia de un \emph{DCR} igual a 0 es un registro copiado desde el conjunto real, esto se puede apreciar en la Tabla \ref{table-example-economicos-a-2-tddpm_mlp-min}.
\begin{table}[H]
\centering
\fontsize{10}{14}\selectfont
\caption{Ejemplos para el modelo Tddpm, minimo, Economicos (A-2)}
\label{table-example-economicos-a-2-tddpm_mlp-min}
\begin{tabular}{|l|r|r|r|}
\hline
\rowcolor[gray]{0.8}
Variable/Distancia & Sintético & DCR1 d(0.00e+00) & DCR2 d(2.26e-07) \\
\hline \_price & \cellcolor[rgb]{0.9, 0.54, 0.52} 2859.676636 & \cellcolor[rgb]{0.9, 0.54, 0.52} 2859.676636 & 2800.000000 \\
\hline bathrooms & \cellcolor[rgb]{0.9, 0.54, 0.52} 2.000000 & \cellcolor[rgb]{0.9, 0.54, 0.52} 2.000000 & \cellcolor[rgb]{0.9, 0.54, 0.52} 2.000000 \\
\hline county & \cellcolor[rgb]{0.9, 0.54, 0.52} Los Vilos & \cellcolor[rgb]{0.9, 0.54, 0.52} Los Vilos & Ovalle \\
\hline m\_built & \cellcolor[rgb]{0.9, 0.54, 0.52} 120.000000 & \cellcolor[rgb]{0.9, 0.54, 0.52} 120.000000 & 122.000000 \\
\hline m\_size & \cellcolor[rgb]{0.9, 0.54, 0.52} 300.000000 & \cellcolor[rgb]{0.9, 0.54, 0.52} 300.000000 & 197.000000 \\
\hline property\_type & \cellcolor[rgb]{0.9, 0.54, 0.52} Casa & \cellcolor[rgb]{0.9, 0.54, 0.52} Casa & \cellcolor[rgb]{0.9, 0.54, 0.52} Casa \\
\hline publication\_date & \cellcolor[rgb]{0.9, 0.54, 0.52} 1545.000000 & \cellcolor[rgb]{0.9, 0.54, 0.52} 1545.000000 & \cellcolor[rgb]{0.9, 0.54, 0.52} 1545.000000 \\
\hline rooms & \cellcolor[rgb]{0.9, 0.54, 0.52} 4.000000 & \cellcolor[rgb]{0.9, 0.54, 0.52} 4.000000 & \cellcolor[rgb]{0.9, 0.54, 0.52} 4.000000 \\
\hline state & \cellcolor[rgb]{0.9, 0.54, 0.52} Coquimbo & \cellcolor[rgb]{0.9, 0.54, 0.52} Coquimbo & \cellcolor[rgb]{0.9, 0.54, 0.52} Coquimbo \\
\hline transaction\_type & \cellcolor[rgb]{0.9, 0.54, 0.52} Venta & \cellcolor[rgb]{0.9, 0.54, 0.52} Venta & \cellcolor[rgb]{0.9, 0.54, 0.52} Venta \\
\hline
\end{tabular}
\end{table}

Ya cuando se observa el percentil 1, se puede apreciar que la diferencia se puede considerar significativa. En el caso mostrado por la Tabla \ref{table-example-economicos-a-2-tddpm_mlp-1p}, los metros cuadrados (\emph{m\_size}) y \emph{\_price} cambian y luego la variable \emph{county} también cambian en el segundo registro más cercano.
\begin{table}[H]
\centering
\fontsize{10}{14}\selectfont
\caption{Ejemplos para el modelo Tddpm, percentil 1, Economicos (A-2)}
\label{table-example-economicos-a-2-tddpm_mlp-1p}
\begin{tabular}{|l|r|r|r|}
\hline
\rowcolor[gray]{0.8}
Variable/Distancia & Sintético & DCR1 d(1.42e-10) & DCR2 d(3.93e-09) \\
\hline \_price & \cellcolor[rgb]{0.9, 0.54, 0.52} 10.747760 & 10.803223 & 9.214514 \\
\hline bathrooms & \cellcolor[rgb]{0.9, 0.54, 0.52} 1.000000 & \cellcolor[rgb]{0.9, 0.54, 0.52} 1.000000 & \cellcolor[rgb]{0.9, 0.54, 0.52} 1.000000 \\
\hline county & \cellcolor[rgb]{0.9, 0.54, 0.52} San Miguel & \cellcolor[rgb]{0.9, 0.54, 0.52} San Miguel & Estación Central \\
\hline m\_built & \cellcolor[rgb]{0.9, 0.54, 0.52} 34.000000 & \cellcolor[rgb]{0.9, 0.54, 0.52} 34.000000 & \cellcolor[rgb]{0.9, 0.54, 0.52} 34.000000 \\
\hline m\_size & \cellcolor[rgb]{0.9, 0.54, 0.52} 35.685290 & 36.000000 & 36.000000 \\
\hline property\_type & \cellcolor[rgb]{0.9, 0.54, 0.52} Departamento & \cellcolor[rgb]{0.9, 0.54, 0.52} Departamento & \cellcolor[rgb]{0.9, 0.54, 0.52} Departamento \\
\hline publication\_date & \cellcolor[rgb]{0.9, 0.54, 0.52} 1545.000000 & \cellcolor[rgb]{0.9, 0.54, 0.52} 1545.000000 & \cellcolor[rgb]{0.9, 0.54, 0.52} 1545.000000 \\
\hline rooms & \cellcolor[rgb]{0.9, 0.54, 0.52} 1.000000 & \cellcolor[rgb]{0.9, 0.54, 0.52} 1.000000 & \cellcolor[rgb]{0.9, 0.54, 0.52} 1.000000 \\
\hline state & \cellcolor[rgb]{0.9, 0.54, 0.52} Metropolitana de Santiago & \cellcolor[rgb]{0.9, 0.54, 0.52} Metropolitana de Santiago & \cellcolor[rgb]{0.9, 0.54, 0.52} Metropolitana de Santiago \\
\hline transaction\_type & \cellcolor[rgb]{0.9, 0.54, 0.52} Arriendo & \cellcolor[rgb]{0.9, 0.54, 0.52} Arriendo & \cellcolor[rgb]{0.9, 0.54, 0.52} Arriendo \\
\hline
\end{tabular}
\end{table}

\newpage
En las Tablas \ref{table-example-economicos-a-1-tddpm_mlp-4p} y \ref{table-example-economicos-a-1-tddpm_mlp-4p-text} se puede observar un registro con coherencia simulada. Por ejemplo, cuando decide generar un número de teléfono, este parece coherente. También menciona que está cerca de un metro, detalle que podría estar presente en una publicación real, a pesar de que el metro indicado no exista.
\begin{table}[H]
\centering
\fontsize{10}{14}\selectfont
\caption{Ejemplos para el modelo Tddpm, percentil 4, Economicos (A-1)}
\label{table-example-economicos-a-1-tddpm_mlp-4p}
\begin{tabular}{|l|r|r|r|}
\hline
\rowcolor[gray]{0.8}
Variable/Distancia & Sintético & DCR1 d(2.84e-09) & DCR2 d(6.74e-09) \\
\hline \_price & \cellcolor[rgb]{0.9, 0.54, 0.52} 11.478674 & 10.485481 & 8.896772 \\
\hline bathrooms & \cellcolor[rgb]{0.9, 0.54, 0.52} 1.000000 & \cellcolor[rgb]{0.9, 0.54, 0.52} 1.000000 & \cellcolor[rgb]{0.9, 0.54, 0.52} 1.000000 \\
\hline county & \cellcolor[rgb]{0.9, 0.54, 0.52} Santiago & \cellcolor[rgb]{0.9, 0.54, 0.52} Santiago & \cellcolor[rgb]{0.9, 0.54, 0.52} Santiago \\
\hline m\_built & \cellcolor[rgb]{0.9, 0.54, 0.52} 38.984909 & 39.000000 & 39.000000 \\
\hline m\_size & \cellcolor[rgb]{0.9, 0.54, 0.52} 45.000000 & 43.000000 & 39.000000 \\
\hline property\_type & \cellcolor[rgb]{0.9, 0.54, 0.52} Departamento & \cellcolor[rgb]{0.9, 0.54, 0.52} Departamento & \cellcolor[rgb]{0.9, 0.54, 0.52} Departamento \\
\hline publication\_date & \cellcolor[rgb]{0.9, 0.54, 0.52} 1545.000000 & \cellcolor[rgb]{0.9, 0.54, 0.52} 1545.000000 & \cellcolor[rgb]{0.9, 0.54, 0.52} 1545.000000 \\
\hline rooms & \cellcolor[rgb]{0.9, 0.54, 0.52} 1.000000 & \cellcolor[rgb]{0.9, 0.54, 0.52} 1.000000 & \cellcolor[rgb]{0.9, 0.54, 0.52} 1.000000 \\
\hline state & \cellcolor[rgb]{0.9, 0.54, 0.52} Metropolitana de Santiago & \cellcolor[rgb]{0.9, 0.54, 0.52} Metropolitana de Santiago & \cellcolor[rgb]{0.9, 0.54, 0.52} Metropolitana de Santiago \\
\hline transaction\_type & \cellcolor[rgb]{0.9, 0.54, 0.52} Arriendo & \cellcolor[rgb]{0.9, 0.54, 0.52} Arriendo & \cellcolor[rgb]{0.9, 0.54, 0.52} Arriendo \\
\hline
\end{tabular}
\end{table}

\begin{table}[H]
\centering
\fontsize{10}{14}\selectfont
\caption{Ejemplos de texto modelo Tddpm, percentil 4, Economicos (A-1)}
\label{table-example-economicos-a-1-tddpm_mlp-4p-text}
\begin{tabular}{|l|m{35em}|}
\hline
\rowcolor[gray]{0.8}
Distancia & description \\
\hline Sintético & Departamento de 2 pisos, 1 ba\~no con vista panor\'amica al centro comercial en Santiago del sector central (centros), cercano a metro Metro San Luis). El edificio cuenta con un gran living comedor que comparte una terraza completamente equipada por lavander{\'\i}a para dos veh{\'\i}culos; el primer piso: Living comedor amplio como sala de estar/cocina amoblada cerrada cubierta de granito cocina techada integrada entre todos los dormitorios m\'as amplias estacionamiento principal se entregan inmediatamente despu\'es de las ventanas termopaneles +56 9 6 7 8 5 4 3 \\
\hline DCR1 d(2.84e-09) & C\'omodo departamento amoblado cercano al metro Santa Lucia  [DISPONIBLE DESDE EL 01 DE MARZO 2022]  CARACTERISTICAS - 39/43 m2 - 1 Dormitorio - 1 Ba\~no - Sin Estacionamiento - Sin Bodega  OTROS - Piso 7 - Orientaci\'on Poniente - Conexi\'on a lavadora - Se aceptan mascotas - Full electric  - Piso flotante - Ventanas Termopanel  CARACTERISTICAS DEL EDIFICIO Edificio cuenta con sala multiuso, gimnasio, lavanderia, bicicletero y un amplio patio interior  Arrienda CUALQUIERA de nuestras propiedades y paga todos los gastos iniciales del arriendo EN CUOTAS con tu tarjeta de cr\'edito - Primer mes de arriendo o proporcional - Mes de garant{\'\i}a - Comisi\'on corredor \\
\hline DCR2 d(6.74e-09) & Departamento de un dormitorio, en San Diego cercano a Avenida Matta y a Pedro Lagos, el metro m\'as cercano es Parque O'higgins, el cual est\'a a 15 minutos caminando, cercano a Universidad De Chile, Universidad Bernardo Ohiggins. Contacto Nicol\'as Ib\'a\~nez Le\'on+569 6577 1999. \\
\hline
\end{tabular}
\end{table}


\newpage
\subsection{Propiedades estadísticas - Conjunto A}
El listado completo de las propiedades estadísticas se encuentra en el Anexo \ref{propiedades-estadisticas-economicos-A}. A continuación, se presentan las propiedades estadísticas en las que los modelos Tddpm y Smote muestran una diferencia mayor al 5\% con respecto al conjunto original de entrenamiento. Como referencia, se incluye el modelo Ctgan. Las variables se seleccionaron por ser 1) las que obtuvieron el peor resultado en cobertura y 2) las que obtuvieron el peor resultado en la distribución, respectivamente.

\begin{table}[H]
\centering
\fontsize{8}{14}\selectfont
\caption{Propiedades estadisticas de variable m\_size con cambio\ensuremath{>}5\%, Economicos (A-1)}
\label{table-stats-economicos-a-1-m_size-short}
\begin{tabular}{|l|m{10em}|m{10em}|m{10em}|m{10em}|}
\hline
 \rowcolor[gray]{0.8}
Variable/Modelo & Real & tddpm\_mlp & smote-enc & ctgan \\
\hline nobs & 22059 & 27574 & 27574 & 27574 \\
\hline mean & 146269 & \cellcolor[rgb]{0.9, 0.54, 0.52} 1875594 & 73666 & \bfseries 138617 \\
\hline std\_err & 105454 & \cellcolor[rgb]{0.9, 0.54, 0.52} 879088 & \bfseries 46661 & 553 \\
\hline upper\_ci & 352956 & \cellcolor[rgb]{0.9, 0.54, 0.52} 3598574 & \bfseries 165120 & 139702 \\
\hline lower\_ci & -60417 & \cellcolor[rgb]{0.9, 0.54, 0.52} 152614 & \bfseries -17788 & 137532 \\
\hline std & 15662334 & \cellcolor[rgb]{0.9, 0.54, 0.52} 145976185 & \bfseries 7748272 & 91900 \\
\hline iqr & 340.500 & \bfseries 322.722 & 365.708 & \cellcolor[rgb]{0.9, 0.54, 0.52} 144556.805 \\
\hline iqr\_normal & 252.413 & \bfseries 239.234 & 271.100 & \cellcolor[rgb]{0.9, 0.54, 0.52} 107160.120 \\
\hline mad & 290635 & \cellcolor[rgb]{0.9, 0.54, 0.52} 3749106 & \bfseries 145665 & 77299 \\
\hline mad\_normal & 364257 & \cellcolor[rgb]{0.9, 0.54, 0.52} 4698808 & \bfseries 182564 & 96880 \\
\hline coef\_var & 107.079 & 77.829 & \bfseries 105.181 & \cellcolor[rgb]{0.9, 0.54, 0.52} 0.663 \\
\hline range & 2.24100e+09 & \cellcolor[rgb]{0.9, 0.54, 0.52} 1.54249e+10 & \bfseries 1.19075e+09 & 3.92397e+05 \\
\hline max & 2.24100e+09 & \cellcolor[rgb]{0.9, 0.54, 0.52} 1.54249e+10 & \bfseries 1.19075e+09 & 3.92397e+05 \\
\hline min & 0.000 & 0.440 & \cellcolor[rgb]{0.9, 0.54, 0.52} 1.546 & \bfseries 0.000 \\
\hline skew & 134.762 & 83.944 & \bfseries 137.694 & \cellcolor[rgb]{0.9, 0.54, 0.52} 0.154 \\
\hline kurtosis & 19053 & 7466 & \bfseries 20489 & \cellcolor[rgb]{0.9, 0.54, 0.52} 2 \\
\hline jarque\_bera & 3.33616e+11 & 6.40295e+10 & \bfseries 4.82276e+11 & \cellcolor[rgb]{0.9, 0.54, 0.52} 1.02812e+03 \\
\hline mode\_freq & 0.027 & \bfseries 0.027 & 0.006 & \cellcolor[rgb]{0.9, 0.54, 0.52} 0.104 \\
\hline median & 145.000 & \bfseries 144.000 & 149.683 & \cellcolor[rgb]{0.9, 0.54, 0.52} 137376.175 \\
\hline 0.1\% & 2.000 & 4.095 & \cellcolor[rgb]{0.9, 0.54, 0.52} 15.010 & \bfseries 0.000 \\
\hline 1.0\% & 22.000 & \bfseries 23.844 & 25.000 & \cellcolor[rgb]{0.9, 0.54, 0.52} 0.000 \\
\hline 25.0\% & 66.000 & 68.000 & \bfseries 67.667 & \cellcolor[rgb]{0.9, 0.54, 0.52} 63804.820 \\
\hline 75.0\% & 406.500 & \bfseries 390.722 & 433.375 & \cellcolor[rgb]{0.9, 0.54, 0.52} 208361.625 \\
\hline 95.0\% & 5000 & \bfseries 5000 & 4313 & \cellcolor[rgb]{0.9, 0.54, 0.52} 292812 \\
\hline 99.0\% & 10200 & 7698 & \bfseries 9396 & \cellcolor[rgb]{0.9, 0.54, 0.52} 334157 \\
\hline 99.9\% & 70000 & 34028 & \bfseries 53878 & \cellcolor[rgb]{0.9, 0.54, 0.52} 364825 \\
\hline
\end{tabular}
\end{table}

\begin{table}[H]
\centering
\fontsize{8}{14}\selectfont
\caption{Propiedades  estadisticas de variable county, Economicos (A-1)}
\label{table-stats-economicos-a-1-county}
\begin{tabular}{|l|m{10em}|m{10em}|m{10em}|m{10em}|}
\hline
 \rowcolor[gray]{0.8}
Variable/Modelo & Real & tddpm\_mlp & smote-enc & ctgan \\
\hline top5 & ['Las Condes' 'Santiago' 'Providencia' 'Vitacura' 'Lo Barnechea'] & ['Las Condes' 'Santiago' 'Providencia' 'Vitacura' 'Lo Barnechea'] & ['Las Condes' 'Santiago' 'Providencia' 'Vitacura' 'Lo Barnechea'] & ['Las Condes' 'Santiago' 'Viña del Mar' 'Vitacura' 'Providencia'] \\
\hline top5\_freq & [3233 2703 1481 1415 1322] & [4149 3211 1910 1871 1740] & [4662 3729 1948 1843 1815] & [3808 3764 2308 1679 1355] \\
\hline top5\_prob & [0.14656149 0.12253502 0.06713813 0.06414615 0.05993019] & [0.15046783 0.11645028 0.06926815 0.06785378 0.06310292] & [0.16907231 0.13523609 0.07064626 0.06683833 0.06582288] & [0.13810111 0.1365054  0.08370204 0.06089069 0.04914049] \\
\hline nobs & 22059 & 27574 & 27574 & 27574 \\
\hline missing & 22059 & 0 & 0 & 0 \\
\hline
\end{tabular}
\end{table}


\newpage
\subsection{SDMetrics Score - Conjunto B}
\label{ds-conjunto-b}
Iniciaría contrastando los resultados entre ambos conjuntos para el modelo Tddpm La Tabla \ref{table-score-economicos-b} muestra mejores \emph{Score}, \emph{Coverage}, \emph{Column Shape} y \emph{Column Pair Trends} comparadas con la Tabla \ref{table-score-economicos-a}. Puede deverse que al ser una cantidad de datos mayor, pudo tener más tiempo de aprender la distribución. Mejoria no notoria en los demás modelos, lo que podría indicar una mayor capacidad de Tddpm. Se puede ver que la cobertura es el indicador más bajo, solo alcanzando el 87\% en el mejor de los casos.

\begin{table}[H]
\centering
\fontsize{7}{14}\selectfont
\caption{Evaluaci\'on de M\'etricas de Rendimiento para Diversos Modelos de Aprendizaje Autom\'atico, Economicos}
\label{table-score-economicos-b}
\begin{tabular}{|l|r|r|r|r|r|r|}
\hline
\rowcolor[gray]{0.8}
Model Name & Column Pair Trends & Column Shapes & Coverage & Boundaries & Synthesis & \textbf{Score} \\
\hline tddpm\_mlp & \bfseries 9.78e-01±2.79e-03 & \bfseries 9.91e-01±1.71e-03 & \bfseries 8.74e-01±3.37e-03 & \bfseries 1.00e+00±0.00e+00 & 9.71e-01±2.17e-03 & \bfseries 9.84e-01±1.85e-03 \\
\hline smote-enc & 9.65e-01±1.01e-03 & 9.20e-01±1.07e-04 & 7.04e-01±3.34e-02 & \bfseries 1.00e+00±0.00e+00 & 9.31e-01±3.00e-03 & 9.43e-01±4.67e-04 \\
\hline copulagan & 7.67e-01±2.32e-02 & 7.81e-01±1.75e-02 & 6.33e-01±5.61e-04 & \bfseries 1.00e+00±0.00e+00 & \bfseries 1.00e+00±0.00e+00 & 7.74e-01±2.02e-02 \\
\hline tvae & 7.77e-01±1.68e-02 & 7.00e-01±1.76e-02 & 2.77e-01±3.70e-03 & \bfseries 1.00e+00±0.00e+00 & \bfseries 1.00e+00±0.00e+00 & 7.38e-01±1.48e-02 \\
\hline ctgan & 7.72e-01±1.35e-02 & 6.96e-01±8.58e-03 & 6.32e-01±9.52e-04 & \bfseries 1.00e+00±0.00e+00 & \bfseries 1.00e+00±0.00e+00 & 7.34e-01±5.42e-03 \\
\hline gaussiancopula & 6.32e-01±0.00e+00 & 6.30e-01±7.85e-17 & 5.63e-01±0.00e+00 & \bfseries 1.00e+00±0.00e+00 & \bfseries 1.00e+00±0.00e+00 & 6.31e-01±0.00e+00 \\
\hline
\end{tabular}
\end{table}


\newpage
\subsection{Correlación - Conjunto B}
Los modelos Smote y Tddpm, al ser comparados con el conjunto original, presentan diferencias marcadas. Los conjuntos sintéticos han creado correlaciones que no se ven presentes en los datos originales. En el caso del modelo Smote, se presentan correlaciones en las variables \emph{bathrooms}-\emph{rooms}, \emph{m\_size}-\emph{m\_built}; mientras que Tddpm adicionalmente genera una correlación entre \emph{\_price}-\emph{m\_size} y \emph{\_price}-\emph{m\_built}.


\begin{figure}[H]
    \centering
    \includesvg[scale=.5,inkscapelatex=false]{datasets/economicos-a-1/pairwise/smote-enc.svg}
    \caption{Correlación de conjunto Real y Modelo: smote-enc}
    \label{pairwise-smote-enc}
\end{figure}
\begin{figure}[H]
    \centering
    \includesvg[scale=.7,inkscapelatex=false]{datasets/kingcounty-a-1/tddpm_mlp/privacy.svg}
    \caption{Frecuencia del campo Privacy en el modelo real y tddpm}
    \label{frecuency-tddpm-privacy}
\end{figure}
\begin{figure}[H]
    \centering
    \includesvg[scale=.7,inkscapelatex=false]{datasets/kingcounty-a-1/tddpm_mlp/bedrooms.svg}
    \caption{Frecuencia del campo Bedrooms en el modelo real y tddpm}
    \label{frecuency-tddpm-bedrooms}
\end{figure}
\begin{figure}[H]
    \centering
    \includesvg[scale=.7,inkscapelatex=false]{datasets/kingcounty-a-1/tddpm_mlp/grade.svg}
    \caption{Frecuencia del campo Grade en el modelo real y tddpm}
    \label{frecuency-tddpm-grade}
\end{figure}
\begin{figure}[H]
    \centering
    \includesvg[scale=.7,inkscapelatex=false]{datasets/kingcounty-a-1/tddpm_mlp/floors.svg}
    \caption{Frecuencia del campo Floors en el modelo real y tddpm, King county (A-1)}
    \label{frecuency-tddpm-floors}
\end{figure}
\begin{figure}[H]
    \centering
    \includesvg[scale=.7,inkscapelatex=false]{datasets/kingcounty-a-2/tddpm_mlp/bathrooms.svg}
    \caption{Frecuencia del campo Bathrooms en el modelo real y tddpm}
    \label{frecuency-tddpm-bathrooms}
\end{figure}
\begin{figure}[H]
    \centering
    \includesvg[scale=.7,inkscapelatex=false]{datasets/kingcounty-a-3/tddpm_mlp/sqft_basement.svg}
    \caption{Frecuencia del campo Sqft basement en el modelo real y tddpm, King county (A-3)}
    \label{frecuency-tddpm-sqft basement}
\end{figure}
\begin{figure}[H]
    \centering
    \includesvg[scale=.7,inkscapelatex=false]{datasets/kingcounty-a-3/tddpm_mlp/sqft_living.svg}
    \caption{Frecuencia del campo Sqft living en el modelo real y tddpm}
    \label{frecuency-tddpm-sqft living}
\end{figure}
\begin{figure}[H]
    \centering
    \includesvg[scale=.7,inkscapelatex=false]{datasets/kingcounty-a-2/tddpm_mlp/waterfront.svg}
    \caption{Frecuencia del campo Waterfront en el modelo real y tddpm}
    \label{frecuency-tddpm-waterfront}
\end{figure}
\begin{figure}[H]
    \centering
    \includesvg[scale=.7,inkscapelatex=false]{datasets/kingcounty-a-3/tddpm_mlp/sqft_lot.svg}
    \caption{Frecuencia del campo Sqft lot en el modelo real y tddpm, King county (A-3)}
    \label{frecuency-tddpm-sqft lot}
\end{figure}
\begin{figure}[H]
    \centering
    \includesvg[scale=.7,inkscapelatex=false]{datasets/kingcounty-a-2/tddpm_mlp/sqft_living15.svg}
    \caption{Frecuencia del campo Sqft living15 en el modelo real y tddpm}
    \label{frecuency-tddpm-sqft living15}
\end{figure}
\begin{figure}[H]
    \centering
    \includesvg[scale=.7,inkscapelatex=false]{datasets/kingcounty-a-1/tddpm_mlp/yr_built.svg}
    \caption{Frecuencia del campo Yr built en el modelo real y tddpm, King county (A-1)}
    \label{frecuency-tddpm-yr built}
\end{figure}
\begin{figure}[H]
    \centering
    \includesvg[scale=.7,inkscapelatex=false]{datasets/kingcounty-a-2/tddpm_mlp/condition.svg}
    \caption{Frecuencia del campo Condition en el modelo real y tddpm, King county (A-2)}
    \label{frecuency-tddpm-condition}
\end{figure}
\begin{figure}[H]
    \centering
    \includesvg[scale=.7,inkscapelatex=false]{datasets/kingcounty-a-3/tddpm_mlp/sqft_above.svg}
    \caption{Frecuencia del campo Sqft above en el modelo real y tddpm, King county (A-3)}
    \label{frecuency-tddpm-sqft above}
\end{figure}
\begin{figure}[H]
    \centering
    \includesvg[scale=.7,inkscapelatex=false]{datasets/kingcounty-a-2/tddpm_mlp/sqft_lot15.svg}
    \caption{Frecuencia del campo Sqft lot15 en el modelo real y tddpm}
    \label{frecuency-tddpm-sqft lot15}
\end{figure}
\begin{figure}[H]
    \centering
    \includesvg[scale=.7,inkscapelatex=false]{datasets/kingcounty-a-1/tddpm_mlp/view.svg}
    \caption{Frecuencia del campo View en el modelo real y tddpm}
    \label{frecuency-tddpm-view}
\end{figure}
\begin{figure}[H]
    \centering
    \includesvg[scale=.7,inkscapelatex=false]{datasets/kingcounty-a-2/tddpm_mlp/price.svg}
    \caption{Frecuencia del campo Price en el modelo real y tddpm, King county (A-2)}
    \label{frecuency-tddpm-price}
\end{figure}

\newpage
\subsection{Reporte diagnóstico - Conjunto B}
La cobertura es notablemente baja en las variables \emph{rooms} y \emph{m\_size} en Smote, y en \emph{bathrooms} y \emph{rooms} en el caso de Tddpm. En general, el modelo Tddpm es ligeramente superior a Smote.

\begin{table}[H]
\centering
\caption{Cobertura Categoría/Rango para Modelos Smote y Tddpm, Economicos}
\label{table-coverage-economicos-b}
\begin{tabular}{|l|l|r|r|}
\hline
\rowcolor[gray]{0.8}
Columna & Metrica & smote-enc & tddpm\_mlp \\
\hline \_price & RangeCoverage & 8.10e-01±1.34e-01 & \bfseries 9.11e-01±1.37e-02 \\
\hline bathrooms & CategoryCoverage & \bfseries 8.63e-01±5.00e-02 & 6.67e-01±1.39e-02 \\
\hline county & CategoryCoverage & 5.90e-01±3.05e-03 & \bfseries 7.99e-01±2.20e-02 \\
\hline m\_built & RangeCoverage & 3.18e-01±1.01e-01 & \bfseries 7.54e-01±1.77e-01 \\
\hline m\_size & RangeCoverage & \cellcolor[rgb]{0.9, 0.54, 0.52} 3.45e-02±1.98e-03 & \bfseries \cellcolor[rgb]{0.9, 0.54, 0.52} 4.00e-01±1.51e-01 \\
\hline property\_type & CategoryCoverage & 6.30e-01±5.24e-02 & \bfseries 9.07e-01±5.24e-02 \\
\hline publication\_date & RangeCoverage & 9.77e-01±6.18e-03 & \bfseries 9.88e-01±4.44e-03 \\
\hline rooms & CategoryCoverage & 7.56e-01±3.98e-02 & \bfseries 7.97e-01±3.04e-02 \\
\hline state & CategoryCoverage & 7.92e-01±2.95e-02 & \bfseries 9.79e-01±2.95e-02 \\
\hline transaction\_type & CategoryCoverage & 5.00e-01±0.00e+00 & \bfseries 9.17e-01±1.18e-01 \\
\hline
\end{tabular}
\end{table}


\newpage
\subsection{Reporte de calidad - Conjunto B}
Ambos modelos presentan buenas métricas, superando el 91\% en términos de distribución y forma. Sin embargo, se observan excepciones en los casos de \emph{m\_built} (85\%) y \emph{m\_size} (55\%).
\begin{table}[H]
\centering
\fontsize{10}{14}\selectfont
\caption{Evaluaci\'on de Similitud de Distribuci\'on para Modelos SMOTE-ENC y TDDPM\_MLP, Economicos}
\label{table-shape-economicos-b}
\begin{tabular}{|l|l|r|r|}
\hline
\rowcolor[gray]{0.8}
Columna & Metrica & smote-enc & tddpm\_mlp \\
\hline \_price & KSComplement & 9.85e-01±1.94e-04 & \bfseries 9.93e-01±8.05e-04 \\
\hline bathrooms & TVComplement & \bfseries 9.98e-01±3.13e-04 & 9.95e-01±4.98e-04 \\
\hline county & TVComplement & 9.10e-01±5.37e-04 & \bfseries 9.84e-01±2.56e-03 \\
\hline m\_built & KSComplement & 8.56e-01±1.32e-03 & \bfseries 9.91e-01±1.44e-03 \\
\hline m\_size & KSComplement & 5.51e-01±8.46e-07 & \bfseries 9.90e-01±2.65e-03 \\
\hline property\_type & TVComplement & 9.79e-01±7.12e-04 & \bfseries 9.89e-01±3.27e-03 \\
\hline publication\_date & KSComplement & 9.66e-01±9.67e-05 & \bfseries 9.91e-01±5.41e-03 \\
\hline rooms & TVComplement & 9.87e-01±9.57e-04 & \bfseries 9.95e-01±7.29e-04 \\
\hline state & TVComplement & 9.78e-01±4.57e-04 & \bfseries 9.90e-01±1.06e-03 \\
\hline transaction\_type & TVComplement & 9.94e-01±1.97e-04 & \bfseries 9.97e-01±1.53e-03 \\
\hline
\end{tabular}
\end{table}


\newpage
\subsection{Privacidad - Conjunto B}
Las distancias mínimas para los percentiles 5 y 1 son varias magnitudes menores en el Conjunto B que en el Conjunto A, pasando de $\times 10^{-9}$ en el Conjunto A a $\times 10^{-15}$ en el Conjunto B, como se puede ver al comparar la Tabla \ref{table-dcr-economicos-b-5th} con la Tabla \ref{table-dcr-economicos-a-5th}. Se puede afirmar que el 95\% de los registros tiene al menos una distancia de $9.12 \times 10^{-15}$.

\begin{table}[H]
\centering
\fontsize{10}{14}\selectfont
\caption{Distancia de registros más cercanos, percentil 5, datos economicos}
\label{table-dcr-economicos-b-5th}
\begin{tabular}{|l|l|r|r|r|r|r|r|r|}
\hline
\rowcolor[gray]{0.8}
Modelo & DCR ST & DCR SH & DCR TH & \textbf{Score} \\
\hline tddpm\_mlp & 9.12e-15±1.09e-15 & 9.99e-15±8.14e-16 & 9.00e-17±0.00e+00 & \bfseries 9.84e-01±1.85e-03 \\
\hline smote-enc & 9.19e-15±6.41e-16 & 1.17e-14±6.96e-16 & 9.00e-17±0.00e+00 & 9.43e-01±4.67e-04 \\
\hline copulagan & \cellcolor[rgb]{0.9, 0.54, 0.52} 2.65e-16±1.60e-16 & \cellcolor[rgb]{0.9, 0.54, 0.52} 2.84e-16±1.73e-16 & 9.00e-17±0.00e+00 & 7.74e-01±2.02e-02 \\
\hline tvae & 1.00e-09±1.74e-09 & 1.00e-09±1.74e-09 & 9.00e-17±0.00e+00 & 7.38e-01±1.48e-02 \\
\hline ctgan & \bfseries 7.29e-09±8.52e-09 & \bfseries 7.35e-09±8.45e-09 & 9.00e-17±0.00e+00 & 7.34e-01±5.42e-03 \\
\hline gaussiancopula & 9.23e-13±0.00e+00 & 1.02e-12±0.00e+00 & 9.00e-17±0.00e+00 & \cellcolor[rgb]{0.9, 0.54, 0.52} 6.31e-01±0.00e+00 \\
\hline
\end{tabular}
\end{table}

\begin{table}[H]
\centering
\fontsize{10}{14}\selectfont
\caption{Distancia de registros más cercanos entre conjuntos Sinteticos, percentil 1, Economicos}
\label{table-dcr-economicos-b-1th}
\begin{tabular}{|l|l|r|r|r|r|}
\hline
\rowcolor[gray]{0.8}
Modelo & DCR ST & DCR SH & DCR TH & \textbf{Score} \\
\hline tddpm\_mlp & 1.44e-10±6.01e-12 & \cellcolor[rgb]{0.9, 0.54, 0.52} 1.40e-09±1.05e-10 & \bfseries \cellcolor[rgb]{0.9, 0.54, 0.52} 0.00e+00±0.00e+00 & \bfseries 9.77e-01±6.88e-04 \\
\hline smote-enc & \cellcolor[rgb]{0.9, 0.54, 0.52} 0.00e+00±0.00e+00 & 1.41e-09±4.21e-10 & \bfseries \cellcolor[rgb]{0.9, 0.54, 0.52} 0.00e+00±0.00e+00 & 9.67e-01±8.19e-04 \\
\hline ctgan & \bfseries 2.20e-06±1.50e-06 & \bfseries 3.24e-06±1.58e-06 & \bfseries \cellcolor[rgb]{0.9, 0.54, 0.52} 0.00e+00±0.00e+00 & 6.96e-01±1.00e-02 \\
\hline copulagan & 2.04e-07±2.85e-08 & 4.37e-07±6.38e-08 & \bfseries \cellcolor[rgb]{0.9, 0.54, 0.52} 0.00e+00±0.00e+00 & 7.81e-01±2.03e-02 \\
\hline gaussiancopula & 8.04e-07±8.64e-23 & 1.93e-06±0.00e+00 & \bfseries \cellcolor[rgb]{0.9, 0.54, 0.52} 0.00e+00±0.00e+00 & 6.91e-01±6.41e-17 \\
\hline tvae & 7.41e-08±1.95e-09 & 1.16e-07±3.43e-09 & \bfseries \cellcolor[rgb]{0.9, 0.54, 0.52} 0.00e+00±0.00e+00 & \cellcolor[rgb]{0.9, 0.54, 0.52} 6.40e-01±3.35e-03 \\
\hline
\end{tabular}
\end{table}

\begin{table}[H]
\centering
\fontsize{10}{14}\selectfont
\caption{Distancia de registros más cercanos, minimo, datos economicos}
\label{table-dcr-economicos-b}
\begin{tabular}{|l|l|r|r|r|r|r|r|r|}
\hline
\rowcolor[gray]{0.8}
Modelo & DCR ST & DCR SH & DCR TH & \textbf{Score} \\
\hline tddpm\_mlp & \cellcolor[rgb]{0.9, 0.54, 0.52} 0.00e+00±0.00e+00 & \cellcolor[rgb]{0.9, 0.54, 0.52} 0.00e+00±0.00e+00 & 0.00e+00±0.00e+00 & \bfseries 9.84e-01±1.85e-03 \\
\hline smote-enc & \cellcolor[rgb]{0.9, 0.54, 0.52} 0.00e+00±0.00e+00 & \cellcolor[rgb]{0.9, 0.54, 0.52} 0.00e+00±0.00e+00 & 0.00e+00±0.00e+00 & 9.43e-01±4.67e-04 \\
\hline copulagan & 4.57e-19±3.77e-21 & \bfseries 5.21e-19±1.82e-22 & 0.00e+00±0.00e+00 & 7.74e-01±2.02e-02 \\
\hline tvae & 8.99e-20±0.00e+00 & 8.99e-20±0.00e+00 & 0.00e+00±0.00e+00 & 7.38e-01±1.48e-02 \\
\hline ctgan & 8.99e-20±0.00e+00 & 8.99e-20±0.00e+00 & 0.00e+00±0.00e+00 & 7.34e-01±5.42e-03 \\
\hline gaussiancopula & \bfseries 5.23e-19±0.00e+00 & 5.09e-19±0.00e+00 & 0.00e+00±0.00e+00 & \cellcolor[rgb]{0.9, 0.54, 0.52} 6.31e-01±0.00e+00 \\
\hline
\end{tabular}
\end{table}

\newpage
De las Tablas \ref{table-nndr-economicos-b-5th}, \ref{table-nndr-economicos-b-1th} y \ref{table-nndr-economicos-b-min} emergen dos características notables. La primera es que en el percentil 1 y el 5, en ambos casos, el modelo Tddpm mantiene la mayor razón entre el primer y el segundo registro más cercano. La segunda es que, al compararse con el Conjunto A (referenciado en la Tabla \ref{table-nndr-economicos-a-1th}), la razón para el modelo Tddpm resulta ser superior.
\begin{table}[H]
\centering
\fontsize{10}{14}\selectfont
\caption{Proporción entre el más cercano y el segundo más cercano, percentil 5, datos economicos}
\label{table-nndr-economicos-b-5th}
\begin{tabular}{|l|l|r|r|r|r|r|r|r|}
\hline
\rowcolor[gray]{0.8}
Modelo & NNDR ST & NNDR SH & NNDR TH & \textbf{Score} \\
\hline tddpm\_mlp & \bfseries 3.03e-01±4.42e-03 & \bfseries 2.96e-01±1.27e-02 & 1.15e-07±0.00e+00 & \bfseries 9.84e-01±1.85e-03 \\
\hline smote-enc & 2.47e-01±3.63e-03 & 2.60e-01±6.24e-03 & 1.15e-07±0.00e+00 & 9.43e-01±4.67e-04 \\
\hline copulagan & \cellcolor[rgb]{0.9, 0.54, 0.52} 1.07e-05±4.91e-06 & \cellcolor[rgb]{0.9, 0.54, 0.52} 2.27e-05±1.82e-05 & 1.15e-07±0.00e+00 & 7.74e-01±2.02e-02 \\
\hline tvae & 4.28e-04±2.75e-04 & 4.49e-04±2.88e-04 & 1.15e-07±0.00e+00 & 7.38e-01±1.48e-02 \\
\hline ctgan & 2.10e-03±7.18e-04 & 7.23e-03±1.01e-02 & 1.15e-07±0.00e+00 & 7.34e-01±5.42e-03 \\
\hline gaussiancopula & 1.52e-02±0.00e+00 & 1.38e-02±0.00e+00 & 1.15e-07±0.00e+00 & \cellcolor[rgb]{0.9, 0.54, 0.52} 6.31e-01±0.00e+00 \\
\hline
\end{tabular}
\end{table}

\begin{table}[H]
\centering
\fontsize{10}{14}\selectfont
\caption{Proporción entre el más cercano y el segundo más cercano, percentil 1, datos economicos}
\label{table-nndr-economicos-b-1th}
\begin{tabular}{|l|l|r|r|r|r|r|r|r|}
\hline
\rowcolor[gray]{0.8}
Modelo & NNDR ST & NNDR SH & NNDR TH & \textbf{Score} \\
\hline tddpm\_mlp & \bfseries 3.14e-02±4.92e-03 & \bfseries 3.08e-02±3.94e-03 & 0.00e+00±0.00e+00 & \bfseries 9.84e-01±1.85e-03 \\
\hline smote-enc & 2.52e-03±1.07e-03 & 3.47e-03±2.68e-04 & 0.00e+00±0.00e+00 & 9.43e-01±4.67e-04 \\
\hline copulagan & \cellcolor[rgb]{0.9, 0.54, 0.52} 5.33e-09±1.38e-09 & \cellcolor[rgb]{0.9, 0.54, 0.52} 1.15e-07±1.65e-07 & 0.00e+00±0.00e+00 & 7.74e-01±2.02e-02 \\
\hline tvae & 3.02e-05±4.14e-05 & 3.04e-05±4.15e-05 & 0.00e+00±0.00e+00 & 7.38e-01±1.48e-02 \\
\hline ctgan & 1.21e-04±1.18e-04 & 1.35e-04±1.66e-04 & 0.00e+00±0.00e+00 & 7.34e-01±5.42e-03 \\
\hline gaussiancopula & 6.43e-06±0.00e+00 & 6.43e-06±0.00e+00 & 0.00e+00±0.00e+00 & \cellcolor[rgb]{0.9, 0.54, 0.52} 6.31e-01±0.00e+00 \\
\hline
\end{tabular}
\end{table}

\begin{table}[H]
\centering
\fontsize{10}{14}\selectfont
\caption{Proporción entre el más cercano y el segundo más cercano, mínimo, Economicos}
\label{table-nndr-economicos-b-min}
\begin{tabular}{|l|l|r|r|r|r|}
\hline
\rowcolor[gray]{0.8}
Modelo & NNDR ST & NNDR SH & NNDR TH & \textbf{Score} \\
\hline tddpm\_mlp & \bfseries 0.00e+00±0.00e+00 & \bfseries 0.00e+00±0.00e+00 & \cellcolor[rgb]{0.9, 0.54, 0.52} \bfseries 0.00e+00±0.00e+00 & \cellcolor[rgb]{0.9, 0.54, 0.52} 9.77e-01±6.88e-04 \\
\hline smote-enc & 0.00e+00±0.00e+00 & 0.00e+00±0.00e+00 & 0.00e+00±0.00e+00 & 9.67e-01±8.19e-04 \\
\hline ctgan & 4.47e-04±2.18e-04 & 1.88e-04±2.25e-04 & 0.00e+00±0.00e+00 & 6.96e-01±1.00e-02 \\
\hline copulagan & 1.79e-04±4.03e-05 & 2.19e-04±3.23e-05 & 0.00e+00±0.00e+00 & 7.81e-01±2.03e-02 \\
\hline gaussiancopula & 1.99e-04±0.00e+00 & 1.04e-04±1.36e-20 & 0.00e+00±0.00e+00 & 6.91e-01±6.41e-17 \\
\hline tvae & \cellcolor[rgb]{0.9, 0.54, 0.52} 8.02e-04±2.62e-04 & \cellcolor[rgb]{0.9, 0.54, 0.52} 8.94e-03±1.12e-03 & 0.00e+00±0.00e+00 & \bfseries 6.40e-01±3.35e-03 \\
\hline
\end{tabular}
\end{table}

\newpage
\subsection{Ejemplos de registros - Conjunto B}
En el ejemplo de las Tablas \ref{table-example-economicos-b-1-tddpm_mlp-2p} y \ref{table-example-economicos-b-1-tddpm_mlp-2p-text}, corresponde a un departamento de dos dormitorios.

\begin{table}[H]
\centering
\fontsize{10}{14}\selectfont
\caption{Ejemplos para el modelo tddpm\_mlp, percentil 2}
\label{table-example-economicos-b-1-tddpm_mlp-2p}
\begin{tabular}{|l|r|r|r|}
\hline
\rowcolor[gray]{0.8}
Variable/Distancia & Sintético & DCR1 d(1.24e-15) & DCR2 d(7.82e-13) \\
\hline \_price & \cellcolor[rgb]{0.9, 0.54, 0.52} 9.128134 & 12.735812 & 2490.000000 \\
\hline bathrooms & \cellcolor[rgb]{0.9, 0.54, 0.52} 1.000000 & \cellcolor[rgb]{0.9, 0.54, 0.52} 1.000000 & \cellcolor[rgb]{0.9, 0.54, 0.52} 1.000000 \\
\hline county & \cellcolor[rgb]{0.9, 0.54, 0.52} Valparaíso & Santiago & Santiago \\
\hline m\_built & \cellcolor[rgb]{0.9, 0.54, 0.52} 50.000000 & 41.000000 & 4929.000000 \\
\hline m\_size & \cellcolor[rgb]{0.9, 0.54, 0.52} 48.000000 & 43.000000 & -1.000000 \\
\hline property\_type & \cellcolor[rgb]{0.9, 0.54, 0.52} Departamento & \cellcolor[rgb]{0.9, 0.54, 0.52} Departamento & \cellcolor[rgb]{0.9, 0.54, 0.52} Departamento \\
\hline publication\_date & \cellcolor[rgb]{0.9, 0.54, 0.52} 350.000000 & \cellcolor[rgb]{0.9, 0.54, 0.52} 350.000000 & \cellcolor[rgb]{0.9, 0.54, 0.52} 350.000000 \\
\hline rooms & \cellcolor[rgb]{0.9, 0.54, 0.52} 2.000000 & \cellcolor[rgb]{0.9, 0.54, 0.52} 2.000000 & \cellcolor[rgb]{0.9, 0.54, 0.52} 2.000000 \\
\hline state & \cellcolor[rgb]{0.9, 0.54, 0.52} Valparaíso & Metropolitana de Santiago & Metropolitana de Santiago \\
\hline transaction\_type & \cellcolor[rgb]{0.9, 0.54, 0.52} Arriendo & \cellcolor[rgb]{0.9, 0.54, 0.52} Arriendo & Venta \\
\hline
\end{tabular}
\end{table}

\begin{table}[H]
\centering
\fontsize{10}{14}\selectfont
\caption{Ejemplos de texto modelo Tddpm, percentil 2, Economicos (B-1)}
\label{table-example-economicos-b-1-tddpm_mlp-2p-text}
\begin{tabular}{|l|m{35em}|}
\hline
\rowcolor[gray]{0.8}
Distancia & description \\
\hline Sintético & Departamento de dos dormitorios, 2 ba\~nos, living comedor con salida a terraza, cocina amoblada equipada (incluye encimera), horno empotrado, campana, cubierta de granito, logia cerrada, estacionamiento subterr\'aneo, bodega \\
\hline DCR1 d(1.24e-15) & Corredor arrienda, disponible inmediato, cercano a metro Franklin linea 2 y 6, supermercado 10, barrio Franklin y Bio Bio, amplio comercio, plazas, otros. Edificio Zenteno Efficient, a\~no 2018, nuevo sin uso. 

San Diego 1721
?
Piso medio, 2 dormitorios, uno grande otro peque\~no para cama de 1 plaza, 1 ba\~no completo,  cocina integrada,espacio para lavadora, sin balc\'on, sistema full electric en cocina, horno y termo electrico.

Requisitos:
1.- Obligatorio 12 Cheques, puede ser del aval
2.- Sueldo TITULAR 3 veces el arriendo
3.- Certificado AFP \'ultimos 12 meses 
4.- Informe Dicom Platinum 
5.- C\'edula por ambos lados 
6.- 1 mes de arriendo, 1 en garant{\'\i}a y comisi\'on 50\%

El Edificio cuenta con lavander{\'\i}a, sala multiuso, seguridad 24/7 \\
\hline DCR2 d(7.82e-13) & SE VENDE, Departamento CONDOMINIO EDIFICIO AVENIDA MATTA PLAZA, accesos controlados 24/7, c\'amaras de seguridad, alarma, timbres de p\'anico en cada Dpto, Ventanas de termopanel, cit\'ofono, cocina equipada con cubierta de granito, Hermosas \'areas de jardines, 2 Dormitorios principal con Woking Closet y 1 Ba\~no, Gimnasio equipado, Sala multiusos, Quinchos, Piscina, Sala primeros auxilios, Terrazas en segundo piso, Sala lavander{\'\i}a, estacionamientos de visitas. Excelente conectividad, Metro Irarrazabal, privacidad y tranquilidad, adem\'as, cerca de supermercados, centros comerciales, jardines y colegios.
Metros Cuadrados
Metros Construidos:  47,29 M{\textasciicircum}2.
Terraza Construida: 3 M{\textasciicircum}2.
Terminaciones
Piso ba\~nos: Cer\'amicos.
Piso Living: Piso Flotante.
Dormitorios: Alfombrados y Porcelanato.
Otros suministros
internet, tel\'efono, tv cable, wi-fi, Gastos Comunes \$ 50.000, No se paga Contribuciones.
Precio: UF 2.490.
{\textexclamdown}{\textexclamdown}{\textexclamdown}NO deje de visitar{\textexclamdown}{\textexclamdown}{\textexclamdown}
Cont\'actanos: 
Carlos Miranda: +569 75894834.
Paulina Montt: +569 96761295.
Daniela Aguirre: +569 93221157.
Email:: contacto@lodgepropiedades.cl
 \\
\hline
\end{tabular}
\end{table}

\newpage
En el ejemplo presentado en las Tablas \ref{table-example-economicos-b-1-tddpm_mlp-4p} y \ref{table-example-economicos-b-1-tddpm_mlp-4p-text}, el registro sintético muestra coherencia con los datos de entrada. Por ejemplo, el texto generado corresponde a un departamento con dos dormitorios, aunque indica la existencia de un baño adicional en comparación con los datos de la publicación. Sin embargo, no proporciona otra información relevante que pueda correlacionarse con los datos estructurados de la publicación.
\begin{table}[H]
\centering
\fontsize{10}{14}\selectfont
\caption{Ejemplos para el modelo tddpm\_mlp, percentil 4}
\label{table-example-economicos-b-1-tddpm_mlp-4p}
\begin{tabular}{|l|r|r|r|}
\hline
\rowcolor[gray]{0.8}
Variable/Distancia & Sintético & DCR1 d(5.13e-15) & DCR2 d(1.00e-09) \\
\hline \_price & \cellcolor[rgb]{0.9, 0.54, 0.52} 16.231131 & 11.115125 & 16.672687 \\
\hline bathrooms & \cellcolor[rgb]{0.9, 0.54, 0.52} 1.000000 & \cellcolor[rgb]{0.9, 0.54, 0.52} 1.000000 & \cellcolor[rgb]{0.9, 0.54, 0.52} 1.000000 \\
\hline county & \cellcolor[rgb]{0.9, 0.54, 0.52} Ñuñoa & Pudahuel & La Florida \\
\hline m\_built & \cellcolor[rgb]{0.9, 0.54, 0.52} 54.023199 & -1.000000 & 200.000000 \\
\hline m\_size & \cellcolor[rgb]{0.9, 0.54, 0.52} 57.000000 & -1.000000 & 270.000000 \\
\hline property\_type & \cellcolor[rgb]{0.9, 0.54, 0.52} Departamento & Casa & Casa \\
\hline publication\_date & \cellcolor[rgb]{0.9, 0.54, 0.52} 142.000000 & \cellcolor[rgb]{0.9, 0.54, 0.52} 142.000000 & \cellcolor[rgb]{0.9, 0.54, 0.52} 142.000000 \\
\hline rooms & \cellcolor[rgb]{0.9, 0.54, 0.52} 2.000000 & \cellcolor[rgb]{0.9, 0.54, 0.52} 2.000000 & 3.000000 \\
\hline state & \cellcolor[rgb]{0.9, 0.54, 0.52} Metropolitana de Santiago & \cellcolor[rgb]{0.9, 0.54, 0.52} Metropolitana de Santiago & \cellcolor[rgb]{0.9, 0.54, 0.52} Metropolitana de Santiago \\
\hline transaction\_type & \cellcolor[rgb]{0.9, 0.54, 0.52} Arriendo & \cellcolor[rgb]{0.9, 0.54, 0.52} Arriendo & \cellcolor[rgb]{0.9, 0.54, 0.52} Arriendo \\
\hline
\end{tabular}
\end{table}

\begin{table}[H]
\centering
\fontsize{10}{14}\selectfont
\caption{Ejemplos de texto modelo Tddpm, percentil 4, Economicos (B-1)}
\label{table-example-economicos-b-1-tddpm_mlp-4p-text}
\begin{tabular}{|l|m{35em}|}
\hline
\rowcolor[gray]{0.8}
Distancia & description \\
\hline Sintético & Departamento de dos dormitorios, 2 ba\~nos, living comedor con salida a terraza, cocina amoblada equipada encimera, horno empotrado, campana, logia cerrada, estacionamiento subterr\'aneo para veh{\'\i}culos (conserjer{\'\i}a las 24 horas) \\
\hline DCR1 d(5.13e-15) & Casa interior, entrada independiente, dos dormitorios, ba\~no, cocina comedor, peque\~no patio, cerca de negocios, supermercados,servicio urgencia, consultorio, cerca metro pudahuel, 300.000 mensual, 1 meses garantia,luz y agua adicional.Solo personas quieran vivir lugar tranquilo.Consultas 9-44104648
llama coordina tu visita, sin estacionamiento
llamar lunes a viernes desde las 17:30, sabado y domingo mismo horario \\
\hline DCR2 d(1.00e-09) & Arriendo Comercial / Habitacional. Casa aislada. Entrada auto (2) prestaci\'on de Servicios computacionales, asesorias  mas informacion al correo mh@rammsy.cl \\
\hline
\end{tabular}
\end{table}


\newpage
\subsection{Propiedades estadísticas - Conjunto B}
El listado completo de las propiedades estadísticas se encuentra en el Anexo \ref{propiedades-estadisticas-economicos-B}. A continuación, se presentan las propiedades estadísticas en las que los modelos Tddpm y Smote muestran una diferencia mayor al 5\% con respecto al conjunto original de entrenamiento. Como referencia, se incluye el modelo Ctgan. Las variables se seleccionaron por ser 1) las que obtuvieron el peor resultado en cobertura y 2) las que obtuvieron el peor resultado en la distribución, respectivamente.

\begin{table}[H]
\centering
\fontsize{8}{14}\selectfont
\caption{Propiedades estadisticas de variable bathrooms con cambio\ensuremath{>}5\%, Economicos (B-1)}
\label{table-stats-economicos-b-1-bathrooms-short}
\begin{tabular}{|l|m{10em}|m{10em}|m{10em}|m{10em}|}
\hline
 \rowcolor[gray]{0.8}
Variable/Modelo & Real & tddpm\_mlp & smote-enc & ctgan \\
\hline nobs & 545870 & 682338 & 682338 & 682338 \\
\hline mean & 0.815 & 0.790 & \bfseries 0.809 & \cellcolor[rgb]{0.9, 0.54, 0.52} 1.507 \\
\hline std\_err & 0.003 & 0.002 & \bfseries 0.002 & \cellcolor[rgb]{0.9, 0.54, 0.52} 0.007 \\
\hline upper\_ci & 0.820 & 0.793 & \bfseries 0.813 & \cellcolor[rgb]{0.9, 0.54, 0.52} 1.521 \\
\hline lower\_ci & 0.810 & 0.786 & \bfseries 0.805 & \cellcolor[rgb]{0.9, 0.54, 0.52} 1.493 \\
\hline std & 1.898 & 1.609 & \bfseries 1.691 & \cellcolor[rgb]{0.9, 0.54, 0.52} 5.740 \\
\hline mad & 1.376 & 1.359 & \bfseries 1.375 & \cellcolor[rgb]{0.9, 0.54, 0.52} 2.494 \\
\hline mad\_normal & 1.725 & 1.703 & \bfseries 1.723 & \cellcolor[rgb]{0.9, 0.54, 0.52} 3.126 \\
\hline coef\_var & 2.328 & 2.037 & \bfseries 2.089 & \cellcolor[rgb]{0.9, 0.54, 0.52} 3.809 \\
\hline range & 437.000 & \cellcolor[rgb]{0.9, 0.54, 0.52} 116.000 & 146.000 & \bfseries 437.000 \\
\hline max & 436.000 & \cellcolor[rgb]{0.9, 0.54, 0.52} 115.000 & 145.000 & \bfseries 436.000 \\
\hline skew & 36.380 & \cellcolor[rgb]{0.9, 0.54, 0.52} 1.000 & 3.134 & \bfseries 15.447 \\
\hline kurtosis & 6629 & \cellcolor[rgb]{0.9, 0.54, 0.52} 41 & 165 & \bfseries 527 \\
\hline jarque\_bera & 9.98582e+11 & \cellcolor[rgb]{0.9, 0.54, 0.52} 4.22514e+07 & 7.44954e+08 & \bfseries 7.84347e+09 \\
\hline 99.9\% & 9.000 & 7.000 & \bfseries 9.000 & \cellcolor[rgb]{0.9, 0.54, 0.52} 83.000 \\
\hline
\end{tabular}
\end{table}

\begin{table}[H]
\centering
\fontsize{8}{14}\selectfont
\caption{Propiedades estadisticas de variable m\_size con cambio\ensuremath{>}5\%, Economicos (B-1)}
\label{table-stats-economicos-b-1-m_size-short}
\begin{tabular}{|l|m{10em}|m{10em}|m{10em}|m{10em}|}
\hline
 \rowcolor[gray]{0.8}
Variable/Modelo & Real & tddpm\_mlp & smote-enc & ctgan \\
\hline nobs & 545870 & 682338 & 682338 & 682338 \\
\hline mean & 2.03551e+16 & \cellcolor[rgb]{0.9, 0.54, 0.52} 1.54755e+18 & 3.16803e+11 & \bfseries 1.86487e+15 \\
\hline std\_err & 2.03549e+16 & \cellcolor[rgb]{0.9, 0.54, 0.52} 1.25490e+17 & 1.82548e+11 & \bfseries 3.05751e+12 \\
\hline upper\_ci & 6.02499e+16 & \cellcolor[rgb]{0.9, 0.54, 0.52} 1.79351e+18 & 6.74592e+11 & \bfseries 1.87086e+15 \\
\hline lower\_ci & -1.95397e+16 & \cellcolor[rgb]{0.9, 0.54, 0.52} 1.30160e+18 & \bfseries -4.09850e+10 & 1.85888e+15 \\
\hline std & 1.50388e+19 & \cellcolor[rgb]{0.9, 0.54, 0.52} 1.03659e+20 & 1.50792e+14 & \bfseries 2.52561e+15 \\
\hline iqr & 181.000 & \bfseries 171.000 & 210.281 & \cellcolor[rgb]{0.9, 0.54, 0.52} 3211862663343589.000 \\
\hline iqr\_normal & 134.176 & \bfseries 126.762 & 155.882 & \cellcolor[rgb]{0.9, 0.54, 0.52} 2380957355104258.000 \\
\hline mad & 4.07100e+16 & \cellcolor[rgb]{0.9, 0.54, 0.52} 3.09404e+18 & 6.33595e+11 & \bfseries 2.03095e+15 \\
\hline mad\_normal & 5.10224e+16 & \cellcolor[rgb]{0.9, 0.54, 0.52} 3.87781e+18 & 7.94093e+11 & \bfseries 2.54542e+15 \\
\hline coef\_var & 738.823 & 66.983 & \bfseries 475.979 & \cellcolor[rgb]{0.9, 0.54, 0.52} 1.354 \\
\hline range & 1.11111e+22 & \bfseries 1.11111e+22 & 8.17891e+16 & \cellcolor[rgb]{0.9, 0.54, 0.52} 2.07066e+16 \\
\hline max & 1.11111e+22 & \bfseries 1.11111e+22 & 8.17891e+16 & \cellcolor[rgb]{0.9, 0.54, 0.52} 2.07066e+16 \\
\hline min & -1000.000 & \bfseries -1000.000 & \cellcolor[rgb]{0.9, 0.54, 0.52} -881.043 & \bfseries -1000.000 \\
\hline skew & 738.828 & 82.168 & \bfseries 507.710 & \cellcolor[rgb]{0.9, 0.54, 0.52} 1.473 \\
\hline kurtosis & 545868 & 7532 & \bfseries 264489 & \cellcolor[rgb]{0.9, 0.54, 0.52} 5 \\
\hline jarque\_bera & 6.77722e+15 & 1.61221e+12 & \bfseries 1.98885e+15 & \cellcolor[rgb]{0.9, 0.54, 0.52} 3.39104e+05 \\
\hline mode\_freq & 0.449 & \bfseries 0.454 & \cellcolor[rgb]{0.9, 0.54, 0.52} 0.279 & 0.454 \\
\hline median & 36.000 & \bfseries 37.000 & 46.170 & \cellcolor[rgb]{0.9, 0.54, 0.52} 472364610529319.625 \\
\hline 75.0\% & 180.000 & \bfseries 170.000 & 209.281 & \cellcolor[rgb]{0.9, 0.54, 0.52} 3211862663342589.000 \\
\hline 95.0\% & 5000 & \bfseries 5000 & 5070 & \cellcolor[rgb]{0.9, 0.54, 0.52} 7129564968780261 \\
\hline 99.0\% & 50000 & \bfseries 44090 & 68451 & \cellcolor[rgb]{0.9, 0.54, 0.52} 9903363003858036 \\
\hline 99.9\% & 4920000 & 1753435 & \bfseries 6574780 & \cellcolor[rgb]{0.9, 0.54, 0.52} 13009696522973218 \\
\hline
\end{tabular}
\end{table}

