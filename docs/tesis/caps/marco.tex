\chapter{Revisión Bibliográfica}
\section{Tipos de Datos}
\label{tipo-de-datos}
Los tipos de datos tienen varias implicancias en su generación, como su representación, almacenamiento y procesamiento. Los datos estructurados estan presentados en Tabla \ref{tabla-tipo-datos}. En el trabajo de 

En el 2012 IDC estimó que para el 2020 más del 95\% sería data no estructurada \cite{gantz_digital_2012}. En seguimiento en el analisis de Kiran Adnan y Rehan Akbar \cite{adnan_analytical_2019} el texto es el tipo de dato no estructurado con mayor crecimiento en las publicaciones. seguidos en orden por imagen, video y finalmente el audio.

La \fullref{tabla-tipo-datos} se resume la lista que figura en \emph{Practical statistics for data scientists} \cite{bruce_practical_2020}.
\begin{table}[H]
	\centering
	\caption{Tipos de datos estructurados}
	\label{tabla-tipo-datos}
    \begin{tabular}{|l|l|m{25em}|m{9em}|}
    \hline
    \rowcolor[gray]{0.8}
    T & Sub tipo & Descripción & Ejemplos \\
    \hline
    \multicolumn{2}{|>{\columncolor[gray]{0.8}}c|}{Numérico} & Datos establecidos como números & \cellcolor[gray]{0.8} - \\
    \hline
    \cellcolor[gray]{0.8} & Continuo & Datos que pueden tomar cualquier valor en un intervalo & 3.14 metros, 1.618 litros \\
    \hline
    \cellcolor[gray]{0.8} & Discreto & Datos que solo pueden tomar valores enteros & 1 habitación, 73 años \\
    \hline
    \multicolumn{2}{|>{\columncolor[gray]{0.8}}c|}{Categórico} & Datos que pueden tomar solo un conjunto específico de valores que representan un conjunto de categorías posibles. & \cellcolor[gray]{0.8} - \\
    \hline
    \cellcolor[gray]{0.8} & Binario & Un caso especial de datos categóricos con solo dos categorías de valores & 0/1, verdadero/falso \\
    \hline
    \cellcolor[gray]{0.8} & Ordinal & Datos categóricos que tienen un ordenamiento explícito. & pequeña/ mediana/ grande \\
    \hline
    \end{tabular}
\end{table}

\newpage
\section{Privacidad de Datos}

La privacidad de datos es un tema crítico en la generación de datos sintéticos. Si los datos pertenecen a recetas o automóviles, entonces la privacidad no es relevante. Sin embargo, la síntesis de datos sobre individuos lo es \cite{bruce_practical_2020}. Es por ello que para Equifax es un tema relevante. Muchos de los conjuntos de datos tienen información personal.

\subsection{Tipo de datos a ser protegidos}
Para determinar qué campos de datos son relevantes en términos de privacidad, se puede utilizar la definición resumida en la Tabla \ref{data-relevante} de \emph{Data privacy: Definitions and techniques} \cite{de_capitani_di_vimercati_data_2012}.


\begin{table}[H]
	\centering
	\caption{Niveles de revelación y ejemplos}
	\label{data-relevante}
    \begin{tabular}{|m{15em}|m{25em}|}
    \hline
    \rowcolor[gray]{0.8}
    Tipo de revelación & Descripción \\
    \hline
    \textbf{Identificadores} 
    & Atributos que identifican de manera única a individuos (por ejemplo, SSN, RUT, DNI). \\
    \hline
    \textbf{Cuasi-identificadores (QI)} 
    & Atributos que, en combinación, pueden identificar a individuos, o reducir la incertidumbre sobre sus identidades (por ejemplo, fecha de nacimiento, género y código postal). \\
    \hline
    \textbf{Atributos confidenciales} 
    & Atributos que representan información sensible (por ejemplo, enfermedad). \\
    \hline
    \textbf{Atributos no confidenciales} 
    & Atributos que los encuestados no consideran sensibles y cuya divulgación es inofensiva (por ejemplo, color favorito). \\
    \hline
    \end{tabular}
\end{table}

\newpage
\subsection{Tipos de riesgos de divulgación}
Los tipos de divulgación definidos de \emph{Practical synthetic data generation} \cite{bruce_practical_2020} se resumen en la Tabla \ref{relevantes-definiciones}

\begin{table}[H]
	\centering
	\caption{Tipos de Riesgos de Divulgación y sus Descripciones}
	\label{relevantes-definiciones}
    \begin{tabular}{|m{15em}|m{25em}|}
    \hline
    \rowcolor[gray]{0.8}
    Tipo de revelación & Descripción \\
    \hline
    \textbf{Divulgación de identidad} 
    & Este riesgo se refiere a la posibilidad de que un atacante pueda identificar la información de un individuo a partir de datos compartidos, utilizando técnicas de filtrado para llegar a una única opción.\\
    \hline
    \textbf{Divulgación de nueva información} 
    & Este riesgo asume el riesgo de Divulgación de Identidad y, además, implica una ganancia de información adicional sobre el individuo a partir de los datos compartidos.\\
    \hline
    \textbf{Divulgación de Atributos} 
    & Este riesgo se presenta cuando, a pesar de no poder identificar a un individuo, se puede identificar un atributo común en varios registros, lo que permite obtener información sensible sobre un grupo de individuos.\\
    \hline
    \textbf{Divulgación Inferencial} 
    & Este riesgo se refiere a la posibilidad de inferir información sensible a partir de los datos compartidos, mediante la utilización de técnicas de análisis estadístico o de aprendizaje automático. Por ejemplo si filtrando todos los registros. 80\% de los registros con las mismas caracteristicas tienen cancer, se podría inferir que el individuo buscado puede tener cancer.\\
    \hline
    \end{tabular}
\end{table}

Adicionalmente se deben establecer dos conceptos relevantes ante el análisis de revelación de información:
\begin{enumerate}
    \item En términos prácticos, normalmente la data sintética busca tener cierta permeabilidad con respecto a la \textbf{Divulgación Inferencial}, ya que queremos que estadisticamente sea similar. adicionalmente se busca proteger la identidad de los individuos, pero no es la unica condición, tambien se busca proteger esos atributos que pueden ser sensibles, por ejemplo enfermedades. A todo este conjunto se le denomina \textbf{Revelación de indentidad significativa}. Particularmente riesgoso por discriminación en ciertos grupos que cumplan los atributos criterios.
    \item Los mismos atributos pueden tener más relevancia para ciertos grupos de la población que para otro. El ejemplo que indica \cite{el_emam_practical_2020} es que debido a que el numero de hijo igual a 2, es menos frecuente en una etnia que otra, una 40\% y la segunda un 10\%, ese dato es más relevante en la segunda. Ya que es un factor que filtra de mejor manera y por ello puede conocerce mejor ese grupo especifico. Esto lo denomina \textbf{Definición de información ganada}
\end{enumerate}
\newpage
\subsection{Regulación de datos sintéticos}
Debido a que los datos sintéticos son basados en datos reales, pueden ser afectos a las regulaciones de sobre protección de datos \cite{bruce_practical_2020}. Los nuevos datos podrían ser afectos por:
\begin{enumerate}
    \item \href{https://dvbi.ru/Portals/0/DOCUMENTS_SHARE/RISK_MANAGEMENT/EBA/GDPR_eng_rus.pdf}{Regulation (EU) 2016/679 of the European Parliament and of the Council} \cite{regulation_regulation_2016}
    \item \href{https://heinonline.org/HOL/LandingPage?handle=hein.journals/jtlp23&div=5&id=&page=}{The California consumer privacy act: Towards a European-style privacy regime in the United States} \cite{pardau_california_2018}
    \item \href{http://www.eolusinc.com/pdf/hipaa.pdf}{Health insurance portability and accountability act of 1996} \cite{act_health_1996}
\end{enumerate}

\subsection{Protección de Privacidad}
TBD

\section{Generación de Datos Sintéticos}

Los datos sintéticos no son datos reales, pero se intenta que conserven algunas propiedades de los datos reales. El grado en que los datos sintéticos pueden servir como proxy de los datos reales es la medida de su utilidad \cite{bruce_practical_2020}. Dependiendo del uso de los datos reales, se pueden separar en tres tipos: los que utilizan datos reales, los que no lo hacen y los híbridos.

\textbf{Datos basados en datos reales}: utilizan modelos que aprenden la distribución de los datos originales para conformar nuevos puntos semejantes.

\textbf{Datos no basados en datos reales}: utilizan el conocimiento del mundo. Por ejemplo, un conjunto de nombres al azar con un apellido al azar para formar un nombre completo.

\textbf{Híbridos}: estos combinan técnicas de imitación de distribución con algunos campos que no provienen de los datos reales. Esto resulta particularmente útil cuando se intenta desligar las distribuciones de datos que podrían ser sensibles o generar discriminación, como la información sobre la etnia.

En la \fullref{tipo-de-datos}, se revisaron los datos estructurados. Si bien cada tipo puede tener muchas representaciones, por ejemplo, los datos continuos podrían considerarse como \emph{float}, \emph{datetime} o incluso intervalos personalizados, como de 0 a 1. Sobre estos datos estructurados, se pueden generar estructuras para unirlos.

Entre las estructuras más comunes se encuentran las matrices bidimensionales (datos tabulares) y los arreglos, que permiten matrices de muchas dimensiones e incluso estructuras complejas que pueden mezclar todas las estructuras previas.

Debido al objetivo, se detallan solo los modelos que permiten abordar la generación de datos tabulares y texto basados en datos reales.

\subsection{Generación de datos tabulares}
En la \fullref{tab-sota-tab}, se resumen las últimas publicaciones sobre generación de datos tabulares, indicando la fecha de publicación y si se puede acceder al código fuente o no, a febrero de 2023.

\begin{table}[H]
	\centering
	\caption{Estado del arte en generación de datos tabulares}
	\label{tab-sota-tab}
    \begin{tabular}{|m{25em}|r|r|}
    \hline
    \rowcolor[gray]{0.8}
    Nombre & Fecha $\downarrow$ & Código \\
    \hline
    REaLTabFormer: Generating Realistic Relational and Tabular Data using Transformers \cite{solatorio_realtabformer_2023}
    & 2023-02-04 & \href{https://github.com/avsolatorio/REaLTabFormer}{Github} \\
    \hline
    PreFair: Privately Generating Justifiably Fair Synthetic Data \cite{pujol_prefair_2022}
    & 2022-12-20 & \\
    \hline
    GenSyn: A Multi-stage Framework for Generating Synthetic Microdata using Macro Data Sources \cite{acharya_gensyn_2022}
    & 2022-12-08 & \href{https://github.com/Angeela03/GenSyn}{Github} \\
    \hline
    TabDDPM: Modelling Tabular Data with Diffusion Models \cite{kotelnikov_tabddpm_2022}
    & 2022-10-30 & \href{https://github.com/rotot0/tab-ddpm}{Github} \\
    \hline
    Language models are realistic tabular data generators \cite{borisov_language_2022}
    & 2022-10-12 & \href{https://github.com/kathrinse/be_great}{Github} \\
    \hline
    Ctab-gan+: Enhancing tabular data synthesis \cite{zhao_ctab-gan_2022}
    & 2022-04-01 & \href{https://github.com/Team-TUD/CTAB-GAN-Plus}{Github} \\
    \hline
    Ctab-gan: Effective table data synthesizing \cite{zhao_ctab-gan_2021}
    & 2021-05-31 & \href{https://github.com/Team-TUD/CTAB-GAN}{Github} \\
    \hline
    Modeling Tabular data using Conditional GAN \cite{xu_modeling_2019}
    & 2019-10-28 & \href{https://github.com/sdv-dev/SDV}{Github} \\
    \hline
    SMOTE: synthetic minority over-sampling technique \cite{chawla_smote_2002}
    & 2002-06-02 & \href{https://github.com/scikit-learn-contrib/imbalanced-learn}{Github} \\
    \hline
    \end{tabular}
\end{table}

\newpage
\subsection{Generación de texto en base de datos tabulares}

En la \fullref{tab-sota-text}, se listan las publicaciones en la generación de texto a partir de datos estructurados.

\begin{table}[H]
	\centering
	\caption{Estado del arte en generación de textos en base a datos}
	\label{tab-sota-text}
    \begin{tabular}{|m{20em}|r|r|}
        \hline
        \rowcolor[gray]{0.8}
        Nombre & Fecha $\downarrow$ & Modelo Base \\
        \hline
        Table-To-Text generation and pre-training with TABT5 \cite{andrejczuk_table--text_2022}
        & 2022-10-17
        & T5 \\
        \hline
        Text-to-text pre-training for data-to-text tasks \cite{kale_text--text_2020}
        & 2021-07-09
        & T5 \\
        \hline
        TaPas: Weakly supervised table parsing via pre-training \cite{herzig_tapas_2020}
        & 2020-04-21
        & Bert \\
        \hline
    \end{tabular}
\end{table}

El estado del arte en la generación de texto a partir de datos tabulares es TabT5. Es importante notar que la tabla mezcla los enfoques de \emph{Table-To-Text} y \emph{Data-To-Text}. Aunque ninguna de las publicaciones incluye código asociado, no es necesario, ya que utilizan modelos abiertos como base (T5 y Bert). Lo más relevante en estos casos es el proceso de \emph{fine-tuning}. Para completar la tarea de generar nuevos textos a partir de información inicial, esta información debe ser codificada para poder ser procesada por el modelo utilizado.

\newpage
En los siguientes ejemplos, se utilizará la \fullref{tabla-ejemplo-inputs} para ilustrar cómo se puede utilizar para generar texto utilizando los modelos de \emph{fine-tuning} mencionados anteriormente. Esta tabla representa información sobre películas, incluyendo el nombre de la película, el director, el año de lanzamiento y el género, y se utilizará para generar preguntas y respuestas a partir de la información proporcionada.
\begin{table}[H]
	\centering
	\caption{Ejemplo de tabla de entrada}
	\label{tabla-ejemplo-inputs}
    \begin{tabular}{|l|l|l|l|}
        \hline
        \rowcolor[gray]{0.8}
        Nombre de la Película & Director & Año de Lanzamiento & Género \\
        \hline
        Star Wars: Una Nueva Esperanza & George Lucas & 1977 & Ciencia ficción \\
        \hline
    \end{tabular}
        
\end{table}

Para los modelos TabT5 y TaPas, se utiliza el mismo preprocesamiento para convertir la tabla de entrada en una pregunta/tarea y respuesta \cite{andrejczuk_table--text_2022, herzig_tapas_2020}. En este ejemplo, la tabla representa información sobre películas, y se utiliza para generar una pregunta y respuesta sobre el director de la película "Star Wars: Una Nueva Esperanza". La pregunta se construye a partir de la información de la tabla, y la respuesta se espera que sea el nombre del director. Una vez que se ha generado la pregunta y la respuesta, se puede utilizar un modelo de \emph{fine-tuning} como TabT5 o TaPas para generar texto a partir de la información proporcionada. En resumen, el proceso de generación de texto a partir de datos tabulares implica la conversión de información tabular en preguntas y respuestas, y luego la utilización de modelos de \emph{fine-tuning} para generar texto a partir de estas preguntas y respuestas.
\begin{tcolorbox}[colback=white,colframe=black!50!white,title=Input]
Table: Películas
Nombre de la Película     | Director                | Año de Lanzamiento | Género 
Star Wars: Una Nueva Esperanza | George Lucas        | 1977              | Ciencia ficción 
\end{tcolorbox}
\begin{tcolorbox}[colback=white,colframe=black!50!white,title=Pregunta]
¿Qué director dirigió la película Star Wars: Una Nueva Esperanza?
\end{tcolorbox}
\begin{tcolorbox}[colback=white,colframe=black!50!white,title=Respuesta esperada]
George Lucas
\end{tcolorbox}

\newpage
En cambio, el modelo \emph{Text-to-text pre-training for data-to-text tasks} utiliza una entrada diferente, que consiste en una serie de tuplas que representan las propiedades de la entidad y sus valores correspondientes. Se espera que el modelo identifique la tupla relevante y genere una pregunta y respuesta correspondientes. Una vez generada la pregunta y respuesta, se puede utilizar el modelo de fine-tuning correspondiente para generar texto a partir de ellas. En conclusión, la generación de texto a partir de datos tabulares implica una conversión adecuada de la información de entrada en un formato apropiado para cada modelo, la identificación de la pregunta o tarea relevante y la utilización del modelo correspondiente para generar el texto resultante.

\begin{tcolorbox}[colback=white,colframe=black!50!white,title=Input]
<Star Wars: Una Nueva Esperanza, Director, George Lucas>, \\
<Star Wars: Una Nueva Esperanza, Año de Lanzamiento, 1977>, \\
<Star Wars: Una Nueva Esperanza, Género, Ciencia ficción>
\end{tcolorbox}
\begin{tcolorbox}[colback=white,colframe=black!50!white,title=Pregunta]
¿Qué director dirigió la película Star Wars: Una Nueva Esperanza?
\end{tcolorbox}
\begin{tcolorbox}[colback=white,colframe=black!50!white,title=Respuesta esperada]
George Lucas
\end{tcolorbox}
    