\chapter{Revisión Bibliográfica}
\section{Tipos de Datos}
\label{tipo-de-datos}
Los tipos de datos tienen diversas implicaciones en su generación, como su representación, almacenamiento y procesamiento. Los datos estructurados se presentan en la \fullref{tabla-tipo-datos}.

En 2012, IDC estimó que para 2020, más del 95\% de los datos serían no estructurados \cite{gantz_digital_2012}. En un análisis posterior, Kiran Adnan y Rehan Akbar \cite{adnan_analytical_2019} encontraron que el texto es el tipo de dato no estructurado que más rápido crece en las publicaciones, seguido por la imagen, el video y finalmente el audio.

La \fullref{tabla-tipo-datos} resume la lista que se encuentra en \emph{Practical Statistics for Data Scientists} \cite{bruce_practical_2020}.

\begin{table}[H]
	\centering
	\caption{Tipos de datos estructurados}
	\label{tabla-tipo-datos}
    \begin{tabular}{|l|l|m{25em}|m{9em}|}
    \hline
    \rowcolor[gray]{0.8}
    T & Sub tipo & Descripción & Ejemplos \\
    \hline
    \multicolumn{2}{|>{\columncolor[gray]{0.8}}c|}{Numérico} & Datos establecidos como números & \cellcolor[gray]{0.8} - \\
    \hline
    \cellcolor[gray]{0.8} & Continuo & Datos que pueden tomar cualquier valor en un intervalo & 3.14 metros, 1.618 litros \\
    \hline
    \cellcolor[gray]{0.8} & Discreto & Datos que solo pueden tomar valores enteros & 1 habitación, 73 años \\
    \hline
    \multicolumn{2}{|>{\columncolor[gray]{0.8}}c|}{Categórico} & Datos que pueden tomar solo un conjunto específico de valores que representan un conjunto de categorías posibles. & \cellcolor[gray]{0.8} - \\
    \hline
    \cellcolor[gray]{0.8} & Binario & Un caso especial de datos categóricos con solo dos categorías de valores & 0/1, verdadero/falso \\
    \hline
    \cellcolor[gray]{0.8} & Ordinal & Datos categóricos que tienen un ordenamiento explícito. & pequeña/ mediana/ grande \\
    \hline
    \end{tabular}
\end{table}

\newpage
\section{Privacidad de Datos}
La protección de la información es un aspecto fundamental en la generación de datos sintéticos. Aunque este aspecto puede no ser crucial cuando los datos corresponden a temas como recetas o automóviles, resulta esencial cuando se trata de información relacionada con individuos \cite{bruce_practical_2020}. Por esta razón, el resguardo de la información es un tema de importancia para entidades como Equifax, que gestionan una gran cantidad de conjuntos de datos con contenido personal.

\subsection{Tipo de datos a ser protegidos}
Para identificar qué campos de datos son significativos desde el punto de vista de la privacidad, se puede recurrir a la definición resumida en la \fullref{data-relevante} del texto \emph{Data privacy: Definitions and techniques} \cite{de_capitani_di_vimercati_data_2012}.

\begin{table}[H]
	\centering
	\caption{Niveles de revelación y ejemplos}
	\label{data-relevante}
    \begin{tabular}{|m{15em}|m{25em}|}
    \hline
    \rowcolor[gray]{0.8}
    Tipo de revelación & Descripción \\
    \hline
    \textbf{Identificadores} 
    & Atributos que identifican de manera única a individuos (por ejemplo, SSN, RUT, DNI). \\
    \hline
    \textbf{Cuasi-identificadores (QI)} 
    & Atributos que, en combinación, pueden identificar a individuos, o reducir la incertidumbre sobre sus identidades (por ejemplo, fecha de nacimiento, género y código postal). \\
    \hline
    \textbf{Atributos confidenciales} 
    & Atributos que representan información sensible (por ejemplo, enfermedad). \\
    \hline
    \textbf{Atributos no confidenciales} 
    & Atributos que los encuestados no consideran sensibles y cuya divulgación es inofensiva (por ejemplo, color favorito). \\
    \hline
    \end{tabular}
\end{table}

\newpage
\subsection{Tipos de riesgos de divulgación}
Los tipos de divulgación definidos en \emph{Practical Synthetic Data Generation} \cite{bruce_practical_2020} están resumidos en la \fullref{relevantes-definiciones}.

\begin{table}[H]
	\centering
	\caption{Tipos de Riesgos de Divulgación y sus Descripciones}
	\label{relevantes-definiciones}
    \begin{tabular}{|m{15em}|m{25em}|}
    \hline
    \rowcolor[gray]{0.8}
    Tipo de revelación & Descripción \\
    \hline
    \textbf{Divulgación de identidad} 
    & Este riesgo se refiere a la posibilidad de que un atacante pueda identificar la información de un individuo a partir de los datos publicados, utilizando técnicas de filtrado para reducir las posibilidades hasta un solo individuo.\\
    \hline
    \textbf{Divulgación de nueva información} 
    & Este riesgo comprende el riesgo de Divulgación de Identidad, y además, implica la adquisición de información adicional sobre el individuo a partir de los datos publicados.\\
    \hline
    \textbf{Divulgación de Atributos} 
    & Este riesgo se da cuando, aunque no se pueda identificar a un individuo, se puede descubrir un atributo común en varios registros, lo que permite obtener información sensible acerca de un grupo de individuos.\\
    \hline
    \textbf{Divulgación Inferencial} 
    & Este riesgo se refiere a la posibilidad de inferir información sensible a partir de los datos publicados, mediante el uso de técnicas de análisis estadístico o de aprendizaje automático. Por ejemplo, si después de filtrar todos los registros, el 80\% de los registros con las mismas características tienen cáncer, se podría inferir que el individuo buscado puede tener cáncer.\\
    \hline
    \end{tabular}
\end{table}


Adicionalmente se deben establecer dos conceptos relevantes ante el análisis de revelación de información:
\begin{enumerate}
    \item En términos prácticos, normalmente los datos sintéticos buscan tener cierta permeabilidad con respecto a la \textbf{Divulgación Inferencial}, ya que se quiere que estadísticamente sean similares. Además, se busca proteger la identidad de los individuos, pero esta no es la única condición, también se busca proteger aquellos atributos que pueden ser sensibles, como las enfermedades. A todo este conjunto se le denomina \textbf{Revelación de identidad significativa}. Es particularmente riesgoso por la posibilidad de discriminación hacia ciertos grupos que cumplen con los atributos criterio.
    \item Los mismos atributos pueden tener más relevancia para ciertos grupos de la población que para otros. El ejemplo que se indica en \cite{el_emam_practical_2020} es que, debido a que el número de hijos igual a 2 es menos frecuente en una etnia que en otra (40\% en la primera y 10\% en la segunda), ese dato es más relevante en la segunda. Esto se debe a que es un factor que filtra mejor y, por lo tanto, puede permitir un mejor conocimiento de ese grupo específico. A esto se le denomina \textbf{Definición de información ganada}.
\end{enumerate}
\newpage

\subsection{Regulación de datos sintéticos}
Debido a que los datos sintéticos son basados en datos reales, pueden ser afectos a las regulaciones de sobre protección de datos \cite{bruce_practical_2020}. Los nuevos datos podrían ser afectos por:
\begin{enumerate}
    \item \href{https://dvbi.ru/Portals/0/DOCUMENTS_SHARE/RISK_MANAGEMENT/EBA/GDPR_eng_rus.pdf}{Regulation (EU) 2016/679 of the European Parliament and of the Council} \cite{regulation_regulation_2016}, si el proceso de generación de datos sintéticos a menudo implica el uso de datos personales reales como entrada. En este caso, el GDPR sería relevante. Las organizaciones que utilicen datos personales para generar datos sintéticos deben garantizar que este proceso cumple con los principios del GDPR, como la minimización de datos (sólo se deben utilizar los datos necesarios) y la limitación de la finalidad (los datos sólo se deben utilizar para el propósito para el que se recogieron).
    \item \href{https://heinonline.org/HOL/LandingPage?handle=hein.journals/jtlp23&div=5&id=&page=}{The California consumer privacy act: Towards a European-style privacy regime in the United States} \cite{pardau_california_2018}
    \item \href{http://www.eolusinc.com/pdf/hipaa.pdf}{Health insurance portability and accountability act of 1996} \cite{act_health_1996}
\end{enumerate}

\subsection{Protección de Privacidad}
En la \fullref{metricas-privacidad} se listas las utilizadas en diferentes publicaciones para determinar la privacidad efectiva de los conjuntos generados.

\begin{table}[H]
	\centering
	\caption{Metricas de privacidad}
	\label{metricas-privacidad}
    \begin{tabular}{|m{15em}|m{25em}|}
    \hline
    \rowcolor[gray]{0.8}
    Tipo de revelación & Descripción \\
    \hline
    \textbf{\emph{Distance to Closest Record} (DCR)} 
    & DRC se utiliza para medir la distancia euclidiana entre cualquier registro sintético y su vecino real más cercano. Idealmente, cuanto mayor sea la DCR, menor será el riesgo de violación de la privacidad. Además, se calcula el percentil 5 de esta métrica para proporcionar una estimación robusta del riesgo de privacidad. \cite{zhao_ctab-gan_2021} \\
    \hline
    \textbf{\emph{Nearest Neighbour Distance Ratio} (NNDR)} 
    & NNDR mide la relación entre la distancia euclidiana del vecino real más cercano y el segundo más cercano para cualquier registro sintético correspondiente. Esta relación se encuentra dentro del intervalo [0, 1]. Los valores más altos indican una mayor privacidad. Los bajos valores de NNDR entre datos sintéticos y reales pueden revelar información sensible del registro de datos reales más cercano. \cite{zhao_ctab-gan_2021} \\
    \hline
    \end{tabular}
\end{table}

\newpage
\section{Generación de Datos Sintéticos}

Los datos sintéticos, aunque no son datos reales, se generan con la intención de preservar ciertas propiedades de los datos originales. La utilidad de los datos sintéticos se mide por su capacidad para servir como un sustituto efectivo de los datos originales \cite{bruce_practical_2020}. Basándose en el uso de los datos originales, los datos sintéticos se pueden clasificar en tres categorías: aquellos que se basan en datos reales, los que no se basan en datos reales, y los híbridos.

\textbf{Datos basados en datos reales}: utilizan modelos que aprenden la distribución de los datos originales para generar nuevos puntos de datos similares.

\textbf{Datos no basados en datos reales}: utilizan conocimientos del mundo real. Por ejemplo, se podría formar un nombre completo seleccionando aleatoriamente un nombre y un apellido de un conjunto predefinido.

\textbf{Híbridos}: estos combinan técnicas de imitación de distribución con algunos campos que no derivan de los datos reales. Esto puede ser especialmente útil cuando se intenta desacoplar las distribuciones de datos que podrían ser sensibles o generar discriminación, como la información sobre la etnia.

En la \fullref{tipo-de-datos}, se revisaron los datos estructurados. Si bien cada tipo puede tener muchas representaciones, por ejemplo, los datos continuos podrían considerarse como \emph{float}, \emph{datetime} o incluso intervalos personalizados, como de 0 a 1. Sobre estos datos estructurados, se pueden generar estructuras para unirlos.

Entre las estructuras más comunes se encuentran las matrices bidimensionales (datos tabulares) y los arreglos, que permiten matrices de muchas dimensiones e incluso estructuras complejas que pueden mezclar todas las estructuras previas.

Debido al objetivo, se detallan solo los modelos que permiten abordar la generación de datos tabulares y texto basados en datos reales.
\newpage
\subsection{Generación de datos tabulares}
En la \fullref{tab-sota-tab}, se resumen las últimas publicaciones sobre generación de datos tabulares, indicando la fecha de publicación y si se puede acceder al código fuente o no, a febrero de 2023.

\begin{table}[H]
	\centering
	\caption{Estado del arte en generación de datos tabulares}
	\label{tab-sota-tab}
    \begin{tabular}{|m{25em}|r|r|}
    \hline
    \rowcolor[gray]{0.8}
    Nombre & Fecha $\downarrow$ & Código \\
    \hline
    REaLTabFormer: Generating Realistic Relational and Tabular Data using Transformers \cite{solatorio_realtabformer_2023}
    & 2023-02-04 & \href{https://github.com/avsolatorio/REaLTabFormer}{Github} \\
    \hline
    PreFair: Privately Generating Justifiably Fair Synthetic Data \cite{pujol_prefair_2022}
    & 2022-12-20 & \\
    \hline
    GenSyn: A Multi-stage Framework for Generating Synthetic Microdata using Macro Data Sources \cite{acharya_gensyn_2022}
    & 2022-12-08 & \href{https://github.com/Angeela03/GenSyn}{Github} \\
    \hline
    TabDDPM: Modelling Tabular Data with Diffusion Models \cite{kotelnikov_tabddpm_2022}
    & 2022-10-30 & \href{https://github.com/rotot0/tab-ddpm}{Github} \\
    \hline
    Language models are realistic tabular data generators \cite{borisov_language_2022}
    & 2022-10-12 & \href{https://github.com/kathrinse/be_great}{Github} \\
    \hline
    Ctab-gan+: Enhancing tabular data synthesis \cite{zhao_ctab-gan_2022}
    & 2022-04-01 & \href{https://github.com/Team-TUD/CTAB-GAN-Plus}{Github} \\
    \hline
    Ctab-gan: Effective table data synthesizing \cite{zhao_ctab-gan_2021}
    & 2021-05-31 & \href{https://github.com/Team-TUD/CTAB-GAN}{Github} \\
    \hline
    Modeling Tabular data using Conditional GAN \cite{xu_modeling_2019}
    & 2019-10-28 & \href{https://github.com/sdv-dev/SDV}{Github} \\
    \hline
    Smote: synthetic minority over-sampling technique \cite{chawla_smote_2002}
    & 2002-06-02 & \href{https://github.com/scikit-learn-contrib/imbalanced-learn}{Github} \\
    \hline
    \end{tabular}
\end{table}

\newpage
\subsection{Generación de texto en base de datos tabulares}

En la \fullref{tab-sota-text}, se listan las publicaciones en la generación de texto a partir de datos estructurados.

\begin{table}[H]
	\centering
	\caption{Estado del arte en generación de textos en base a datos}
	\label{tab-sota-text}
    \begin{tabular}{|m{20em}|r|r|}
        \hline
        \rowcolor[gray]{0.8}
        Nombre & Fecha $\downarrow$ & Modelo Base \\
        \hline
        Table-To-Text generation and pre-training with TABT5 \cite{andrejczuk_table--text_2022}
        & 2022-10-17
        & T5 \\
        \hline
        Text-to-text pre-training for data-to-text tasks \cite{kale_text--text_2020}
        & 2021-07-09
        & T5 \\
        \hline
        TaPas: Weakly supervised table parsing via pre-training \cite{herzig_tapas_2020}
        & 2020-04-21
        & Bert \\
        \hline
    \end{tabular}
\end{table}

El estado del arte en la generación de texto a partir de datos tabulares es TabT5. Es importante notar que la tabla mezcla los enfoques de \emph{Table-To-Text} y \emph{Data-To-Text}. Aunque ninguna de las publicaciones incluye código asociado, no es necesario, ya que utilizan modelos abiertos como base (T5 y Bert). Lo más relevante en estos casos es el proceso de \emph{fine-tuning}. Para completar la tarea de generar nuevos textos a partir de información inicial, esta información debe ser codificada para poder ser procesada por el modelo utilizado.

La diferencia entre \emph{Table-To-Text} y \emph{Data-To-Text} radica en el formato de información de entrada. en \emph{Table-To-Text} es una tabla con multiples filas y en \emph{Data-To-Text} corresponde a un solo objeto con sus propiedades. A continuación ejemplos de entradas de los modelos.

\newpage
En los siguientes ejemplos, se utilizará la \fullref{tabla-ejemplo-inputs} para ilustrar cómo se puede utilizar para generar texto utilizando los modelos de \emph{fine-tuning} mencionados anteriormente. Esta tabla representa información sobre películas, incluyendo el nombre de la película, el director, el año de lanzamiento y el género, y se utilizará para generar preguntas y respuestas a partir de la información proporcionada.
\begin{table}[H]
	\centering
	\caption{Ejemplo de tabla de entrada}
	\label{tabla-ejemplo-inputs}
    \begin{tabular}{|l|l|l|l|}
        \hline
        \rowcolor[gray]{0.8}
        Nombre de la Película & Director & Año de Lanzamiento & Género \\
        \hline
        Star Wars: Una Nueva Esperanza & George Lucas & 1977 & Ciencia ficción \\
        \hline
    \end{tabular}
        
\end{table}

Para los modelos TabT5 y TaPas, se utiliza el mismo preprocesamiento para convertir la tabla de entrada en una pregunta/tarea y respuesta \cite{andrejczuk_table--text_2022, herzig_tapas_2020}. En este ejemplo, la tabla representa información sobre películas, y se utiliza para generar una pregunta y respuesta sobre el director de la película "Star Wars: Una Nueva Esperanza". La pregunta se construye a partir de la información de la tabla, y la respuesta se espera que sea el nombre del director. Una vez que se ha generado la pregunta y la respuesta, se puede utilizar un modelo de \emph{fine-tuning} como TabT5 o TaPas para generar texto a partir de la información proporcionada. En resumen, el proceso de generación de texto a partir de datos tabulares implica la conversión de información tabular en preguntas y respuestas, y luego la utilización de modelos de \emph{fine-tuning} para generar texto a partir de estas preguntas y respuestas.
\begin{tcolorbox}[colback=white,colframe=black!50!white,title=Input]
Table: Películas
Nombre de la Película     | Director                | Año de Lanzamiento | Género 
Star Wars: Una Nueva Esperanza | George Lucas        | 1977              | Ciencia ficción 
\end{tcolorbox}
\begin{tcolorbox}[colback=white,colframe=black!50!white,title=Pregunta]
¿Qué director dirigió la película Star Wars: Una Nueva Esperanza?
\end{tcolorbox}
\begin{tcolorbox}[colback=white,colframe=black!50!white,title=Respuesta esperada]
George Lucas
\end{tcolorbox}

\newpage
En cambio, el modelo \emph{Text-to-text pre-training for data-to-text tasks} \cite{kale_text--text_2020} utiliza una entrada diferente, que consiste en una serie de tuplas que representan las propiedades de la entidad y sus valores correspondientes. Se espera que el modelo identifique la tupla relevante y genere una pregunta y respuesta correspondientes. Una vez generada la pregunta y respuesta, se puede utilizar el modelo de fine-tuning correspondiente para generar texto a partir de ellas. En conclusión, la generación de texto a partir de datos tabulares implica una conversión adecuada de la información de entrada en un formato apropiado para cada modelo, la identificación de la pregunta o tarea relevante y la utilización del modelo correspondiente para generar el texto resultante.

\begin{tcolorbox}[colback=white,colframe=black!50!white,title=Input]
<Star Wars: Una Nueva Esperanza, Director, George Lucas>, \\
<Star Wars: Una Nueva Esperanza, Año de Lanzamiento, 1977>, \\
<Star Wars: Una Nueva Esperanza, Género, Ciencia ficción>
\end{tcolorbox}
\begin{tcolorbox}[colback=white,colframe=black!50!white,title=Pregunta]
¿Qué director dirigió la película Star Wars: Una Nueva Esperanza?
\end{tcolorbox}
\begin{tcolorbox}[colback=white,colframe=black!50!white,title=Respuesta esperada]
George Lucas
\end{tcolorbox}

\newpage
\section{Metricas de evaluación}
Es importante destacar que no todas estas métricas son aplicables a todos los tipos de datos y modelos, y que la selección de las métricas a utilizar debe ser cuidadosamente considerada en función de las necesidades y objetivos específicos de cada caso de estudio.
A continuación presentan algunas de las posibles a considerar para medir la similitud, privacidad y utilidad en la evaluación de los conjuntos de datos sintéticos generados.

\subsection{SDMetrics}
SDMetrics es una herramienta integral para la evaluación de conjuntos de datos sintéticos. Esta herramienta implementa dos enfoques distintos para el cálculo de métricas: el Reporte y el Diagnóstico \cite{noauthor_sdmetrics_nodate}.


\subsubsection{SDMetrics Report}

El informe de SDMetrics genera una puntuación de evaluación para un conjunto de datos sintéticos al compararlo con el conjunto de datos reales. Esta puntuación se compone de dos componentes: \emph{Column Shapes} y \emph{Column Pair Trends}.

\textbf{Column Shapes} se compone de las puntuaciones \emph{KSComplement} para columnas numéricas y \emph{TVComplement} para columnas categóricas.

\textbf{TVComplement} se centra en la similitud entre una columna real y una columna sintética en términos de sus formas de distribución.

\textbf{KSComplement} utiliza la estadística de Kolmogorov-Smirnov para calcular la máxima diferencia entre las funciones de distribución acumulativa de dos distribuciones numéricas. 

\rule{\textwidth}{0.5pt} 

\textbf{Column Pair Trends} se compone de \emph{CorrelationSimilarity} para columnas numéricas y \emph{ContingencySimilarity} para columnas categóricas o combinaciones de columnas categóricas y numéricas.

\textbf{CorrelationSimilarity} mide la correlación entre un par de columnas numéricas y calcula la similitud entre los datos reales y sintéticos al comparar las tendencias de las distribuciones bidimensionales. 

\textbf{ContingencySimilarity} mide la similitud entre dos variables categóricas utilizando la tabla de contingencia y la estadística del coeficiente de contingencia, proporcionando una medida de la dependencia entre las dos variables.

Cada una de estas métricas evalúa un aspecto diferente de la calidad de los datos sintéticos, proporcionando información valiosa sobre diversas características de los datos. \textbf{TVComplement} se enfoca en la distribución marginal o el histograma unidimensional de la columna, mientras que \textbf{KSComplement} se centra en la diferencia entre las funciones de distribución acumulativa de dos distribuciones numéricas. \textbf{CorrelationSimilarity} mide la similitud en la correlación entre un par de columnas numéricas en los datos reales y sintéticos, y \textbf{ContingencySimilarity} mide la similitud en la dependencia entre dos variables categóricas en los datos reales y sintéticos utilizando la tabla de contingencia y la estadística del coeficiente de contingencia. Juntas, estas métricas ofrecen una evaluación completa de la calidad de los datos sintéticos.

\subsubsection{SDMetrics Diagnostic}

SDMetrics Diagnostic emplea una serie de métricas para arrojar luz sobre la calidad de los datos sintéticos, haciendo especial énfasis en \emph{Coverage} y \emph{Boundaries}.

\textbf{Coverage} comprende la media aritmética de \emph{RangeCoverage} y \emph{CategoryCoverage}.

\textbf{RangeCoverage} cuantifica la proporción del rango de valores posibles de una característica que está representada en los datos. Se define como la relación entre el rango de valores observados y el rango de valores posibles para la característica en cuestión. Esta métrica es útil para determinar si los datos proporcionan una representación adecuada de la diversidad de valores que la característica puede adoptar.

\textbf{CategoryCoverage} mide la proporción de categorías posibles de una característica categórica que están representadas en los datos. Se calcula como la relación entre el número de categorías observadas y el número total de categorías posibles para dicha característica. Esta métrica es útil para evaluar si los datos proporcionan una representación adecuada de la diversidad de categorías que la característica categórica puede adoptar.

\rule{\textwidth}{0.5pt} 

\textbf{Boundaries} se considera únicamente para la métrica \emph{BoundaryAdherence}, calculada como el promedio para todas las columnas numéricas.

\textbf{BoundaryAdherence} evalúa la proporción de puntos de datos que caen dentro de los límites especificados para una característica. Se determina como la relación entre el número de puntos de datos que caen dentro de los límites y el número total de puntos de datos. Esta métrica es útil para evaluar si los datos se ajustan a los límites especificados para una característica, lo cual puede ser crucial en contextos donde se espera que la característica observe ciertos valores o límites específicos.

En resumen, \emph{SDMetrics Diagnostic} utiliza RangeCoverage, BoundaryAdherence y CategoryCoverage para evaluar la calidad de los datos tabulares sintéticos. Estas métricas aportan una visión detallada de la cobertura de los datos en términos de rango de valores, límites y categorías posibles, lo que puede ayudar a detectar potenciales problemas en la calidad de los datos.
\newpage

\subsection{Conjuntos Estadísticos}

\begin{longtable}{|m{7em}|m{30em}|}
    \caption{Listado de conjunto estadísticos} 
    \label{tab-stats} \\
    \hline
    \rowcolor[gray]{0.8}
    Nombre & Descripción \\
    \hline
    \endfirsthead
    
    \hline
    \rowcolor[gray]{0.8}
    Nombre & Descripción \\
    \hline
    \endhead
    
    \hline \multicolumn{2}{|r|}{{Continúa en la siguiente página}} \\ \hline
    \endfoot
    
    \hline \hline
    \endlastfoot
    
    \hline
    Media (Mean) & La suma de todos los valores dividido por el número total de valores \\
    \hline
    Mediana (Median) & El valor que se encuentra en el centro de un conjunto de datos ordenados de menor a mayor. Es decir, la mitad de los valores son mayores que la mediana y la otra mitad son menores \\
    \hline
    Moda (Mode) & El valor que aparece con mayor frecuencia en un conjunto de datos \\
    \hline
    Mínimo (Min) & El valor más pequeño en un conjunto de datos \\
    \hline
    Máximo (Max) & El valor más grande en un conjunto de datos \\
    \hline
    Percentil (25, 75) (Percentile) & El valor tal que P (25 o 75) por ciento de los datos son menores que él, y el restante (100 - P) por ciento son mayores. Cuando P = 50, el percentil es la mediana \\
    \hline
    Media Truncada (Trimmed Mean) & El promedio de todos los valores, una vez que se han eliminado un porcentaje de los valores más bajos y un porcentaje de los valores más altos \\
    \hline
    Outlier & Un valor que se encuentra muy lejos de la mayoría de los valores en un conjunto de datos \\
    \hline
    Desviación (Deviation) & La diferencia entre un valor observado y la estimación de ese valor \\
    \hline
    Varianza (Variance) & La medida de cuán dispersos están los valores en un conjunto de datos. Es la suma de los cuadrados de las desviaciones desde la media dividido por n - 1, donde n es el número de valores \\
    \hline
    Desviación Estándar (SD) & La raíz cuadrada de la varianza \\
    \hline
    Desviación Absoluta Media (MAD) & La media de los valores absolutos de las desviaciones desde la media \\
    \hline
    Rango (Range) & La diferencia entre el valor más grande y el valor más pequeño en un conjunto de datos \\
    \hline
    Tablas de Frecuencia (Frequency Tables) & Un método para resumir los datos al contar cuántas veces ocurre cada valor en un conjunto de datos \\
    \hline
    Probabilidad (Probability) & La medida de la posibilidad de que un evento ocurra. Se establece como el número de ocurrencias de un valor dividido por el número total de ocurrencias \\
    \hline
    Tabla de Contingencia (Contingency Table) & Una tabla que muestra la distribución conjunta de dos o más variables categóricas \\
    \hline
    Correlación& Una medida estadística que indica cómo dos variables numéricas están relacionadas entre sí. Puede variar entre -1 y 1 \\
    \hline
    Distribución Estratificada & Una comparación de la distribución de datos para diferentes estratos \\
    \hline
    Comparación de Modelos Predictivos Multivariables & Un método para comparar varios modelos predictivos que involucran múltiples variables. Implica la construcción de modelos separados para cada variable objetivo y comparar la curva ROC (Receiver Operating Characteristic) para cada modelo \\
    \hline
    Distinguibilidad & Un método para evaluar la calidad de los conjuntos de datos sintéticos. Implica la creación de un modelo que intenta distinguir entre conjuntos de datos reales y sintéticos. Un buen conjunto sintético es aquel que el modelo no puede distinguir de los datos reales \\
    \hline
    Kullback-Leibler & Una medida de la divergencia entre dos distribuciones de probabilidad \\
    \hline
    Pairwise Correlation & Una medida de la similitud entre dos conjuntos de datos que compara las correlaciones de cada par de variables en los conjuntos de datos \\
    \hline
    Log-Cluster & Un método para evaluar la calidad de los conjuntos de datos sintéticos que compara la estructura de los conjuntos de datos reales y sintéticos mediante el uso de clustering \\
    \hline
    Cobertura de Soporte (Support Coverage) & Una medida de qué tan bien los datos sintéticos representan la distribución de los datos reales. Se mide como la proporción de variables en el conjunto de datos real que están representadas en el conjunto de datos sintéticos \\
    \hline
    Cross-Classification & Un método para evaluar la calidad de los conjuntos de datos sintéticos que compara la precisión de los modelos predictivos construidos a partir de los conjuntos de datos reales y sintéticos \\
    \hline
    Métrica de Revelación Involuntaria & Una medida de qué tan bien se protege la privacidad de los datos en un conjunto de datos sintético. Se mide como la tasa de predicciones correctas de atributos sensibles de un individuo en un conjunto de datos sintético \\
    \hline
    
    
\end{longtable}
